\documentclass[11pt,oneside]{book}

% --------------------------------------------------
% Basic geometry and typography
% --------------------------------------------------
\usepackage[a4paper,margin=1in]{geometry}
\usepackage[utf8]{inputenc}
\usepackage[T1]{fontenc}
\usepackage[sc]{mathpazo} % Palatino-like text + maths
\linespread{1.05}
\setlength{\parskip}{0.6em}
\setlength{\parindent}{0pt}

% --------------------------------------------------
% Maths and symbols
% --------------------------------------------------
\usepackage{amsmath,amssymb,amsthm,mathtools}
\usepackage{bm}

\newcommand{\vect}[1]{\mathbf{#1}}
\newcommand{\mat}[1]{\mathbf{#1}}
\newcommand{\E}{\mathbb{E}}
\newcommand{\Var}{\mathrm{Var}}
\newcommand{\Cov}{\mathrm{Cov}}
\newcommand{\tr}{\mathrm{tr}}

% Key idea box (using tcolorbox)
\usepackage{tcolorbox}
\newtcolorbox{keyidea}{
  colback=blue!5!white,
  colframe=blue!75!black,
  fonttitle=\bfseries,
  title=Key Idea
}

% --------------------------------------------------
% Graphics
% --------------------------------------------------
\usepackage{graphicx}
\graphicspath{{figures/}}  % Look for figures in this subdirectory

% --------------------------------------------------
% Code listings
% --------------------------------------------------
\usepackage{listings}

\lstdefinestyle{pythonstyle}{
  language=Python,
  basicstyle=\ttfamily\small,
  keywordstyle=\bfseries,
  commentstyle=\itshape,
  showstringspaces=false,
  frame=single,
  framerule=0.3pt,
  xleftmargin=1em,
  xrightmargin=1em,
  belowskip=1em,
  aboveskip=1em,
  breaklines=true,
  breakatwhitespace=true,
}

\lstnewenvironment{pycode}[1][]
  {\lstset{style=pythonstyle,numbers=left,numberstyle=\tiny,#1}}
  {}

% --------------------------------------------------
% Colours, links, and boxes
% --------------------------------------------------
\usepackage{xcolor}
\definecolor{myblue}{RGB}{0,60,130}
\definecolor{mygreen}{RGB}{0,120,80}

\usepackage[
  colorlinks=true,
  linkcolor=myblue,
  citecolor=mygreen,
  urlcolor=myblue
]{hyperref}
\usepackage[capitalize,nameinlink]{cleveref}

\newtcolorbox{examplebox}{
  colback=mygreen!3,
  colframe=mygreen,
  title={Example},
}

% --------------------------------------------------
% Theorem-like environments (optional)
% --------------------------------------------------
\theoremstyle{definition}
\newtheorem{definition}{Definition}[chapter]
\newtheorem{example}{Example}[chapter]

\theoremstyle{plain}
\newtheorem{theorem}{Theorem}[chapter]

% --------------------------------------------------
% Indexes (subject index + author index)
% --------------------------------------------------
\usepackage{imakeidx}
\makeindex[name=subject,title=Subject Index,columns=2]
\makeindex[name=authors,title=Author Index,columns=2]

\usepackage{epigraph}
\setlength{\epigraphwidth}{0.85\textwidth}
\renewcommand{\epigraphflush}{center}
\renewcommand{\sourceflush}{center}

% --------------------------------------------------
% Title and author
% --------------------------------------------------

\title{\textbf{Seeing the Shape}\\[0.5em]
  \large A Geometric Introduction to Multivariate Quantitative Genetics}
\author{Daniel Ortiz-Barrientos\\[0.5em]
  \normalsize School of the Environment\\
  \normalsize The University of Queensland\\[0.3em]
  \normalsize ARC Centre of Excellence for Plant Success\\
  \normalsize in Nature and Agriculture}
\date{\today}

\begin{document}

\frontmatter
\maketitle

\epigraph{\textit{The shape of the ellipsoid and the direction of the arrow---these two things, together, determine what will happen. The G matrix is potential; selection is actuality. Their interaction is evolution.}}{}

% --------------------------------------------------
% Copyright page
% --------------------------------------------------
\thispagestyle{empty}
\vspace*{\fill}

\begin{flushleft}
\textbf{Seeing the Shape: A Geometric Introduction to Multivariate Quantitative Genetics}

\vspace{1em}
\textcopyright{} 2025 Daniel Ortiz-Barrientos

\vspace{1em}
This work is licensed under a Creative Commons Attribution-NonCommercial-ShareAlike 4.0 International License (CC BY-NC-SA 4.0). You are free to share and adapt this material for non-commercial purposes, provided you give appropriate credit and distribute any derivative works under the same license.

\vspace{0.5em}
\url{https://creativecommons.org/licenses/by-nc-sa/4.0/}

\vspace{2em}
School of the Environment\\
The University of Queensland\\
Brisbane, Queensland, Australia

\vspace{2em}
\texttt{https://www.ortizbarrientoslab.org}

\vspace{2em}
First edition: December 2025

\vspace{1em}
\small{Typeset in Palatino using \LaTeX.}
\end{flushleft}

\vspace*{\fill}
\cleardoublepage

\chapter*{Preface}

These notes began as a companion to my reading of Mark Blows' paper \emph{``A tale of two matrices.''} They grew while teaching Biostatistics at The University of Queensland, where I found that students comfortable with regression, $t$-tests, and ANOVA would lose their footing as soon as covariance matrices, eigenvalues, and diagonalisation appeared. The notes have expanded through conversations with Nicholas O'Brien, Mark Blows, Mark Cooper, Jan Engesltaedter, Pamela Burrage, and Kevin Burrage.

The aim here is to build a different path into that material. We start from distance, which is intuitive, and we show how the standard tools of multivariate evolutionary biology---phenotypic and genetic covariance matrices, Mahalanobis distance, PCA, canonical analyses, and the matrices underlying multivariate selection---are all expressions of the same geometric story.

By the end of these notes you should be able to
\begin{itemize}
  \item interpret eigenvectors and eigenvalues geometrically and biologically;
  \item perform diagonalisation in simple cases and understand each step;
  \item decide when diagonalisation is the right tool for a given biological question.
\end{itemize}

The guiding principle is simple: \emph{symmetric matrices describe shapes}. The algebra is a precise language for those shapes. Whenever the symbols become opaque, the right move is to go back to the picture and draw the ellipse.

These are living lecture notes. They will change. Feedback is welcome.

\subsection*{Intended audience and prerequisites}

This book is written for biologists who are already comfortable with basic
statistics and are willing to learn some linear algebra along the way.
A good starting point is the level of a typical second- or third-year
biostatistics course: means and variances, covariance and correlation,
simple linear regression, $t$-tests, and ANOVA.

You do \emph{not} need a full course in linear algebra before starting,
but you will get more from the later chapters if you have met vectors,
matrices, and the idea of an eigenvalue at least once. The appendix
\emph{Mathematical and Statistical Background} collects the minimum
machinery assumed in the main text: vectors and matrices as things you can
multiply, eigenvalues and eigenvectors for symmetric matrices, basic
probability language (variance, covariance, multivariate normal), and the
definition of selection gradients and Lande's equation. Readers who feel
rusty on any of these topics are encouraged to skim that appendix early and
return to it as needed.

The book is designed to be readable in two passes. On a first pass, you can
focus on the geometric story: pictures of trait space, ellipses for
covariance, the idea of whitening, and the shapes defined by $\mat{G}$,
$\mat{P}$, and $\boldsymbol{\gamma}$. On a second pass, you can pay closer
attention to the algebra, work through the derivations in detail, and use
the worked examples and code in the later chapters to practise complete
analyses from data to interpretation.

\subsection*{How this book is organised}

This book is written for biologists who want to understand the geometry behind quantitative genetics. It is organised in four parts that build on each other: we start with pictures of trait space, move to distance and covariance, then to natural axes and whitening, and finally to genetic and fitness objects such as the $\mat{G}$ matrix and curved fitness surfaces.:contentReference[oaicite:1]{index=1}

\medskip

\textbf{Part I: Geometry of trait space}

The first part introduces the basic geometric language.

In \textbf{Chapter 1: Points and Trait Space} we learn to see individual phenotypes as points in a trait space, samples as clouds of points, and trait differences as arrows between points. 

In \textbf{Chapter 2: Vectors, Coordinates, and Angles} we put coordinates on this space, introduce vectors as directed differences, and use the dot product to talk about lengths and angles between trait combinations. 

In \textbf{Chapter 3: Matrices as Machines That Move Vectors} we treat matrices as machines that move vectors, and we see how simple matrices can stretch, rotate, and shear trait space in ways that we will later use to describe variance, covariance, and selection.

\medskip

\textbf{Part II: Distance, variance, and covariance}

The second part explains why distance matters for variation and why simple Euclidean distance is sometimes misleading.

In \textbf{Chapter 4: Distance and Why We Square It} we link straight-line distance to Pythagoras’ theorem, show how squaring distance leads naturally to variance as “average squared distance from the mean”, and connect this idea to the familiar one-dimensional formulas used in statistics. 

In \textbf{Chapter 5: When Euclidean Distance Fails} we see concrete examples where Euclidean distance ignores scale differences between traits, misses correlations between traits, and does not match the probability structure of our data, motivating the need for a more flexible metric. 

In \textbf{Chapter 6: Covariance and Mahalanobis Distance} we introduce the covariance matrix as a shape that summarises how traits vary together and define Mahalanobis distance by placing this matrix between two vectors, leading to ellipses (and ellipsoids) that match the spread and correlation of the data.

\medskip

\textbf{Part III: Natural axes, diagonalisation, and whitening}

The third part shows how to find natural axes of variation and how to rescale trait space so that phenotypic variance looks spherical.

In \textbf{Chapter 7: Diagonalisation and Natural Axes} we introduce eigenvalues and eigenvectors as directions in which a matrix only stretches and does not rotate, and we relate these natural axes to principal components and to the shape of covariance ellipses. 

In \textbf{Chapter 8: Whitening and the P-sphere} we transform trait space using the phenotypic covariance matrix $\mat{P}$ so that phenotypic variation becomes a sphere (the P-sphere), and in this whitened space we see how a transformed genetic matrix $\mat{G}^\ast$ encodes directional heritability along different directions of trait change.

\medskip

\textbf{Part IV: Genetic and fitness geometry}

The final part applies these geometric tools to genetic and fitness objects.

In \textbf{Chapter 9: The $\mat{G}$ Matrix and the Genetic Ellipsoid} we describe the additive genetic covariance matrix $\mat{G}$ as an ellipsoid that channels evolutionary change, introduce key quantities such as the leading eigenvector (often called $g_{\max}$), and connect this geometric view to evolvability, constraint, and effective dimensionality. 

In \textbf{Chapter 10: The Fitness Surface and $\boldsymbol{\gamma}$} we represent local fitness surfaces as curved (paraboloid) shapes over trait space, introduce the curvature matrix $\boldsymbol{\gamma}$, and explain how curvature interacts with genetic variation to influence the paths and limits of evolutionary change.

In \textbf{Chapter 11: PCA, MANOVA, and Projections} we connect the geometric picture to standard multivariate methods, showing how eigenstructure underlies PCA, MANOVA, discriminant analysis, and related projection techniques.

\medskip

\textbf{Part V: Practice and extensions}

In \textbf{Chapter 12: Worked Examples: Complete Analyses} we walk through complete analyses from raw data to biological interpretation, combining $\mat{G}$, $\mat{P}$, $\boldsymbol{\beta}$, and $\boldsymbol{\gamma}$.

In \textbf{Chapter 13: Directional Heritability and the Geometry of Constraint} we develop the distributional view of directional heritability, link it to the eigenstructure of $\mat{G}^\ast$, and discuss constraint heterogeneity and its implications for evolution and breeding.

A final epilogue reflects on the geometric perspective, and the back matter collects references, the mathematical appendix, and hints for selected exercises.

\subsection*{Code companion and reproducibility}

All code used to generate figures and worked examples is available in a companion repository:

\begin{center}
\url{https://github.com/dortizbarrientos/seetheshape}
\end{center}

The repository contains parallel \texttt{python/} and \texttt{R/} directories with annotated scripts organised by chapter, together with figure-generation scripts in \texttt{figures/}. The naming convention matches the chapter numbering in this book:

\begin{itemize}
  \item Chapter 1 (\emph{Points and Trait Space}) corresponds to files such as \texttt{python/ch01\_points\_trait\_space.py} and \texttt{R/ch01\_points\_trait\_space.R}.
  \item Chapter 2 (\emph{Vectors, Coordinates, and Angles}) to \texttt{ch02\_vectors\_coordinates.*}.
  \item \dots
  \item Chapter 13 (\emph{Directional Heritability and the Geometry of Constraint}) to \texttt{ch13\_directional\_heritability.*}.
\end{itemize}

Each script mirrors the structure of the corresponding chapter: it implements the algebra, reproduces the main figures, and includes additional comments and small exercises. The top-level \texttt{README.md} in the repository summarises the directory layout and lists the main functions provided for each chapter.

Readers who prefer to learn by doing are encouraged to keep the book and the code side by side: read a section, run the matching code, and adjust parameters or trait combinations to see how the geometry changes.

\subsection*{Notation used in this book}

This book uses a small, consistent set of symbols. The aim is to reduce the
need to hunt through previous chapters when you forget what a symbol means.
Roughly speaking, plain italics such as $a$ or $x$ denote single numbers
(scalars), bold symbols such as $\vect{x}$ denote vectors, and bold capitals
such as $\mat{A}$ denote matrices. The commands \verb|\vect{}| and
\verb|\mat{}| in the source simply typeset these in bold.

\medskip

\begin{table}[ht]
\centering
\begin{tabular}{ll}
\hline
Symbol & Meaning \\
\hline
$a,b,c$            & Single numbers (scalars) \\
$x,y,z$            & Scalar trait values or coordinates \\
$i,j,k$            & Indices for individuals or traits \\
$n$                & Number of individuals in a sample \\
$p$                & Number of traits \\
\hline
$\vect{x}$         & Column vector of trait values for one individual \\
$\vect{x}_i$       & Phenotype vector of individual $i$ \\
$\bar{\vect{x}}$   & Mean phenotype vector in a sample or population \\
$\vect{0}$         & Zero vector (all components equal to 0) \\
$\vect{e}_i$       & Unit vector along trait $i$ (1 in position $i$, 0 elsewhere) \\
\hline
$\mat{A}, \mat{B}$ & General matrices (linear transformations) \\
$\mat{I}$          & Identity matrix (leaves every vector unchanged) \\
${\Sigma}$         & Generic covariance matrix \\
$\mat{P}$          & Phenotypic covariance matrix \\
$\mat{G}$          & Additive genetic covariance matrix \\
$\mat{G}^\ast$     & Whitened genetic matrix: $\mat{P}^{-1/2} \mat{G} \mat{P}^{-1/2}$ \\
${\Lambda}$        & Diagonal matrix of eigenvalues \\
$\mat{Q}$          & Matrix whose columns are eigenvectors \\
\hline
$\vect{x}^\top$    & Transpose of $\vect{x}$ (row vector) \\
$\langle \vect{x}, \vect{y} \rangle$ & Inner (dot) product of two vectors \\
$\|\vect{x}\|$     & Length (Euclidean norm) of vector $\vect{x}$ \\
$\vect{x}^\top \mat{A} \vect{x}$ & Quadratic form defined by $\mat{A}$ \\
\hline
$\lambda_i$        & $i$th eigenvalue of a matrix (stretch along $\vect{v}_i$) \\
$\vect{v}_i$       & $i$th eigenvector (direction associated with $\lambda_i$) \\
\hline
$X,Y$              & Scalar random variables \\
$\E[X]$            & Expectation (mean) of $X$ \\
$\Var(X)$          & Variance of $X$ \\
$\Cov(X,Y)$        & Covariance between $X$ and $Y$ \\
$\mathcal{N}(\vect{\mu},{\Sigma})$ & Multivariate normal with mean ${\mu}$ and covariance ${\Sigma}$ \\
\hline
$\vect{\beta}$     & Vector of directional selection gradients \\
$\vect{s}$         & Vector of selection differentials \\
$\Delta \bar{\vect{z}}$ & Change in mean phenotype under selection \\
$h^2$              & Heritability of a single trait \\
$h^2(\vect{u})$    & Directional heritability along direction $\vect{u}$ \\
$e(\vect{u})$      & Evolvability along direction $\vect{u}$ (additive variance in that direction) \\
\hline
$w$                & Fitness of an individual \\
$\bar{w}$          & Mean fitness in the population \\
$\boldsymbol{\gamma}$ & Matrix of quadratic selection gradients (curvature of the fitness surface) \\
\hline
\end{tabular}
\end{table}

\medskip

This table does not try to list every symbol used in a specific example. Its
role is to give you a quick reminder of the main objects that appear
throughout the book: vectors and matrices for traits, the key covariance
matrices $\mat{P}$ and $\mat{G}$, and the quantities used to describe
selection, response, and fitness.

\tableofcontents

\mainmatter

% ==================================================
% Part I -- Arrows, Directions, and Tables of Numbers
% ==================================================
\part{Arrows, Directions, and Tables of Numbers}

\chapter{Points and Trait Space}

These notes will use pictures of points, arrows, and clouds in space
throughout. Before we mention vectors or matrices, we need a clear picture
of what the space is and what it means to move around in it.

\section{Traits as axes, individuals as points}

Start with a single quantitative trait, such as body length. We can draw a
line and mark each individual on that line according to its measured value.
This is the usual number line from school (\cref{fig:two-points}).

\begin{figure}[ht]
  \centering
  \includegraphics[width=0.7\textwidth]{fig01_two_points.png}
  \caption[Two points on a number line]{
    Distance begins with two points. On a single trait axis, the distance
    between individuals A and B is the absolute difference of their values.
  }
  \label{fig:two-points}
\end{figure}

Now take two traits, for example body length and wing span. Instead of a
single line, we draw a horizontal axis for body length and a vertical axis
for wing span. Every individual is now a point in the resulting plane. If
individual $i$ has body length $x_i$ and wing span $y_i$, we draw it at the
point $(x_i, y_i)$.

With three traits, we would have three axes and each individual would be a
point in three-dimensional space. We cannot draw that easily on paper, but
the idea is the same. In general, if we measure $p$ traits on each
individual, then each individual is a point in a $p$-dimensional
\emph{trait space}.

\begin{keyidea}
A multivariate phenotype is a point in trait space. The number of
dimensions equals the number of traits.
\end{keyidea}

This is our basic mental model: a sample is a cloud of points in trait
space. Much of quantitative genetics is about describing the shape of this
cloud and how it moves under selection, drift, and other processes.

\section{Differences between individuals as arrows}

We often care about how two individuals differ. On a single trait, the
difference between individuals $i$ and $j$ is just
\[
  x_j - x_i,
\]
a signed distance along the line. When we have a reference point, every
individual can be described by its displacement from that reference
(\cref{fig:reference-point}).

\begin{figure}[ht]
  \centering
  \includegraphics[width=0.7\textwidth]{fig02_reference_point.png}
  \caption[Distance from a reference point]{
    Choosing a reference point lets us describe each individual by its
    displacement. Here, individual A sits 2 units from the reference,
    B sits 7 units away. The distance between A and B is still $|7-2|=5$.
  }
  \label{fig:reference-point}
\end{figure}

In two traits, the situation is similar but now differences have two
components. If individual $i$ is at $(x_i, y_i)$ and individual $j$ is at
$(x_j, y_j)$, then the difference between them can be drawn as an arrow
from $i$ to $j$.

To construct this arrow, we subtract coordinates:
\[
  \text{change in body length} = x_j - x_i,
\]
\[
  \text{change in wing span} = y_j - y_i.
\]

The arrow itself records two pieces of information:

\begin{itemize}
  \item a \emph{direction} in the plane (where you would walk if you were
        trying to go from individual $i$ to individual $j$);
  \item a \emph{length} (how far you would need to walk).
\end{itemize}

We can describe the same arrow in words:

\begin{quote}
  ``Starting at individual $i$, add $(x_j - x_i)$ units of body length and
  $(y_j - y_i)$ units of wing span.''
\end{quote}

In coordinates, we might write this as
\[
  \begin{pmatrix}
    x_j - x_i \\
    y_j - y_i
  \end{pmatrix}.
\]

At this stage we do not need any new jargon. The important point is that
these arrows behave in a regular way.

\section{The mean as natural reference}

In statistics, we typically use the sample mean as our reference point
(\cref{fig:mean-reference}). This choice is not arbitrary: measuring
deviations from the mean is the foundation of variance.

\begin{figure}[ht]
  \centering
  \includegraphics[width=0.7\textwidth]{fig03_mean_as_reference.png}
  \caption[The mean as reference]{
    Using the mean as reference point. Each individual's position is now
    a deviation from the mean. The sum of all deviations (with signs) is
    zero---this is a defining property of the mean.
  }
  \label{fig:mean-reference}
\end{figure}

When we summarise a sample, we face a problem: how do we reduce many
individual deviations to a single number that captures the ``spread'' of
the cloud? (\cref{fig:summarizing-problem})

\begin{figure}[ht]
  \centering
  \includegraphics[width=0.7\textwidth]{fig04_summarizing_problem.png}
  \caption[The summarising problem]{
    Many individuals, many deviations. We need a single number to describe
    how spread out the cloud is. Simply averaging the deviations will not
    work---positive and negative values cancel.
  }
  \label{fig:summarizing-problem}
\end{figure}

\section{Adding and stretching arrows}

If arrows represent changes in phenotype, then we can combine changes.

Imagine one change that goes from phenotype A to phenotype B, and another
change that goes from phenotype B to phenotype C. If we draw both arrows
head-to-tail, the overall change from A to C is the arrow from the start of
the first to the end of the second.

Algebraically, if the first change has components $(\Delta x_1, \Delta y_1)$
and the second has components $(\Delta x_2, \Delta y_2)$, then the combined
change has components
\[
  (\Delta x_1 + \Delta x_2,\; \Delta y_1 + \Delta y_2).
\]

Similarly, we can stretch or shrink an arrow. If a change in phenotype is
described by $(\Delta x, \Delta y)$, then half that change is
$(\tfrac{1}{2}\Delta x, \tfrac{1}{2}\Delta y)$, and twice that change is
$(2\Delta x, 2\Delta y)$.

\begin{keyidea}
Changes in phenotype can be added and scaled. Geometrically, this means we
can join arrows head-to-tail and stretch or shrink them. Algebraically, this
corresponds to adding and scaling their coordinate pairs.
\end{keyidea}

This simple behaviour is what makes these objects so useful. It is the
reason we will eventually give them a special name.

\begin{figure}[htbp]
    \centering
    \includegraphics[width=\textwidth]{figures/fig_ch1_vector_addition.pdf}
    \caption{Vector addition by the head-to-tail method. (a) Two vectors 
    $\mathbf{u}$ and $\mathbf{v}$ originating from the origin, representing 
    independent changes in phenotype. (b) Head-to-tail construction: the tail 
    of $\mathbf{v}$ is placed at the head of $\mathbf{u}$. (c) The resulting 
    sum vector $\mathbf{u} + \mathbf{v}$ connects the origin to the endpoint.
    This construction underlies how selection differentials accumulate across 
    episodes of selection and how evolutionary responses combine across 
    generations.}
    \label{fig:vector_addition}
\end{figure}

\section{From arrows to vectors}

We are now ready to introduce the word ``vector''.

\begin{definition}
A \emph{vector} in trait space is a quantity that has both direction and
length and that can be added and scaled in the way just described.
\end{definition}

The difference between two phenotypes is a vector. A selection gradient is
also a vector: it points in the direction of steepest increase in fitness in
trait space. A response to selection is a vector: it describes how the mean
phenotype moves.

Each vector can be viewed in three equivalent ways:

\begin{itemize}
  \item as an arrow in trait space;
  \item as a verbal instruction for how to change each trait;
  \item as a list of numbers, one for each trait.
\end{itemize}

For example, in two traits we might write
\[
  \vect{v}
    =
  \begin{pmatrix}
    2 \\
    -1
  \end{pmatrix}
\]
to represent the change ``add 2 units of trait 1 and subtract 1 unit of
trait 2''. In three traits, a vector would have three components, and so
on.

When you see a bold symbol like $\vect{z}$ or $\boldsymbol{\beta}$ in later
chapters, it will always stand for such an arrow in trait space.

\section{Distances and lengths of vectors}

Once we have arrows, it is natural to ask how long they are. For a vector
with components $(\Delta x, \Delta y)$, the length is given by the
Pythagorean theorem:
\[
  \|\vect{v}\|
    = \sqrt{(\Delta x)^2 + (\Delta y)^2}.
\]

We can also use this to define the distance between two phenotypes:
draw the arrow from one to the other and take its length. This recovers
the usual Euclidean distance\index[subject]{distance!Euclidean} distance in the plane.

In higher dimensions, the same idea applies. If a vector has components
$(\Delta x_1, \Delta x_2, \dots, \Delta x_p)$, then its length is
\[
  \|\vect{v}\|
    = \sqrt{
        (\Delta x_1)^2 + (\Delta x_2)^2 + \dots + (\Delta x_p)^2
      }.
\]

The square of this length,
\[
  \|\vect{v}\|^2
    = (\Delta x_1)^2 + (\Delta x_2)^2 + \dots + (\Delta x_p)^2,
\]
is a sum of squared components. This will connect directly to variance
and, later, to matrix notation.

\section{Summary}

In this chapter we have:

\begin{itemize}
  \item represented multivariate phenotypes as points in trait space;
  \item represented differences as arrows, and named them vectors;
  \item linked vector length to sums of squared components;
  \item shown how to generate a simple trait space plot with code.
\end{itemize}

In the next chapters we will move from these arrows to distances from a
mean, and from there to variance and covariance.

\section*{Exercises}

\paragraph{Exercise 0.1 (Plotting a phenotype cloud).}
Five plants are measured for leaf length (cm) and leaf width (cm):

\begin{center}
\begin{tabular}{ccc}
\hline
Plant & Length & Width \\
\hline
A & 4.2 & 2.1 \\
B & 5.1 & 2.8 \\
C & 3.8 & 1.9 \\
D & 4.7 & 2.4 \\
E & 4.2 & 2.3 \\
\hline
\end{tabular}
\end{center}

\begin{enumerate}
  \item Plot these five individuals as points in a two-dimensional trait space.
  \item Estimate the centroid (mean phenotype) by eye from your plot.
  \item Calculate the centroid exactly. How close was your estimate?
  \item A sixth plant F has measurements (6.0, 3.5). Add it to your plot.
        How does the centroid shift?
\end{enumerate}

\paragraph{Exercise 0.2 (Trait space dimensions).}
A bird ecologist measures wing length, tarsus length, bill depth, and body
mass on each individual.

\begin{enumerate}
  \item How many dimensions does this trait space have?
  \item Can you visualise this space directly? If not, what strategies
        might help you understand the distribution of individuals?
  \item If you added bill width as a fifth trait, how would the
        dimensionality change?
\end{enumerate}

\paragraph{Exercise 0.3 (Phenotype as position).}
Consider two fish: Fish 1 has length 15 cm and mass 50 g; Fish 2 has
length 20 cm and mass 80 g.

\begin{enumerate}
  \item Represent each fish as a point in (length, mass) space.
  \item Draw the arrow from Fish 1 to Fish 2. What does this arrow
        represent biologically?
  \item If a third fish lies exactly halfway along this arrow, what are
        its length and mass?
\end{enumerate}

\paragraph{Exercise 0.4 (The meaning of ``distance'' in trait space).}
Two flowers differ in petal length by 2 mm and in petal width by 3 mm.

\begin{enumerate}
  \item What is the straight-line (Euclidean) distance between them in
        trait space?
  \item Does this number have a direct biological interpretation?
  \item What might make two flowers ``far apart'' in trait space but
        similar in fitness?
\end{enumerate}

\chapter{Vectors, Coordinates, and Angles}

In the previous chapter we met vectors as arrows in trait space: they have
direction and length and can be added and scaled. In this chapter we make
their algebra a bit more precise. We introduce

\begin{itemize}
  \item coordinates for vectors;
  \item the dot product (inner product);
  \item the idea of an angle between two vectors;
  \item distance between phenotypes written in vector notation.
\end{itemize}

\section{Column vectors and coordinates}

Suppose we measure $p$ traits on each individual. A phenotype is then a
point in a $p$-dimensional trait space. When we choose axes (one per trait),
we can record the position of that point as a list of $p$ numbers.

It is convenient to write this list as a column vector:
\[
  \vect{z}
    =
  \begin{pmatrix}
    z_1 \\
    z_2 \\
    \vdots \\
    z_p
  \end{pmatrix},
\]
where $z_1$ is trait~1, $z_2$ is trait~2, and so on.

A change in phenotype is also a vector. If an evolutionary process moves the
mean phenotype by $\Delta z_1$ units in trait~1, $\Delta z_2$ in trait~2,
and so on, we can record this as
\[
  \Delta\vect{z}
    =
  \begin{pmatrix}
    \Delta z_1 \\
    \Delta z_2 \\
    \vdots \\
    \Delta z_p
  \end{pmatrix}.
\]

\begin{keyidea}
Writing vectors as columns of numbers does not change what they are. It is
a compact way to store ``change in each trait'' and to perform calculations.
\end{keyidea}

\section{Unit vectors and decomposing changes}

In two traits, define
\[
  \vect{e}_1 =
  \begin{pmatrix}
    1 \\ 0
  \end{pmatrix},
  \qquad
  \vect{e}_2 =
  \begin{pmatrix}
    0 \\ 1
  \end{pmatrix}.
\]

Geometrically, $\vect{e}_1$ is a step of length~1 along trait~1, and
$\vect{e}_2$ is a step of length~1 along trait~2. Any change in phenotype
can be written as a combination of these unit steps. For example,
\[
  \begin{pmatrix}
    2 \\ -1
  \end{pmatrix}
  = 2\,\vect{e}_1 - 1\,\vect{e}_2.
\]

In $p$ traits, we have $p$ unit vectors, and any vector can be written as
\[
  \vect{v}
    = v_1 \vect{e}_1 + v_2 \vect{e}_2 + \dots + v_p \vect{e}_p.
\]

This just says that the components $v_1, \dots, v_p$ are the coordinates of
the vector along the trait axes.

\section{Lengths written in vector notation}

Previously we defined the length of a vector in $p$ traits as
\[
  \|\vect{v}\|
    = \sqrt{
        v_1^2 + v_2^2 + \dots + v_p^2
      }.
\]

If we form the transpose $\vect{v}^\top$,
\[
  \vect{v}^\top
    =
  \begin{pmatrix}
    v_1 & v_2 & \dots & v_p
  \end{pmatrix},
\]
then
\[
  \vect{v}^\top\vect{v}
    =
  \begin{pmatrix}
    v_1 & v_2 & \dots & v_p
  \end{pmatrix}
  \begin{pmatrix}
    v_1 \\ v_2 \\ \vdots \\ v_p
  \end{pmatrix}
  = v_1^2 + v_2^2 + \dots + v_p^2.
\]

\begin{keyidea}
The squared length of a vector can be written compactly as
\[
  \|\vect{v}\|^2 = \vect{v}^\top \vect{v}.
\]
\end{keyidea}

This is our first piece of matrix notation. For now, it is just a compact
way to write a sum of squares.

\section{The dot product and angles}

For two vectors $\vect{v}$ and $\vect{w}$, the dot product is
\[
  \vect{v}^\top\vect{w}
    = v_1 w_1 + v_2 w_2 + \dots + v_p w_p.
\]

When $\vect{v} = \vect{w}$ we recover the squared length. More generally,
if $\theta$ is the angle between $\vect{v}$ and $\vect{w}$, then
\[
  \vect{v}^\top\vect{w}
    = \|\vect{v}\|\,\|\vect{w}\|\cos\theta.
\]

So:

\begin{itemize}
  \item if $\vect{v}$ and $\vect{w}$ point in the same direction,
        $\theta \approx 0$ and the dot product is large and positive;
  \item if they are at right angles, $\theta = \pi/2$ and the dot product is
        zero;
  \item if they point in opposite directions, the dot product is negative.
\end{itemize}

\begin{keyidea}
The dot product measures how much two vectors point in the same direction.
\end{keyidea}

This will let us talk about selection along a particular trait combination,
or response along a given direction.

\section{Projections onto a direction}

Take a non-zero vector $\vect{u}$ and make the unit vector in its direction:
\[
  \hat{\vect{u}} = \frac{\vect{u}}{\|\vect{u}\|}.
\]

For any vector $\vect{v}$, the quantity $\hat{\vect{u}}^\top \vect{v}$
measures the component of $\vect{v}$ along $\hat{\vect{u}}$. It is the
length of the ``shadow'' of $\vect{v}$ when projected onto that direction.

In evolutionary terms, if $\hat{\vect{u}}$ represents a particular trait
combination (say, an eigenvector\index[subject]{eigenvector} of $G$), then $\hat{\vect{u}}^\top
\boldsymbol{\beta}$ measures how strong selection is along that combination.

\begin{figure}[htbp]
    \centering
    \includegraphics[width=0.85\textwidth]{figures/fig_ch2_projection.pdf}
    \caption{\textbf{Projection as shadow.} The projection of $\vect{v}$ onto 
    direction $\hat{\vect{u}}$ is the component of $\vect{v}$ lying along 
    $\hat{\vect{u}}$. The dashed line shows the perpendicular (residual) 
    component. The length of the projection is $\vect{v}^\top\hat{\vect{u}} = 
    \|\vect{v}\| \cos \theta$. In evolutionary terms, if $\hat{\vect{u}}$ is 
    an eigenvector of $\mat{G}$, then $\boldsymbol{\beta}^\top\hat{\vect{u}}$ 
    measures how strongly selection aligns with that genetic axis.}
    \label{fig:projection}
\end{figure}

\section{Distances between phenotypes}

Let $\vect{z}_i$ and $\vect{z}_j$ be the phenotypes of individuals $i$ and
$j$. The difference between them is
\[
  \vect{z}_j - \vect{z}_i,
\]
and the squared Euclidean distance\index[subject]{distance!Euclidean} distance is
\[
  d_{\text{Euc}}^2(i, j)
    = \|\vect{z}_j - \vect{z}_i\|^2
    = (\vect{z}_j - \vect{z}_i)^\top (\vect{z}_j - \vect{z}_i).
\]

\begin{keyidea}
Euclidean distance\index[subject]{distance!Euclidean} squared distance between two phenotypes can be written as
\[
  d_{\text{Euc}}^2(i, j)
    = (\vect{z}_j - \vect{z}_i)^\top (\vect{z}_j - \vect{z}_i).
\]
\end{keyidea}

Later, when we introduce covariance matrices, a matrix will appear between
the transpose and the vector. The pattern ``row $\times$ matrix $\times$
column'' will keep returning.

\section{Summary}

In this chapter we have:

\begin{itemize}
  \item written vectors as columns of trait values or changes;
  \item introduced unit vectors and decomposed vectors into components;
  \item defined the dot product and linked it to angles between directions;
  \item expressed squared lengths and distances using $\vect{v}^\top\vect{v}$
        and $(\vect{z}_j - \vect{z}_i)^\top(\vect{z}_j - \vect{z}_i)$;
  \item introduced projections onto chosen directions.
\end{itemize}

We now have enough language to connect variance to squared distances from a
mean, and to see what changes when multiple traits interact. That is the
next step.



%%%%%%%%%%%%%%%%%%%%%%%%%%%%%%%%%%%%%%%%%%%%%%%%%%%%%%%%%%%%%%%%%%%%%%%%%%%%%%%
% CHAPTER 01: VECTORS AND COORDINATES
%%%%%%%%%%%%%%%%%%%%%%%%%%%%%%%%%%%%%%%%%%%%%%%%%%%%%%%%%%%%%%%%%%%%%%%%%%%%%%%

\section*{Exercises}

\paragraph{Exercise 1.1 (Vector length).}
Compute the length (magnitude) of each vector:
\begin{enumerate}
  \item $\vect{a} = (3, 4)$
  \item $\vect{b} = (1, 1, 1)$
  \item $\vect{c} = (2, -2, 1)$
  \item $\vect{d} = (1, 0, 0, 0, 1)$
\end{enumerate}

\paragraph{Exercise 1.2 (Normalising vectors).}
A unit vector has length 1. For each vector below, find the unit vector
pointing in the same direction:
\begin{enumerate}
  \item $(3, 4)$
  \item $(1, 1)$
  \item $(5, 0)$
  \item $(1, 2, 2)$
\end{enumerate}

\paragraph{Exercise 1.3 (Dot product and angles).}
The dot product of $\vect{a}$ and $\vect{b}$ is
$\vect{a} \cdot \vect{b} = \|\vect{a}\| \|\vect{b}\| \cos\theta$,
where $\theta$ is the angle between them.

\begin{enumerate}
  \item Compute the dot product of $(1, 0)$ and $(1, 1)$.
  \item Find the angle between these vectors.
  \item Compute the dot product of $(1, 2)$ and $(-2, 1)$. What does
        this tell you about the angle between them?
  \item Two vectors are orthogonal if their dot product is zero. Find
        a vector orthogonal to $(3, 4)$.
\end{enumerate}

\paragraph{Exercise 1.4 (Projection).}
The projection of $\vect{a}$ onto $\vect{b}$ is the ``shadow'' of
$\vect{a}$ in the direction of $\vect{b}$:
\[
  \text{proj}_{\vect{b}}(\vect{a}) = 
    \frac{\vect{a} \cdot \vect{b}}{\vect{b} \cdot \vect{b}} \vect{b}.
\]

\begin{enumerate}
  \item Project $(3, 4)$ onto $(1, 0)$. Interpret geometrically.
  \item Project $(3, 4)$ onto $(1, 1)$.
  \item Project $(3, 4)$ onto $(0, 1)$.
  \item What is the projection of any vector onto itself?
\end{enumerate}

\paragraph{Exercise 1.5 (Biological interpretation).}
In a two-trait system, the vector $(1, 1)/\sqrt{2}$ represents the
direction where both traits increase equally.

\begin{enumerate}
  \item What direction does $(1, -1)/\sqrt{2}$ represent?
  \item If selection acts in the direction $(0.8, 0.6)$, is it favouring
        both traits equally? Which trait is favoured more?
  \item A population's mean breeding value shifts by $\Delta\bar{\vect{z}} = (0.5, 0.3)$.
        What is the magnitude of this response? What is its direction?
\end{enumerate}

\chapter{Matrices as Machines That Move Vectors}

In the previous chapters we represented phenotypes as points in trait space
and changes in phenotype as vectors. We learned to measure lengths and angles
using dot products. Now we need a language for transformations---rules that
take one vector and return another.

This chapter introduces matrices as such rules. The key insight is
geometric: a matrix does not merely store numbers in a rectangular array; it
describes a transformation of space. Once you see this, covariance matrices,
genetic variance matrices, and selection gradients all become visualisable.

\section{A motivating example: scaling traits differently}

Suppose we measure body size in centimetres and wing length in millimetres.
A fly with body size 0.3\,cm and wing length 2.5\,mm sits at the point
$(0.3, 2.5)$ in our trait space.

Now imagine we want to convert both measurements to the same units---say,
millimetres. Body size must be multiplied by 10; wing length stays as it is.
The new coordinates are $(3.0, 2.5)$.

We can write this conversion as a rule:
\[
  \begin{pmatrix}
    z_1' \\ z_2'
  \end{pmatrix}
  =
  \begin{pmatrix}
    10 \cdot z_1 + 0 \cdot z_2 \\
    0 \cdot z_1 + 1 \cdot z_2
  \end{pmatrix}
  =
  \begin{pmatrix}
    10 & 0 \\
    0 & 1
  \end{pmatrix}
  \begin{pmatrix}
    z_1 \\ z_2
  \end{pmatrix}.
\]

The array of numbers in the middle is a \emph{matrix}. It encodes the rule:
``multiply trait~1 by 10, leave trait~2 alone.'' When we apply this rule to
every point in trait space, the entire cloud of phenotypes stretches
horizontally by a factor of 10 while remaining unchanged vertically.

This is the central idea: a matrix is a machine that moves vectors.

\section{Matrix--vector multiplication}

Let us make the rule precise. A $2 \times 2$ matrix $\mat{A}$ has four
entries arranged in two rows and two columns:
\[
  \mat{A}
  =
  \begin{pmatrix}
    a_{11} & a_{12} \\
    a_{21} & a_{22}
  \end{pmatrix}.
\]

When we multiply $\mat{A}$ by a column vector $\vect{v}$, we obtain a new
vector $\vect{w} = \mat{A}\vect{v}$:
\[
  \begin{pmatrix}
    w_1 \\ w_2
  \end{pmatrix}
  =
  \begin{pmatrix}
    a_{11} & a_{12} \\
    a_{21} & a_{22}
  \end{pmatrix}
  \begin{pmatrix}
    v_1 \\ v_2
  \end{pmatrix}
  =
  \begin{pmatrix}
    a_{11} v_1 + a_{12} v_2 \\
    a_{21} v_1 + a_{22} v_2
  \end{pmatrix}.
\]

Each entry of the output is a dot product: row~$i$ of the matrix dotted with
the input vector gives entry~$i$ of the output.

In words: the matrix takes each input component, weights it according to its
entries, and combines those weighted contributions to produce each output
component. This is a \emph{linear combination}---no squares, no products of
different components, just weighted sums.

\begin{keyidea}
Matrix--vector multiplication produces a new vector whose components are
linear combinations of the original components. The matrix entries are the
weights.
\end{keyidea}

\section{What happens to the unit vectors?}

A powerful way to understand any matrix is to ask: what does it do to the
standard unit vectors?

Recall from Chapter~1 that in two dimensions the unit vectors are
\[
  \vect{e}_1 =
  \begin{pmatrix}
    1 \\ 0
  \end{pmatrix},
  \qquad
  \vect{e}_2 =
  \begin{pmatrix}
    0 \\ 1
  \end{pmatrix}.
\]

Apply the matrix $\mat{A}$ to $\vect{e}_1$:
\[
  \mat{A}\vect{e}_1
  =
  \begin{pmatrix}
    a_{11} & a_{12} \\
    a_{21} & a_{22}
  \end{pmatrix}
  \begin{pmatrix}
    1 \\ 0
  \end{pmatrix}
  =
  \begin{pmatrix}
    a_{11} \\ a_{21}
  \end{pmatrix}.
\]

This is simply the first column of $\mat{A}$. Similarly,
\[
  \mat{A}\vect{e}_2
  =
  \begin{pmatrix}
    a_{12} \\ a_{22}
  \end{pmatrix},
\]
the second column.

\begin{keyidea}
The columns of a matrix are the images of the unit vectors. Column~$j$ tells
you where $\vect{e}_j$ lands after the transformation.
\end{keyidea}

This observation is the key to visualising what any matrix does. If you know
where the coordinate axes go, you know everything---because every other
vector is a combination of those axes.

\section{Geometric vocabulary: stretch, rotate, shear}

Different matrices produce different geometric effects. Here are the main
types, illustrated in two dimensions.

\paragraph{Scaling (stretching or compressing).}
A diagonal matrix stretches each axis independently:
\[
  \begin{pmatrix}
    3 & 0 \\
    0 & 2
  \end{pmatrix}
\]
stretches trait~1 by a factor of 3 and trait~2 by a factor of 2. A circle
of points becomes an ellipse aligned with the axes. If one diagonal entry
is less than 1, that axis is compressed rather than stretched.

\paragraph{Rotation.}
The matrix
\[
  \begin{pmatrix}
    \cos\theta & -\sin\theta \\
    \sin\theta & \cos\theta
  \end{pmatrix}
\]
rotates every vector by angle $\theta$ anticlockwise. A circle remains a
circle; only its orientation changes. Lengths and angles between vectors
are preserved.

\paragraph{Shear.}
The matrix
\[
  \begin{pmatrix}
    1 & k \\
    0 & 1
  \end{pmatrix}
\]
slides points horizontally in proportion to their vertical coordinate. A
square becomes a parallelogram. Trait~1 gains a contribution from trait~2,
but not vice versa.

\paragraph{Reflection.}
The matrix
\[
  \begin{pmatrix}
    -1 & 0 \\
    0 & 1
  \end{pmatrix}
\]
flips points across the vertical axis. Reflections reverse orientation:
a clockwise path around a triangle becomes anticlockwise after reflection.

Most matrices combine several of these effects. The covariance matrices we
will meet shortly turn out to be pure stretches along rotated axes---no
shear, no reflection. That special structure is what makes them
diagonalisable.

\begin{figure}[htbp]
    \centering
    \includegraphics[width=\textwidth]{figures/fig_ch3_four_transformations.pdf}
    \caption{\textbf{The four basic linear transformations.} Each panel shows 
    the unit square (dashed gray) and its image (solid colour) under a different 
    transformation. (a)~\textbf{Scaling} stretches each axis independently. 
    (b)~\textbf{Rotation} preserves lengths and angles. (c)~\textbf{Shear} 
    slides points parallel to one axis---note the parallelogram. 
    (d)~\textbf{Reflection} reverses orientation. Covariance matrices, being 
    symmetric and positive definite, produce only scaling along rotated 
    axes---no shear or reflection.}
    \label{fig:four-transformations}
\end{figure}



\section{Linearity: the defining property}

Matrix transformations have a crucial property: they are \emph{linear}. This
means two things hold for any matrix $\mat{A}$ and any vectors $\vect{u}$,
$\vect{v}$:

\begin{enumerate}
  \item \textbf{Additivity.} $\mat{A}(\vect{u} + \vect{v}) = \mat{A}\vect{u}
        + \mat{A}\vect{v}$.
  \item \textbf{Scaling.} $\mat{A}(c\,\vect{v}) = c\,(\mat{A}\vect{v})$ for
        any scalar $c$.
\end{enumerate}

In plain language: if you transform two vectors and then add them, you get
the same result as adding first and then transforming. And scaling a vector
before or after the transformation gives the same answer.

Why does this matter biologically? Selection gradients, breeding values, and
responses to selection are all defined as linear combinations of underlying
quantities. The machinery of quantitative genetics is built on linearity.
When nonlinear effects enter---epistasis, dominance, genotype-by-environment
interaction---the linear framework becomes an approximation, and we must ask
how good that approximation is. That question will occupy us in later
chapters.

\section{Composing transformations: matrix multiplication}

Suppose we apply one transformation $\mat{A}$ and then another $\mat{B}$.
What is the combined effect?

Start with a vector $\vect{v}$. After $\mat{A}$, we have $\mat{A}\vect{v}$.
After $\mat{B}$, we have $\mat{B}(\mat{A}\vect{v})$.

It turns out this combined transformation is itself a matrix, written
$\mat{B}\mat{A}$ (note the order: $\mat{A}$ acts first, then $\mat{B}$). The
entries of the product matrix are computed by the row-times-column rule:
\[
  (\mat{B}\mat{A})_{ij} = \sum_k B_{ik} A_{kj}.
\]

We will not dwell on the mechanics of matrix multiplication here. The
conceptual point is that composing linear transformations yields another
linear transformation. This is why matrix algebra is so powerful: complex
sequences of operations can be encoded as single matrices.

\begin{keyidea}
The product $\mat{B}\mat{A}$ represents ``first $\mat{A}$, then $\mat{B}$.''
Order matters: in general, $\mat{A}\mat{B} \neq \mat{B}\mat{A}$.
\end{keyidea}

\begin{figure}[htbp]
    \centering
    \includegraphics[width=\textwidth]{figures/fig_ch3_columns_unit_vectors.pdf}
    \caption{\textbf{The columns of a matrix are the images of the unit 
    vectors.} Left: The standard unit vectors $\vect{e}_1$ and $\vect{e}_2$. 
    Right: Their images under the matrix $\mat{A}$. Notice that 
    $\mat{A}\vect{e}_1 = (2,1)^\top$ is exactly the first column of $\mat{A}$ 
    (red), and $\mat{A}\vect{e}_2 = (1,3)^\top$ is the second column (blue). 
    This is always true: column~$j$ tells you where $\vect{e}_j$ lands.}
    \label{fig:columns-unit-vectors}
\end{figure}

\section{The identity and inverse}

Some matrices do nothing at all. The \emph{identity matrix}
\[
  \mat{I} =
  \begin{pmatrix}
    1 & 0 \\
    0 & 1
  \end{pmatrix}
\]
sends every vector to itself: $\mat{I}\vect{v} = \vect{v}$. It stretches
each axis by a factor of 1---that is, it leaves everything unchanged.

If a matrix $\mat{A}$ has an \emph{inverse} $\mat{A}^{-1}$, then applying
$\mat{A}$ followed by $\mat{A}^{-1}$ returns every vector to its starting
point:
\[
  \mat{A}^{-1}\mat{A} = \mat{I}.
\]

Geometrically, $\mat{A}^{-1}$ undoes whatever $\mat{A}$ did. If $\mat{A}$
stretches trait~1 by 3, then $\mat{A}^{-1}$ compresses it by $1/3$. If
$\mat{A}$ rotates by $30°$, then $\mat{A}^{-1}$ rotates by $-30°$.

Not every matrix has an inverse. A matrix that collapses the plane onto a
line, for instance, loses information and cannot be undone. Such matrices
are called \emph{singular}. In evolutionary applications, singular
covariance matrices indicate that some trait combinations have zero
variance---the population has no variation in those directions.

\section{Symmetric matrices: a special and important case}

A matrix is \emph{symmetric} if it equals its own transpose:
\[
  \mat{A} = \mat{A}^\top,
\]
which means $a_{ij} = a_{ji}$ for all $i$ and $j$. The matrix is unchanged
when you reflect it across its main diagonal.

Symmetric matrices have remarkable properties that we will exploit
throughout these notes:

\begin{enumerate}
  \item Their eigenvalue\index[subject]{eigenvalue}s (stretching factors along special directions) are
        always real numbers, never complex.
  \item Their eigenvector\index[subject]{eigenvector}s (those special directions) are always
        perpendicular to one another.
  \item They can always be diagonalised by a rotation---no shear is needed.
\end{enumerate}

These properties mean that a symmetric matrix describes a pure stretch along
a set of perpendicular axes. In two dimensions, this is an ellipse aligned
with those axes. In three dimensions, an ellipsoid.

Covariance matrices are symmetric by construction: the covariance of trait~1
with trait~2 equals the covariance of trait~2 with trait~1. This is why
ellipses appear everywhere in multivariate statistics and why
diagonalisation is the natural tool for understanding them.

\begin{keyidea}
Symmetric matrices describe shapes---ellipses in 2D, ellipsoids in higher
dimensions. The eigenvalue\index[subject]{eigenvalue}s give the lengths of the principal axes; the
eigenvector\index[subject]{eigenvector}s give their directions.
\end{keyidea}

We are not yet ready to define eigenvalue\index[subject]{eigenvalue}s and eigenvector\index[subject]{eigenvector}s formally. For
now, hold onto the geometric picture: a symmetric matrix stretches space
along perpendicular axes, and the amount of stretch along each axis is what
we will call an eigenvalue\index[subject]{eigenvalue}.

\section{Preview: the quadratic form\index[subject]{quadratic form}}

There is one more construction we need before connecting matrices to
variance. Given a symmetric matrix $\mat{A}$ and a vector $\vect{v}$, the
quantity
\[
  \vect{v}^\top \mat{A} \vect{v}
\]
is called a \emph{quadratic form\index[subject]{quadratic form}}. It takes a vector and returns a single
number.

Let us unpack this in two dimensions. If
\[
  \mat{A} =
  \begin{pmatrix}
    a & b \\
    b & c
  \end{pmatrix},
  \qquad
  \vect{v} =
  \begin{pmatrix}
    v_1 \\ v_2
  \end{pmatrix},
\]
then
\[
  \vect{v}^\top \mat{A} \vect{v}
  = a\, v_1^2 + 2b\, v_1 v_2 + c\, v_2^2.
\]

This is a weighted sum of squared terms and cross-products. When $\mat{A}$
is a covariance matrix\index[subject]{covariance matrix}, this quadratic form\index[subject]{quadratic form} will give us the variance of a
linear combination of traits. When $\mat{A}$ is a selection matrix, it will
give us the curvature of the fitness surface.

The pattern ``row vector $\times$ matrix $\times$ column vector'' will
appear repeatedly:

\begin{itemize}
  \item Variance of a trait combination: $\vect{a}^\top{\Sigma}
        \vect{a}$
  \item Mahalanobis distance\index[subject]{distance!Mahalanobis} distance: $(\vect{z} - \boldsymbol{\mu})^\top
        {\Sigma}^{-1} (\vect{z} - \boldsymbol{\mu})$
  \item Quadratic selection: $\vect{z}^\top \boldsymbol{\gamma} \vect{z}$
\end{itemize}

Understanding the quadratic form\index[subject]{quadratic form} geometrically---as measuring how much a
vector aligns with the axes of stretch encoded by the matrix---is the key to
reading these expressions fluently.

\begin{figure}[htbp]
    \centering
    \includegraphics[width=\textwidth]{figures/fig_ch3_quadratic_form.pdf}
    \caption{The quadratic form $\mathbf{x}^\top \mathbf{A} \mathbf{x}$ as a 
    surface over trait space. (a) Three-dimensional view showing the paraboloid 
    surface; the height at any point $\mathbf{x}$ equals the quadratic form 
    value. The red point illustrates how a specific phenotype maps to a scalar 
    value. (b) Top-down view showing level curves, which are ellipses whose 
    axes align with the eigenvectors of $\mathbf{A}$. The eigenvalues determine 
    how steeply the surface rises along each principal direction. For covariance 
    matrices, this surface represents variance; for selection matrices $\gamma$, 
    it represents fitness curvature.}
    \label{fig:quadratic_form_surface}
\end{figure}

\section{A biological example: the G matrix as a transformation}

To make these ideas concrete, consider the additive genetic covariance
matrix $\mat{G}$. In two traits,
\[
  \mat{G} =
  \begin{pmatrix}
    V_{A1} & \text{Cov}_{A}(z_1, z_2) \\
    \text{Cov}_{A}(z_1, z_2) & V_{A2}
  \end{pmatrix}.
\]

What does $\mat{G}$ do when we treat it as a transformation?

Take the selection gradient $\boldsymbol{\beta}$, which points in the
direction of steepest fitness increase. The response to selection is
\[
  \Delta\bar{\vect{z}} = \mat{G}\boldsymbol{\beta}.
\]

This is the multivariate breeder's equation\index[subject]{breeder's equation} equation. The matrix $\mat{G}$
transforms the direction of selection into the direction of evolutionary
response.

Geometrically: $\boldsymbol{\beta}$ tells you where selection wants to go;
$\mat{G}$ tells you where genetic variation allows you to go. The response
$\mat{G}\boldsymbol{\beta}$ is a compromise between these.

If $\mat{G}$ is a diagonal matrix (no genetic correlations), the response is
parallel to selection---you go where selection pushes. If $\mat{G}$ has
strong off-diagonal elements (genetic correlations), the response is
deflected toward the direction of greatest genetic variance.

This deflection is not a bug; it is the central phenomenon of multivariate
evolution. Understanding it requires understanding what $\mat{G}$ does as a
transformation, which in turn requires the diagonalisation\index[subject]{diagonalisation} tools we will
develop in Part~III.

\section{Summary}

In this chapter we have introduced matrices as rules that transform vectors.
The main ideas are:

\begin{itemize}
  \item A matrix acts on a vector to produce a new vector, with each output
        component being a linear combination of the input components.
  \item The columns of a matrix are the images of the unit vectors; they
        tell you where the coordinate axes go.
  \item Common transformations include scaling, rotation, shear, and
        reflection. Most matrices combine several of these.
  \item Symmetric matrices are special: they describe pure stretches along
        perpendicular axes, producing elliptical shapes.
  \item The quadratic form\index[subject]{quadratic form} $\vect{v}^\top\mat{A}\vect{v}$ measures how a
        vector interacts with the shape encoded by a symmetric matrix.
  \item The $\mat{G}$ matrix transforms selection gradients into
        evolutionary responses, making it a concrete biological example of
        matrix-as-machine.
\end{itemize}

We now have the vocabulary to ask precise questions about distance and
shape. In Part~II, we will see why the usual Euclidean distance\index[subject]{distance!Euclidean} distance fails when
traits are correlated, and how the covariance matrix\index[subject]{covariance matrix} provides a remedy.

%%%%%%%%%%%%%%%%%%%%%%%%%%%%%%%%%%%%%%%%%%%%%%%%%%%%%%%%%%%%%%%%%%%%%%%%%%%%%%%
% CHAPTER 02: MATRICES AS MACHINES
%%%%%%%%%%%%%%%%%%%%%%%%%%%%%%%%%%%%%%%%%%%%%%%%%%%%%%%%%%%%%%%%%%%%%%%%%%%%%%%

\section*{Exercises}

\paragraph{Exercise 2.1 (Stretching).}
Consider the matrix
\[
  \mat{A} = \begin{pmatrix} 2 & 0 \\ 0 & 1 \end{pmatrix}.
\]

\begin{enumerate}
  \item Apply $\mat{A}$ to the vectors $(1, 0)$, $(0, 1)$, and $(1, 1)$.
  \item Describe in words what $\mat{A}$ does to any vector.
  \item Sketch the unit circle and its image under $\mat{A}$. What shape
        results?
  \item Find a matrix that stretches by 3 in the $x$-direction and by 2
        in the $y$-direction.
\end{enumerate}

\paragraph{Exercise 2.2 (Rotation).}
The matrix
\[
  \mat{R} = \begin{pmatrix} 0 & -1 \\ 1 & 0 \end{pmatrix}
\]
represents a rotation.

\begin{enumerate}
  \item Apply $\mat{R}$ to $(1, 0)$. Where does it go?
  \item Apply $\mat{R}$ to $(0, 1)$. Where does it go?
  \item By what angle does $\mat{R}$ rotate vectors?
  \item Apply $\mat{R}$ twice (compute $\mat{R}^2 = \mat{R}\mat{R}$).
        What transformation is $\mat{R}^2$?
\end{enumerate}

\paragraph{Exercise 2.3 (Shearing).}
Consider the shear matrix
\[
  \mat{S} = \begin{pmatrix} 1 & 1 \\ 0 & 1 \end{pmatrix}.
\]

\begin{enumerate}
  \item Apply $\mat{S}$ to $(1, 0)$, $(0, 1)$, and $(1, 1)$.
  \item Sketch the unit square with corners at $(0,0)$, $(1,0)$, $(0,1)$,
        $(1,1)$. Then sketch its image under $\mat{S}$.
  \item Compute $\mat{S}^2$. How does the shear accumulate?
  \item Is $\mat{S}$ symmetric? Does it change lengths?
\end{enumerate}

\paragraph{Exercise 2.4 (Matrix multiplication).}
Let
\[
  \mat{A} = \begin{pmatrix} 2 & 0 \\ 0 & 3 \end{pmatrix}, \quad
  \mat{B} = \begin{pmatrix} 0 & 1 \\ 1 & 0 \end{pmatrix}.
\]

\begin{enumerate}
  \item Compute $\mat{A}\mat{B}$ and $\mat{B}\mat{A}$. Are they equal?
  \item $\mat{B}$ swaps the two coordinates. Describe what $\mat{A}\mat{B}$
        does (first $\mat{B}$, then $\mat{A}$).
  \item Describe what $\mat{B}\mat{A}$ does (first $\mat{A}$, then $\mat{B}$).
  \item Find the inverse of $\mat{A}$. Verify by computing $\mat{A}\mat{A}^{-1}$.
\end{enumerate}

\paragraph{Exercise 2.5 (Symmetric matrices).}
A symmetric matrix satisfies $\mat{M} = \mat{M}^\top$.

\begin{enumerate}
  \item Which of the following are symmetric?
        \[
          \begin{pmatrix} 1 & 2 \\ 2 & 3 \end{pmatrix}, \quad
          \begin{pmatrix} 1 & 2 \\ 3 & 4 \end{pmatrix}, \quad
          \begin{pmatrix} 5 & 0 \\ 0 & 5 \end{pmatrix}
        \]
  \item If $\mat{M}$ is any matrix, show that $\mat{M}^\top\mat{M}$ is
        symmetric.
  \item Covariance matrices are always symmetric. Why does this make
        biological sense?
\end{enumerate}


% ==================================================
% Part II -- Distance and Shape
% ==================================================
\part{Distance and Shape}

\chapter{Distance and Why We Square It}

In Part~I we built a language for trait space: phenotypes are points,
differences are vectors, and matrices are machines that transform vectors.
Now we turn to a deceptively simple question: how do we measure how
different two phenotypes are?

This chapter develops the idea of distance and explains why squaring plays
such a central role in statistics. The answer is not arbitrary convention.
Squaring has a geometric meaning that connects individual differences to
population summaries in a way that no other operation does.

\section{Why distance matters}

Distance is fundamental to biology. When we ask whether two species have
diverged, we are asking about distance in phenotype space. When we measure
the strength of selection, we compare phenotypes that survive to those that
do not---again, a question of distance. When we estimate heritability\index[subject]{heritability}, we
ask whether offspring are closer to their parents than to random individuals
in the population.

In all these cases, we need a number that captures ``how different'' two
phenotypes are. That number should have sensible properties:

\begin{itemize}
  \item The distance from A to B should equal the distance from B to A.
  \item The distance from any phenotype to itself should be zero.
  \item If A, B, and C lie on a straight line with B between them, the
        distance from A to C should equal the sum of distances A to B and
        B to C.
\end{itemize}

These are the axioms of a \emph{metric}. Different metrics give different
answers to ``how different,'' and choosing the right metric is not a
mathematical formality---it changes what we see in our data.

\section{One trait: distance on a line}

Start with a single trait. Individual $i$ has value $z_i$, individual $j$
has value $z_j$. The natural measure of difference is
\[
  z_j - z_i,
\]
a signed quantity that tells us both the magnitude and direction of the
difference.

But for many purposes we want an unsigned measure---just ``how far apart,''
regardless of which is larger. We could take the absolute value:
\[
  |z_j - z_i|.
\]

Or we could square:
\[
  (z_j - z_i)^2.
\]

Both give non-negative numbers that are zero only when $z_i = z_j$. Why
might we prefer one over the other?

\section{Two traits: the Pythagorean formula}

With two traits, the situation becomes geometric (\cref{fig:pythagoras}).
Individual $i$ sits at $(x_i, y_i)$, individual $j$ at $(x_j, y_j)$. The
difference in trait~1 is $\Delta x = x_j - x_i$; the difference in trait~2
is $\Delta y = y_j - y_i$.

\begin{figure}[ht]
  \centering
  \includegraphics[width=0.6\textwidth]{fig10_pythagoras.png}
  \caption[Pythagorean distance]{
    The distance between two phenotypes in trait space. The horizontal leg
    is $\Delta x$, the vertical leg is $\Delta y$, and the hypotenuse has
    length $\sqrt{(\Delta x)^2 + (\Delta y)^2}$.
  }
  \label{fig:pythagoras}
\end{figure}

The Pythagorean theorem gives the straight-line distance:
\[
  d = \sqrt{(\Delta x)^2 + (\Delta y)^2}.
\]

This is the length of the arrow from $i$ to $j$. In vector notation, if
$\vect{v} = \vect{z}_j - \vect{z}_i$, then
\[
  d = \|\vect{v}\| = \sqrt{\vect{v}^\top \vect{v}}.
\]

The squared distance is simpler:
\[
  d^2 = (\Delta x)^2 + (\Delta y)^2 = \vect{v}^\top \vect{v}.
\]

No square root, just a sum of squared components. This pattern extends to
any number of traits: if we have $p$ traits, the squared Euclidean distance\index[subject]{distance!Euclidean} distance
is
\[
  d^2 = \sum_{k=1}^{p} (\Delta z_k)^2.
\]

\section{Why do we square?}

Here is the central question of this chapter. The Pythagorean formula
involves squares. Variance involves squares. Why?

The answer has three parts, each revealing a different reason why squaring
is not arbitrary.

\subsection*{Reason 1: Geometry demands it}

The Pythagorean theorem is not a human invention; it is a property of flat
space. If you want the distance along the hypotenuse of a right triangle,
you must add the squares of the legs and take the square root. This is what
``straight-line distance'' means in Euclidean distance\index[subject]{distance!Euclidean} geometry.

Using absolute values instead---$|\Delta x| + |\Delta y|$---gives a
different metric, sometimes called ``Manhattan distance'' or ``taxicab
distance'' because it measures how far you would walk along a grid of
streets. This is a perfectly valid metric, but it does not measure
straight-line distance. In trait space, we usually want the length of the
arrow, not the length of a path that only moves parallel to the axes.

\subsection*{Reason 2: Algebra rewards it}

Squared quantities have a remarkable property: they decompose additively
when the underlying components are independent.

Consider two independent random variables $X$ and $Y$. The variance of their
sum satisfies
\[
  \Var(X + Y) = \Var(X) + \Var(Y).
\]

This additivity is the foundation of ANOVA, of partitioning variance into
genetic and environmental components, of combining information across
independent sources. It works because variance is defined using squares.

If we used absolute deviations instead, we would not get this clean
decomposition. The mean absolute deviation of a sum is not, in general, the
sum of the mean absolute deviations. The algebra becomes intractable.

\begin{keyidea}
Squaring is the price we pay for additivity. Variance decomposes into
components precisely because it is built from squared deviations.
\end{keyidea}

\subsection*{Reason 3: Calculus prefers it}

Squared functions are smooth. The function $f(x) = x^2$ has a derivative
everywhere; the function $f(x) = |x|$ has a kink at zero where the
derivative is undefined.

This matters when we optimise. Least-squares fitting works because the sum
of squared residuals is a smooth, bowl-shaped function with a unique
minimum that calculus can find. Least-absolute-deviation fitting is harder:
the objective function has corners, and the minimum is not always unique.

In evolutionary biology, we often model fitness surfaces, selection
gradients, and breeding values using derivatives. Squared quantities fit
naturally into this calculus-based framework.

\begin{figure}[htbp]
    \centering
    \includegraphics[width=\textwidth]{figures/fig_ch4_variance_accumulation.pdf}
    \caption{\textbf{Variance as the mean of squared distances from the mean.}
    (a)~A sample of eight individuals (blue points) on a trait axis, with 
    deviation arrows showing each point's displacement from the mean 
    $\bar{x} = 5.38$ (red diamond). Orange arrows indicate positive deviations; 
    magenta arrows indicate negative deviations.
    (b)~The same deviations represented as squares, where the area of each square 
    equals $d_i^2 = (x_i - \bar{x})^2$. Larger deviations contribute 
    disproportionately more to the variance because of the squaring operation.
    (c)~Bar chart of individual squared deviations. The horizontal red line 
    marks the variance---the mean of these squared deviations. The shaded region 
    emphasizes that variance is this average, not the sum. This figure illustrates 
    why squaring is essential: it makes all contributions positive while 
    maintaining the geometric connection to distance via Pythagoras.}
    \label{fig:variance_accumulation}
\end{figure}


\section{From individual distances to population spread}

Now we make the connection to variance.

Take a sample of $n$ individuals with values $z_1, z_2, \ldots, z_n$ on a
single trait. The sample mean is
\[
  \bar{z} = \frac{1}{n} \sum_{i=1}^{n} z_i.
\]

Each individual's deviation from the mean is $z_i - \bar{z}$. The sample
variance is the average of the squared deviations:
\[
  s^2 = \frac{1}{n-1} \sum_{i=1}^{n} (z_i - \bar{z})^2.
\]

\begin{keyidea}
Variance is the mean squared distance from the mean. It measures how spread
out the sample is by averaging how far each individual sits from the centre.
\end{keyidea}

The geometric picture is clear (\cref{fig:variance-as-distance}): each
individual has an arrow pointing from the mean to its location. Variance
averages the squared lengths of these arrows.

\begin{figure}[ht]
  \centering
  \includegraphics[width=0.7\textwidth]{fig10_variance_as_distance.png}
  \caption[Variance as mean squared distance]{
    Variance as the mean squared distance from the mean. Each arrow connects
    an individual to the sample mean. Squaring and averaging these lengths
    gives the variance.
  }
  \label{fig:variance-as-distance}
\end{figure}

Why $n-1$ in the denominator rather than $n$? This is Bessel's correction,
which makes the sample variance an unbiased estimator of the population
variance. The geometric reason: the deviations $z_i - \bar{z}$ are
constrained to sum to zero (by definition of the mean), so only $n-1$ of
them are free to vary. We are averaging over $n-1$ independent pieces of
information, not $n$.

\section{Variance in two traits: the covariance matrix\index[subject]{covariance matrix} appears}

With two traits, each individual is a point in the plane. The mean is now a
point $(\bar{x}, \bar{y})$, and each individual's deviation is a vector:
\[
  \vect{d}_i =
  \begin{pmatrix}
    x_i - \bar{x} \\
    y_i - \bar{y}
  \end{pmatrix}.
\]

The squared distance from individual $i$ to the mean is
\[
  \|\vect{d}_i\|^2 = (x_i - \bar{x})^2 + (y_i - \bar{y})^2.
\]

If we average these squared distances, we get a single number---the total
variance, summed across traits. But this discards information. It does not
tell us whether the cloud is elongated, or in which direction.

To capture the shape of the cloud, we need to keep track of more than just
the sum of squares. We need the individual squared deviations in each trait
\emph{and} the cross-products between traits. This leads to the covariance
matrix:
\[
  \mat{S} =
  \frac{1}{n-1}
  \sum_{i=1}^{n}
  \vect{d}_i \vect{d}_i^\top
  =
  \frac{1}{n-1}
  \sum_{i=1}^{n}
  \begin{pmatrix}
    (x_i - \bar{x})^2 & (x_i - \bar{x})(y_i - \bar{y}) \\
    (x_i - \bar{x})(y_i - \bar{y}) & (y_i - \bar{y})^2
  \end{pmatrix}.
\]

The diagonal entries are the variances of each trait. The off-diagonal
entries are the covariances---they measure how the two traits vary together.

\begin{keyidea}
The covariance matrix\index[subject]{covariance matrix} packages the variances and covariances into a single
object. It captures not just how spread out the data are, but the shape and
orientation of the cloud.
\end{keyidea}

Notice the construction: each deviation vector $\vect{d}_i$ is multiplied by
its own transpose to form a $2 \times 2$ matrix, and these matrices are
averaged. This is the multivariate generalisation of ``average of squared
deviations.''

\section{A worked example}

Consider five individuals measured on two traits:

\begin{center}
\begin{tabular}{ccc}
  \hline
  Individual & Trait 1 ($x$) & Trait 2 ($y$) \\
  \hline
  1 & 2 & 3 \\
  2 & 4 & 5 \\
  3 & 3 & 4 \\
  4 & 5 & 7 \\
  5 & 6 & 6 \\
  \hline
\end{tabular}
\end{center}

The means are $\bar{x} = 4$ and $\bar{y} = 5$. The deviations are:

\begin{center}
\begin{tabular}{ccc}
  \hline
  Individual & $x_i - \bar{x}$ & $y_i - \bar{y}$ \\
  \hline
  1 & $-2$ & $-2$ \\
  2 & $0$ & $0$ \\
  3 & $-1$ & $-1$ \\
  4 & $1$ & $2$ \\
  5 & $2$ & $1$ \\
  \hline
\end{tabular}
\end{center}

The sums of squares and cross-products are:
\begin{align*}
  \sum (x_i - \bar{x})^2 &= 4 + 0 + 1 + 1 + 4 = 10, \\
  \sum (y_i - \bar{y})^2 &= 4 + 0 + 1 + 4 + 1 = 10, \\
  \sum (x_i - \bar{x})(y_i - \bar{y}) &= 4 + 0 + 1 + 2 + 2 = 9.
\end{align*}

Dividing by $n - 1 = 4$:
\[
  \mat{S} =
  \begin{pmatrix}
    2.5 & 2.25 \\
    2.25 & 2.5
  \end{pmatrix}.
\]

The variances are equal (2.5 each), and the covariance is positive (2.25),
indicating that the traits tend to increase together. The cloud is elongated
along a diagonal.

\section{The covariance matrix\index[subject]{covariance matrix} as a shape}

The covariance matrix\index[subject]{covariance matrix} $\mat{S}$ is symmetric: the $(1,2)$ entry equals the
$(2,1)$ entry. From Chapter~2, we know that symmetric matrices describe
shapes---ellipses in two dimensions, ellipsoids in higher dimensions.

What shape does $\mat{S}$ describe? Consider the set of all points
$\vect{z}$ satisfying
\[
  (\vect{z} - \bar{\vect{z}})^\top \mat{S}^{-1} (\vect{z} - \bar{\vect{z}}) = c
\]
for some constant $c$. This is an ellipse centred at the mean
(\cref{fig:covariance-ellipse}).

\begin{figure}[ht]
  \centering
  \includegraphics[width=0.7\textwidth]{fig10_covariance_ellipse.png}
  \caption[The covariance ellipse]{
    The covariance matrix\index[subject]{covariance matrix} defines an ellipse. Points on the ellipse are
    equidistant from the mean in a sense we will make precise. The
    orientation and elongation of the ellipse encode the correlations
    between traits.
  }
  \label{fig:covariance-ellipse}
\end{figure}

The eigenvalue\index[subject]{eigenvalue}s of $\mat{S}$ determine the lengths of the ellipse's axes.
The eigenvector\index[subject]{eigenvector}s determine the directions of those axes. If the eigenvalue\index[subject]{eigenvalue}s
are equal, the ellipse is a circle. If they are very different, the ellipse
is elongated.

We will develop this connection fully in Chapter~12. For now, the key point
is that squaring gives us access to this geometric structure. The covariance
matrix is not just a table of numbers; it is a description of shape.

\section{What squared distance assumes}

Euclidean distance\index[subject]{distance!Euclidean} distance treats all traits equally. One unit of body length counts
the same as one unit of wing span, regardless of how variable each trait is
or how they correlate.

This assumption is often unreasonable:

\begin{itemize}
  \item If body length varies over a range of 10 units and wing span varies
        over 0.1 units, a difference of 1 unit in body length is tiny
        (within normal variation), while a difference of 1 unit in wing span
        is enormous (ten times the typical range).
  \item If body length and wing span are positively correlated, an individual
        with large body and small wings is unusual not because either trait
        is extreme, but because the \emph{combination} is rare.
\end{itemize}

Euclidean distance\index[subject]{distance!Euclidean} distance ignores both issues. It can badly misrepresent how
``different'' two phenotypes really are.

\begin{keyidea}
Euclidean distance\index[subject]{distance!Euclidean} distance assumes all traits are measured on comparable scales and
are uncorrelated. When these assumptions fail, we need a better metric.
\end{keyidea}

This sets up the problem that the next chapter will solve.

\section{Summary}

In this chapter we have:

\begin{itemize}
  \item Motivated distance as a fundamental biological quantity---the answer
        to ``how different are these phenotypes?''
  \item Derived the Pythagorean formula and connected it to
        $\vect{v}^\top\vect{v}$.
  \item Explained why we square: geometry demands it for straight-line
        distance; algebra rewards it with additive decomposition; calculus
        prefers the smoothness.
  \item Reframed variance as mean squared distance from the mean.
  \item Introduced the covariance matrix\index[subject]{covariance matrix} as a summary of spread that captures
        shape, not just magnitude.
  \item Noted that Euclidean distance\index[subject]{distance!Euclidean} distance assumes comparable scales and no
        correlation---assumptions that often fail.
\end{itemize}

The stage is set. In the next chapter, we will see concrete examples where
Euclidean distance\index[subject]{distance!Euclidean} distance gives misleading answers, and we will introduce the
Mahalanobis distance\index[subject]{distance!Mahalanobis} distance as the remedy. The covariance matrix\index[subject]{covariance matrix} will move from
being a summary statistic to being a tool for measuring distance itself.

%%%%%%%%%%%%%%%%%%%%%%%%%%%%%%%%%%%%%%%%%%%%%%%%%%%%%%%%%%%%%%%%%%%%%%%%%%%%%%%
% CHAPTER 10: DISTANCE AND WHY WE SQUARE IT
%%%%%%%%%%%%%%%%%%%%%%%%%%%%%%%%%%%%%%%%%%%%%%%%%%%%%%%%%%%%%%%%%%%%%%%%%%%%%%%

\section*{Exercises}

\paragraph{Exercise 10.1 (Euclidean distance).}
Compute the Euclidean distance between:
\begin{enumerate}
  \item $(0, 0)$ and $(3, 4)$
  \item $(1, 2)$ and $(4, 6)$
  \item $(0, 0, 0)$ and $(1, 2, 2)$
  \item $(1, 1, 1, 1)$ and $(2, 2, 2, 2)$
\end{enumerate}

\paragraph{Exercise 10.2 (Variance as mean squared distance).}
Consider the data set $\{2, 4, 6, 8, 10\}$.

\begin{enumerate}
  \item Compute the mean.
  \item Compute each observation's deviation from the mean.
  \item Square these deviations and compute their average. This is the
        variance.
  \item Verify using the formula $\Var(X) = \E[X^2] - (\E[X])^2$.
\end{enumerate}

\paragraph{Exercise 10.3 (Why squaring?).}
Suppose we defined ``variance'' using absolute deviations instead of
squared deviations: $\text{MAD} = \frac{1}{n}\sum_i |x_i - \bar{x}|$.

\begin{enumerate}
  \item Compute MAD for the data $\{2, 4, 6, 8, 10\}$.
  \item The function $f(x) = |x|$ has a corner at $x = 0$. Why does this
        make calculus difficult?
  \item The function $g(x) = x^2$ is smooth everywhere. Why does this
        matter for optimisation?
  \item Give one advantage and one disadvantage of using squared
        deviations.
\end{enumerate}

\paragraph{Exercise 10.4 (The mean minimises squared distance).}
Consider three points on a line: $x_1 = 1$, $x_2 = 3$, $x_3 = 5$.

\begin{enumerate}
  \item Compute the sum of squared distances from each point to $c = 2$:
        $\sum_i (x_i - 2)^2$.
  \item Compute the sum of squared distances from each point to $c = 3$
        (the mean).
  \item Compute the sum of squared distances from each point to $c = 4$.
  \item Which value of $c$ minimises the sum of squared distances?
  \item (Challenge) For general data $x_1, \ldots, x_n$, use calculus to
        show that the sum of squared distances is minimised when $c = \bar{x}$.
\end{enumerate}

\paragraph{Exercise 10.5 (Distance in trait space).}
Two birds are measured:
\begin{itemize}
  \item Bird A: wing = 10 cm, tarsus = 2 cm
  \item Bird B: wing = 12 cm, tarsus = 2.5 cm
\end{itemize}

\begin{enumerate}
  \item Compute the Euclidean distance between them.
  \item Now express wing in mm instead of cm. Recompute the distance.
  \item Why did the distance change? What does this tell us about using
        raw Euclidean distance for traits measured on different scales?
\end{enumerate}

\chapter{When Euclidean distance\index[subject]{distance!Euclidean} Distance Fails}

In the previous chapter we derived the Euclidean distance\index[subject]{distance!Euclidean} distance formula and saw
why squaring plays a central role in statistics. But we ended with a
warning: Euclidean distance\index[subject]{distance!Euclidean} distance assumes that all traits are measured on
comparable scales and that traits are uncorrelated. When these assumptions
fail, Euclidean distance\index[subject]{distance!Euclidean} distance can give badly misleading answers.

This chapter presents three examples where Euclidean distance\index[subject]{distance!Euclidean} distance leads us
astray. Each example reveals a different way the assumption can break. By
the end, you will be convinced that we need a better metric---one that
accounts for the structure of variation in the data.

\section{Example 1: The problem of scale}

Consider two morphological traits measured on a sample of beetles:

\begin{itemize}
  \item Elytra length, measured in millimetres, with a sample mean of 12\,mm
        and a standard deviation of 2\,mm.
  \item Body mass, measured in milligrams, with a sample mean of 450\,mg
        and a standard deviation of 80\,mg.
\end{itemize}

Now compare two beetles to a reference individual at the population mean
(12\,mm, 450\,mg):

\begin{center}
\begin{tabular}{lccc}
  \hline
  Beetle & Elytra (mm) & Mass (mg) & Euclidean distance\index[subject]{distance!Euclidean} distance \\
  \hline
  Reference & 12 & 450 & 0 \\
  A & 14 & 450 & 2 \\
  B & 12 & 530 & 80 \\
  \hline
\end{tabular}
\end{center}

Beetle A differs from the reference by 2\,mm in elytra length. Beetle B
differs by 80\,mg in mass. The Euclidean distance\index[subject]{distance!Euclidean} distances are 2 and 80,
respectively. By this measure, beetle B is 40 times more different from
the reference than beetle A.

But wait. Beetle A is one standard deviation above the mean in elytra
length ($2\,\text{mm} / 2\,\text{mm} = 1$). Beetle B is also one standard
deviation above the mean in mass ($80\,\text{mg} / 80\,\text{mg} = 1$). In
terms of how unusual each beetle is within the population, they are
\emph{equally extreme}.

\begin{figure}[ht]
  \centering
  \includegraphics[width=0.75\textwidth]{fig11_scale_problem.png}
  \caption[The scale problem]{
    Euclidean distance\index[subject]{distance!Euclidean} distance depends on measurement units. Beetles A and B are
    each one standard deviation from the mean, but Euclidean distance\index[subject]{distance!Euclidean} distance
    makes B appear 40 times more extreme because mass is measured in
    larger numbers.
  }
  \label{fig:scale-problem}
\end{figure}

The problem is clear: Euclidean distance\index[subject]{distance!Euclidean} distance treats a difference of 1\,mm the
same as a difference of 1\,mg, even though 1\,mm is a large deviation for
elytra length while 1\,mg is tiny for body mass. The units of measurement
contaminate our notion of ``how different.''

\begin{keyidea}
Euclidean distance\index[subject]{distance!Euclidean} distance is not unit-free. If you change from millimetres to
metres, or from milligrams to grams, the distances change. This is a
problem because biological questions should not depend on arbitrary choices
of measurement scale.
\end{keyidea}

\subsection*{A partial fix: standardisation}

One common remedy is to standardise each trait by its standard deviation
before computing distances. Define
\[
  z_{\text{std}} = \frac{z - \bar{z}}{s},
\]
where $s$ is the sample standard deviation. After standardisation, each
trait has mean zero and standard deviation one.

For our beetles:
\begin{align*}
  \text{Beetle A (standardised):} &\quad (1, 0) \\
  \text{Beetle B (standardised):} &\quad (0, 1)
\end{align*}

The Euclidean distance\index[subject]{distance!Euclidean} distances from the origin (the standardised mean) are now
both equal to 1. Problem solved?

Not quite. Standardisation fixes the scale problem, but it ignores
something else: the correlation\index[subject]{correlation}between traits. The next example shows why
this matters.

\section{Example 2: The problem of correlation}

Suppose elytra length and body mass are positively correlated---larger
beetles tend to be both longer and heavier. This is biologically
reasonable; body parts scale together.

\begin{figure}[ht]
  \centering
  \includegraphics[width=0.75\textwidth]{fig11_correlation_problem.png}
  \caption[The correlation\index[subject]{correlation}problem]{
    When traits are correlated, the cloud of individuals is elongated.
    Points C and D are equally far from the mean in Euclidean distance\index[subject]{distance!Euclidean} terms, but
    C lies along the main axis of variation (common) while D lies
    perpendicular to it (rare).
  }
  \label{fig:correlation-problem}
\end{figure}

Consider two more beetles, both one unit of Euclidean distance\index[subject]{distance!Euclidean} distance from the
mean after standardisation (\cref{fig:correlation-problem}):

\begin{itemize}
  \item Beetle C: large elytra and heavy body---both traits elevated
        together, in the direction of the correlation.
  \item Beetle D: large elytra but light body---one trait elevated, the
        other depressed, against the correlation.
\end{itemize}

Euclidean distance\index[subject]{distance!Euclidean} distance says C and D are equally different from the mean. But
look at the data cloud. Beetles like C are common; they lie along the
elongated axis of the ellipse. Beetles like D are rare; they lie in a
direction where the population shows little variation.

In biological terms, a beetle with large elytra and heavy body is just a
``big beetle''---unusual in size but not in proportion. A beetle with large
elytra but light body has an unusual \emph{combination} of traits. It is a
genuine outlier, not just an extreme of normal variation.

\begin{keyidea}
Euclidean distance\index[subject]{distance!Euclidean} distance ignores correlation\index[subject]{correlation}structure. It treats directions of
high variation the same as directions of low variation. Two points can have
the same Euclidean distance\index[subject]{distance!Euclidean} distance from the mean but very different probabilities
under the population distribution.
\end{keyidea}

\subsection*{Why standardisation does not help}

Standardising each trait individually does not fix this problem.
Standardisation rescales the axes so that each trait has variance 1, but it
does not rotate the axes to align with the correlation\index[subject]{correlation}structure. The
ellipse becomes a different ellipse (closer to circular along the original
axes), but it is still tilted.

What we need is a distance measure that accounts for the full covariance
structure---both the variances and the correlations. That measure is the
Mahalanobis distance\index[subject]{distance!Mahalanobis} distance, which we will develop in the next chapter.

\begin{figure}[htbp]
    \centering
    \includegraphics[width=\textwidth]{figures/fig_ch5_quantile_comparison.pdf}
    \caption{\textbf{Probability calibration: Euclidean versus Mahalanobis distance.}
    Data drawn from a bivariate normal with correlation $\rho = 0.8$.
    (a)~Scatter plot with contours of constant distance. Red dashed circles 
    show Euclidean distance contours; blue ellipses show Mahalanobis distance 
    contours aligned with the covariance structure. Point A lies along the 
    correlation direction (common phenotype); point B lies against it (rare 
    phenotype). Despite similar Euclidean distances, their Mahalanobis distances 
    differ dramatically: $D_{\text{Mah}}(A) = 1.05$ versus $D_{\text{Mah}}(B) = 2.39$.
    (b)~Distribution of squared distances. Mahalanobis $D^2$ follows the 
    theoretical $\chi^2_2$ distribution (black curve); Euclidean $D^2$ does not.
    (c)~Tail probabilities $P(D^2 > c)$. At the 95th percentile of $\chi^2_2$ 
    ($c = 5.99$), Mahalanobis correctly identifies 5\% of points as extreme, 
    while Euclidean identifies 17\%---a threefold miscalibration. This is why 
    Mahalanobis distance is essential for probability-based inference: it is 
    the only metric properly calibrated to the data's covariance structure.}
    \label{fig:quantile_comparison}
\end{figure}

\section{Example 3: The probability perspective}

Here is another way to see the problem. Suppose our two traits follow a
bivariate normal distribution with means $\mu_1 = 0$, $\mu_2 = 0$, standard
deviations $\sigma_1 = 1$, $\sigma_2 = 1$, and correlation\index[subject]{correlation}$\rho = 0.8$.

The probability density at a point $(z_1, z_2)$ depends on how ``central''
that point is. Points near the mean have high density; points far from the
mean have low density. But ``far from the mean'' must be measured in a way
that respects the correlation.

\begin{figure}[ht]
  \centering
  \includegraphics[width=0.8\textwidth]{fig11_probability_contours.png}
  \caption[Probability contours]{
    Contours of equal probability density for a correlated bivariate
    normal. The contours are ellipses, not circles. Points on the same
    Euclidean distance\index[subject]{distance!Euclidean} circle (dashed) can have very different probability
    densities.
  }
  \label{fig:probability-contours}
\end{figure}

\Cref{fig:probability-contours} shows contours of equal probability density.
These contours are ellipses aligned with the correlation\index[subject]{correlation}structure. The
dashed circle shows points at constant Euclidean distance\index[subject]{distance!Euclidean} distance from the mean.
Notice that:

\begin{itemize}
  \item Some points on the circle lie well inside the 95\% probability
        contour---they are not unusual at all.
  \item Other points on the same circle lie far outside---they are extremely
        rare.
\end{itemize}

Euclidean distance\index[subject]{distance!Euclidean} distance cannot distinguish these cases. It sees only the radius
of the circle, not where on the circle the point lies.

\begin{keyidea}
Euclidean distance\index[subject]{distance!Euclidean} distance is blind to the shape of the distribution. Points at the
same Euclidean distance\index[subject]{distance!Euclidean} distance can have wildly different probabilities. A proper
distance metric should make ``equally distant'' mean ``equally probable.''
\end{keyidea}

\section{A biological interlude: why this matters for selection}

These are not just statistical curiosities. The failure of Euclidean distance\index[subject]{distance!Euclidean}
distance has direct consequences for how we think about natural selection.

Consider stabilising selection on multiple traits. Fitness decreases as
phenotypes deviate from an optimum. If we model this using Euclidean distance\index[subject]{distance!Euclidean}
distance from the optimum,
\[
  w(\vect{z}) = \exp\left( -\frac{1}{2} \|\vect{z} - \boldsymbol{\theta}\|^2 \right),
\]
we are implicitly assuming that all directions of deviation are equally
costly. A beetle that is too long receives the same fitness penalty as a
beetle that is too heavy, unit for unit.

But if the population has abundant genetic variation in the ``too long and
too heavy'' direction and little variation in the ``too long but too
light'' direction, these deviations are not biologically equivalent. The
first represents a common, easily produced phenotype; the second represents
a rare, difficult-to-produce phenotype.

A more realistic fitness function would penalise deviations according to
how unusual they are in the population---that is, according to Mahalanobis distance\index[subject]{distance!Mahalanobis}
distance, not Euclidean distance\index[subject]{distance!Euclidean} distance. We will return to this point when we
discuss selection surfaces in Part~IV.

\section{What we need from a better metric}

Let us summarise what Euclidean distance\index[subject]{distance!Euclidean} distance gets wrong and what a better
metric should provide.

\paragraph{Problem 1: Scale dependence.}
Euclidean distance\index[subject]{distance!Euclidean} distance changes when we change units. A better metric should be
\emph{scale-invariant}---the answer should not depend on whether we measure
length in millimetres or metres.

\paragraph{Problem 2: Ignoring correlation.}
Euclidean distance\index[subject]{distance!Euclidean} distance treats all directions equally. A better metric should
weight directions according to how variable the population is in those
directions. Deviations in low-variance directions should count more than
deviations in high-variance directions.

\paragraph{Problem 3: Disconnection from probability.}
Euclidean distance\index[subject]{distance!Euclidean} distance does not correspond to probability. A better metric
should have the property that points at the same distance from the mean are
equally probable under the population distribution.

All three problems have a common solution: use the covariance matrix\index[subject]{covariance matrix} to
define distance. The covariance matrix\index[subject]{covariance matrix} knows about both variances (which
address the scale problem) and correlations (which address the shape
problem). Using it correctly will give us a distance that corresponds to
probability.

\section{A geometric preview}

Before we derive the Mahalanobis distance\index[subject]{distance!Mahalanobis} distance formally, let us see geometrically
what we are aiming for.

\begin{figure}[ht]
  \centering
  \includegraphics[width=0.9\textwidth]{fig11_geometric_preview.png}
  \caption[Geometric preview of the solution]{
    Left: Euclidean distance\index[subject]{distance!Euclidean} distance uses circles centred on the mean. Points on
    the same circle are ``equally far'' even if some are common and others
    rare. Right: Mahalanobis distance\index[subject]{distance!Mahalanobis} distance uses ellipses that match the shape
    of the data. Points on the same ellipse are equally probable---this is
    the right notion of ``equally far'' for a population.
  }
  \label{fig:geometric-preview}
\end{figure}

The left panel of \cref{fig:geometric-preview} shows Euclidean distance\index[subject]{distance!Euclidean} distance:
circles centred on the mean. The right panel shows what we want: ellipses
that match the shape of the covariance structure. Points on the same ellipse
are equally probable; they represent the same degree of ``unusualness.''

The transformation from circles to ellipses is exactly what the covariance
matrix encodes. In the next chapter, we will see how to use the inverse of
the covariance matrix\index[subject]{covariance matrix} to define a new distance---the Mahalanobis distance\index[subject]{distance!Mahalanobis}
distance---that has all the properties we want.

\section{A worked comparison}

To make the contrast concrete, consider the following covariance matrix\index[subject]{covariance matrix}:
\[
  \mat{S} =
  \begin{pmatrix}
    1.0 & 0.8 \\
    0.8 & 1.0
  \end{pmatrix}.
\]

Both traits have variance 1, and the correlation\index[subject]{correlation}is 0.8. Consider three
points, all at Euclidean distance\index[subject]{distance!Euclidean} distance 1 from the origin:

\begin{center}
\begin{tabular}{lcccc}
  \hline
  Point & $z_1$ & $z_2$ & Euclidean distance\index[subject]{distance!Euclidean} dist & Direction \\
  \hline
  E & $1/\sqrt{2}$ & $1/\sqrt{2}$ & 1.0 & Along correlation\index[subject]{correlation}\\
  F & $1/\sqrt{2}$ & $-1/\sqrt{2}$ & 1.0 & Against correlation\index[subject]{correlation}\\
  G & $1$ & $0$ & 1.0 & Along trait 1 only \\
  \hline
\end{tabular}
\end{center}

Point E lies along the major axis of the ellipse (both traits elevated
together). Point F lies along the minor axis (traits in opposition). Point G
lies along the first trait axis.

All three have Euclidean distance\index[subject]{distance!Euclidean} distance 1. But their Mahalanobis distance\index[subject]{distance!Mahalanobis} distances (which
we will compute properly in the next chapter) are:

\begin{center}
\begin{tabular}{lcc}
  \hline
  Point & Euclidean distance\index[subject]{distance!Euclidean} dist & Mahalanobis distance\index[subject]{distance!Mahalanobis} dist \\
  \hline
  E & 1.0 & 0.53 \\
  F & 1.0 & 2.36 \\
  G & 1.0 & 1.00 \\
  \hline
\end{tabular}
\end{center}

Point E, which lies in the direction of maximum variation, has a small
Mahalanobis distance\index[subject]{distance!Mahalanobis} distance---it is not unusual. Point F, which lies in the
direction of minimum variation, has a large Mahalanobis distance\index[subject]{distance!Mahalanobis} distance---it is
very unusual. Point G is intermediate.

\begin{keyidea}
Mahalanobis distance\index[subject]{distance!Mahalanobis} distance rescales directions by how variable the population is
in those directions. Large deviations in high-variance directions are less
surprising than small deviations in low-variance directions.
\end{keyidea}

\section{The connection to standardisation}

You might wonder: if we standardise each trait to have variance 1, and then
rotate to align with the principal axes of the correlation\index[subject]{correlation}matrix, would
that fix the problem?

Yes---and that is exactly what Mahalanobis distance\index[subject]{distance!Mahalanobis} distance does, algebraically.
It can be understood as:

\begin{enumerate}
  \item Standardise each trait by its standard deviation (fixing the scale
        problem).
  \item Rotate to the principal axes of the covariance matrix\index[subject]{covariance matrix} (fixing the
        correlation\index[subject]{correlation}problem).
  \item Measure ordinary Euclidean distance\index[subject]{distance!Euclidean} distance in this transformed space.
\end{enumerate}

The matrix that performs this combined standardisation-and-rotation is
related to the inverse (or inverse square root) of the covariance matrix\index[subject]{covariance matrix}.
We will derive this in the next chapter.

This preview should help you see that Mahalanobis distance\index[subject]{distance!Mahalanobis} distance is not some
arbitrary alternative to Euclidean distance\index[subject]{distance!Euclidean} distance. It is the natural distance in
a space that has been ``whitened''---transformed so that the covariance
matrix becomes the identity. In whitened space, Euclidean distance\index[subject]{distance!Euclidean} distance works
correctly because the assumptions behind it are satisfied.

\section{Summary}

In this chapter we have seen three ways that Euclidean distance\index[subject]{distance!Euclidean} distance fails:

\begin{itemize}
  \item \textbf{Scale dependence:} Distances change with measurement units.
        Two individuals equally extreme in their traits can have very
        different Euclidean distance\index[subject]{distance!Euclidean} distances.
  \item \textbf{Ignoring correlation:} Directions of high and low variation
        are treated equally. Points along the major axis of variation
        appear just as extreme as points along the minor axis.
  \item \textbf{Disconnection from probability:} Points at the same
        Euclidean distance\index[subject]{distance!Euclidean} distance can have very different probabilities under
        the population distribution.
\end{itemize}

We have also previewed the solution:

\begin{itemize}
  \item The covariance matrix\index[subject]{covariance matrix} encodes both scale (variances) and shape
        (correlations).
  \item Using the covariance matrix\index[subject]{covariance matrix} to define distance gives us contours
        that are ellipses matching the data, not circles ignoring it.
  \item This distance---the Mahalanobis distance\index[subject]{distance!Mahalanobis} distance---makes ``equally far''
        mean ``equally probable.''
\end{itemize}

In the next chapter, we derive the Mahalanobis distance\index[subject]{distance!Mahalanobis} distance formally and show
how the inverse covariance matrix\index[subject]{covariance matrix} enters the formula. The key insight will
be that inserting a matrix between the vectors in our distance formula
changes the shape of the ``unit ball'' from a circle to an ellipse.


Copy

%%%%%%%%%%%%%%%%%%%%%%%%%%%%%%%%%%%%%%%%%%%%%%%%%%%%%%%%%%%%%%%%%%%%%%%%%%%%%%%
% EXERCISE SECTIONS FOR CHAPTERS 00-02, 10-12, AND 32
%%%%%%%%%%%%%%%%%%%%%%%%%%%%%%%%%%%%%%%%%%%%%%%%%%%%%%%%%%%%%%%%%%%%%%%%%%%%%%%
%
% Instructions: Copy each section to the end of the corresponding chapter,
% just before the final \end{document} or before the next \chapter command.
%
% These exercises are designed to reinforce key concepts without solutions.
% A companion "Hints and Selected Solutions" appendix follows at the end
% of this file.
%
%%%%%%%%%%%%%%%%%%%%%%%%%%%%%%%%%%%%%%%%%%%%%%%%%%%%%%%%%%%%%%%%%%%%%%%%%%%%%%%


%%%%%%%%%%%%%%%%%%%%%%%%%%%%%%%%%%%%%%%%%%%%%%%%%%%%%%%%%%%%%%%%%%%%%%%%%%%%%%%
% CHAPTER 00: POINTS AND TRAIT SPACE
%%%%%%%%%%%%%%%%%%%%%%%%%%%%%%%%%%%%%%%%%%%%%%%%%%%%%%%%%%%%%%%%%%%%%%%%%%%%%%%

\section*{Exercises}

\paragraph{Exercise 0.1 (Plotting a phenotype cloud).}
Five plants are measured for leaf length (cm) and leaf width (cm):

\begin{center}
\begin{tabular}{ccc}
\hline
Plant & Length & Width \\
\hline
A & 4.2 & 2.1 \\
B & 5.1 & 2.8 \\
C & 3.8 & 1.9 \\
D & 4.7 & 2.4 \\
E & 4.2 & 2.3 \\
\hline
\end{tabular}
\end{center}

\begin{enumerate}
  \item Plot these five individuals as points in a two-dimensional trait space.
  \item Estimate the centroid (mean phenotype) by eye from your plot.
  \item Calculate the centroid exactly. How close was your estimate?
  \item A sixth plant F has measurements (6.0, 3.5). Add it to your plot.
        How does the centroid shift?
\end{enumerate}

\paragraph{Exercise 0.2 (Trait space dimensions).}
A bird ecologist measures wing length, tarsus length, bill depth, and body
mass on each individual.

\begin{enumerate}
  \item How many dimensions does this trait space have?
  \item Can you visualise this space directly? If not, what strategies
        might help you understand the distribution of individuals?
  \item If you added bill width as a fifth trait, how would the
        dimensionality change?
\end{enumerate}

\paragraph{Exercise 0.3 (Phenotype as position).}
Consider two fish: Fish 1 has length 15 cm and mass 50 g; Fish 2 has
length 20 cm and mass 80 g.

\begin{enumerate}
  \item Represent each fish as a point in (length, mass) space.
  \item Draw the arrow from Fish 1 to Fish 2. What does this arrow
        represent biologically?
  \item If a third fish lies exactly halfway along this arrow, what are
        its length and mass?
\end{enumerate}

\paragraph{Exercise 0.4 (The meaning of ``distance'' in trait space).}
Two flowers differ in petal length by 2 mm and in petal width by 3 mm.

\begin{enumerate}
  \item What is the straight-line (Euclidean) distance between them in
        trait space?
  \item Does this number have a direct biological interpretation?
  \item What might make two flowers ``far apart'' in trait space but
        similar in fitness?
\end{enumerate}


%%%%%%%%%%%%%%%%%%%%%%%%%%%%%%%%%%%%%%%%%%%%%%%%%%%%%%%%%%%%%%%%%%%%%%%%%%%%%%%
% CHAPTER 11: WHEN EUCLIDEAN DISTANCE FAILS
%%%%%%%%%%%%%%%%%%%%%%%%%%%%%%%%%%%%%%%%%%%%%%%%%%%%%%%%%%%%%%%%%%%%%%%%%%%%%%%

\section*{Exercises I}

\paragraph{Exercise 11.1 (Scale dependence).}
A data set contains height (in metres) and weight (in kg):

\begin{center}
\begin{tabular}{ccc}
\hline
Individual & Height (m) & Weight (kg) \\
\hline
A & 1.70 & 70 \\
B & 1.75 & 72 \\
C & 1.80 & 90 \\
\hline
\end{tabular}
\end{center}

\begin{enumerate}
  \item Compute the Euclidean distance from A to B and from A to C.
  \item Convert height to centimetres. Recompute both distances.
  \item Which distance changed more? Why?
  \item Propose a way to make the distance independent of measurement units.
\end{enumerate}

\paragraph{Exercise 11.2 (Ignoring correlation).}
Imagine a bivariate distribution where traits $X$ and $Y$ are strongly
positively correlated ($r = 0.95$). Most individuals lie near the line
$Y = X$.

\begin{enumerate}
  \item Sketch this distribution as an elliptical cloud.
  \item Mark two points that are equidistant (in Euclidean terms) from the
        centre: one along the major axis, one along the minor axis.
  \item Which point is more ``unusual'' given the shape of the distribution?
  \item Why does Euclidean distance fail to capture this?
\end{enumerate}

\paragraph{Exercise 11.3 (A concrete failure).}
Consider a population where leaf length ($L$) and leaf width ($W$) have:
\begin{itemize}
  \item Mean: $\bar{L} = 10$ cm, $\bar{W} = 5$ cm
  \item Standard deviation: $s_L = 2$ cm, $s_W = 1$ cm
  \item Correlation: $r = 0.8$
\end{itemize}

Three individuals are measured:
\begin{itemize}
  \item Plant X: $(L, W) = (12, 6)$
  \item Plant Y: $(L, W) = (10, 7)$
  \item Plant Z: $(L, W) = (14, 5)$
\end{itemize}

\begin{enumerate}
  \item Compute the Euclidean distance of each plant from the mean.
  \item Standardise each trait (subtract mean, divide by SD) and recompute
        distances.
  \item Which plant is most unusual given the correlation structure?
        (Hint: think about which point lies furthest from the major axis
        of the ellipse.)
\end{enumerate}

\paragraph{Exercise 11.4 (When Euclidean distance works).}
Under what conditions would Euclidean distance be a reasonable measure
of dissimilarity?

\begin{enumerate}
  \item List two conditions on the traits.
  \item Give a biological example where these conditions might hold.
  \item Give a biological example where they clearly do not hold.
\end{enumerate}


\chapter{Covariance and Mahalanobis \index[subject]{distance!Mahalanobis} Distance}

The previous chapter showed three ways that Euclidean\index[subject]{distance!Euclidean} distance fails: it
depends on measurement units, it ignores correlations, and it does not
correspond to probability. We previewed the solution---ellipses that match
the shape of the data---but did not derive it.

This chapter develops that solution. We will see how the covariance matrix
enters the distance formula, why the \emph{inverse} of the covariance matrix\index[subject]{covariance matrix}
appears, and what it means geometrically. By the end, you will understand
the Mahalanobis \index[subject]{distance!Mahalanobis} distance not as an arbitrary formula but as the natural way
to measure ``how unusual'' a phenotype is.

\section{The key insight: a matrix between the vectors}

Recall from Chapter~1 that the squared Euclidean\index[subject]{distance!Euclidean} distance between two
points $\vect{z}_i$ and $\vect{z}_j$ can be written as
\[
  d^2_{\text{Euc}} 
    = (\vect{z}_j - \vect{z}_i)^\top (\vect{z}_j - \vect{z}_i).
\]

This is just the dot product of the difference vector with itself---the
sum of squared components.

Now consider what happens if we insert a matrix $\mat{M}$ between the
transpose and the vector:
\[
  d^2_{\mat{M}} 
    = (\vect{z}_j - \vect{z}_i)^\top \mat{M} (\vect{z}_j - \vect{z}_i).
\]

This is still a quadratic form\index[subject]{quadratic form}. It still takes a vector and returns a
non-negative number (provided $\mat{M}$ is positive definite\index[subject]{matrix!positive definite}). But the
matrix $\mat{M}$ changes how different directions are weighted.

\begin{keyidea}
Inserting a matrix into the distance formula changes the shape of the
``unit ball.'' With the identity matrix, the unit ball is a sphere. With
a general positive definite\index[subject]{matrix!positive definite} matrix, the unit ball becomes an ellipsoid.
\end{keyidea}

The question is: which matrix $\mat{M}$ should we use?

\section{The covariance matrix\index[subject]{covariance matrix} and its inverse}

If the problem with Euclidean\index[subject]{distance!Euclidean} distance is that it ignores the covariance
structure of the data, then the solution should involve the covariance
matrix ${\Sigma}$.

But should we use ${\Sigma}$ itself, or its inverse ${\Sigma}^{-1}$?

Consider what we want. We want deviations in high-variance directions to
count \emph{less} (because they are common) and deviations in low-variance
directions to count \emph{more} (because they are rare). This means we want
to \emph{downweight} directions of large variance and \emph{upweight}
directions of small variance.

The covariance matrix\index[subject]{covariance matrix} ${\Sigma}$ has large eigenvalue\index[subject]{eigenvalue}s in directions of
large variance. Its inverse ${\Sigma}^{-1}$ has \emph{small} eigenvalue\index[subject]{eigenvalue}s
in those same directions (since the eigenvalue\index[subject]{eigenvalue}s of the inverse are the
reciprocals of the original eigenvalue\index[subject]{eigenvalue}s).

Therefore, to downweight high-variance directions, we use the inverse:
\[
  d^2_{\text{Mah}} 
    = (\vect{z} - \boldsymbol{\mu})^\top {\Sigma}^{-1} (\vect{z} - \boldsymbol{\mu}).
\]

This is the \textbf{Mahalanobis distance\index[subject]{distance!Mahalanobis} distance} (squared) from the point
$\vect{z}$ to the mean $\boldsymbol{\mu}$.

\begin{keyidea}
The Mahalanobis distance\index[subject]{distance!Mahalanobis} distance uses the \emph{inverse} covariance matrix\index[subject]{covariance matrix} because
we want to penalise deviations in low-variance directions more heavily than
deviations in high-variance directions.
\end{keyidea}

\section{A one-dimensional sanity check}

Before tackling multiple traits, let us verify that the formula makes sense
in one dimension.

With a single trait, the covariance matrix\index[subject]{covariance matrix} is just the variance:
${\Sigma} = \sigma^2$. Its inverse is $1/\sigma^2$. The Mahalanobis distance\index[subject]{distance!Mahalanobis}
distance from a value $z$ to the mean $\mu$ is
\[
  d^2_{\text{Mah}} 
    = (z - \mu)^\top \cdot \frac{1}{\sigma^2} \cdot (z - \mu)
    = \frac{(z - \mu)^2}{\sigma^2}.
\]

Taking the square root:
\[
  d_{\text{Mah}} = \frac{|z - \mu|}{\sigma}.
\]

This is simply the number of standard deviations from the mean---the
familiar $z$-score! The Mahalanobis distance\index[subject]{distance!Mahalanobis} distance generalises the $z$-score to
multiple dimensions.

\begin{keyidea}
In one dimension, the Mahalanobis distance\index[subject]{distance!Mahalanobis} distance equals the absolute $z$-score.
In multiple dimensions, it generalises this idea by accounting for both
variances and covariances.
\end{keyidea}

\section{Geometry: how the inverse reshapes space}

Let us see geometrically why the inverse covariance matrix\index[subject]{covariance matrix} produces
ellipses that match the data.

Suppose the covariance matrix\index[subject]{covariance matrix} is
\[
  {\Sigma} =
  \begin{pmatrix}
    4 & 0 \\
    0 & 1
  \end{pmatrix}.
\]

Trait~1 has variance 4 (standard deviation 2), and trait~2 has variance 1
(standard deviation 1). There is no correlation, so the ellipse is aligned
with the axes (\cref{fig:inverse-geometry}).

\begin{figure}[ht]
  \centering
  \includegraphics[width=0.85\textwidth]{fig12_inverse_geometry.png}
  \caption[Geometry of the inverse]{
    Left: The covariance ellipse shows the shape of variation. Trait~1
    (horizontal) has more variance. Right: Using ${\Sigma}^{-1}$ in
    the distance formula produces contours that are circles in the
    \emph{standardised} space---ellipses matching the data in the original
    space.
  }
  \label{fig:inverse-geometry}
\end{figure}

The inverse covariance matrix\index[subject]{covariance matrix} is
\[
  {\Sigma}^{-1} =
  \begin{pmatrix}
    1/4 & 0 \\
    0 & 1
  \end{pmatrix}.
\]

When we compute $\vect{v}^\top {\Sigma}^{-1} \vect{v}$, the component
along trait~1 is divided by 4, while the component along trait~2 is left
unchanged. This shrinks distances in the high-variance direction and
leaves distances in the low-variance direction alone.

The result: a deviation of 2 units in trait~1 contributes the same to the
Mahalanobis distance\index[subject]{distance!Mahalanobis} distance as a deviation of 1 unit in trait~2. Both represent
one standard deviation from the mean.

\section{The formula in components}

For two traits with covariance matrix\index[subject]{covariance matrix}
\[
  {\Sigma} =
  \begin{pmatrix}
    \sigma_1^2 & \rho\sigma_1\sigma_2 \\
    \rho\sigma_1\sigma_2 & \sigma_2^2
  \end{pmatrix},
\]
the inverse is
\[
  {\Sigma}^{-1} =
  \frac{1}{\sigma_1^2\sigma_2^2(1-\rho^2)}
  \begin{pmatrix}
    \sigma_2^2 & -\rho\sigma_1\sigma_2 \\
    -\rho\sigma_1\sigma_2 & \sigma_1^2
  \end{pmatrix}.
\]

The squared Mahalanobis distance\index[subject]{distance!Mahalanobis} distance from a point $(z_1, z_2)$ to the mean
$(\mu_1, \mu_2)$ is
\[
  d^2_{\text{Mah}} =
  \frac{1}{1 - \rho^2}
  \left[
    \left(\frac{z_1 - \mu_1}{\sigma_1}\right)^2
    - 2\rho \left(\frac{z_1 - \mu_1}{\sigma_1}\right)
            \left(\frac{z_2 - \mu_2}{\sigma_2}\right)
    + \left(\frac{z_2 - \mu_2}{\sigma_2}\right)^2
  \right].
\]

This formula has three parts:

\begin{enumerate}
  \item The squared $z$-scores for each trait: 
        $\left(\frac{z_1 - \mu_1}{\sigma_1}\right)^2$ and
        $\left(\frac{z_2 - \mu_2}{\sigma_2}\right)^2$.
  \item A cross-term that subtracts when the correlation\index[subject]{correlation}is positive and
        both deviations have the same sign (reducing distance for points
        along the correlation) or adds when they have opposite signs
        (increasing distance for points against the correlation).
  \item A factor $\frac{1}{1-\rho^2}$ that inflates everything when
        correlation\index[subject]{correlation}is high, reflecting the reduced ``effective
        dimensionality'' of the data.
\end{enumerate}

When $\rho = 0$, the cross-term vanishes and we get
\[
  d^2_{\text{Mah}} =
    \left(\frac{z_1 - \mu_1}{\sigma_1}\right)^2
    + \left(\frac{z_2 - \mu_2}{\sigma_2}\right)^2,
\]
which is just the sum of squared $z$-scores---Euclidean\index[subject]{distance!Euclidean} distance in
standardised space.

\section{Mahalanobis distance\index[subject]{distance!Mahalanobis} distance and probability}

For multivariate normal data, the Mahalanobis distance\index[subject]{distance!Mahalanobis} distance has a direct
connection to probability.

If $\vect{z}$ follows a $p$-variate normal distribution with mean
$\boldsymbol{\mu}$ and covariance ${\Sigma}$, then the squared
Mahalanobis distance\index[subject]{distance!Mahalanobis} distance
\[
  d^2_{\text{Mah}} 
    = (\vect{z} - \boldsymbol{\mu})^\top {\Sigma}^{-1} (\vect{z} - \boldsymbol{\mu})
\]
follows a chi-squared distribution with $p$ degrees of freedom.

This means:

\begin{itemize}
  \item Points with $d^2_{\text{Mah}} < \chi^2_{p,0.95}$ lie within the
        95\% probability ellipse.
  \item The probability of observing a point at least as extreme as
        $\vect{z}$ can be computed directly from the chi-squared
        distribution.
  \item Contours of equal Mahalanobis distance\index[subject]{distance!Mahalanobis} distance are contours of equal
        probability density.
\end{itemize}

\begin{keyidea}
For multivariate normal data, Mahalanobis distance\index[subject]{distance!Mahalanobis} distance is directly tied to
probability. Equal Mahalanobis distance\index[subject]{distance!Mahalanobis} distance means equal probability density.
This is exactly what we wanted from a ``proper'' distance metric.
\end{keyidea}

\section{A worked example}

Let us compute Mahalanobis distance\index[subject]{distance!Mahalanobis} distances for the three points from
Chapter~11. The covariance matrix\index[subject]{covariance matrix} was
\[
  {\Sigma} =
  \begin{pmatrix}
    1.0 & 0.8 \\
    0.8 & 1.0
  \end{pmatrix},
  \qquad
  {\Sigma}^{-1} =
  \begin{pmatrix}
    2.778 & -2.222 \\
    -2.222 & 2.778
  \end{pmatrix}.
\]

Consider three points, all at Euclidean\index[subject]{distance!Euclidean} distance 1 from the origin:

\paragraph{Point E: $(1/\sqrt{2}, 1/\sqrt{2}) \approx (0.707, 0.707)$.}
This point lies along the major axis of the ellipse (both traits elevated
together).

\begin{align*}
  d^2_{\text{Mah}} &= 
    \begin{pmatrix} 0.707 & 0.707 \end{pmatrix}
    \begin{pmatrix} 2.778 & -2.222 \\ -2.222 & 2.778 \end{pmatrix}
    \begin{pmatrix} 0.707 \\ 0.707 \end{pmatrix} \\
  &= \begin{pmatrix} 0.707 & 0.707 \end{pmatrix}
     \begin{pmatrix} 0.393 \\ 0.393 \end{pmatrix} \\
  &= 0.556.
\end{align*}

So $d_{\text{Mah}} = \sqrt{0.556} \approx 0.75$.

\paragraph{Point F: $(1/\sqrt{2}, -1/\sqrt{2}) \approx (0.707, -0.707)$.}
This point lies along the minor axis (traits in opposition).

\begin{align*}
  d^2_{\text{Mah}} &= 
    \begin{pmatrix} 0.707 & -0.707 \end{pmatrix}
    \begin{pmatrix} 2.778 & -2.222 \\ -2.222 & 2.778 \end{pmatrix}
    \begin{pmatrix} 0.707 \\ -0.707 \end{pmatrix} \\
  &= \begin{pmatrix} 0.707 & -0.707 \end{pmatrix}
     \begin{pmatrix} 3.536 \\ -3.536 \end{pmatrix} \\
  &= 5.0.
\end{align*}

So $d_{\text{Mah}} = \sqrt{5.0} \approx 2.24$.

\paragraph{Point G: $(1, 0)$.}
This point lies along the first trait axis.

\begin{align*}
  d^2_{\text{Mah}} &= 
    \begin{pmatrix} 1 & 0 \end{pmatrix}
    \begin{pmatrix} 2.778 & -2.222 \\ -2.222 & 2.778 \end{pmatrix}
    \begin{pmatrix} 1 \\ 0 \end{pmatrix} \\
  &= 2.778.
\end{align*}

So $d_{\text{Mah}} = \sqrt{2.778} \approx 1.67$.

\begin{figure}[ht]
  \centering
  \includegraphics[width=0.75\textwidth]{fig12_worked_example.png}
  \caption[Worked example]{
    Three points at Euclidean\index[subject]{distance!Euclidean} distance 1 from the origin have very
    different Mahalanobis distance\index[subject]{distance!Mahalanobis} distances: E (along correlation) is closest,
    F (against correlation) is farthest, G is intermediate.
  }
  \label{fig:worked-example}
\end{figure}

\begin{center}
\begin{tabular}{lccc}
  \hline
  Point & Direction & Euclidean\index[subject]{distance!Euclidean} & Mahalanobis distance\index[subject]{distance!Mahalanobis} \\
  \hline
  E & Along correlation\index[subject]{correlation}& 1.0 & 0.75 \\
  F & Against correlation\index[subject]{correlation}& 1.0 & 2.24 \\
  G & Trait 1 only & 1.0 & 1.67 \\
  \hline
\end{tabular}
\end{center}

Point E, lying in the direction of maximum variance, is the \emph{least}
unusual---Mahalanobis distance\index[subject]{distance!Mahalanobis} distance is less than 1. Point F, lying in the
direction of minimum variance, is the \emph{most} unusual---Mahalanobis distance\index[subject]{distance!Mahalanobis}
distance exceeds 2. The Mahalanobis distance\index[subject]{distance!Mahalanobis} distance correctly identifies which
points are rare and which are common.

\section{The Mahalanobis \index[subject]{distance!Mahalanobis} distance as a transformation}

There is another way to understand Mahalanobis distance\index[subject]{distance!Mahalanobis} distance: as Euclidean\index[subject]{distance!Euclidean}
distance after a particular transformation.

Any positive definite\index[subject]{matrix!positive definite} matrix ${\Sigma}$ can be factored as
\[
  {\Sigma} = {\Sigma}^{1/2} {\Sigma}^{1/2},
\]
where ${\Sigma}^{1/2}$ is the matrix square root (the unique positive
definite matrix whose square is ${\Sigma}$).

Define the transformed variable
\[
  \vect{w} = {\Sigma}^{-1/2} (\vect{z} - \boldsymbol{\mu}).
\]

This transformation does two things:

\begin{enumerate}
  \item Centres the data at the origin.
  \item Rescales and rotates so that the covariance matrix\index[subject]{covariance matrix} of $\vect{w}$
        is the identity matrix $\mat{I}$.
\end{enumerate}

In the $\vect{w}$ space, the data form a spherical cloud with unit variance
in all directions and no correlations. This is called ``whitening\index[subject]{whitening transformation}'' or
``sphering'' the data.

Now compute the squared Euclidean\index[subject]{distance!Euclidean} length of $\vect{w}$:
\begin{align*}
  \|\vect{w}\|^2 
    &= \vect{w}^\top \vect{w} \\
    &= \left[ {\Sigma}^{-1/2} (\vect{z} - \boldsymbol{\mu}) \right]^\top
       \left[ {\Sigma}^{-1/2} (\vect{z} - \boldsymbol{\mu}) \right] \\
    &= (\vect{z} - \boldsymbol{\mu})^\top 
       {\Sigma}^{-1/2} {\Sigma}^{-1/2}
       (\vect{z} - \boldsymbol{\mu}) \\
    &= (\vect{z} - \boldsymbol{\mu})^\top {\Sigma}^{-1} 
       (\vect{z} - \boldsymbol{\mu}) \\
    &= d^2_{\text{Mah}}.
\end{align*}

\begin{keyidea}
Mahalanobis distance\index[subject]{distance!Mahalanobis} distance is ordinary Euclidean\index[subject]{distance!Euclidean} distance in ``whitened'' space
---the space where the data have been transformed to have identity
covariance. The transformation that achieves this is ${\Sigma}^{-1/2}$.
\end{keyidea}

This insight will be central to Part~III, where we develop whitening\index[subject]{whitening transformation} and
the ``P-sphere\index[subject]{P-sphere}'' as tools for understanding evolutionary constraints.

\begin{figure}[htbp]
    \centering
    \includegraphics[width=\textwidth]{figures/fig_ch6_whitening_steps.pdf}
    \caption{The whitening transformation step by step. (a) Original data with 
    correlated traits; the covariance ellipse ($\mathbf{P}$) is tilted. 
    (b) After rotation by $\mathbf{V}^\top$: data align with eigenvector axes; 
    covariance is now diagonal ($\boldsymbol{\Lambda}$). (c) After scaling by 
    $\boldsymbol{\Lambda}^{-1/2}$: variances are equalized; covariance becomes 
    identity. (d) The full transformation $\mathbf{P}^{-1/2}\mathbf{z}$ produces 
    spherical data. In this whitened space, the $\mathbf{P}$-sphere becomes 
    the unit circle, and Mahalanobis distance equals Euclidean distance.}
    \label{fig:whitening_steps}
\end{figure}

\section{Mahalanobis distance\index[subject]{distance!Mahalanobis} distance between two points}

So far we have measured distance from a point to the mean. But we can
also measure the Mahalanobis distance\index[subject]{distance!Mahalanobis} distance between any two points:
\[
  d^2_{\text{Mah}}(\vect{z}_i, \vect{z}_j) =
    (\vect{z}_j - \vect{z}_i)^\top {\Sigma}^{-1} (\vect{z}_j - \vect{z}_i).
\]

This asks: how different are these two phenotypes, measured in units that
account for the population's variance structure?

When comparing phenotypes, this is often more appropriate than asking how
far each is from the mean. It tells us whether the difference between two
individuals is large or small relative to typical variation in the
population.

\section{Connection to discriminant analysis\index[subject]{discriminant analysis}}

Mahalanobis \index[subject]{distance!Mahalanobis} distance has a natural application in classification. Suppose
we have two groups (e.g., two species, or survivors versus non-survivors)
with means $\boldsymbol{\mu}_1$ and $\boldsymbol{\mu}_2$ and a common
within-group covariance matrix\index[subject]{covariance matrix} ${\Sigma}_W$.

To classify a new individual with phenotype $\vect{z}$, we compute the
Mahalanobis \index[subject]{distance!Mahalanobis} distance from $\vect{z}$ to each group mean:
\begin{align*}
  d^2_1 &= (\vect{z} - \boldsymbol{\mu}_1)^\top {\Sigma}_W^{-1} 
           (\vect{z} - \boldsymbol{\mu}_1), \\
  d^2_2 &= (\vect{z} - \boldsymbol{\mu}_2)^\top {\Sigma}_W^{-1} 
           (\vect{z} - \boldsymbol{\mu}_2),
\end{align*}
and assign the individual to the closer group.

This is the basis of linear discriminant analysis\index[subject]{discriminant analysis} (LDA). The Mahalanobis distance\index[subject]{distance!Mahalanobis}
distance ensures that classification respects the shape of within-group
variation: a large difference along a high-variance direction counts less
than a small difference along a low-variance direction.

\section{Biological interpretation}

In evolutionary biology, the Mahalanobis \index[subject]{distance!Mahalanobis} distance has a natural
interpretation. If we use the phenotypic covariance matrix\index[subject]{covariance matrix} $\mat{P}$, then
\[
  d^2_{\mat{P}} = (\vect{z} - \boldsymbol{\mu})^\top \mat{P}^{-1} 
                  (\vect{z} - \boldsymbol{\mu})
\]
measures how unusual a phenotype is relative to the variation present in
the population.

If we use the genetic covariance matrix\index[subject]{covariance matrix} $\mat{G}$, then
\[
  d^2_{\mat{G}} = (\vect{z} - \boldsymbol{\mu})^\top \mat{G}^{-1} 
                  (\vect{z} - \boldsymbol{\mu})
\]
measures how unusual a phenotype is relative to the \emph{genetic}
variation available. This is relevant for asking: how difficult would it
be to evolve to this phenotype? Phenotypes far from the mean in
low-genetic-variance directions are harder to reach by selection than
phenotypes far from the mean in high-genetic-variance directions.

\begin{keyidea}
The choice of covariance matrix\index[subject]{covariance matrix} changes the question being asked:
\begin{itemize}
  \item $\mat{P}^{-1}$: How unusual is this phenotype given the total
        variation in the population?
  \item $\mat{G}^{-1}$: How unusual is this phenotype given the genetic
        variation available for selection to act on?
\end{itemize}
\end{keyidea}

\section{What Mahalanobis \index[subject]{distance!Mahalanobis} distance requires}

The Mahalanobis \index[subject]{distance!Mahalanobis} distance is well-defined only when the covariance matrix\index[subject]{covariance matrix}
is invertible. This fails when:

\begin{itemize}
  \item The covariance matrix\index[subject]{covariance matrix} is singular (has zero eigenvalue\index[subject]{eigenvalue}s), meaning
        some linear combination of traits has zero variance.
  \item The sample size is smaller than the number of traits, making the
        sample covariance matrix\index[subject]{covariance matrix} rank-deficient.
\end{itemize}

In practice, researchers often use regularised covariance estimates or
reduce dimensionality (e.g., via PCA\index[subject]{PCA}) before computing Mahalanobis \index[subject]{distance!Mahalanobis}
distances. We will discuss these issues further in Part~IV.

\section{Summary}

In this chapter we have:

\begin{itemize}
  \item Derived the Mahalanobis \index[subject]{distance!Mahalanobis} distance by inserting the inverse covariance
        matrix into the distance formula.
  \item Verified that in one dimension, Mahalanobis \index[subject]{distance!Mahalanobis} distance reduces to the
        absolute $z$-score.
  \item Seen geometrically how the inverse covariance matrix\index[subject]{covariance matrix} reshapes the
        unit ball from a sphere to an ellipsoid matching the data.
  \item Connected Mahalanobis \index[subject]{distance!Mahalanobis} distance to probability: for multivariate normal data, equal Mahalanobis \index[subject]{distance!Mahalanobis} distance means equal probability density.
  \item Computed a worked example showing how points at the same Euclidean \index[subject]{distance!Euclidean}
        distance have very different Mahalanobis \index[subject]{distance!Mahalanobis} distances.
  \item Understood Mahalanobis \index[subject]{distance!Mahalanobis} distance as Euclidean \index[subject]{distance!Euclidean} distance in whitened
        space---space where the covariance has been transformed to the
        identity.
  \item Noted the biological interpretations: $\mat{P}^{-1}$ measures
        phenotypic unusualness, $\mat{G}^{-1}$ measures genetic
        ``difficulty to reach.''
\end{itemize}

We now have the conceptual foundation for Part~III. The covariance matrix\index[subject]{covariance matrix}
encodes the shape of the data cloud; its inverse defines a natural distance.
But we have not yet said how to \emph{find} the axes of the ellipse or
compute the eigenvalue\index[subject]{eigenvalue}s that determine its shape. That is the work of
diagonalisation\index[subject]{diagonalisation}, which we take up next.

\section*{Exercises}

\paragraph{Exercise 12.1 (Computing a covariance matrix).}
Five individuals are measured for two traits:

\begin{center}
\begin{tabular}{ccc}
\hline
Individual & $X$ & $Y$ \\
\hline
1 & 2 & 4 \\
2 & 3 & 5 \\
3 & 5 & 7 \\
4 & 4 & 6 \\
5 & 6 & 8 \\
\hline
\end{tabular}
\end{center}

\begin{enumerate}
  \item Compute the mean of $X$ and the mean of $Y$.
  \item Compute the variance of $X$ and the variance of $Y$.
  \item Compute the covariance of $X$ and $Y$ using
        $\Cov(X,Y) = \frac{1}{n-1}\sum_i (x_i - \bar{x})(y_i - \bar{y})$.
  \item Assemble the $2 \times 2$ covariance matrix $\mat{S}$.
  \item Verify that $\mat{S}$ is symmetric.
\end{enumerate}

\paragraph{Exercise 12.2 (Covariance matrix properties).}
Consider the covariance matrix
\[
  \mat{S} = \begin{pmatrix} 4 & 2 \\ 2 & 5 \end{pmatrix}.
\]

\begin{enumerate}
  \item What is the variance of trait 1? Of trait 2?
  \item What is the covariance between the traits?
  \item Compute the correlation: $r = \Cov(X,Y) / (s_X s_Y)$.
  \item Is this matrix positive definite? (Hint: compute its eigenvalues
        or check that $\det(\mat{S}) > 0$ and $\tr(\mat{S}) > 0$.)
\end{enumerate}

\paragraph{Exercise 12.3 (Mahalanobis distance by hand).}
Using the covariance matrix from Exercise 12.2, compute the Mahalanobis
distance from the mean $(0, 0)$ to the point $(2, 1)$.

\begin{enumerate}
  \item First, compute the inverse of $\mat{S}$.
  \item Then compute $D^2 = \vect{x}^\top \mat{S}^{-1} \vect{x}$ where
        $\vect{x} = (2, 1)^\top$.
  \item Take the square root to get $D$.
  \item Compare to the Euclidean $\|\vect{x}\| = \sqrt{2^2 + 1^2}$.
\end{enumerate}

\paragraph{Exercise 12.4 (Mahalanobis equals Euclidean when\ldots).}
Show that Mahalanobis distance equals Euclidean distance when the
covariance matrix is the identity matrix $\mat{I}$.

\begin{enumerate}
  \item Write down the $2 \times 2$ identity matrix.
  \item What does $\mat{S} = \mat{I}$ imply about the variances and
        covariance?
  \item Compute $D^2 = \vect{x}^\top \mat{I}^{-1} \vect{x}$ and simplify.
  \item Under what biological conditions might $\mat{S} \approx \mat{I}$?
\end{enumerate}

\paragraph{Exercise 12.5 (Ellipses and probability).}
The set of points with Mahalanobis distance $D = 1$ from the mean forms
an ellipse.

\begin{enumerate}
  \item For a bivariate normal distribution, approximately what percentage
        of points lie within the $D = 1$ ellipse? (Hint: it's not 68\%.)
  \item How does this compare to the univariate case where about 68\% of
        observations lie within 1 SD of the mean?
  \item The $D = \sqrt{2}$ ellipse contains about 63\% of observations
        for a bivariate normal. Why does the ``1 SD'' intuition not
        transfer directly to multiple dimensions?
\end{enumerate}

% ==================================================
% Part III -- Diagonalisation and Natural Axes
% ==================================================
\part{Diagonalisation and Natural Axes}

\chapter{Diagonalisation\index[subject]{diagonalisation} and Natural Axes}

In Part~II we saw that covariance matrices define ellipses and that the Mahalanobis distance\index[subject]{distance!Mahalanobis} distance uses the inverse covariance matrix\index[subject]{covariance matrix} to measure ``unusualness.'' But we repeatedly invoked eigenvalue\index[subject]{eigenvalue}s and eigenvectors without explaining how to find them or what they mean.

This chapter fills that gap. We develop diagonalisation\index[subject]{diagonalisation}---the process of finding the natural axes of an ellipse---from first principles. By the end, you will understand eigenvalue\index[subject]{eigenvalue}s and eigenvector\index[subject]{eigenvector}s not as abstract algebra but as answers to a concrete geometric question: \emph{in which directions does a matrix act as pure stretching?}

\section{The question that leads to eigenvalue\index[subject]{eigenvalue}s}

Consider a symmetric matrix $\mat{A}$ acting on vectors in the plane. For most vectors $\vect{v}$, the output $\mat{A}\vect{v}$ points in a different direction from $\vect{v}$. The matrix rotates as well as stretches.

But for some special vectors, the output points in the \emph{same direction} as the input---or exactly opposite. The matrix stretches (or compresses) without rotating. These special vectors are called \textbf{eigenvector\index[subject]{eigenvector}s},
and the stretching factors are called \textbf{eigenvalue\index[subject]{eigenvalue}s}.

Formally, $\vect{v}$ is an eigenvector\index[subject]{eigenvector} of $\mat{A}$ with eigenvalue\index[subject]{eigenvalue}
$\lambda$ if
\[
  \mat{A}\vect{v} = \lambda\vect{v}.
\]

The matrix $\mat{A}$ acting on $\vect{v}$ produces the same result as simply multiplying $\vect{v}$ by the scalar $\lambda$.

\begin{keyidea}
eigenvector\index[subject]{eigenvector}s are the directions along which a matrix acts as pure scaling. eigenvalue\index[subject]{eigenvalue}s are the scaling factors. Finding them reveals the natural axes of the transformation.
\end{keyidea}

\section{A concrete example}

Let us find the eigenvector\index[subject]{eigenvector}s and eigenvalue\index[subject]{eigenvalue}s of
\[
  \mat{A} =
  \begin{pmatrix}
    3 & 1 \\
    1 & 3
  \end{pmatrix}.
\]

We seek vectors $\vect{v}$ and scalars $\lambda$ such that
$\mat{A}\vect{v} = \lambda\vect{v}$.

Rearranging:
\[
  \mat{A}\vect{v} - \lambda\vect{v} = \vect{0}
  \quad\Rightarrow\quad
  (\mat{A} - \lambda\mat{I})\vect{v} = \vect{0}.
\]

For a non-zero solution $\vect{v}$ to exist, the matrix
$(\mat{A} - \lambda\mat{I})$ must be singular. This happens when its determinant is zero:
\[
  \det(\mat{A} - \lambda\mat{I}) = 0.
\]

This is the \textbf{characteristic equation\index[subject]{characteristic equation}}. For our matrix:
\[
  \det
  \begin{pmatrix}
    3 - \lambda & 1 \\
    1 & 3 - \lambda
  \end{pmatrix}
  = (3 - \lambda)^2 - 1
  = \lambda^2 - 6\lambda + 8
  = (\lambda - 4)(\lambda - 2)
  = 0.
\]

The eigenvalue\index[subject]{eigenvalue}s are $\lambda_1 = 4$ and $\lambda_2 = 2$.

\subsection*{Finding the eigenvector\index[subject]{eigenvector}s}

For $\lambda_1 = 4$:
\[
  (\mat{A} - 4\mat{I})\vect{v} =
  \begin{pmatrix}
    -1 & 1 \\
    1 & -1
  \end{pmatrix}
  \begin{pmatrix}
    v_1 \\ v_2
  \end{pmatrix}
  = \vect{0}.
\]

This gives $-v_1 + v_2 = 0$, so $v_1 = v_2$. The eigenvector\index[subject]{eigenvector} (normalised to unit length) is
\[
  \vect{v}_1 = \frac{1}{\sqrt{2}}
  \begin{pmatrix}
    1 \\ 1
  \end{pmatrix}.
\]

For $\lambda_2 = 2$:
\[
  (\mat{A} - 2\mat{I})\vect{v} =
  \begin{pmatrix}
    1 & 1 \\
    1 & 1
  \end{pmatrix}
  \begin{pmatrix}
    v_1 \\ v_2
  \end{pmatrix}
  = \vect{0}.
\]

This gives $v_1 + v_2 = 0$, so $v_2 = -v_1$. The eigenvector\index[subject]{eigenvector} is
\[
  \vect{v}_2 = \frac{1}{\sqrt{2}}
  \begin{pmatrix}
    1 \\ -1
  \end{pmatrix}.
\]

\begin{figure}[ht]
  \centering
  \includegraphics[width=0.8\textwidth]{fig20_eigenvectors.png}
  \caption[Eigenvectors and eigenvalue\index[subject]{eigenvalue}s]{
    The matrix $\mat{A}$ stretches space by factor 4 along $\vect{v}_1$ (the diagonal where both traits increase together) and by factor 2 along $\vect{v}_2$ (the diagonal where traits move in opposite directions). These are the natural axes of the transformation.
  }
  \label{fig:eigenvectors}
\end{figure}

Notice that $\vect{v}_1$ and $\vect{v}_2$ are perpendicular (their dot product is zero). This is not a coincidence---it is guaranteed for symmetric matrices.

\section{The spectral theorem: why symmetric matrices are special}

Symmetric matrices have three remarkable properties that make them central to statistics and biology:

\begin{enumerate}
  \item \textbf{Real eigenvalue\index[subject]{eigenvalue}s.} The eigenvalue\index[subject]{eigenvalue}s of a symmetric matrix are always real numbers, never complex.
  \item \textbf{Orthogonal eigenvector\index[subject]{eigenvector}s.} eigenvector\index[subject]{eigenvector}s corresponding to different eigenvalue\index[subject]{eigenvalue}s are perpendicular.
  \item \textbf{Complete set.} A $p \times p$ symmetric matrix always has $p$ eigenvector\index[subject]{eigenvector}s that form an orthonormal basis for $\mathbb{R}^p$.
\end{enumerate}

These properties are collectively known as the \textbf{spectral theorem}.

\begin{keyidea}
Every symmetric matrix can be understood as pure stretching along perpendicular axes. There is no rotation mixed in, no shear, no reflection---just stretching (or compressing) along $p$ orthogonal directions.
\end{keyidea}

Covariance matrices are symmetric by construction (the covariance of $X$ with $Y$ equals the covariance of $Y$ with $X$). This is why ellipses, not parallelograms, describe their geometry.

\section{diagonalisation\index[subject]{diagonalisation}: the matrix factorisation}

Collect the eigenvector\index[subject]{eigenvector}s as columns of a matrix $\mat{V}$:
\[
  \mat{V} =
  \begin{pmatrix}
    | & | \\
    \vect{v}_1 & \vect{v}_2 \\
    | & |
  \end{pmatrix}.
\]

For our example:
\[
  \mat{V} = \frac{1}{\sqrt{2}}
  \begin{pmatrix}
    1 & 1 \\
    1 & -1
  \end{pmatrix}.
\]

Collect the eigenvalue\index[subject]{eigenvalue}s in a diagonal matrix ${\Lambda}$:
\[
  {\Lambda} =
  \begin{pmatrix}
    \lambda_1 & 0 \\
    0 & \lambda_2
  \end{pmatrix}
  =
  \begin{pmatrix}
    4 & 0 \\
    0 & 2
  \end{pmatrix}.
\]

The spectral theorem guarantees that
\[
  \mat{A} = \mat{V}{\Lambda}\mat{V}^\top.
\]

This is the \textbf{eigendecomposition} or \textbf{spectral decomposition}
of $\mat{A}$.

\begin{keyidea}
The eigendecomposition $\mat{A} = \mat{V}{\Lambda}\mat{V}^\top$
factorises a symmetric matrix into three parts:
\begin{itemize}
  \item $\mat{V}^\top$: rotate from original axes to eigenvector\index[subject]{eigenvector} axes;
  \item ${\Lambda}$: stretch along each eigenvector\index[subject]{eigenvector} axis;
  \item $\mat{V}$: rotate back to original axes.
\end{itemize}
\end{keyidea}

Let us verify this for our example:
\begin{align*}
  \mat{V}{\Lambda}\mat{V}^\top
  &= \frac{1}{\sqrt{2}}
     \begin{pmatrix} 1 & 1 \\ 1 & -1 \end{pmatrix}
     \begin{pmatrix} 4 & 0 \\ 0 & 2 \end{pmatrix}
     \frac{1}{\sqrt{2}}
     \begin{pmatrix} 1 & 1 \\ 1 & -1 \end{pmatrix} \\
  &= \frac{1}{2}
     \begin{pmatrix} 1 & 1 \\ 1 & -1 \end{pmatrix}
     \begin{pmatrix} 4 & 4 \\ 2 & -2 \end{pmatrix} \\
  &= \frac{1}{2}
     \begin{pmatrix} 6 & 2 \\ 2 & 6 \end{pmatrix}
  = \begin{pmatrix} 3 & 1 \\ 1 & 3 \end{pmatrix}
  = \mat{A}. \quad\checkmark
\end{align*}

\begin{figure}[htbp]
    \centering
    \includegraphics[width=\textwidth]{figures/fig_ch7_rotate_stretch_rotate.pdf}
    \caption{\textbf{The eigendecomposition as rotate--stretch--rotate.} The 
    matrix $\mat{A} = \mat{V}\mat{\Lambda}\mat{V}^\top$ acts in three stages. 
    (a)~Original unit circle with eigenvectors marked (dashed). 
    (b)~Multiply by $\mat{V}^\top$: rotate the coordinate system so that the 
    eigenvectors align with the axes. (c)~Multiply by $\mat{\Lambda}$: stretch 
    by $\lambda_1 = 4$ along the first axis and $\lambda_2 = 2$ along the 
    second. (d)~Multiply by $\mat{V}$: rotate back to the original coordinates. 
    The unit circle becomes an ellipse whose axes align with the eigenvectors 
    and whose semi-axis lengths are $\sqrt{\lambda_1}$ and $\sqrt{\lambda_2}$.}
    \label{fig:rotate-stretch-rotate}
\end{figure}

\section{Geometric interpretation: the ellipse revealed}

Now we can see exactly what a covariance matrix\index[subject]{covariance matrix} describes geometrically.

If ${\Sigma}$ is a covariance matrix\index[subject]{covariance matrix} with eigendecomposition
${\Sigma} = \mat{V}{\Lambda}\mat{V}^\top$, then:

\begin{itemize}
  \item The \textbf{eigenvector\index[subject]{eigenvector}s} $\vect{v}_1, \vect{v}_2, \ldots, \vect{v}_p$
        are the directions of the principal axes of the covariance ellipse.
  \item The \textbf{eigenvalue\index[subject]{eigenvalue}s} $\lambda_1, \lambda_2, \ldots, \lambda_p$
        are the variances along those axes.
  \item The \textbf{semi-axis lengths} of the ellipse are
        $\sqrt{\lambda_1}, \sqrt{\lambda_2}, \ldots, \sqrt{\lambda_p}$.
\end{itemize}

\begin{figure}[ht]
  \centering
  \includegraphics[width=0.8\textwidth]{fig20_ellipse_axes.png}
  \caption[eigenvalue\index[subject]{eigenvalue}s and eigenvector\index[subject]{eigenvector}s of a covariance matrix\index[subject]{covariance matrix}]{
    A covariance matrix\index[subject]{covariance matrix} defines an ellipse. The eigenvector\index[subject]{eigenvector}s point along the principal axes; the eigenvalue\index[subject]{eigenvalue}s are the variances (squared semi-axis lengths) along those axes.
  }
  \label{fig:ellipse-axes}
\end{figure}

For the covariance matrix\index[subject]{covariance matrix} from Chapter~12,
\[
  {\Sigma} =
  \begin{pmatrix}
    1.0 & 0.8 \\
    0.8 & 1.0
  \end{pmatrix},
\]
the eigenvalue\index[subject]{eigenvalue}s are $\lambda_1 = 1.8$ and $\lambda_2 = 0.2$. The ratio $\lambda_1 / \lambda_2 = 9$ tells us the ellipse is elongated: its major axis is three times longer than its minor axis ($\sqrt{9} = 3$).

The eigenvector\index[subject]{eigenvector}s are at 45° angles to the original axes---one along the direction where both traits increase together (the correlation\index[subject]{correlation}direction), and one perpendicular to it.

\section{Why diagonalisation\index[subject]{diagonalisation} simplifies everything}

In the eigenvector\index[subject]{eigenvector} coordinate system, the covariance matrix\index[subject]{covariance matrix} becomes diagonal:
\[
  {\Lambda} =
  \begin{pmatrix}
    \lambda_1 & 0 & \cdots & 0 \\
    0 & \lambda_2 & \cdots & 0 \\
    \vdots & & \ddots & \vdots \\
    0 & 0 & \cdots & \lambda_p
  \end{pmatrix}.
\]

A diagonal matrix is trivial to work with:

\begin{itemize}
  \item \textbf{Inverse:} ${\Lambda}^{-1}$ has entries
        $1/\lambda_1, 1/\lambda_2, \ldots$
  \item \textbf{Square root:} ${\Lambda}^{1/2}$ has entries
        $\sqrt{\lambda_1}, \sqrt{\lambda_2}, \ldots$
  \item \textbf{Powers:} ${\Lambda}^k$ has entries
        $\lambda_1^k, \lambda_2^k, \ldots$
  \item \textbf{Determinant:} $\det({\Lambda}) = \lambda_1 \lambda_2
        \cdots \lambda_p$
  \item \textbf{Trace:} $\tr({\Lambda}) = \lambda_1 + \lambda_2 +
        \cdots + \lambda_p$
\end{itemize}

Since ${\Sigma} = \mat{V}{\Lambda}\mat{V}^\top$, these operations
extend to the original matrix. For example:
\[
  {\Sigma}^{-1} = \mat{V}{\Lambda}^{-1}\mat{V}^\top,
  \qquad
  {\Sigma}^{1/2} = \mat{V}{\Lambda}^{1/2}\mat{V}^\top.
\]

\begin{keyidea}
diagonalisation\index[subject]{diagonalisation} converts hard matrix problems into easy scalar problems. Once you know the eigenvalue\index[subject]{eigenvalue}s and eigenvector\index[subject]{eigenvector}s, operations like inversion, taking square roots, and computing powers become simple.
\end{keyidea}

\section{The trace and determinant as summaries}

Two numbers summarise the eigenvalue\index[subject]{eigenvalue} spectrum:

\paragraph{The trace.}
The trace of a matrix is the sum of its diagonal entries, which equals the sum of its eigenvalue\index[subject]{eigenvalue}s:
\[
  \tr({\Sigma}) = \sigma_1^2 + \sigma_2^2 + \cdots + \sigma_p^2
    = \lambda_1 + \lambda_2 + \cdots + \lambda_p.
\]

For a covariance matrix\index[subject]{covariance matrix}, the trace is the \textbf{total variance}---the sum of variances across all traits. Geometrically, it measures the overall ``size'' of the ellipse.

\paragraph{The determinant.}
The determinant is the product of the eigenvalue\index[subject]{eigenvalue}s:
\[
  \det({\Sigma}) = \lambda_1 \lambda_2 \cdots \lambda_p.
\]

Geometrically, the determinant is the squared volume of the ellipsoid. For a covariance matrix\index[subject]{covariance matrix}, it measures \textbf{generalised variance}---a single number capturing the overall spread, accounting for correlations.

If any eigenvalue\index[subject]{eigenvalue} is zero, the determinant is zero, indicating that the data lie in a lower-dimensional subspace. The matrix is then singular and
cannot be inverted.

\section{Variance in any direction: the quadratic form\index[subject]{quadratic form} revisited}

In Chapter~2 we introduced the quadratic form\index[subject]{quadratic form} $\vect{v}^\top{\Sigma}\vect{v}$. Now we can interpret it precisely.

Let $\boldsymbol{\beta}$ be a unit vector representing a direction in trait space. The variance of the population in that direction is
\[
  \sigma^2_{\boldsymbol{\beta}} = \boldsymbol{\beta}^\top {\Sigma} \boldsymbol{\beta}.
\]

Using the eigendecomposition ${\Sigma} = \mat{V}{\Lambda}\mat{V}^\top$:
\begin{align*}
  \boldsymbol{\beta}^\top {\Sigma} \boldsymbol{\beta}
    &= \boldsymbol{\beta}^\top \mat{V}{\Lambda}\mat{V}^\top \boldsymbol{\beta} \\
    &= (\mat{V}^\top\boldsymbol{\beta})^\top {\Lambda} (\mat{V}^\top\boldsymbol{\beta}).
\end{align*}

Let $\vect{c} = \mat{V}^\top\boldsymbol{\beta}$ be the coordinates of $\boldsymbol{\beta}$ in the eigenvector\index[subject]{eigenvector} basis. Then
\[
  \boldsymbol{\beta}^\top {\Sigma} \boldsymbol{\beta}
    = \vect{c}^\top {\Lambda} \vect{c}
    = \sum_{i=1}^{p} \lambda_i c_i^2.
\]

\begin{keyidea}
The variance in direction $\boldsymbol{\beta}$ is a weighted average of the eigenvalue\index[subject]{eigenvalue}s, with weights $c_i^2 = (\boldsymbol{\beta} \cdot \vect{v}_i)^2$ ---the squared projections of $\boldsymbol{\beta}$ onto the eigenvector\index[subject]{eigenvector}s.
\end{keyidea}

This formula explains why:
\begin{itemize}
  \item Variance is maximised ($= \lambda_1$) when $\boldsymbol{\beta}$ 
        aligns with $\vect{v}_1$, the first eigenvector\index[subject]{eigenvector}.
  \item Variance is minimised ($= \lambda_p$) when $\boldsymbol{\beta}$
        aligns with $\vect{v}_p$, the last eigenvector\index[subject]{eigenvector}.
  \item For any other direction, variance is intermediate.
\end{itemize}

The eigenvalue\index[subject]{eigenvalue}s bound the variance: $\lambda_{\min} \le \boldsymbol{\beta}^\top{\Sigma}\boldsymbol{\beta} \le \lambda_{\max}$.

\section{Principal Component Analysis (PCA\index[subject]{PCA}) in one paragraph}

PCA\index[subject]{PCA} is simply this: project the data onto the eigenvector\index[subject]{eigenvector}s of the covariance matrix\index[subject]{covariance matrix}. The first principal component is the projection onto $\vect{v}_1$; it captures the direction of maximum variance. The second principal component is the projection onto $\vect{v}_2$; it captures the most variance in the subspace orthogonal to the first. And so on.

The eigenvalue\index[subject]{eigenvalue}s tell you how much variance each component captures. If $\lambda_1$ is much larger than the others, the first principal component carries most of the information, and the data are approximately one-dimensional despite having $p$ measured traits.

We will explore PCA\index[subject]{PCA} and related methods in Chapter~32.

\section{Computing eigenvalue\index[subject]{eigenvalue}s and eigenvector\index[subject]{eigenvector}s in practice}

For $2 \times 2$ matrices, you can solve the characteristic equation\index[subject]{characteristic equation} by hand. For larger matrices, use numerical algorithms. In R:

\begin{verbatim}
Sigma <- matrix(c(1.0, 0.8, 0.8, 1.0), nrow = 2)
eig <- eigen(Sigma)
eig$values      # eigenvalues
eig$vectors     # eigenvectors (columns)
\end{verbatim}

In Python:

\begin{verbatim}
import numpy as np
Sigma = np.array([[1.0, 0.8], [0.8, 1.0]])
eigenvalues, eigenvectors = np.linalg.eigh(Sigma)
\end{verbatim}

Note: \texttt{eigh} is for symmetric (Hermitian) matrices and guarantees real eigenvalue\index[subject]{eigenvalue}s and orthogonal eigenvector\index[subject]{eigenvector}s. Use it for covariance matrices.

\section{A biological example: the G matrix}

The additive genetic covariance matrix\index[subject]{covariance matrix} $\mat{G}$ describes the genetic architecture underlying multiple traits. Its eigendecomposition reveals:

\begin{itemize}
  \item \textbf{$\mathbf{g}_{\max}$:} The first eigenvector\index[subject]{eigenvector}, pointing in
        the direction of maximum genetic variance. This is the ``line of
        least resistance''---the direction evolution finds easiest.
  \item \textbf{Higher eigenvector\index[subject]{eigenvector}s:} Directions of progressively less
        genetic variance. Evolution in these directions requires stronger
        selection to achieve the same response.
  \item \textbf{eigenvalue\index[subject]{eigenvalue}s:} The genetic variances along each principal
        axis. Large eigenvalue\index[subject]{eigenvalue} ratios indicate that $\mat{G}$ strongly
        channels evolution along certain directions.
\end{itemize}

\begin{figure}[ht]
  \centering
  \includegraphics[width=0.8\textwidth]{fig20_gmax.png}
  \caption[The G matrix and $\mathbf{g}_{\max}$]{
    The G matrix defines a genetic ellipse. The first eigenvector\index[subject]{eigenvector}
    $\mathbf{g}_{\max}$ points along the direction of maximum genetic
    variance---the ``line of least evolutionary resistance.''
  }
  \label{fig:gmax}
\end{figure}

When the G matrix is highly eccentric (eigenvalue\index[subject]{eigenvalue}s very unequal), the population can respond quickly to selection along $\mathbf{g}_{\max}$ but responds sluggishly---or not at all---to selection perpendicular to it. This is the geometric foundation of evolutionary constraint.

\section{positive definite\index[subject]{matrix!positive definite}ness and what eigenvalue\index[subject]{eigenvalue}s tell us}

A matrix is \textbf{positive definite\index[subject]{matrix!positive definite}} if all its eigenvalue\index[subject]{eigenvalue}s are strictly positive. For covariance matrices, this means:

\begin{itemize}
  \item Every direction has positive variance.
  \item The matrix can be inverted.
  \item The ellipse is a proper ellipse, not degenerate.
\end{itemize}

A matrix is \textbf{positive semi-definite\index[subject]{matrix!positive semi-definite}} if all eigenvalue\index[subject]{eigenvalue}s are non-negative (some may be zero). A zero eigenvalue\index[subject]{eigenvalue} means:

\begin{itemize}
  \item Some linear combination of traits has zero variance.
  \item The data lie in a lower-dimensional subspace.
  \item The matrix cannot be inverted (it is singular).
\end{itemize}

In practice, estimated covariance matrices from finite samples may have small or even slightly negative eigenvalues due to sampling error. This can cause numerical problems and may require regularisation.

\section{The condition number: how ``ill-behaved'' is the matrix?}

The ratio of largest to smallest eigenvalue\index[subject]{eigenvalue} is the \textbf{condition number}:
\[
  \kappa = \frac{\lambda_{\max}}{\lambda_{\min}}.
\]

A large condition number indicates:

\begin{itemize}
  \item The ellipse is highly elongated.
  \item The matrix is nearly singular.
  \item Numerical computations (like inversion) may be unstable.
  \item Small errors in estimating the matrix can cause large errors in
        derived quantities.
\end{itemize}

In evolutionary terms, a G matrix with large condition number channels evolution strongly along certain directions. In statistical terms, it makes estimation difficult.

\section{Summary}

In this chapter we have:

\begin{itemize}
  \item Defined eigenvalues and eigenvector\index[subject]{eigenvector}s as the answers to ``in which directions does a matrix act as pure scaling?''
  \item Worked through a complete example: finding eigenvalue\index[subject]{eigenvalue}s from the characteristic equation\index[subject]{characteristic equation}, then finding eigenvector\index[subject]{eigenvector}s.
  \item Stated the spectral theorem: symmetric matrices have real eigenvalue\index[subject]{eigenvalue}s and orthogonal eigenvector\index[subject]{eigenvector}s.
  \item Introduced the eigendecomposition
        $\mat{A} = \mat{V}{\Lambda}\mat{V}^\top$ and interpreted it as rotate--stretch--rotate-back.
  \item Connected eigenvalues to ellipse geometry: eigenvector\index[subject]{eigenvector}s are axes, eigenvalue\index[subject]{eigenvalue}s are variances along those axes.
  \item Showed that variance in any direction is a weighted average of eigenvalue\index[subject]{eigenvalue}s, with weights given by squared projections.
  \item Applied these ideas to the G matrix: $\mathbf{g}_{\max}$ is the direction of maximum genetic variance, and eigenvalue\index[subject]{eigenvalue} ratios quantify constraint.
\end{itemize}

We now have the tools to understand any symmetric matrix geometrically. In the next chapter, we use diagonalisation\index[subject]{diagonalisation} to construct the whitening\index[subject]{whitening transformation} transformation---the key to understanding directional heritability\index[subject]{heritability} and the P-sphere\index[subject]{P-sphere}.

%%%%%%%%%%%%%%%%%%%%%%%%%%%%%%%%%%%%%%%%%%%%%%%%%%%%%%%%%%%%%%%%%%%%%%%%%%%%%%%
% CHAPTER 32: PCA, MANOVA, AND PROJECTIONS
%%%%%%%%%%%%%%%%%%%%%%%%%%%%%%%%%%%%%%%%%%%%%%%%%%%%%%%%%%%%%%%%%%%%%%%%%%%%%%%

\section*{Exercises}

\paragraph{Exercise 32.1 (PCA by hand).}
Consider the covariance matrix
\[
  \mat{S} = \begin{pmatrix} 5 & 3 \\ 3 & 5 \end{pmatrix}.
\]

\begin{enumerate}
  \item Find the eigenvalues of $\mat{S}$.
  \item Find the eigenvectors of $\mat{S}$ (normalised to unit length).
  \item What proportion of total variance does PC1 explain?
  \item Interpret PC1 and PC2 in terms of the original traits.
\end{enumerate}

\paragraph{Exercise 32.2 (When all loadings are positive).}
In many morphological data sets, PC1 has all positive loadings.

\begin{enumerate}
  \item What biological interpretation does this suggest?
  \item Give an example of a trait set where you would \emph{not} expect
        all PC1 loadings to be positive.
  \item If PC1 is ``size,'' what does PC2 typically represent?
\end{enumerate}

\paragraph{Exercise 32.3 (Covariance vs.\ correlation PCA).}
A data set has three traits with very different variances:
\begin{itemize}
  \item Trait A: variance = 100
  \item Trait B: variance = 10
  \item Trait C: variance = 1
\end{itemize}

\begin{enumerate}
  \item If you run PCA on the covariance matrix, which trait will
        dominate PC1?
  \item What happens if you standardise each trait to unit variance
        before computing PCA (equivalently, PCA on the correlation matrix)?
  \item When is covariance-based PCA appropriate? When is correlation-based
        PCA better?
\end{enumerate}

\paragraph{Exercise 32.4 (MANOVA intuition).}
Two species of iris are measured for sepal length and sepal width. The
within-species variation forms two overlapping ellipses; the between-species
variation is the distance between their centroids.

\begin{enumerate}
  \item Sketch this scenario with two partially overlapping ellipses.
  \item MANOVA compares the ``size'' of between-group variation to
        within-group variation. In your sketch, are the groups well
        separated?
  \item How would the separation change if the within-group ellipses
        were narrower?
  \item How would it change if the centroids were farther apart?
\end{enumerate}

\paragraph{Exercise 32.5 (Discriminant analysis).}
Using the iris scenario from Exercise 32.4:

\begin{enumerate}
  \item The first discriminant function (DF1) is the direction that
        maximises between-group variance relative to within-group variance.
        Sketch the direction of DF1 on your diagram.
  \item If you project all individuals onto DF1, what do you expect the
        resulting univariate distributions to look like?
  \item How does DF1 relate to the eigenvectors of $\mat{W}^{-1}\mat{B}$?
  \item If there are $g$ groups, what is the maximum number of non-trivial
        discriminant functions?
\end{enumerate}

\paragraph{Exercise 32.6 (Scree plots).}
A PCA of 10 traits yields eigenvalues: 5.2, 2.1, 1.0, 0.6, 0.4, 0.3, 0.2,
0.1, 0.08, 0.02.

\begin{enumerate}
  \item Compute the proportion of variance explained by each PC.
  \item Compute the cumulative proportion of variance.
  \item Sketch a scree plot (eigenvalue vs.\ PC number).
  \item How many PCs would you retain? Justify your choice.
  \item What percentage of total variance do your retained PCs explain?
\end{enumerate}

\chapter{Whitening and the P-sphere\index[subject]{P-sphere}}

In the previous chapter we learned to diagonalise symmetric matrices and
interpret eigenvalues geometrically. Now we apply these tools to a specific
problem: how do we compare genetic and phenotypic variation across
directions in trait space?

This chapter introduces the whitening\index[subject]{whitening transformation} transformation and the concept of the
P-sphere\index[subject]{P-sphere}. These ideas unify several threads from earlier chapters and set
the stage for understanding directional heritability and evolutionary
constraint.

\section{The problem: comparing G and P}

Quantitative genetics gives us two fundamental matrices:

\begin{itemize}
  \item The genetic covariance matrix\index[subject]{covariance matrix} $\mat{G}$, describing heritable
        variation.
  \item The phenotypic covariance matrix\index[subject]{covariance matrix} $\mat{P}$, describing total
        observed variation (genetic plus environmental).
\end{itemize}

For a single trait, heritability\index[subject]{heritability} is the ratio $h^2 = V_G / V_P$. But with
multiple traits, both $\mat{G}$ and $\mat{P}$ are matrices, not scalars.
How do we generalise heritability\index[subject]{heritability} to multiple dimensions?

One approach is to pick a direction $\boldsymbol{\beta}$ in trait space and
ask: what fraction of phenotypic variance in that direction is genetic?
This gives us the \textbf{directional heritability\index[subject]{heritability}}:
\[
  h^2(\boldsymbol{\beta}) 
    = \frac{\boldsymbol{\beta}^\top \mat{G} \boldsymbol{\beta}}
           {\boldsymbol{\beta}^\top \mat{P} \boldsymbol{\beta}}.
\]

The numerator is the genetic variance in direction $\boldsymbol{\beta}$;
the denominator is the phenotypic variance in the same direction. Their
ratio is a number between 0 and 1 (assuming $\mat{G}$ and $\mat{P}$ are
properly estimated).

\begin{keyidea}
Directional heritability\index[subject]{heritability} $h^2(\boldsymbol{\beta})$ measures what fraction
of phenotypic variance is genetic along a specific direction in trait
space. It generalises the scalar heritability\index[subject]{heritability} $h^2 = V_G/V_P$ to multiple
traits.
\end{keyidea}

But here is the difficulty: the value of $h^2(\boldsymbol{\beta})$ depends
on which direction we choose. Some directions may have high heritability\index[subject]{heritability}
(most variation is genetic), while others have low heritability\index[subject]{heritability} (most
variation is environmental). How do we summarise this variation across
directions? And how do we sample directions ``fairly''?

\section{The naive approach and its problem}

A natural idea is to sample directions uniformly from the unit sphere---all
directions equally likely---and compute $h^2(\boldsymbol{\beta})$ for each.

But ``uniform on the unit sphere'' is ambiguous when traits have different
scales. Consider two traits: body mass in kilograms and wing length in
millimetres. A ``uniform'' sample in the original coordinates would be
dominated by directions that emphasise the trait with larger numerical
values.

Even after standardising each trait to have unit variance, there is still
a problem. If traits are correlated, the phenotypic covariance matrix\index[subject]{covariance matrix}
$\mat{P}$ is not the identity. The directions that ``look uniform'' in
Euclidean distance\index[subject]{distance!Euclidean} space are not uniform with respect to phenotypic variation.

\begin{figure}[ht]
  \centering
  \includegraphics[width=0.85\textwidth]{fig21_sphere_vs_psphere.png}
  \caption[The Euclidean distance\index[subject]{distance!Euclidean} sphere versus the P-sphere\index[subject]{P-sphere}]{
    Left: The unit sphere in original coordinates. Uniform sampling here
    ignores the correlation\index[subject]{correlation}structure. Right: The P-sphere\index[subject]{P-sphere}, where points
    are equidistant in Mahalanobis distance\index[subject]{distance!Mahalanobis} distance. Uniform sampling on the
    P-sphere\index[subject]{P-sphere} respects the phenotypic covariance structure.
  }
  \label{fig:sphere-vs-psphere}
\end{figure}

\section{The P-sphere\index[subject]{P-sphere}: uniform with respect to phenotype}

The solution is to define ``uniform'' with respect to the phenotypic
covariance matrix\index[subject]{covariance matrix}. Instead of the Euclidean distance\index[subject]{distance!Euclidean} unit sphere
\[
  \{ \boldsymbol{\beta} : \|\boldsymbol{\beta}\|^2 = 1 \}
    = \{ \boldsymbol{\beta} : \boldsymbol{\beta}^\top \boldsymbol{\beta} = 1 \},
\]
we use the \textbf{P-sphere\index[subject]{P-sphere}}:
\[
  \{ \boldsymbol{\beta} : \boldsymbol{\beta}^\top \mat{P} \boldsymbol{\beta} = 1 \}.
\]

Points on the P-sphere\index[subject]{P-sphere} all have unit phenotypic variance. This is the
natural normalisation for comparing directions: we are asking ``per unit
of phenotypic variance, how much is genetic?''

The P-sphere\index[subject]{P-sphere} is an ellipsoid in the original coordinate system, but it
becomes a true sphere after the whitening\index[subject]{whitening transformation} transformation.

\begin{keyidea}
The P-sphere\index[subject]{P-sphere} is the set of all directions with unit phenotypic variance.
Sampling uniformly from the P-sphere\index[subject]{P-sphere} means treating all phenotypically
equivalent directions equally.
\end{keyidea}

\section{The whitening\index[subject]{whitening transformation} transformation}

In Chapter~12 we saw that the Mahalanobis distance\index[subject]{distance!Mahalanobis} distance can be understood as
Euclidean distance\index[subject]{distance!Euclidean} distance after a whitening\index[subject]{whitening transformation} transformation. Now we develop this
idea systematically.

The whitening\index[subject]{whitening transformation} transformation uses the matrix square root of $\mat{P}^{-1}$.
Define
\[
  \mat{P}^{-1/2} = \mat{V}_P {\Lambda}_P^{-1/2} \mat{V}_P^\top,
\]
where $\mat{V}_P$ contains the eigenvectors of $\mat{P}$ and
${\Lambda}_P$ is the diagonal matrix of eigenvalue\index[subject]{eigenvalue}s. The matrix
${\Lambda}_P^{-1/2}$ has entries $1/\sqrt{\lambda_i}$ on the diagonal.

Apply this transformation to both the genetic and phenotypic matrices:
\begin{align*}
  \mat{P}^* &= \mat{P}^{-1/2} \mat{P} \mat{P}^{-1/2} = \mat{I}, \\
  \mat{G}^* &= \mat{P}^{-1/2} \mat{G} \mat{P}^{-1/2}.
\end{align*}

The phenotypic matrix becomes the identity---this is what ``whitening\index[subject]{whitening transformation}''
means. The genetic matrix becomes $\mat{G}^*$, sometimes called the
\textbf{P-standardised genetic matrix} or the \textbf{G-P matrix}.

\begin{keyidea}
whitening\index[subject]{whitening transformation} by $\mat{P}^{-1/2}$ transforms the phenotypic matrix to the
identity. In whitened space, the P-sphere\index[subject]{P-sphere} becomes the ordinary unit sphere,
and uniform sampling is straightforward.
\end{keyidea}

\section{A remarkable fact: eigenvalue\index[subject]{eigenvalue}s of G* are directional heritabilities}

Here is the key result. In whitened coordinates, let
$\boldsymbol{\beta}^*$ be a unit vector (on the ordinary sphere, which is
now also the P-sphere\index[subject]{P-sphere}). The directional heritability\index[subject]{heritability} is
\begin{align*}
  h^2(\boldsymbol{\beta}^*)
    &= \frac{(\boldsymbol{\beta}^*)^\top \mat{G}^* \boldsymbol{\beta}^*}
            {(\boldsymbol{\beta}^*)^\top \mat{I} \boldsymbol{\beta}^*} \\
    &= (\boldsymbol{\beta}^*)^\top \mat{G}^* \boldsymbol{\beta}^*.
\end{align*}

This is a quadratic form\index[subject]{quadratic form} in $\mat{G}^*$. From Chapter~20, we know that:
\begin{itemize}
  \item The maximum value is the largest eigenvalue\index[subject]{eigenvalue} of $\mat{G}^*$.
  \item The minimum value is the smallest eigenvalue\index[subject]{eigenvalue} of $\mat{G}^*$.
  \item The eigenvector\index[subject]{eigenvector}s of $\mat{G}^*$ are the directions that achieve
        these extremes.
\end{itemize}

\begin{keyidea}
The eigenvalue\index[subject]{eigenvalue}s of $\mat{G}^* = \mat{P}^{-1/2}\mat{G}\mat{P}^{-1/2}$ are
the maximum and minimum directional heritabilities. The eigenvector\index[subject]{eigenvector}s are
the directions that achieve them.
\end{keyidea}

This is a powerful result. It tells us that to understand the range of
possible directional heritabilities, we need only diagonalise $\mat{G}^*$.
The eigenvalue\index[subject]{eigenvalue}s give us the bounds; the eigenvector\index[subject]{eigenvector}s tell us where those
bounds are achieved.

\begin{figure}[htbp]
    \centering
    \includegraphics[width=\textwidth]{figures/fig_ch8_h2_distribution.pdf}
    \caption{Distribution of directional heritability $h^2(\boldsymbol{\beta})$ 
    across directions. (a) The $\mathbf{G}^*$ ellipse (magenta) inside the 
    $\mathbf{P}$-sphere (green) in whitened space. Arrows indicate directions 
    of maximum and minimum heritability, which are the eigenvectors of 
    $\mathbf{G}^*$. (b) Histogram of $h^2$ values from 10,000 random directions 
    sampled uniformly from the $\mathbf{P}$-sphere. The shaded region indicates 
    the ``constraint trap zone'' where heritability is well below average. 
    (c) Heritability as a continuous function of direction angle, showing the 
    180° periodicity (opposite directions have identical $h^2$). The eigenvalues 
    of $\mathbf{G}^*$ bound the distribution.}
    \label{fig:h2_distribution}
\end{figure}

\section{The distribution of directional heritability\index[subject]{heritability}}

If we sample directions uniformly from the P-sphere\index[subject]{P-sphere} (equivalently, the
unit sphere in whitened space), what distribution of $h^2$ values do we
get?

From Chapter~20, the quadratic form\index[subject]{quadratic form}
$(\boldsymbol{\beta}^*)^\top \mat{G}^* \boldsymbol{\beta}^*$
is a weighted average of the eigenvalue\index[subject]{eigenvalue}s $\lambda_i^*$ of $\mat{G}^*$,
with weights given by squared projections onto the eigenvector\index[subject]{eigenvector}s.

For uniform random directions on the sphere, there is a known formula
for the variance of this quadratic form\index[subject]{quadratic form}:
\[
  \Var[h^2(\boldsymbol{\beta})] 
    = \frac{2}{p+2} \cdot \Var(\lambda^*),
\]
where $\Var(\lambda^*)$ is the variance of the eigenvalue\index[subject]{eigenvalue}s of $\mat{G}^*$,
and $p$ is the number of traits.

The coefficient of variation of directional heritability\index[subject]{heritability} is therefore
\[
  \text{CV}[h^2] = \sqrt{\frac{2}{p+2} \cdot V_{\text{rel}}(\mat{G}^*)},
\]
where $V_{\text{rel}}(\mat{G}^*) = \Var(\lambda^*) / \bar{\lambda}^{*2}$
is the relative variance of the eigenvalue\index[subject]{eigenvalue}s.

\begin{keyidea}
The variability of directional heritability\index[subject]{heritability} across directions depends on
two factors:
\begin{enumerate}
  \item The relative variance of the eigenvalue\index[subject]{eigenvalue}s of $\mat{G}^*$---how
        eccentric is the P-standardised genetic ellipsoid?
  \item The number of traits $p$---more traits mean more ``averaging''
        and less variability.
\end{enumerate}
\end{keyidea}

\section{Constraint traps}

A \textbf{constraint trap} occurs when a direction has low heritability\index[subject]{heritability}
despite having substantial phenotypic variance. Selection in that direction
produces little evolutionary response because the genetic variance is low
relative to environmental variance.

In the G* framework, constraint traps correspond to directions near the
eigenvector\index[subject]{eigenvector}s of $\mat{G}^*$ with small eigenvalue\index[subject]{eigenvalue}s. These are directions
where:
\begin{itemize}
  \item The phenotypic variance is typical (by construction, we are on the
        P-sphere\index[subject]{P-sphere}).
  \item The genetic variance is unusually low.
  \item The heritability\index[subject]{heritability} $h^2(\boldsymbol{\beta})$ is near its minimum.
\end{itemize}

\begin{figure}[ht]
  \centering
  \includegraphics[width=0.8\textwidth]{fig21_constraint_trap.png}
  \caption[Constraint traps]{
    In whitened space, the G* ellipse sits inside the P-sphere\index[subject]{P-sphere} (which is
    now the unit circle). Directions where G* is thin relative to the
    sphere are constraint traps: plenty of phenotypic variance, but little
    genetic variance.
  }
  \label{fig:constraint-trap}
\end{figure}

The danger is subtle. A breeder or natural selection might target a
direction with plenty of phenotypic variation, expecting a response. But
if that direction happens to be a constraint trap, the response will be
disappointing---the variation is mostly environmental, not genetic.

\section{Visualising G inside P}

A useful visualisation is to plot the G ellipse and P ellipse together,
centred at the same point. In two dimensions:

\begin{itemize}
  \item The P ellipse shows where phenotypic variation extends.
  \item The G ellipse shows where genetic variation extends.
  \item Directions where G is thin relative to P are low-heritability\index[subject]{heritability}
        directions.
  \item Directions where G nearly fills P are high-heritability\index[subject]{heritability} directions.
\end{itemize}

After whitening\index[subject]{whitening transformation}, P becomes the unit circle. The G* ellipse sits inside it
(assuming $h^2 \le 1$ in all directions). The shape of G* relative to the
circle reveals the constraint structure.

\begin{figure}[ht]
  \centering
  \includegraphics[width=0.85\textwidth]{fig21_g_inside_p.png}
  \caption[G inside P]{
    Left: In original coordinates, both G and P are ellipses. Right: After
    whitening\index[subject]{whitening transformation}, P becomes the unit circle, and G* reveals where heritability\index[subject]{heritability}
    is high (G* close to the circle) or low (G* far inside).
  }
  \label{fig:g-inside-p}
\end{figure}

\section{Connection to the breeder's equation\index[subject]{breeder's equation} equation}

Recall the multivariate breeder's equation\index[subject]{breeder's equation} equation:
\[
  \Delta\bar{\vect{z}} = \mat{G}\mat{P}^{-1}\vect{S} = \mat{G}\boldsymbol{\beta},
\]
where $\vect{S}$ is the selection differential and
$\boldsymbol{\beta} = \mat{P}^{-1}\vect{S}$ is the selection gradient.

The response to selection depends on both $\mat{G}$ and the direction of
$\boldsymbol{\beta}$. If $\boldsymbol{\beta}$ points in a high-heritability\index[subject]{heritability}
direction (large $\boldsymbol{\beta}^\top\mat{G}\boldsymbol{\beta}$ relative
to $\boldsymbol{\beta}^\top\mat{P}\boldsymbol{\beta}$), the response is
strong. If it points in a constraint trap, the response is weak.

The P-whitening\index[subject]{whitening transformation} framework makes this explicit. In whitened coordinates:
\[
  \Delta\bar{\vect{z}}^* = \mat{G}^* \boldsymbol{\beta}^*.
\]

The response in whitened space is simply $\mat{G}^*$ acting on the
whitened selection gradient. The eigenstructure of $\mat{G}^*$ directly
determines how selection translates to response.

\section{Computing G* in practice}

Given estimates of $\mat{G}$ and $\mat{P}$, here is how to compute
$\mat{G}^*$:

\paragraph{Step 1: Eigendecompose P.}
\[
  \mat{P} = \mat{V}_P{\Lambda}_P \mat{V}_P^\top.
\]

\paragraph{Step 2: Compute $\mat{P}^{-1/2}$.}
\[
  \mat{P}^{-1/2} = \mat{V}_P {\Lambda}_P^{-1/2} \mat{V}_P^\top,
\]
where ${\Lambda}_P^{-1/2}$ has diagonal entries $1/\sqrt{\lambda_i}$.

\paragraph{Step 3: Transform G.}
\[
  \mat{G}^* = \mat{P}^{-1/2} \mat{G} \mat{P}^{-1/2}.
\]

\paragraph{Step 4: Eigendecompose G*.}
The eigenvalue\index[subject]{eigenvalue}\index[subject]{eigenvalue}s of $\mat{G}^*$ are the directional heritabilities along
the principal axes. The eigenvector\index[subject]{eigenvector}s (transformed back to original
coordinates by $\mat{P}^{1/2}$) are those principal axes.

In R:
\begin{verbatim}
# Eigendecompose P
eig_P <- eigen(P)
V_P <- eig_P$vectors
Lambda_P <- diag(eig_P$values)

# Compute P^{-1/2}
P_inv_sqrt <- V_P %*% diag(1/sqrt(eig_P$values)) %*% t(V_P)

# Transform G
G_star <- P_inv_sqrt %*% G %*% P_inv_sqrt

# Eigendecompose G*
eig_Gstar <- eigen(G_star)
# eig_Gstar$values are directional heritabilities
\end{verbatim}

\section{Why whitening\index[subject]{whitening transformation} matters}

The whitening\index[subject]{whitening transformation} transformation is not just a mathematical convenience. It
changes how we think about constraint.

Without whitening\index[subject]{whitening transformation}, we might compare directions using Euclidean distance\index[subject]{distance!Euclidean} angles and
conclude that a direction is ``close to $\mathbf{g}_{\max}$'' when it is
actually far from it in phenotypic terms. whitening\index[subject]{whitening transformation} ensures that our
notion of ``close'' respects the phenotypic covariance structure.

It also simplifies sampling. To study the distribution of heritability\index[subject]{heritability}
across directions, we can sample uniformly from the ordinary sphere in
whitened space, which is easy. Sampling uniformly from the P-sphere\index[subject]{P-sphere} in
original coordinates would require accounting for the elliptical geometry.

\begin{keyidea}
whitening\index[subject]{whitening transformation} by $\mat{P}$ is the multivariate generalisation of dividing by
the phenotypic standard deviation. It puts all directions on an equal
footing, so that comparisons are fair.
\end{keyidea}

\section{Summary}

In this chapter we have:

\begin{itemize}
  \item Introduced directional heritability\index[subject]{heritability} $h^2(\boldsymbol{\beta})$ as
        the ratio of genetic to phenotypic variance in a given direction.
  \item Defined the P-sphere\index[subject]{P-sphere} as the set of directions with unit phenotypic
        variance, and explained why sampling uniformly from the P-sphere\index[subject]{P-sphere}
        is the right notion of ``uniform.''
  \item Developed the whitening\index[subject]{whitening transformation} transformation using $\mat{P}^{-1/2}$,
        which converts the P-sphere\index[subject]{P-sphere} to the ordinary unit sphere.
  \item Shown that the eigenvalue\index[subject]{eigenvalue}s of $\mat{G}^* = \mat{P}^{-1/2}\mat{G}\mat{P}^{-1/2}$
        are the extreme directional heritabilities, and the eigenvector\index[subject]{eigenvector}s
        are the directions that achieve them.
  \item Connected the variance of directional heritability\index[subject]{heritability} to the relative
        variance of the eigenvalue\index[subject]{eigenvalue}s of $\mat{G}^*$.
  \item Defined constraint traps as directions where phenotypic variance
        is normal but genetic variance (and hence heritability\index[subject]{heritability}) is low.
  \item Provided practical code for computing $\mat{G}^*$ from estimates
        of $\mat{G}$ and $\mat{P}$.
\end{itemize}

The whitening\index[subject]{whitening transformation} framework unifies the geometry of G and P into a single
picture. In Part~IV, we will apply these ideas to real biological
questions: the G matrix and its eigenstructure, fitness surfaces, and
the analysis of selection and response.


% ==================================================
% Part IV -- Evolutionary Applications
% ==================================================
\part{Evolutionary Applications}

\chapter{The G Matrix and the Genetic Ellipsoid}

We now have a complete geometric toolkit: vectors, matrices, eigenvalue\index[subject]{eigenvalue}s,
the Mahalanobis distance\index[subject]{distance!Mahalanobis} distance, and the whitening transformation. In this chapter
we apply these tools to the central object of multivariate quantitative
genetics: the additive genetic covariance matrix\index[subject]{covariance matrix}, $\mat{G}$.

The G matrix is not merely a table of numbers. It is a shape---an ellipsoid
in trait space that determines how populations can and cannot evolve. By
understanding G geometrically, we gain insight into evolutionary constraint,
the ``line of least resistance,'' and why some trait combinations respond
readily to selection while others do not.

\section{What the G matrix represents}

The additive genetic covariance matrix\index[subject]{covariance matrix} $\mat{G}$ summarises the heritable
variation in a population. Its entries are:

\begin{itemize}
  \item \textbf{Diagonal entries} $G_{ii} = V_{A,i}$: the additive genetic
        variance of trait $i$.
  \item \textbf{Off-diagonal entries} $G_{ij} = \text{Cov}_A(z_i, z_j)$:
        the additive genetic covariance between traits $i$ and $j$.
\end{itemize}

These covariances arise from pleiotropy (single genes affecting multiple
traits) and linkage disequilibrium (non-random association of alleles at
different loci). When $G_{ij} > 0$, alleles that increase trait $i$ tend
also to increase trait $j$. When $G_{ij} < 0$, they tend to have opposite
effects.

\begin{keyidea}
The G matrix encodes the genetic architecture underlying multiple traits.
It determines which trait combinations can be easily assembled by selection
and which cannot.
\end{keyidea}

\section{The genetic ellipsoid}

Like any symmetric positive semi-definite\index[subject]{matrix!positive semi-definite} matrix, $\mat{G}$ defines an
ellipsoid. In two traits, this is an ellipse; in three traits, an
ellipsoid; in $p$ traits, a $p$-dimensional hyperellipsoid.

The ellipsoid has a concrete interpretation: it shows where genetic
variation extends in trait space. Directions along the long axes of the
ellipsoid have high genetic variance; directions along the short axes have
low genetic variance.

\begin{figure}[ht]
  \centering
  \includegraphics[width=0.8\textwidth]{fig30_genetic_ellipsoid.png}
  \caption[The genetic ellipsoid]{
    The G matrix defines an ellipsoid in trait space. The eigenvector\index[subject]{eigenvector}s
    point along the principal axes; the eigenvalue\index[subject]{eigenvalue}s are the genetic
    variances along those axes. The shape reveals which directions have
    abundant genetic variation and which are constrained.
  }
  \label{fig:genetic-ellipsoid}
\end{figure}

Formally, if we diagonalise $\mat{G} = \mat{V}\mat{\Lambda}\mat{V}^\top$:

\begin{itemize}
  \item The eigenvector\index[subject]{eigenvector}s $\vect{g}_1, \vect{g}_2, \ldots, \vect{g}_p$ are
        the principal axes of the genetic ellipsoid.
  \item The eigenvalue\index[subject]{eigenvalue}s $\lambda_1 \ge \lambda_2 \ge \cdots \ge \lambda_p$
        are the genetic variances along those axes.
  \item The semi-axis lengths are $\sqrt{\lambda_1}, \sqrt{\lambda_2},
        \ldots, \sqrt{\lambda_p}$.
\end{itemize}

\section[gmax: the line of least evolutionary resistance]{\texorpdfstring{$\mathbf{g}_{\max}$}{gmax}: the line of least evolutionary resistance}

The first eigenvector\index[subject]{eigenvector} of $\mat{G}$, denoted $\mathbf{g}_{\max}$, points in
the direction of maximum genetic variance. This direction has been called
the ``line of least evolutionary resistance'' because:

\begin{enumerate}
  \item Selection along $\mathbf{g}_{\max}$ produces the largest possible
        response per unit of selection intensity.
  \item Even when selection targets a different direction, the response
        tends to be deflected toward $\mathbf{g}_{\max}$.
  \item Over evolutionary time, populations may diverge primarily along
        $\mathbf{g}_{\max}$, not along the direction of selection.
\end{enumerate}

\begin{keyidea}
$\mathbf{g}_{\max}$ is the direction of maximum genetic variance. It
represents the path of least resistance for evolutionary change---the
direction the population ``wants'' to go, regardless of where selection
points.
\end{keyidea}

The smallest eigenvector\index[subject]{eigenvector}, $\mathbf{g}_{\min}$, represents the direction of
minimum genetic variance. Evolution in this direction is difficult: even
strong selection produces little response because the necessary genetic
variation is scarce.

\section{The multivariate breeder's equation\index[subject]{breeder's equation} equation revisited}

The importance of G geometry becomes clear through the breeder's equation\index[subject]{breeder's equation} equation.
In its multivariate form:
\[
  \Delta\bar{\vect{z}} = \mat{G}\boldsymbol{\beta},
\]
where $\boldsymbol{\beta} = \mat{P}^{-1}\vect{S}$ is the selection gradient
and $\vect{S}$ is the selection differential.

The response $\Delta\bar{\vect{z}}$ is not parallel to $\boldsymbol{\beta}$
unless $\mat{G}$ is a scalar multiple of the identity (equal variances, no
covariances). In general, $\mat{G}$ rotates and stretches the selection
gradient, deflecting the response toward directions of high genetic
variance.

\begin{figure}[ht]
  \centering
  \includegraphics[width=0.85\textwidth]{fig30_deflection.png}
  \caption[Selection deflected by G]{
    Selection (red arrow) aims in one direction, but the response (blue
    arrow) is deflected toward $\mathbf{g}_{\max}$. The more eccentric the
    G ellipse, the stronger the deflection.
  }
  \label{fig:deflection}
\end{figure}

\subsection*{A worked example}

Consider a G matrix with strong positive genetic correlation:
\[
  \mat{G} =
  \begin{pmatrix}
    1.0 & 0.8 \\
    0.8 & 1.0
  \end{pmatrix}.
\]

The eigenvalue\index[subject]{eigenvalue}s are $\lambda_1 = 1.8$ and $\lambda_2 = 0.2$. The first
eigenvector\index[subject]{eigenvector} points at 45° (both traits increasing together); the second
points at 135° (traits in opposition).

Suppose selection favours increased trait 2 only:
$\boldsymbol{\beta} = (0, 1)^\top$.

The response is:
\[
  \Delta\bar{\vect{z}} = \mat{G}\boldsymbol{\beta}
  = \begin{pmatrix} 1.0 & 0.8 \\ 0.8 & 1.0 \end{pmatrix}
    \begin{pmatrix} 0 \\ 1 \end{pmatrix}
  = \begin{pmatrix} 0.8 \\ 1.0 \end{pmatrix}.
\]

Selection targeted only trait 2, but the response includes an increase in
trait 1 as well. The genetic correlation\index[subject]{correlation}has ``dragged'' trait 1 along.
The response vector $(0.8, 1.0)$ is deflected toward the 45° direction of
$\mathbf{g}_{\max}$.

\begin{figure}[htbp]
    \centering
    \includegraphics[width=0.9\textwidth]{figures/fig_ch9_evolvability_vs_respondability.pdf}
    \caption{\textbf{Evolvability versus respondability.} Direction 
    $\vect{u}_1$ (red) has high evolvability (genetic variance = 1.21) but 
    lower heritability ($h^2 = 0.30$) because phenotypic variance is even 
    higher. Direction $\vect{u}_2$ (blue) has lower evolvability (genetic 
    variance = 0.40) but higher heritability ($h^2 = 0.37$). Evolvability 
    tells you how much genetic variance is available; respondability 
    (heritability) tells you what fraction of phenotypic variance is genetic. 
    Both matter for predicting response to selection.}
    \label{fig:evolvability-respondability}
\end{figure}

\section{Evolvability: genetic variance in the direction of selection}

Hansen and Houle introduced \textbf{evolvability\index[subject]{evolvability}} as the genetic variance
in the direction of selection, scaled appropriately. For a selection
gradient $\boldsymbol{\beta}$, the evolvability\index[subject]{evolvability} is:
\[
  e(\boldsymbol{\beta}) = \boldsymbol{\beta}^\top \mat{G} \boldsymbol{\beta}.
\]

When $\boldsymbol{\beta}$ is a unit vector, this is simply the genetic
variance in that direction. When $\boldsymbol{\beta}$ is standardised by
trait means (for mean-scaled evolvability\index[subject]{evolvability}), it measures the proportional
response to proportional selection.

\begin{keyidea}
evolvability\index[subject]{evolvability} $e(\boldsymbol{\beta}) = \boldsymbol{\beta}^\top\mat{G}\boldsymbol{\beta}$
measures the genetic variance available in the direction selection is
pushing. High evolvability\index[subject]{evolvability} means large response; low evolvability\index[subject]{evolvability} means
the population is constrained in that direction.
\end{keyidea}

From Chapter~20, we know that:
\[
  \lambda_{\min} \le \boldsymbol{\beta}^\top\mat{G}\boldsymbol{\beta} \le \lambda_{\max}.
\]

evolvability\index[subject]{evolvability} ranges from the smallest to the largest eigenvalue\index[subject]{eigenvalue} of $\mat{G}$.
The ratio $\lambda_{\max}/\lambda_{\min}$ quantifies how much evolvability\index[subject]{evolvability}
varies across directions---how ``eccentric'' the genetic ellipsoid is.

\section{Respondability and the comparison with P}

evolvability\index[subject]{evolvability} measures genetic variance alone. But response to selection
also depends on how much of the phenotypic variance is genetic. This leads
to \textbf{respondability}:
\[
  r(\boldsymbol{\beta}) 
    = \frac{\boldsymbol{\beta}^\top \mat{G} \boldsymbol{\beta}}
           {\boldsymbol{\beta}^\top \mat{P} \boldsymbol{\beta}}
    = h^2(\boldsymbol{\beta}),
\]
which is the directional heritability\index[subject]{heritability} from Chapter~21.

Respondability asks: of the phenotypic variance in direction
$\boldsymbol{\beta}$, what fraction is genetic? A direction can have high
evolvability\index[subject]{evolvability} (lots of genetic variance) but low respondability (even more
environmental variance), or vice versa.

The P-whitening\index[subject]{whitening transformation} framework from Chapter~21 lets us study respondability
systematically. The eigenvalue\index[subject]{eigenvalue}s of
$\mat{G}^* = \mat{P}^{-1/2}\mat{G}\mat{P}^{-1/2}$
are the respondabilities along the principal axes.

\section{Constraint: when G limits evolution}

The G matrix constrains evolution when:

\begin{enumerate}
  \item \textbf{Low eigenvalue\index[subject]{eigenvalue}s exist.} Some directions have little genetic
        variance. Evolution in those directions is slow or impossible.
  \item \textbf{eigenvalue\index[subject]{eigenvalue}s are unequal.} The genetic ellipsoid is
        eccentric. Response is channelled along $\mathbf{g}_{\max}$ even
        when selection points elsewhere.
  \item \textbf{G and selection are misaligned.} If selection targets a
        direction near $\mathbf{g}_{\min}$, response will be weak.
\end{enumerate}

\begin{figure}[ht]
  \centering
  \includegraphics[width=0.8\textwidth]{fig30_constraint_scenarios.png}
  \caption[Constraint scenarios]{
    Three scenarios. Left: G nearly spherical---little constraint, response
    tracks selection. Centre: G eccentric but aligned with selection---
    response is strong. Right: G eccentric and misaligned---response is
    weak and deflected.
  }
  \label{fig:constraint-scenarios}
\end{figure}

A useful summary statistic is the \textbf{eccentricity} of G, measured by
the relative variance of its eigenvalue\index[subject]{eigenvalue}s:
\[
  V_{\text{rel}}(\mat{G}) 
    = \frac{\Var(\lambda)}{\bar{\lambda}^2}
    = \frac{\sum_i (\lambda_i - \bar{\lambda})^2 / p}
           {\left(\sum_i \lambda_i / p\right)^2}.
\]

When $V_{\text{rel}} = 0$, all eigenvalue\index[subject]{eigenvalue}s are equal and G is spherical
(no constraint). When $V_{\text{rel}}$ is large, G is highly eccentric
(strong constraint).

\section{Effective dimensionality}

Another way to quantify constraint is through \textbf{effective
dimensionality}---how many independent directions of genetic variation
exist.

If all eigenvalue\index[subject]{eigenvalue}s were equal, $\lambda_i = \bar{\lambda}$, we would have
$p$ effective dimensions. If one eigenvalue\index[subject]{eigenvalue} dominates, the effective
dimensionality is closer to 1.

One common measure is:
\[
  n_{\text{eff}} = \frac{(\tr \mat{G})^2}{\tr(\mat{G}^2)}
                 = \frac{\left(\sum_i \lambda_i\right)^2}{\sum_i \lambda_i^2}.
\]

This equals $p$ when all eigenvalue\index[subject]{eigenvalue}s are equal and approaches 1 when one
eigenvalue\index[subject]{eigenvalue} dominates.

\begin{keyidea}
Effective dimensionality measures how many ``independent'' directions of
genetic variation the population has. Low effective dimensionality means
genetic variation is concentrated in a few directions---the population is
genetically constrained.
\end{keyidea}

\section{Empirical G matrices: what do they look like?}

Decades of empirical work have revealed some patterns:

\begin{itemize}
  \item \textbf{G matrices are often eccentric.} In many studies, the first
        few eigenvalue\index[subject]{eigenvalue}s account for most of the genetic variance.
        Effective dimensionality is typically much less than $p$.
  \item \textbf{$\mathbf{g}_{\max}$ often aligns with body size.} For
        morphological traits, the direction of maximum genetic variance
        frequently corresponds to overall size---all traits scaling
        together.
  \item \textbf{Genetic correlations can be strong.} Off-diagonal elements
        of G are often substantial, reflecting pervasive pleiotropy.
  \item \textbf{G varies among populations.} The G matrix is not fixed; it
        evolves and can differ between populations, species, and
        environments.
\end{itemize}

These patterns suggest that genetic constraint is common. Populations do
not have equal access to all directions in trait space; evolution is
channelled along particular paths.

\section{Stability and estimation of G}

Estimating G requires breeding designs (parent-offspring regression,
half-sib designs, animal models) or genomic data. Estimation is
challenging because:

\begin{itemize}
  \item \textbf{Sample sizes are often small.} Estimating a $p \times p$
        matrix requires estimating $p(p+1)/2$ unique elements. With many
        traits, sampling error can be severe.
  \item \textbf{G can be singular or nearly singular.} If some trait
        combinations have near-zero genetic variance, the estimated G may
        have zero or negative eigenvalue\index[subject]{eigenvalue}s due to sampling error.
  \item \textbf{G may not be stable.} If genetic architecture changes
        over time or differs among environments, a single G matrix may
        not capture the population's evolutionary potential.
\end{itemize}

These issues motivate careful statistical treatment: regularisation,
Bayesian estimation, and sensitivity analyses. We should interpret G
matrices with appropriate caution, especially their smaller eigenvalue\index[subject]{eigenvalue}s.

\section{G in the context of P}

Throughout these notes we have emphasised comparing G and P. The P matrix
describes total phenotypic variation; G describes the heritable component.
Their relationship determines:

\begin{itemize}
  \item \textbf{heritability\index[subject]{heritability}:} The ratio of G to P, generalised to multiple
        traits through directional heritability\index[subject]{heritability} $h^2(\boldsymbol{\beta})$.
  \item \textbf{Selection response:} The breeder's equation\index[subject]{breeder's equation} equation
        $\Delta\bar{\vect{z}} = \mat{G}\mat{P}^{-1}\vect{S}$ involves both.
  \item \textbf{Constraint traps:} Directions where G is small relative to
        P---phenotypic variation exists, but it is mostly environmental.
\end{itemize}

The P-whitening\index[subject]{whitening transformation} transformation $\mat{G}^* = \mat{P}^{-1/2}\mat{G}\mat{P}^{-1/2}$
places G and P on a common footing. In whitened space, P is the identity,
and G* directly reveals the heritability\index[subject]{heritability} structure.

\section{Summary}

In this chapter we have:

\begin{itemize}
  \item Interpreted the G matrix as an ellipsoid in trait space, with
        eigenvector\index[subject]{eigenvector}s as principal axes and eigenvalue\index[subject]{eigenvalue}s as genetic
        variances along those axes.
  \item Identified $\mathbf{g}_{\max}$ as the direction of maximum genetic
        variance---the line of least evolutionary resistance.
  \item Shown how the breeder's equation\index[subject]{breeder's equation} equation $\Delta\bar{\vect{z}} = \mat{G}\boldsymbol{\beta}$
        deflects selection response toward $\mathbf{g}_{\max}$.
  \item Defined evolvability\index[subject]{evolvability} as $\boldsymbol{\beta}^\top\mat{G}\boldsymbol{\beta}$
        and respondability (directional heritability\index[subject]{heritability}) as its ratio to
        phenotypic variance.
  \item Discussed constraint in terms of eigenvalue\index[subject]{eigenvalue} eccentricity and
        effective dimensionality.
  \item Summarised empirical patterns: G matrices are often eccentric,
        with $\mathbf{g}_{\max}$ frequently aligned with body size.
  \item Noted challenges in estimating G and the importance of comparing
        G to P through P-whitening\index[subject]{whitening transformation}.
\end{itemize}

The G matrix is the engine of evolutionary response. Its shape determines
which paths are open and which are blocked. In the next chapter, we turn
to the other side of the equation: the fitness surface, encoded in the
$\boldsymbol{\gamma}$ matrix, which determines where selection is pushing.
\chapter[The Fitness Surface and Gamma]{The Fitness Surface and \texorpdfstring{$\boldsymbol{\gamma}$}{γ}}

The previous chapter explored the G matrix---the genetic variation that
fuels evolutionary response. Now we turn to the other side of the equation:
the fitness surface that determines where selection pushes the population.

The fitness surface is a landscape in trait space, with fitness as elevation.
Peaks represent optimal phenotypes; valleys represent maladaptive ones. The
shape of this surface---its slopes and curvatures---determines the direction
and strength of natural selection. The matrix $\boldsymbol{\gamma}$ captures
the curvature, revealing whether selection is stabilising, disruptive, or
acting differently along different trait combinations.

\section{Fitness as a surface over trait space}

Imagine a population of organisms, each with a phenotype that can be
represented as a point in trait space. Each phenotype has an associated
fitness---the expected reproductive success of an individual with that
phenotype.

We can think of fitness as a surface over trait space. In one trait, this
is a curve: fitness $w$ as a function of phenotype $z$. In two traits, it
is a surface: $w(z_1, z_2)$. In $p$ traits, it is a hypersurface that we
cannot visualise directly but can analyse mathematically.

\begin{figure}[ht]
  \centering
  \includegraphics[width=0.85\textwidth]{fig31_fitness_surface.png}
  \caption[A fitness surface]{
    A fitness surface over two traits. The peak represents the optimum
    phenotype. Contour lines show phenotypes of equal fitness. The surface
    may be symmetric (circular contours) or asymmetric (elliptical contours),
    depending on how selection acts on different trait combinations.
  }
  \label{fig:fitness-surface}
\end{figure}

The key insight is that the local shape of the fitness surface determines
selection. Near a peak, fitness decreases in all directions---this is
stabilising selection. Near a saddle point, fitness increases in some
directions and decreases in others---this is correlational selection. On
a slope, fitness increases in one direction---this is directional selection.

\section[The selection gradient beta]{The selection gradient \texorpdfstring{$\boldsymbol{\beta}$}{β}}

The first-order description of the fitness surface is its slope. At any
point in trait space, we can ask: in which direction does fitness increase
most steeply?

The answer is the \textbf{selection gradient} $\boldsymbol{\beta}$, defined
as the vector of partial derivatives of relative fitness with respect to
each trait:
\[
  \beta_i = \frac{\partial \ln w}{\partial z_i}
          = \frac{1}{w} \frac{\partial w}{\partial z_i}.
\]

In matrix notation:
\[
  \boldsymbol{\beta} = \nabla \ln w.
\]

The selection gradient points ``uphill'' on the fitness surface. Its
magnitude indicates how steep the slope is; its direction indicates where
selection is pushing.

\begin{keyidea}
The selection gradient $\boldsymbol{\beta}$ is the direction of steepest
ascent on the fitness surface. It describes directional selection---the
tendency for the population mean to move toward higher fitness.
\end{keyidea}

From the Lande equation\index[subject]{breeder's equation}, the response to selection is
$\Delta\bar{\vect{z}} = \mat{G}\boldsymbol{\beta}$.
The selection gradient tells us where selection wants to go; the G matrix
determines how much of that desire is realised.

\section[Beyond slopes: the curvature matrix gamma]{Beyond slopes: the curvature matrix \texorpdfstring{$\boldsymbol{\gamma}$}{γ}}

Slopes tell us about directional selection, but they miss an important
aspect: is the fitness surface curved? Is selection pushing the population
toward a peak (stabilising) or away from a valley (disruptive)?

Curvature is captured by second derivatives. The matrix of second partial
derivatives of relative fitness is:
\[
  \gamma_{ij} = \frac{\partial^2 \ln w}{\partial z_i \partial z_j}.
\]

This is the \textbf{quadratic selection gradient} or \textbf{gamma matrix},
denoted $\boldsymbol{\gamma}$.

\begin{keyidea}
The $\boldsymbol{\gamma}$ matrix describes the curvature of the fitness
surface. Its eigenvalue\index[subject]{eigenvalue}s tell us whether selection is stabilising
(negative curvature) or disruptive (positive curvature) along each
principal axis.
\end{keyidea}

Like G and P, $\boldsymbol{\gamma}$ is a symmetric matrix. It can be
diagonalised, and its eigenstructure reveals the geometry of selection.

\section[Interpreting gamma: the sign of eigenvalue\index[subject]{eigenvalue}s]{Interpreting \texorpdfstring{$\boldsymbol{\gamma}$}{γ}: the sign of eigenvalue\index[subject]{eigenvalue}s}

Suppose we diagonalise $\boldsymbol{\gamma}$:
\[
  \boldsymbol{\gamma} = \mat{V}\mat{\Lambda}\mat{V}^\top.
\]

The eigenvalue\index[subject]{eigenvalue}s $\lambda_1, \lambda_2, \ldots, \lambda_p$ describe the
curvature along the principal axes (eigenvector\index[subject]{eigenvector}s) of the fitness surface.

\paragraph{Negative eigenvalue\index[subject]{eigenvalue}: stabilising selection.}
If $\lambda_i < 0$, the fitness surface curves downward along eigenvector\index[subject]{eigenvector}
$\vect{v}_i$. Moving away from the current mean in that direction decreases
fitness. Selection is stabilising: it pushes the population back toward
the centre.

\paragraph{Positive eigenvalue\index[subject]{eigenvalue}: disruptive selection.}
If $\lambda_i > 0$, the fitness surface curves upward along $\vect{v}_i$.
Moving away from the mean increases fitness. Selection is disruptive: it
pushes the population away from the centre, potentially toward two or more
distinct phenotypes.

\paragraph{Zero eigenvalue\index[subject]{eigenvalue}: no curvature.}
If $\lambda_i = 0$, the fitness surface is flat along $\vect{v}_i$ (at
least locally). Selection has no stabilising or disruptive component in
that direction---only directional selection (if any) from $\boldsymbol{\beta}$.

\begin{figure}[ht]
  \centering
  \includegraphics[width=0.9\textwidth]{fig31_curvature_types.png}
  \caption[Types of curvature]{
    Three types of curvature in one dimension. Left: negative curvature
    (stabilising selection, fitness peak). Centre: zero curvature
    (directional selection, constant slope). Right: positive curvature
    (disruptive selection, fitness valley).
  }
  \label{fig:curvature-types}
\end{figure}

\section{Correlational selection: the off-diagonal elements}

The diagonal elements of $\boldsymbol{\gamma}$ describe curvature along
each individual trait. The off-diagonal elements describe something more
subtle: \textbf{correlational selection}.

If $\gamma_{12} \neq 0$, then the curvature of the fitness surface depends
on trait combinations, not just individual traits. A positive $\gamma_{12}$
means that fitness is higher when both traits are simultaneously high or
simultaneously low (positive correlation\index[subject]{correlation}favoured). A negative $\gamma_{12}$
means that fitness is higher when traits are in opposition (negative
correlation\index[subject]{correlation}favoured).

\begin{keyidea}
Correlational selection ($\gamma_{ij} \neq 0$ for $i \neq j$) favours
particular trait combinations. It can build or break genetic correlations
over evolutionary time.
\end{keyidea}

The eigenvector\index[subject]{eigenvector}s of $\boldsymbol{\gamma}$ reveal the trait combinations
along which selection acts most strongly (largest $|\lambda_i|$) and the
directions along which selection is neutral (small $|\lambda_i|$).

\begin{figure}[htbp]
    \centering
    \includegraphics[width=\textwidth]{figures/fig_ch10_saddle_correlational.pdf}
    \caption{\textbf{A fitness saddle from correlational selection.} When 
    $\boldsymbol{\gamma}$ has both positive and negative eigenvalues, the 
    fitness surface is a saddle. (a)~Three-dimensional view: fitness increases 
    along one axis (the ridge, $\lambda = 0.3 > 0$, disruptive) and decreases 
    along the other (the valley, $\lambda = -0.7 < 0$, stabilising). 
    (b)~Contour view from above: hyperbolic level curves indicate a saddle 
    point. The eigenvectors of $\boldsymbol{\gamma}$ (dashed lines) point 
    along the ridge and valley directions. Correlational selection 
    ($\gamma_{12} \neq 0$) rotates these directions away from the original 
    trait axes.}
    \label{fig:saddle-correlational}
\end{figure}

\section{Quadratic fitness functions}

A common model for the fitness surface is a quadratic function centred on
an optimum $\boldsymbol{\theta}$:
\[
  w(\vect{z}) = w_{\max} \exp\left(
    -\frac{1}{2}(\vect{z} - \boldsymbol{\theta})^\top
    \boldsymbol{\omega}^{-1}
    (\vect{z} - \boldsymbol{\theta})
  \right).
\]

Here $\boldsymbol{\omega}$ is a matrix that describes the ``width'' of
the fitness peak. The quadratic form\index[subject]{quadratic form} in the exponent is a Mahalanobis distance\index[subject]{distance!Mahalanobis}
distance from the optimum, using $\boldsymbol{\omega}^{-1}$ as the metric.

For this Gaussian fitness function:
\begin{itemize}
  \item The selection gradient at phenotype $\vect{z}$ is
        $\boldsymbol{\beta} = \boldsymbol{\omega}^{-1}(\boldsymbol{\theta} - \vect{z})$.
  \item The curvature matrix is $\boldsymbol{\gamma} = -\boldsymbol{\omega}^{-1}$.
\end{itemize}

The curvature is constant everywhere and equals the negative of the inverse
width matrix. Narrow peaks (small $\boldsymbol{\omega}$) have strong
curvature (large $|\boldsymbol{\gamma}|$); wide peaks have weak curvature.

\begin{figure}[ht]
  \centering
  \includegraphics[width=0.8\textwidth]{fig31_gaussian_surface.png}
  \caption[Gaussian fitness surface]{
    A Gaussian fitness surface with elliptical contours. The optimum is at
    $\boldsymbol{\theta}$. The width matrix $\boldsymbol{\omega}$ determines
    how rapidly fitness declines away from the optimum in different
    directions.
  }
  \label{fig:gaussian-surface}
\end{figure}

\section[Estimating gamma from data]{Estimating \texorpdfstring{$\boldsymbol{\gamma}$}{γ} from data}

In practice, $\boldsymbol{\gamma}$ is estimated by regressing fitness on
traits and their products. The standard approach (Lande and Arnold 1983)
uses:
\[
  w = \alpha + \sum_i \beta_i z_i 
    + \frac{1}{2}\sum_i \gamma_{ii} z_i^2 
    + \sum_{i < j} \gamma_{ij} z_i z_j + \epsilon.
\]

The linear coefficients estimate directional selection ($\boldsymbol{\beta}$);
the quadratic and cross-product coefficients estimate the curvature
($\boldsymbol{\gamma}$).

There are subtleties:

\begin{itemize}
  \item \textbf{Standardisation.} Traits are typically standardised to have
        mean zero and unit variance before analysis, so that coefficients
        are comparable across traits.
  \item \textbf{The factor of 2.} Note the $\frac{1}{2}$ in front of the
        diagonal quadratic terms. This ensures that $\gamma_{ii}$ equals
        the second derivative $\partial^2 w / \partial z_i^2$, not half of
        it.
  \item \textbf{Relative fitness.} Fitness should be relativised (divided
        by mean fitness) so that the regression estimates selection
        gradients, not absolute fitness effects.
  \item \textbf{Sample size.} Estimating $\boldsymbol{\gamma}$ requires
        large samples because quadratic terms have less power than linear
        terms.
\end{itemize}

\section[The geometry of gamma]{The geometry of \texorpdfstring{$\boldsymbol{\gamma}$}{γ}}

Like G and P, we can visualise $\boldsymbol{\gamma}$ as an ellipse (in two
traits) or ellipsoid (in higher dimensions). But the interpretation is
different:

\begin{itemize}
  \item For G and P, the ellipse shows where \emph{variation} extends.
  \item For $\boldsymbol{\gamma}$, the ellipse shows where \emph{selection
        curvature} is strongest.
\end{itemize}

The eigenvector\index[subject]{eigenvector}s of $\boldsymbol{\gamma}$ are the principal axes of the
fitness surface---the directions along which curvature is purely stabilising
or disruptive, with no correlational component. The eigenvalue\index[subject]{eigenvalue}s are the
curvatures along those axes.

If all eigenvalue\index[subject]{eigenvalue}s of $\boldsymbol{\gamma}$ are negative and equal, the
fitness surface is a symmetric peak with circular contours. If eigenvalue\index[subject]{eigenvalue}s
differ, the peak is elongated---selection is stronger in some directions
than others.

\section[Comparing gamma and G: alignment matters]{Comparing \texorpdfstring{$\boldsymbol{\gamma}$}{γ} and G: alignment matters}

A central question in evolutionary biology is: how does the geometry of
selection (encoded in $\boldsymbol{\gamma}$) interact with the geometry
of genetic variation (encoded in G)?

\paragraph{Aligned.}
If the eigenvector\index[subject]{eigenvector}s of $\boldsymbol{\gamma}$ align with those of G, then
selection is strongest along directions where genetic variation is abundant
or scarce. This can either accelerate evolution (if strong selection meets
abundant variation) or frustrate it (if strong selection meets scarce
variation).

\paragraph{Misaligned.}
If $\boldsymbol{\gamma}$ and G are misaligned, the situation is more complex.
Selection may push in one direction, but genetic variation may only be
available in another. The response is a compromise, deflected by G away
from where $\boldsymbol{\gamma}$ would drive it.

\begin{keyidea}
The interaction between $\boldsymbol{\gamma}$ (the geometry of selection)
and G (the geometry of genetic variation) determines whether evolution is
fast or slow, aligned or deflected. Understanding both matrices is essential
for predicting evolutionary trajectories.
\end{keyidea}

\section[Gamma and the maintenance of genetic variation]{\texorpdfstring{$\boldsymbol{\gamma}$}{γ} and the maintenance of genetic variation}

The curvature matrix also influences the maintenance of genetic variation.
Under stabilising selection ($\boldsymbol{\gamma}$ with negative eigenvalue\index[subject]{eigenvalue}s),
selection removes variation by favouring intermediate phenotypes. The
stronger the curvature (more negative eigenvalue\index[subject]{eigenvalue}s), the faster variation
is eroded.

This creates a puzzle: if stabilising selection is common, why do
populations retain genetic variation? Possible answers include:
\begin{itemize}
  \item mutation-selection balance\index[subject]{mutation-selection balance}.
  \item Fluctuating selection (the optimum moves over time).
  \item Correlational constraints (selection on one trait limited by
        correlated response in others).
  \item Frequency-dependent or spatially varying selection.
\end{itemize}

The eigenstructure of $\boldsymbol{\gamma}$ helps quantify how rapidly
selection should erode variation along different axes, informing these
debates.

\section{A worked example}

Consider a study measuring survival as a function of two morphological
traits in a bird population. After standardising traits and relativising
fitness, a quadratic regression yields:

\[
  \boldsymbol{\beta} = \begin{pmatrix} 0.15 \\ 0.08 \end{pmatrix},
  \qquad
  \boldsymbol{\gamma} = \begin{pmatrix} -0.12 & 0.06 \\ 0.06 & -0.08 \end{pmatrix}.
\]

\paragraph{Interpreting $\boldsymbol{\beta}$.}
Both selection gradients are positive: directional selection favours
larger values of both traits. Trait 1 is under stronger directional
selection than trait 2.

\paragraph{Interpreting $\boldsymbol{\gamma}$.}
Both diagonal elements are negative: stabilising selection on each trait
individually. The off-diagonal element is positive: correlational selection
favours positive trait combinations (both high or both low).

\paragraph{Eigendecomposition.}
\[
  \lambda_1 = -0.04, \quad \lambda_2 = -0.16.
\]

Both eigenvalue\index[subject]{eigenvalue}s are negative, confirming overall stabilising selection.
But selection is much stronger ($|\lambda_2| = 0.16$) along the second
eigenvector\index[subject]{eigenvector} than the first ($|\lambda_1| = 0.04$). The fitness peak is
elongated.

The eigenvector\index[subject]{eigenvector}s reveal that the direction of weakest stabilising selection
($\lambda_1$) is approximately the positive diagonal (both traits high
together), while the direction of strongest stabilising selection
($\lambda_2$) is the negative diagonal (traits in opposition).

This makes biological sense: the population can tolerate variation in
overall size (both traits scaling together) but is strongly selected
against unusual proportions (one trait high, the other low).

\section{Canonical analysis of the fitness surface}

Phillips and Arnold (1989) introduced \textbf{canonical analysis} to
understand the geometry of $\boldsymbol{\gamma}$. The idea is to:

\begin{enumerate}
  \item Diagonalise $\boldsymbol{\gamma}$ to find its eigenvector\index[subject]{eigenvector}s and
        eigenvalue\index[subject]{eigenvalue}s.
  \item Express the fitness surface in terms of these principal axes.
  \item Interpret which trait combinations are under strong versus weak
        stabilising or disruptive selection.
\end{enumerate}

In the canonical basis, the fitness function becomes:
\[
  w = \bar{w} + \sum_i \theta_i m_i + \frac{1}{2}\sum_i \lambda_i m_i^2,
\]
where $m_i$ is the projection of the phenotype onto the $i$th eigenvector\index[subject]{eigenvector},
$\theta_i$ is the directional selection along that axis (the projection
of $\boldsymbol{\beta}$), and $\lambda_i$ is the curvature.

This separates selection into independent components along orthogonal axes,
making interpretation cleaner.

\section{Summary}

In this chapter we have:

\begin{itemize}
  \item Introduced the fitness surface as a landscape over trait space,
        with slopes (directional selection) and curvatures (stabilising or
        disruptive selection).
  \item Defined the selection gradient $\boldsymbol{\beta}$ as the slope
        of the fitness surface and the curvature matrix $\boldsymbol{\gamma}$
        as its second derivatives.
  \item Interpreted eigenvalue\index[subject]{eigenvalue}s of $\boldsymbol{\gamma}$: negative means
        stabilising, positive means disruptive, zero means flat.
  \item Explained correlational selection through off-diagonal elements
        of $\boldsymbol{\gamma}$.
  \item Described Gaussian fitness surfaces, where
        $\boldsymbol{\gamma} = -\boldsymbol{\omega}^{-1}$.
  \item Outlined how to estimate $\boldsymbol{\gamma}$ from fitness data
        using quadratic regression.
  \item Emphasised that the interaction between $\boldsymbol{\gamma}$ and
        G determines evolutionary trajectories.
  \item Worked through an example showing how to interpret eigenvalue\index[subject]{eigenvalue}s
        and eigenvector\index[subject]{eigenvector}s of $\boldsymbol{\gamma}$.
  \item Introduced canonical analysis as a tool for decomposing selection
        along principal axes.
\end{itemize}

With G describing what evolution \emph{can} do and $\boldsymbol{\gamma}$
describing what selection \emph{wants}, we now have both pieces of the
puzzle. In the next chapter, we turn to the statistical tools---PCA\index[subject]{PCA},
MANOVA\index[subject]{MANOVA}, and related methods---that let us estimate and compare these
matrices from data.
\chapter{PCA, MANOVA, and Projections}

The previous chapters developed the geometry of G, P, and $\boldsymbol{\gamma}$.
We saw that these matrices are ellipsoids, that their eigenstructure reveals
principal axes, and that comparing matrices illuminates evolutionary
constraint and selection. But how do we estimate these matrices and test
hypotheses about them?

This chapter connects the geometric framework to statistical practice. We
cover Principal Component Analysis (PCA\index[subject]{PCA}), Multivariate Analysis of Variance
(MANOVA\index[subject]{MANOVA}), and related projection methods. The unifying theme is that all
these techniques are applications of eigendecomposition to biological
questions.

\section{The bridge from geometry to statistics}

Every statistical method in this chapter does the same thing at its core:
it takes a covariance matrix (or a ratio of covariance matrices), finds its
eigenvalue\index[subject]{eigenvalue}s and eigenvector\index[subject]{eigenvector}s, and interprets them biologically.

\begin{itemize}
  \item \textbf{PCA\index[subject]{PCA}} eigendecomposes a single covariance matrix\index[subject]{covariance matrix} to find
        directions of maximum variance.
  \item \textbf{MANOVA\index[subject]{MANOVA}} compares covariance matrices (among-group vs.
        within-group) to test whether groups differ.
  \item \textbf{Canonical correlation\index[subject]{correlation}Analysis (CCA)} finds directions
        that maximise correlation\index[subject]{correlation}between two sets of variables.
  \item \textbf{Discriminant Analysis} finds directions that best separate
        groups.
\end{itemize}

Once you understand diagonalisation\index[subject]{diagonalisation}, these methods become variations on a
single theme.

\begin{keyidea}
PCA\index[subject]{PCA}, MANOVA\index[subject]{MANOVA}, CCA, and discriminant analysis\index[subject]{discriminant analysis} are all eigendecompositions
of covariance matrices or their ratios. The geometry we have developed
provides the interpretive framework for all of them.
\end{keyidea}

\section{Principal Component Analysis (PCA\index[subject]{PCA})}

PCA\index[subject]{PCA} is the most widely used multivariate technique in biology. Its goal is
dimensionality reduction: represent the variation in $p$ traits using fewer
than $p$ derived variables, while losing as little information as possible.

\subsection*{The procedure}

Given data on $p$ traits measured on $n$ individuals:

\begin{enumerate}
  \item Compute the sample covariance matrix\index[subject]{covariance matrix} $\mat{S}$ (or correlation
        matrix $\mat{R}$ if traits are on different scales).
  \item Eigendecompose: $\mat{S} = \mat{V}\mat{\Lambda}\mat{V}^\top$.
  \item The eigenvector\index[subject]{eigenvector}s $\vect{v}_1, \vect{v}_2, \ldots, \vect{v}_p$ are
        the \textbf{principal component loadings}.
  \item The eigenvalue\index[subject]{eigenvalue}s $\lambda_1 \ge \lambda_2 \ge \cdots \ge \lambda_p$
        are the variances along each principal component.
  \item Project each individual onto the eigenvector\index[subject]{eigenvector}s to get
        \textbf{principal component scores}.
\end{enumerate}

The first principal component (PC1) is the direction of maximum variance
in the data---exactly the first eigenvector\index[subject]{eigenvector} of $\mat{S}$. PC2 is the
direction of maximum variance orthogonal to PC1, and so on.

\begin{figure}[ht]
  \centering
  \includegraphics[width=0.85\textwidth]{fig32_pca_geometry.png}
  \caption[PCA\index[subject]{PCA} geometry]{
    PCA\index[subject]{PCA} finds the principal axes of the data ellipse. PC1 points along
    the direction of maximum variance; PC2 is orthogonal. Projecting
    data onto these axes gives the principal component scores.
  }
  \label{fig:pca-geometry}
\end{figure}

\subsection*{Interpretation}

The eigenvalue\index[subject]{eigenvalue}s tell us how much variance each PC captures. A common
summary is the proportion of variance explained:
\[
  \text{Proportion for PC}_k = \frac{\lambda_k}{\sum_{i=1}^{p} \lambda_i}
    = \frac{\lambda_k}{\tr(\mat{S})}.
\]

If the first few eigenvalue\index[subject]{eigenvalue}s are large and the rest are small, most
variation lies in a low-dimensional subspace. We can then work with just
PC1 and PC2 (for example) without losing much information.

The eigenvector\index[subject]{eigenvector}s (loadings) tell us what each PC represents biologically.
If all loadings have the same sign, PC1 represents overall ``size.'' If
loadings have mixed signs, the PC represents a contrast between trait
groups.

\begin{keyidea}
PCA\index[subject]{PCA} is eigendecomposition of the covariance matrix\index[subject]{covariance matrix}. The eigenvalue\index[subject]{eigenvalue}s measure
how much variance each direction captures; the eigenvector\index[subject]{eigenvector}s define those
directions. PCA\index[subject]{PCA} does not assume any group structure---it describes the
total variation in the sample.
\end{keyidea}

\subsection*{Covariance vs. correlation\index[subject]{correlation}PCA\index[subject]{PCA}}

A crucial choice is whether to analyse the covariance matrix\index[subject]{covariance matrix} $\mat{S}$ or
the correlation\index[subject]{correlation}matrix $\mat{R}$.

\begin{itemize}
  \item \textbf{Covariance PCA\index[subject]{PCA}:} Uses raw (centred) data. Traits with
        larger variances dominate the analysis. Appropriate when traits
        are on comparable scales and absolute variances are meaningful.
  \item \textbf{correlation\index[subject]{correlation}PCA\index[subject]{PCA}:} Standardises each trait to unit variance
        before analysis. All traits contribute equally regardless of
        original scale. Appropriate when traits are on different scales
        (e.g., length in mm vs. mass in g).
\end{itemize}

In evolutionary biology, covariance PCA is often preferred when analysing
G or P matrices because the absolute magnitudes of genetic variances are
biologically meaningful. correlation\index[subject]{correlation}PCA\index[subject]{PCA} is useful for exploratory analysis
of phenotypic data on mixed scales.

\section{MANOVA\index[subject]{MANOVA}: comparing groups}

While PCA\index[subject]{PCA} describes variation within a single sample, MANOVA\index[subject]{MANOVA} asks whether
multiple groups differ in their multivariate means. It is the multivariate
generalisation of ANOVA.

\subsection*{The setup}

Suppose we have $k$ groups (e.g., populations, treatments, species) and
measure $p$ traits on individuals within each group. MANOVA\index[subject]{MANOVA} tests the null
hypothesis that all group means are equal:
\[
  H_0: \boldsymbol{\mu}_1 = \boldsymbol{\mu}_2 = \cdots = \boldsymbol{\mu}_k.
\]

\subsection*{The geometry}

MANOVA\index[subject]{MANOVA} decomposes total variation into among-group and within-group
components, just as univariate ANOVA does. The key objects are:

\begin{itemize}
  \item \textbf{Among-group matrix $\mat{B}$:} Measures how group means
        differ from the grand mean. Large $\mat{B}$ indicates groups are
        spread out in trait space.
  \item \textbf{Within-group matrix $\mat{W}$:} Measures variation within
        groups, pooled across groups. This is the ``error'' variation.
  \item \textbf{Total matrix $\mat{T} = \mat{B} + \mat{W}$:} Total
        variation ignoring group structure.
\end{itemize}

The MANOVA\index[subject]{MANOVA} test asks: is $\mat{B}$ large relative to $\mat{W}$? If groups
are very different (large $\mat{B}$) and within-group variation is small
(small $\mat{W}$), we reject the null hypothesis.

\subsection*{Test statistics}

Several test statistics are used, all based on eigenvalue\index[subject]{eigenvalue}s of
$\mat{W}^{-1}\mat{B}$:

\begin{itemize}
  \item \textbf{Wilks' $\Lambda$:} $\Lambda = \det(\mat{W}) / \det(\mat{T})
        = \prod_i (1 + \lambda_i)^{-1}$, where $\lambda_i$ are eigenvalue\index[subject]{eigenvalue}s
        of $\mat{W}^{-1}\mat{B}$.
  \item \textbf{Pillai's trace:} $\sum_i \lambda_i / (1 + \lambda_i)$.
  \item \textbf{Hotelling-Lawley trace:} $\sum_i \lambda_i$.
  \item \textbf{Roy's largest root:} $\lambda_1$ (the largest eigenvalue\index[subject]{eigenvalue}).
\end{itemize}

These statistics have different properties. Wilks' $\Lambda$ is most common;
Pillai's trace is most robust to violations of assumptions; Roy's largest
root is most powerful when groups differ along a single dimension.

\begin{keyidea}
MANOVA\index[subject]{MANOVA} tests whether groups differ by comparing among-group variation
($\mat{B}$) to within-group variation ($\mat{W}$). The eigenvalue\index[subject]{eigenvalue}s of
$\mat{W}^{-1}\mat{B}$ quantify how much groups differ along each
discriminant axis.
\end{keyidea}

\section{discriminant analysis\index[subject]{discriminant analysis}}

discriminant analysis\index[subject]{discriminant analysis} is closely related to MANOVA\index[subject]{MANOVA}. While MANOVA\index[subject]{MANOVA} tests
\emph{whether} groups differ, discriminant analysis\index[subject]{discriminant analysis} finds the directions
along which they differ most and uses these for classification.

\subsection*{Linear discriminant analysis\index[subject]{discriminant analysis} (LDA)}

LDA finds linear combinations of traits that maximise the ratio of
among-group to within-group variance. These are the eigenvector\index[subject]{eigenvector}s of
$\mat{W}^{-1}\mat{B}$.

The first discriminant function (DF1) is the direction along which groups
are most separated relative to within-group spread. DF2 is the next-best
direction orthogonal to DF1, and so on.

\begin{figure}[ht]
  \centering
  \includegraphics[width=0.85\textwidth]{fig32_discriminant.png}
  \caption[discriminant analysis\index[subject]{discriminant analysis}]{
    discriminant analysis\index[subject]{discriminant analysis} finds directions that separate groups. The
    first discriminant function maximises the ratio of among-group to
    within-group variance. Projecting data onto this axis best reveals
    group differences.
  }
  \label{fig:discriminant}
\end{figure}

\subsection*{Connection to Mahalanobis distance\index[subject]{distance!Mahalanobis} distance}

The Mahalanobis distance\index[subject]{distance!Mahalanobis} distance between two group means, using the pooled
within-group covariance, is:
\[
  D^2 = (\boldsymbol{\mu}_1 - \boldsymbol{\mu}_2)^\top \mat{W}^{-1}
        (\boldsymbol{\mu}_1 - \boldsymbol{\mu}_2).
\]

This is exactly the squared distance along the discriminant axis connecting
the two groups. discriminant analysis\index[subject]{discriminant analysis} and Mahalanobis distance\index[subject]{distance!Mahalanobis} distance are two views
of the same geometry.

\section{Canonical correlation\index[subject]{correlation}Analysis (CCA)}

CCA extends correlation\index[subject]{correlation}to multiple variables on each side. Given two sets
of variables (e.g., morphological traits and physiological traits), CCA
finds linear combinations of each set that are maximally correlated.

\subsection*{The setup}

Let $\vect{x}$ be a vector of $p$ variables and $\vect{y}$ be a vector of
$q$ variables. We seek coefficients $\vect{a}$ and $\vect{b}$ such that
the correlation\index[subject]{correlation}between $\vect{a}^\top\vect{x}$ and $\vect{b}^\top\vect{y}$
is maximised.

The solution involves eigendecomposition of matrices built from the
covariance structure. The eigenvalue\index[subject]{eigenvalue}s are the squared canonical
correlations; the eigenvector\index[subject]{eigenvector}s define the canonical variates.

\subsection*{Biological applications}

CCA is useful for:
\begin{itemize}
  \item Relating genotype to phenotype (which genetic combinations predict
        which phenotypic combinations?).
  \item Relating morphology to performance (which body shapes predict which
        functional capacities?).
  \item Relating traits to environmental variables.
\end{itemize}

\begin{figure}[htbp]
    \centering
    \includegraphics[width=\textwidth]{figures/fig_ch11_cca_geometry.pdf}
    \caption{Geometry of Canonical Correlation Analysis (CCA). (a) The 
    $\mathbf{X}$ variable set (e.g., morphological traits) with its covariance 
    ellipse and first canonical direction $\mathbf{a}_1$. (b) The $\mathbf{Y}$ 
    variable set (e.g., performance traits) with its covariance ellipse and 
    corresponding canonical direction $\mathbf{b}_1$. (c) First canonical 
    variate pair: projections onto $\mathbf{a}_1$ and $\mathbf{b}_1$ achieve 
    maximum correlation ($r_1$). (d) Second canonical variate pair: orthogonal 
    to the first, capturing residual correlation ($r_2$). CCA finds linear 
    combinations of each variable set that are maximally correlated, revealing 
    the underlying dimensions linking morphology to performance, or genotype 
    to phenotype.}
    \label{fig:cca_geometry}
\end{figure}

\section{Projection pursuit: the general principle}

All these methods share a common structure: they find ``interesting''
directions in high-dimensional space by optimising some criterion.

\begin{itemize}
  \item PCA\index[subject]{PCA} maximises variance.
  \item discriminant analysis\index[subject]{discriminant analysis} maximises group separation.
  \item CCA maximises correlation\index[subject]{correlation}between variable sets.
\end{itemize}

This perspective, called \textbf{projection pursuit}, suggests that we can
invent new methods by defining new criteria for ``interesting'' directions.
In evolutionary biology, natural criteria include:

\begin{itemize}
  \item Directions of maximum genetic variance ($\mathbf{g}_{\max}$).
  \item Directions of maximum selection ($\boldsymbol{\gamma}$ eigenvector\index[subject]{eigenvector}s).
  \item Directions of maximum or minimum heritability\index[subject]{heritability} ($\mat{G}^*$
        eigenvector\index[subject]{eigenvector}s).
\end{itemize}

\begin{keyidea}
Projection pursuit is the general framework: find directions that optimise
some criterion. PCA\index[subject]{PCA}, discriminant analysis\index[subject]{discriminant analysis}, and CCA are special cases.
Evolutionary biologists can define biologically motivated criteria.
\end{keyidea}

\section{Comparing G matrices}

A major application of these methods is comparing G matrices across
populations or species. Do different populations have the same pattern
of genetic constraints?

\subsection*{Common principal components}

Flury's method tests whether two or more G matrices share the same
eigenvector\index[subject]{eigenvector}s (common principal components) even if eigenvalue\index[subject]{eigenvalue}s differ.
This tests whether the ``shape'' of genetic constraint is conserved.

Hierarchical models allow partial sharing:
\begin{itemize}
  \item \textbf{Equal matrices:} Same eigenvector\index[subject]{eigenvector}s and eigenvalue\index[subject]{eigenvalue}s.
  \item \textbf{Proportional matrices:} Same eigenvector\index[subject]{eigenvector}s, eigenvalue\index[subject]{eigenvalue}s
        differ by a constant factor.
  \item \textbf{Common principal components:} Same eigenvector\index[subject]{eigenvector}s, different
        eigenvalue\index[subject]{eigenvalue}s.
  \item \textbf{Partial CPC:} Some eigenvector\index[subject]{eigenvector}s shared, others not.
  \item \textbf{Unrelated:} No structural similarity.
\end{itemize}

\subsection*{Random skewers}

An alternative approach is to compare how matrices respond to random
selection vectors. Generate many random unit vectors $\boldsymbol{\beta}$,
compute the response $\mat{G}\boldsymbol{\beta}$ for each matrix, and
correlate the responses.

If two G matrices give similar responses to the same selection pressures,
they are functionally similar even if their eigenvector\index[subject]{eigenvector}s differ.

\subsection*{Krzanowski's subspace comparison}

Krzanowski's method compares the subspaces spanned by the first $k$
eigenvector\index[subject]{eigenvector}s of two matrices. It asks: do the major axes of variation
span similar directions?

The comparison uses angles between subspaces. If the leading eigenvector\index[subject]{eigenvector}s
of two matrices point in similar directions, the subspaces overlap
substantially.

\section{Estimation issues}

Estimating covariance matrices is statistically challenging, especially
with many traits.

\subsection*{Sample size requirements}

A $p \times p$ covariance matrix\index[subject]{covariance matrix} has $p(p+1)/2$ unique elements. With $p$
traits, you need far more than $p$ observations to estimate the matrix
reliably. Rules of thumb suggest $n > 10p$ or even $n > 20p$ for stable
estimates.

When $n < p$, the sample covariance matrix\index[subject]{covariance matrix} is singular (has zero
eigenvalue\index[subject]{eigenvalue}s). This occurs frequently in genomic studies where thousands
of markers are measured on hundreds of individuals.

\subsection*{Shrinkage and regularisation}

When sample sizes are small relative to dimensionality, shrinkage
estimators improve accuracy by pulling eigenvalue\index[subject]{eigenvalue}s toward a common value
(reducing the extremes). Common approaches include:

\begin{itemize}
  \item \textbf{Ledoit-Wolf shrinkage:} Shrinks toward a scaled identity
        matrix.
  \item \textbf{Ridge regularisation:} Adds a small constant to the
        diagonal before inversion.
  \item \textbf{Factor models:} Assume the covariance arises from a few
        latent factors.
\end{itemize}

These methods trade bias for reduced variance, often improving predictions
and avoiding numerical instability.

\subsection*{Bayesian estimation}

Bayesian methods place prior distributions on covariance matrices and
estimate posterior distributions. This naturally handles small samples
and provides uncertainty quantification for eigenvalue\index[subject]{eigenvalue}s and eigenvector\index[subject]{eigenvector}s.

Animal models in quantitative genetics typically use Bayesian MCMC to
estimate G matrices, providing credible intervals for genetic parameters.

\section{Visualising high-dimensional patterns}

With more than three traits, direct visualisation is impossible. Several
approaches help:

\begin{itemize}
  \item \textbf{Scree plots:} Plot eigenvalue\index[subject]{eigenvalue}s against their rank to see
        how many dimensions capture most variation.
  \item \textbf{Biplots:} Overlay individuals (as points) and variables
        (as arrows) in the PC1-PC2 plane.
  \item \textbf{Heatmaps:} Display the covariance or correlation\index[subject]{correlation}matrix
        as a coloured grid.
  \item \textbf{Parallel coordinates:} Draw each individual as a line
        connecting their values on parallel trait axes.
\end{itemize}

\begin{figure}[ht]
  \centering
  \includegraphics[width=0.85\textwidth]{fig32_visualisation.png}
  \caption[Visualisation methods]{
    Methods for visualising multivariate data. Top left: scree plot of
    eigenvalue\index[subject]{eigenvalue}s. Top right: biplot showing individuals and variable
    loadings. Bottom: correlation\index[subject]{correlation}heatmap.
  }
  \label{fig:visualisation}
\end{figure}

\section{Worked example: PCA\index[subject]{PCA} of a phenotypic dataset}

Consider measurements of four traits on 150 individuals from three
populations. We perform PCA\index[subject]{PCA} on the pooled covariance matrix\index[subject]{covariance matrix}.

\paragraph{Step 1: Compute covariance matrix\index[subject]{covariance matrix}.}
\[
  \mat{S} =
  \begin{pmatrix}
    1.20 & 0.85 & 0.42 & 0.31 \\
    0.85 & 1.05 & 0.51 & 0.28 \\
    0.42 & 0.51 & 0.78 & 0.15 \\
    0.31 & 0.28 & 0.15 & 0.55
  \end{pmatrix}.
\]

\paragraph{Step 2: Eigendecomposition.}
eigenvalue\index[subject]{eigenvalue}s: $\lambda_1 = 2.35$, $\lambda_2 = 0.72$, $\lambda_3 = 0.38$,
$\lambda_4 = 0.13$.

Proportion of variance: PC1 captures $2.35/3.58 = 66\%$, PC2 captures
$20\%$, PC3 captures $11\%$, PC4 captures $4\%$.

\paragraph{Step 3: Interpret loadings.}
The first eigenvector\index[subject]{eigenvector} has all positive loadings: $(0.58, 0.56, 0.42, 0.31)$.
This represents overall ``size''---individuals with high PC1 scores are
large on all traits.

The second eigenvector\index[subject]{eigenvector} has mixed signs: $(0.21, 0.18, -0.65, 0.71)$.
This contrasts traits 3 and 4---individuals with high PC2 scores have
relatively low trait 3 and high trait 4.

\paragraph{Step 4: Biological interpretation.}
With 66\% of variance in PC1 (size), the dominant pattern is allometric
scaling. PC2 captures shape variation independent of size. For many
biological questions, PC1 and PC2 together (86\% of variance) provide
an adequate low-dimensional summary.

\section{Summary}

In this chapter we have:

\begin{itemize}
  \item Connected the geometric framework to statistical methods: PCA\index[subject]{PCA},
        MANOVA\index[subject]{MANOVA}, discriminant analysis, and CCA are all eigendecompositions.
  \item Detailed PCA\index[subject]{PCA} as eigendecomposition of the covariance matrix\index[subject]{covariance matrix},
        finding directions of maximum variance.
  \item Explained MANOVA\index[subject]{MANOVA} as a comparison of among-group ($\mat{B}$) to
        within-group ($\mat{W}$) variation, testing whether groups differ.
  \item Introduced discriminant analysis\index[subject]{discriminant analysis} for finding directions that best
        separate groups and for classification.
  \item Described CCA for finding maximally correlated combinations of
        two variable sets.
  \item Presented the projection pursuit framework as a unifying
        perspective: all methods find ``interesting'' directions by
        optimising some criterion.
  \item Discussed methods for comparing G matrices: common principal
        components, random skewers, and subspace comparison.
  \item Addressed estimation challenges: sample size requirements,
        shrinkage, regularisation, and Bayesian approaches.
  \item Introduced visualisation tools: scree plots, biplots, heatmaps.
  \item Worked through a PCA\index[subject]{PCA} example, interpreting eigenvalue\index[subject]{eigenvalue}s as variance
        captured and eigenvector\index[subject]{eigenvector}s as biological patterns.
\end{itemize}

With this chapter, we have completed the core of Part~IV. The matrices
G, P, and $\boldsymbol{\gamma}$ are no longer abstract---they are objects
we can estimate, visualise, compare, and interpret. In Part~V, we turn
to practice: worked examples that integrate all these ideas, and extensions
that connect to current research on directional heritability\index[subject]{heritability} and
evolutionary constraint.


% ==================================================
% Part V -- Practice and Extensions
% ==================================================
\part{Practice and Extensions}

\chapter{Worked Examples: Complete Analyses}
\label{ch:worked-examples}

This chapter brings together everything we have learned. We work through
complete analyses from raw data to biological interpretation, showing each
step explicitly. The goal is not just to demonstrate techniques, but to
illustrate the thought process: when to use each tool, how to check
assumptions, and how to connect mathematical results to biological meaning.

We present three examples of increasing complexity:
\begin{enumerate}
  \item A two-trait analysis of a G matrix, computing directional
        heritabilities by hand.
  \item A four-trait analysis comparing G and P, with P-whitening.
  \item A selection analysis combining $\boldsymbol{\beta}$ and
        $\boldsymbol{\gamma}$ with the G matrix.
\end{enumerate}

Each example follows the same arc: state the data, check assumptions,
compute the relevant eigendecompositions, and interpret the results
biologically.

%-------------------------------------------------------------------
\section{Example 1: Two-trait G matrix analysis}
\index[subject]{worked example!two-trait G matrix}
%-------------------------------------------------------------------

\subsection*{The data}

A plant breeding program has estimated the following additive genetic
covariance matrix\index[subject]{covariance matrix}\index[subject]{G matrix@$\mathbf{G}$ matrix} for 
flowering time (days) and plant height (cm):
\[
  \mat{G} =
  \begin{pmatrix}
    25 & 15 \\
    15 & 36
  \end{pmatrix}.
\]

The phenotypic covariance matrix\index[subject]{covariance matrix}\index[subject]{P matrix@$\mathbf{P}$ matrix} is:
\[
  \mat{P} =
  \begin{pmatrix}
    50 & 20 \\
    20 & 60
  \end{pmatrix}.
\]

Our goals are to:
\begin{enumerate}
  \item Find the principal axes of genetic variation.
  \item Compute heritability in several directions.
  \item Identify the directions of maximum and minimum heritability\index[subject]{heritability}.
\end{enumerate}

\begin{figure}[htbp]
    \centering
    \includegraphics[width=\textwidth]{figures/fig_ch12_worked_example.pdf}
    \caption{Complete analysis of the two-trait worked example. (a) Genetic 
    ($\mathbf{G}$, blue) and phenotypic ($\mathbf{P}$, green) ellipses in 
    original trait space, with $\mathbf{g}_{\max}$ and $\mathbf{g}_{\min}$ 
    marked. (b) The whitened view: $\mathbf{G}^*$ ellipse inside the 
    $\mathbf{P}$-sphere, showing directions of extreme heritability. 
    (c) Directional heritability $h^2(\theta)$ as a function of direction, 
    ranging from 0.42 to 0.62. (d) Summary statistics from the analysis. 
    (e) The breeder's equation in action: selection gradient $\boldsymbol{\beta}$ 
    is deflected toward $\mathbf{g}_{\max}$ in the response $\Delta\bar{\mathbf{z}}$. 
    (f) Variance decomposition by direction, showing the gap between phenotypic 
    and genetic variance (environmental variance) varies with direction.}
    \label{fig:worked_example}
\end{figure}

\subsection*{Step 1: Eigendecompose G}
\index[subject]{eigendecomposition!of G matrix}

The characteristic equation\index[subject]{characteristic equation}\index[subject]{characteristic equation\index[subject]{characteristic equation}} for $\mat{G}$ is:
\[
  \det(\mat{G} - \lambda\mat{I}) = (25 - \lambda)(36 - \lambda) - 15^2 = 0.
\]

Expanding:
\[
  \lambda^2 - 61\lambda + (25 \times 36 - 225) = \lambda^2 - 61\lambda + 675 = 0.
\]

Using the quadratic form\index[subject]{quadratic form}ula:
\[
  \lambda = \frac{61 \pm \sqrt{61^2 - 4 \times 675}}{2}
          = \frac{61 \pm \sqrt{3721 - 2700}}{2}
          = \frac{61 \pm \sqrt{1021}}{2}
          = \frac{61 \pm 31.95}{2}.
\]

So $\lambda_1 = 46.48$ and $\lambda_2 = 14.52$.

For $\lambda_1 = 46.48$, the eigenvector satisfies:
\[
  \begin{pmatrix} 25 - 46.48 & 15 \\ 15 & 36 - 46.48 \end{pmatrix}
  \begin{pmatrix} v_1 \\ v_2 \end{pmatrix} = \vect{0}.
\]

From the first row: $-21.48 v_1 + 15 v_2 = 0$, so $v_2/v_1 = 21.48/15 = 1.43$.

Normalising: $\mathbf{g}_{\max} = (0.573, 0.820)^\top$.
\index[subject]{gmax@$\mathbf{g}_{\max}$}

Similarly, $\mathbf{g}_{\min} = (0.820, -0.573)^\top$ (orthogonal to
$\mathbf{g}_{\max}$).

\paragraph{Interpretation.}
The direction of maximum genetic variance points mostly toward height
(coefficient 0.820) with a positive contribution from flowering time
(0.573). Plants with high breeding values tend to be both tall and
late-flowering. The genetic variance along this axis is 46.48.

The direction of minimum genetic variance contrasts the traits: tall but
early-flowering, or short but late-flowering. Genetic variance along this
axis is only 14.52---about one-third of the maximum.

\subsection*{Step 2: Compute univariate heritabilities}
\index[subject]{heritability}

For each trait separately:
\[
  h^2_{\text{time}} = \frac{G_{11}}{P_{11}} = \frac{25}{50} = 0.50,
  \qquad
  h^2_{\text{height}} = \frac{G_{22}}{P_{22}} = \frac{36}{60} = 0.60.
\]

Both traits have moderate heritability\index[subject]{heritability}.

\subsection*{Step 3: Compute directional heritabilities}
\index[subject]{heritability!directional}

For an arbitrary direction $\boldsymbol{\beta}$, the directional
heritability\index[subject]{heritability} is:
\[
  h^2(\boldsymbol{\beta}) = 
    \frac{\boldsymbol{\beta}^\top \mat{G} \boldsymbol{\beta}}
         {\boldsymbol{\beta}^\top \mat{P} \boldsymbol{\beta}}.
\]

\paragraph{Along $\mathbf{g}_{\max}$.}
Let $\boldsymbol{\beta} = (0.573, 0.820)^\top$.

Numerator:
\begin{align*}
  \boldsymbol{\beta}^\top \mat{G} \boldsymbol{\beta}
    &= (0.573, 0.820) 
       \begin{pmatrix} 25 & 15 \\ 15 & 36 \end{pmatrix}
       \begin{pmatrix} 0.573 \\ 0.820 \end{pmatrix} \\
    &= (0.573, 0.820) \begin{pmatrix} 26.63 \\ 38.12 \end{pmatrix}
     = 46.48.
\end{align*}

Denominator:
\begin{align*}
  \boldsymbol{\beta}^\top \mat{P} \boldsymbol{\beta}
    &= (0.573, 0.820) 
       \begin{pmatrix} 50 & 20 \\ 20 & 60 \end{pmatrix}
       \begin{pmatrix} 0.573 \\ 0.820 \end{pmatrix} \\
    &= (0.573, 0.820) \begin{pmatrix} 45.05 \\ 60.66 \end{pmatrix}
     = 75.55.
\end{align*}

So $h^2(\mathbf{g}_{\max}) = 46.48 / 75.55 = 0.615$.

\paragraph{Along $\mathbf{g}_{\min}$.}
Let $\boldsymbol{\beta} = (0.820, -0.573)^\top$.

By similar calculation:
\[
  \boldsymbol{\beta}^\top \mat{G} \boldsymbol{\beta} = 14.52,
  \qquad
  \boldsymbol{\beta}^\top \mat{P} \boldsymbol{\beta} = 34.45.
\]

So $h^2(\mathbf{g}_{\min}) = 14.52 / 34.45 = 0.421$.

\paragraph{Along flowering time only.}
Let $\boldsymbol{\beta} = (1, 0)^\top$.

Then $h^2 = G_{11}/P_{11} = 25/50 = 0.50$.

\subsection*{Step 4: Find extreme heritabilities via G*}
\index[subject]{G* matrix@$\mathbf{G}^*$ matrix}

To find the true maximum and minimum heritabilities across all directions,
we compute $\mat{G}^* = \mat{P}^{-1/2}\mat{G}\mat{P}^{-1/2}$ and
find its eigenvalues.

First, eigendecompose $\mat{P}$:
\[
  \mat{P} = \mat{V}_P {\Lambda}_P \mat{V}_P^\top.
\]

The eigenvalue\index[subject]{eigenvalue}s of $\mat{P}$ are $\lambda_{P,1} = 75.62$ and
$\lambda_{P,2} = 34.38$, with corresponding eigenvector\index[subject]{eigenvector}s.

Then:
\[
  \mat{P}^{-1/2} = \mat{V}_P {\Lambda}_P^{-1/2} \mat{V}_P^\top.
\]

Computing $\mat{G}^* = \mat{P}^{-1/2}\mat{G}\mat{P}^{-1/2}$ and
finding its eigenvalue\index[subject]{eigenvalue}s gives:
\[
  \lambda_1^* = 0.619, \qquad \lambda_2^* = 0.419.
\]

These are the maximum and minimum directional heritabilities.

\begin{keyidea}
The eigenvalue\index[subject]{eigenvalue}s of $\mat{G}^* = \mat{P}^{-1/2}\mat{G}\mat{P}^{-1/2}$ give
the extreme directional heritabilities. Here, heritability\index[subject]{heritability} ranges from
0.42 to 0.62 depending on direction---a range of 0.20, or about 40\%
of the minimum value.
\end{keyidea}

\subsection*{Summary table}

\begin{center}
\begin{tabular}{lcc}
\hline
Direction & Genetic variance & heritability\index[subject]{heritability} \\
\hline
Flowering time only & 25.0 & 0.50 \\
Height only & 36.0 & 0.60 \\
$\mathbf{g}_{\max}$ & 46.5 & 0.62 \\
$\mathbf{g}_{\min}$ & 14.5 & 0.42 \\
Maximum $h^2$ direction & --- & 0.62 \\
Minimum $h^2$ direction & --- & 0.42 \\
\hline
\end{tabular}
\end{center}

In this example, the direction of maximum genetic variance 
($\mathbf{g}_{\max}$) is close to the direction of maximum heritability\index[subject]{heritability},
but they need not coincide in general. The former maximises
$\boldsymbol{\beta}^\top\mat{G}\boldsymbol{\beta}$; the latter maximises
the ratio
$\boldsymbol{\beta}^\top\mat{G}\boldsymbol{\beta} /
 \boldsymbol{\beta}^\top\mat{P}\boldsymbol{\beta}$.
When $\mat{G}$ and $\mat{P}$ have different orientations, these directions
can differ substantially.

%-------------------------------------------------------------------
\section{Example 2: Four-trait G-P comparison with whitening\index[subject]{whitening transformation}}
\index[subject]{worked example!four-trait G-P comparison}
\index[subject]{whitening\index[subject]{whitening transformation} transformation}
%-------------------------------------------------------------------

\subsection*{The data}

A study of a passerine bird population estimates G and P for four
morphological traits: wing length, tarsus length, bill depth, and bill
width. The matrices are:

\[
  \mat{G} =
  \begin{pmatrix}
    0.80 & 0.45 & 0.20 & 0.15 \\
    0.45 & 0.60 & 0.25 & 0.18 \\
    0.20 & 0.25 & 0.35 & 0.28 \\
    0.15 & 0.18 & 0.28 & 0.30
  \end{pmatrix},
\]
\[
  \mat{P} =
  \begin{pmatrix}
    1.20 & 0.55 & 0.30 & 0.22 \\
    0.55 & 0.95 & 0.35 & 0.25 \\
    0.30 & 0.35 & 0.55 & 0.40 \\
    0.22 & 0.25 & 0.40 & 0.50
  \end{pmatrix}.
\]

\subsection*{Step 1: Basic checks}
\index[subject]{positive definite\index[subject]{matrix!positive definite}ness}

Before analysis, we verify that both matrices are positive definite\index[subject]{matrix!positive definite}
(all eigenvalue\index[subject]{eigenvalue}s positive) and that $G_{ii} \le P_{ii}$ for each trait
(genetic variance should not exceed phenotypic variance).

eigenvalue\index[subject]{eigenvalue}s of $\mat{G}$: 1.36, 0.44, 0.21, 0.04.
All positive---$\mat{G}$ is positive definite\index[subject]{matrix!positive definite}.

eigenvalue\index[subject]{eigenvalue}s of $\mat{P}$: 1.95, 0.67, 0.46, 0.12.
All positive---$\mat{P}$ is positive definite\index[subject]{matrix!positive definite}.

Diagonal check: $G_{ii} \le P_{ii}$ for all $i$. Yes: 0.80 < 1.20,
0.60 < 0.95, 0.35 < 0.55, 0.30 < 0.50.

\subsection*{Step 2: Univariate heritabilities}

\begin{center}
\begin{tabular}{lccc}
\hline
Trait & $G_{ii}$ & $P_{ii}$ & $h^2$ \\
\hline
Wing length  & 0.80 & 1.20 & 0.67 \\
Tarsus length & 0.60 & 0.95 & 0.63 \\
Bill depth   & 0.35 & 0.55 & 0.64 \\
Bill width   & 0.30 & 0.50 & 0.60 \\
\hline
\end{tabular}
\end{center}

All traits have similar, moderately high heritabilities (0.60--0.67).
A univariate analysis would conclude that all four traits are roughly
equally heritable. But this masks important directional variation.

\subsection*{Step 3: Compute G* and its eigenstructure}

We compute $\mat{G}^* = \mat{P}^{-1/2}\mat{G}\mat{P}^{-1/2}$ using the
eigendecomposition of $\mat{P}$.

The eigenvalue\index[subject]{eigenvalue}s of $\mat{G}^*$ are the directional heritabilities along
the principal axes of P-whitened space:

\begin{center}
\begin{tabular}{cccc}
\hline
$\lambda_1^*$ & $\lambda_2^*$ & $\lambda_3^*$ & $\lambda_4^*$ \\
\hline
0.71 & 0.65 & 0.45 & 0.34 \\
\hline
\end{tabular}
\end{center}

\subsection*{Step 4: Interpret the heritability\index[subject]{heritability} distribution}

The eigenvalue\index[subject]{eigenvalue}s range from 0.34 to 0.71. This means:
\begin{itemize}
  \item Maximum directional heritability\index[subject]{heritability}: 0.71 (71\% of variance genetic).
  \item Minimum directional heritability\index[subject]{heritability}: 0.34 (34\% genetic).
  \item Range: 0.37 (more than double the minimum).
\end{itemize}

Mean heritability\index[subject]{heritability} (average of eigenvalue\index[subject]{eigenvalue}s): $\bar{h}^2 = 0.54$.

Coefficient of variation of eigenvalue\index[subject]{eigenvalue}s:
\[
  \text{CV}(\lambda^*) = \frac{\text{SD}(\lambda^*)}{\text{mean}(\lambda^*)}
    = \frac{0.150}{0.54} = 0.28.
\]

From the formula $\text{CV}(h^2) = \sqrt{2/(p+2)} \times \text{CV}(\lambda^*)$:
\[
  \text{CV}(h^2) = \sqrt{2/6} \times 0.28 = 0.577 \times 0.28 = 0.16.
\]

The coefficient of variation of directional heritability\index[subject]{heritability} is about 16\%.
This indicates moderate constraint heterogeneity---some directions are
substantially more heritable than others.

\subsection*{Step 5: Identify constraint traps}
\index[subject]{constraint!trap}

The eigenvector\index[subject]{eigenvector} corresponding to $\lambda_4^* = 0.34$ defines the direction
of minimum heritability\index[subject]{heritability}. Examining its loadings:
\[
  \vect{v}_4^* = (0.14, -0.19, 0.66, -0.71)^\top.
\]

This direction contrasts the bill traits (depth positive, width negative)
with small contributions from the body-size traits. Selection for birds
with deep but narrow bills, or vice versa, would face a constraint trap:
only 34\% of phenotypic variance in this direction is genetic.

In contrast, the direction of maximum heritability\index[subject]{heritability} ($\lambda_1^* = 0.71$)
has loadings:
\[
  \vect{v}_1^* = (0.79, 0.61, 0.05, 0.04)^\top.
\]

This direction loads heavily on wing and tarsus---overall body ``size.''
Selection for larger or smaller birds has high heritability\index[subject]{heritability}; 71\% of
variance is genetic.

\begin{keyidea}
Size variation has high heritability\index[subject]{heritability} (71\%); bill-shape variation has
lower heritability\index[subject]{heritability} (34\%). A breeding program targeting bill proportions
would face stronger constraints than one targeting overall body size.
\end{keyidea}

\subsection*{Step 6: Geometric interpretation}

Figure~\ref{fig:bird_case_study} illustrates these results geometrically.

In the original trait coordinates, the phenotypic covariance matrix\index[subject]{covariance matrix}
$\mat{P}$ defines an ellipsoid whose cross-sections in any two-trait
plane are ellipses. The genetic covariance matrix\index[subject]{covariance matrix} $\mat{G}$ defines a
second ellipsoid nested inside the first. For this bird population, the
$\mat{G}$ ellipsoid is somewhat narrower along directions involving bill
shape than along overall size.

whitening\index[subject]{whitening transformation} by $\mat{P}^{-1/2}$ maps the $\mat{P}$ ellipsoid to a sphere:
in whitened coordinates every direction has unit phenotypic variance.
In this whitened space, $\mat{G}^\ast$ appears as an ellipsoid whose
axes have lengths given by the square roots of the eigenvalue\index[subject]{eigenvalue}s
$\lambda_i^\ast$. The long axis corresponds to the high-heritability\index[subject]{heritability}
``size'' direction ($h^2 = 0.71$); the short axis corresponds to the
low-heritability\index[subject]{heritability} ``shape'' direction ($h^2 = 0.34$).

\begin{figure}[t]
  \centering
  \includegraphics[width=0.95\textwidth]{fig_bird_case_study.pdf}
  \caption[Geometric summary of the bird G--P example]{
    Schematic view of the four-trait bird example.
    (a) Genetic ($\mat{G}$) and phenotypic ($\mat{P}$) ellipses in a
        two-trait projection of the original trait space.
    (b) Whitened trait space: the P-sphere\index[subject]{P-sphere} (unit circle) and the
        $\mat{G}^\ast$ ellipse showing directional heritabilities. The
        long axis corresponds to the high-heritability\index[subject]{heritability} size direction
        ($h^2 = 0.71$); the short axis corresponds to the 
        low-heritability\index[subject]{heritability} bill-shape direction ($h^2 = 0.34$), a 
        constraint trap.
  }
  \label{fig:bird_case_study}
\end{figure}

%-------------------------------------------------------------------
\section[Example 3: Selection analysis with G and gamma]{Example 3: Selection analysis with G and \texorpdfstring{$\boldsymbol{\gamma}$}{γ}}
\index[subject]{worked example!selection analysis}
\index[subject]{gamma matrix@$\boldsymbol{\gamma}$ matrix}
%-------------------------------------------------------------------

\subsection*{The data}

A study measures survival in relation to two traits (standardised to mean
zero, unit variance). The estimated selection gradients are:
\[
  \boldsymbol{\beta} = \begin{pmatrix} 0.18 \\ 0.12 \end{pmatrix},
  \qquad
  \boldsymbol{\gamma} = \begin{pmatrix} -0.15 & 0.08 \\ 0.08 & -0.10 \end{pmatrix}.
\]

The G matrix (in standardised units) is:
\[
  \mat{G} = \begin{pmatrix} 0.45 & 0.30 \\ 0.30 & 0.35 \end{pmatrix}.
\]

\subsection*{Step 1: Interpret $\boldsymbol{\beta}$}
\index[subject]{selection gradient}

Both elements of $\boldsymbol{\beta}$ are positive: directional selection
favours increases in both traits. Trait 1 is under stronger directional
selection (0.18 vs.\ 0.12).

The magnitude $\|\boldsymbol{\beta}\| = \sqrt{0.18^2 + 0.12^2} = 0.216$
gives overall strength of directional selection.

The direction of selection:
\[
  \frac{\boldsymbol{\beta}}{\|\boldsymbol{\beta}\|} = (0.832, 0.555)^\top.
\]

\subsection*{Step 2: Interpret $\boldsymbol{\gamma}$}
\index[subject]{selection!stabilising}
\index[subject]{selection!correlational}

Both diagonal elements are negative: stabilising selection on each trait
individually. The off-diagonal is positive: correlational selection favours
positive trait combinations (both high or both low together).

Eigendecomposition of $\boldsymbol{\gamma}$:
\[
  \lambda_1^\gamma = -0.04, \qquad \lambda_2^\gamma = -0.21.
\]

Both negative, confirming overall stabilising selection. But selection is
much stronger ($|\lambda| = 0.21$) along the second eigenvector\index[subject]{eigenvector} than the
first ($|\lambda| = 0.04$).

The eigenvector\index[subject]{eigenvector}s are:
\begin{align*}
  \vect{v}_1^\gamma &= (0.59, 0.81)^\top \quad \text{(weak stabilising)}, \\
  \vect{v}_2^\gamma &= (0.81, -0.59)^\top \quad \text{(strong stabilising)}.
\end{align*}

\paragraph{Interpretation.}
Stabilising selection is weak along the direction where both traits
increase together---a ``size'' axis. Stabilising selection is strong
along the direction where traits oppose each other---a ``contrast'' axis.
The fitness surface is elongated: a ridge running along the positive
diagonal, with steep sides.

\subsection*{Step 3: Predict response to selection}

Using the Lande equation\index[subject]{breeder's equation}:
\[
  \Delta\bar{\vect{z}} = \mat{G}\boldsymbol{\beta}
    = \begin{pmatrix} 0.45 & 0.30 \\ 0.30 & 0.35 \end{pmatrix}
      \begin{pmatrix} 0.18 \\ 0.12 \end{pmatrix}
    = \begin{pmatrix} 0.117 \\ 0.096 \end{pmatrix}.
\]

The predicted response is $(0.117, 0.096)$---increases in both traits, with
trait 1 responding slightly more.

The response direction is:
\[
  \frac{\Delta\bar{\vect{z}}}{\|\Delta\bar{\vect{z}}\|}
    = (0.773, 0.634)^\top.
\]

Comparing to the selection direction $(0.832, 0.555)^\top$, we see the
response is deflected toward the direction of higher genetic variance
(trait 1), but the deflection is modest because the genetic correlation
is positive and selection favours both traits.

\subsection*{Step 4: Alignment of G and $\boldsymbol{\gamma}$}
\index[subject]{G-gamma alignment@G--$\gamma$ alignment}

Eigendecompose $\mat{G}$:
\[
  \lambda_1^G = 0.70, \quad \lambda_2^G = 0.10.
\]

The eigenvector\index[subject]{eigenvector} for $\lambda_1^G$ is $(0.76, 0.65)^\top$---similar to the
direction of weak stabilising selection ($\vect{v}_1^\gamma$).

\begin{keyidea}
The direction of maximum genetic variance ($\mathbf{g}_{\max}$) aligns
with the direction of weak stabilising selection. This is favourable:
the population can move along the fitness ridge without fighting strong
curvature. Evolution is channeled but not blocked.
\end{keyidea}

Conversely, the direction of minimum genetic variance aligns with the
direction of strong stabilising selection. Even if selection pushed toward
unusual trait combinations (one high, one low), the population would
struggle to respond because:
\begin{enumerate}
  \item Genetic variance is low in that direction ($\lambda_2^G = 0.10$).
  \item Stabilising selection is strong ($\lambda_2^\gamma = -0.21$).
\end{enumerate}

This alignment is likely not coincidental. Theory predicts that
mutation-selection balance\index[subject]{mutation-selection balance} tends to erode variance in directions of strong
stabilising selection while preserving variance along fitness ridges.

\subsection*{Step 5: Long-term prediction}

Under mutation-selection balance\index[subject]{mutation-selection balance}, the population is expected to maintain
variation primarily along $\mathbf{g}_{\max}$ (the fitness ridge). The
combination of high genetic variance, weak stabilising selection, and
positive correlational selection along the size axis suggests that size
variation will persist. Variation in trait contrast will be more rapidly
eroded.

This example illustrates why both $\mat{G}$ and $\boldsymbol{\gamma}$
matter. Knowing $\mat{G}$ alone tells us about evolutionary potential;
knowing $\boldsymbol{\gamma}$ alone tells us about the fitness landscape.
Only by comparing their geometries can we predict whether evolution will
be fast or slow, direct or deflected.

%-------------------------------------------------------------------
\section{Computational tools}
\index[subject]{R code}
%-------------------------------------------------------------------

All calculations in this chapter can be performed by hand for two traits,
but become tedious for more. Here is R code implementing the key steps:

\begin{verbatim}
# Given G and P matrices (bird example)
G <- matrix(c(0.80, 0.45, 0.20, 0.15,
              0.45, 0.60, 0.25, 0.18,
              0.20, 0.25, 0.35, 0.28,
              0.15, 0.18, 0.28, 0.30), 4, 4)

P <- matrix(c(1.20, 0.55, 0.30, 0.22,
              0.55, 0.95, 0.35, 0.25,
              0.30, 0.35, 0.55, 0.40,
              0.22, 0.25, 0.40, 0.50), 4, 4)

# Step 1: Basic checks
cat("G eigenvalues:", round(eigen(G)$values, 3), "\n")
cat("P eigenvalues:", round(eigen(P)$values, 3), "\n")
cat("All positive?", all(eigen(G)$values > 0) & 
                     all(eigen(P)$values > 0), "\n")

# Step 2: Compute P^{-1/2}
eig_P <- eigen(P)
V_P <- eig_P$vectors
P_inv_sqrt <- V_P %*% diag(1/sqrt(eig_P$values)) %*% t(V_P)

# Step 3: Compute G*
G_star <- P_inv_sqrt %*% G %*% P_inv_sqrt

# Step 4: Eigenvalues of G* are directional heritabilities
eig_Gstar <- eigen(G_star)
h2_dir <- eig_Gstar$values
cat("Directional heritabilities:", round(h2_dir, 3), "\n")
cat("Max h2:", round(max(h2_dir), 3), "\n")
cat("Min h2:", round(min(h2_dir), 3), "\n")

# Step 5: CV of directional heritability
mean_h2 <- mean(h2_dir)
sd_h2 <- sd(h2_dir)
cv_lambda <- sd_h2 / mean_h2
p <- length(h2_dir)
cv_h2 <- sqrt(2/(p+2)) * cv_lambda
cat("Mean h2:", round(mean_h2, 3), "\n")
cat("CV(h2):", round(cv_h2, 3), "\n")

# Step 6: Identify constraint directions
cat("\nMax h2 direction (loadings):\n")
print(round(eig_Gstar$vectors[, 1], 2))
cat("\nMin h2 direction (loadings):\n")
print(round(eig_Gstar$vectors[, p], 2))
\end{verbatim}

%-------------------------------------------------------------------
\section{Summary}
%-------------------------------------------------------------------

In this chapter we have:

\begin{itemize}
  \item Worked through a complete two-trait analysis: eigendecomposition
        of G, calculation of directional heritabilities, and identification
        of extreme values via G*.
  \item Extended to four traits, demonstrating P-whitening\index[subject]{whitening transformation} and
        interpretation of the eigenvalue\index[subject]{eigenvalue} spectrum as the distribution of
        directional heritability\index[subject]{heritability}.
  \item Combined G with $\boldsymbol{\gamma}$ to analyse how genetic
        constraint interacts with the fitness surface geometry.
  \item Shown that alignment between G and $\boldsymbol{\gamma}$ determines
        whether evolution is facilitated or frustrated.
  \item Provided R code for computing G* and its eigenstructure.
\end{itemize}

These examples illustrate the payoff of the geometric perspective. Matrices
are not just tables of numbers---they are shapes that constrain and channel
evolution. By visualising G, P, and $\boldsymbol{\gamma}$ as ellipsoids and
understanding their eigenstructure, we gain insight into evolutionary
potential and constraint that would be invisible from univariate analyses
alone.

%-------------------------------------------------------------------
\section*{Exercises}
%-------------------------------------------------------------------

\paragraph{Exercise 1 (Covariance ellipses with the same trace).}
Construct two $2 \times 2$ covariance matrices that have the same trace
(sum of diagonal elements) but different eigenvalue\index[subject]{eigenvalue}s. For each matrix:
\begin{enumerate}
  \item Compute the eigenvalue\index[subject]{eigenvalue}s and eigenvector\index[subject]{eigenvector}s.
  \item Sketch the corresponding covariance ellipse.
  \item Explain how the trace can be the same while the shape differs.
\end{enumerate}

\paragraph{Exercise 2 (G* and the P-sphere\index[subject]{P-sphere} for two traits).}
Using the plant example from Example~1:
\begin{enumerate}
  \item Compute $\mat{P}^{-1/2}$ explicitly.
  \item Compute $\mat{G}^\ast = \mat{P}^{-1/2} \mat{G} \mat{P}^{-1/2}$.
  \item Draw the unit circle (the P-sphere\index[subject]{P-sphere}) and sketch the ellipse
        defined by $\mat{G}^\ast$.
  \item Mark the directions of maximum and minimum heritability\index[subject]{heritability}.
\end{enumerate}

\paragraph{Exercise 3 (Directional heritability\index[subject]{heritability} in a chosen direction).}
In the bird example from Example~2:
\begin{enumerate}
  \item Define a direction corresponding to ``bill shape'' (e.g.,
        increasing bill depth while decreasing bill width).
  \item Compute $h^2(\vect{u})$ for this direction.
  \item Compare to the extreme values from $\mat{G}^\ast$.
\end{enumerate}

\paragraph{Exercise 4 (G-$\gamma$ alignment).}
Consider the selection example in Example~3:
\begin{enumerate}
  \item Compute the angle between $\mathbf{g}_{\max}$ and
        $\vect{v}_1^\gamma$.
  \item How would the evolutionary trajectory change if these were
        perpendicular?
  \item Describe a scenario where misalignment would strongly
        frustrate evolutionary change.
\end{enumerate}
\chapter{directional heritability\index[subject]{heritability!directional} and the Geometry of Constraint}

This final chapter connects the geometric framework to a frontier research
question: how does heritability\index[subject]{heritability} vary across directions in trait space, and
what does this variation tell us about evolutionary constraint?

We have seen that the eigenvalue\index[subject]{eigenvalue}s of
$\mat{G}^* = \mat{P}^{-1/2}\mat{G}\mat{P}^{-1/2}$ are the directional
heritabilities along principal axes. But most selection does not align
with principal axes. What is the \emph{distribution} of heritability\index[subject]{heritability} across
all possible directions? How do we characterise, measure, and interpret
this distribution?

These questions lead to the concept of \textbf{constraint heterogeneity}:
the degree to which heritability\index[subject]{heritability} varies across directions. When constraint
heterogeneity is high, some directions are evolutionary highways while
others are dead ends. Understanding this heterogeneity is essential for
predicting evolutionary trajectories and designing effective breeding
programs.

\section{From eigenvalue\index[subject]{eigenvalue}s to distributions}

The eigenvalue\index[subject]{eigenvalue}s of $\mat{G}^*$ give us the extreme directional
heritabilities:
\[
  h^2_{\min} = \lambda_p^* \le h^2(\boldsymbol{\beta}) \le \lambda_1^* = h^2_{\max}.
\]

But what about all the directions in between? If we sample directions
uniformly from the P-sphere\index[subject]{P-sphere} (the set of unit phenotypic variance directions),
what distribution of $h^2$ values do we observe?

From Chapter~21, we know that the quadratic form\index[subject]{quadratic form}
$h^2(\boldsymbol{\beta}) = \boldsymbol{\beta}^\top\mat{G}^*\boldsymbol{\beta}$
(for unit vectors in whitened space) is a weighted average of the
eigenvalue\index[subject]{eigenvalue}s $\lambda_i^*$, with weights given by squared projections onto
eigenvectors.

The mathematics of random quadratic form\index[subject]{quadratic form}s gives us a precise characterisation.
For a random unit vector uniformly distributed on the sphere, the variance
of the quadratic form\index[subject]{quadratic form} is:
\[
  \Var[h^2(\boldsymbol{\beta})] = \frac{2}{p(p+2)} 
    \sum_{i < j} (\lambda_i^* - \lambda_j^*)^2
    = \frac{2}{p+2} \Var(\lambda^*),
\]
where $\Var(\lambda^*)$ is the variance of the eigenvalue\index[subject]{eigenvalue}s of $\mat{G}^*$.

\begin{keyidea}
The variance of directional heritability\index[subject]{heritability} across random directions is
proportional to the variance of the eigenvalue\index[subject]{eigenvalue}s of $\mat{G}^*$. If
eigenvalue\index[subject]{eigenvalue}s are similar, heritability\index[subject]{heritability} is similar in all directions. If
eigenvalue\index[subject]{eigenvalue}s differ greatly, heritability\index[subject]{heritability} varies dramatically with direction.
\end{keyidea}

\begin{figure}[htbp]
    \centering
    \includegraphics[width=\textwidth]{figures/fig_ch13_cv_dimensionality.pdf}
    \caption{\textbf{How dimensionality affects the variation in directional heritability.}
    (a)~The dimensionality factor $\sqrt{2/(p+2)}$ decreases with the number of 
    traits $p$. In two dimensions, this factor equals 0.707; by $p = 50$, it 
    has fallen to 0.196. This decay occurs because random directions in high 
    dimensions tend to ``average out'' the eigenvalues of $\mathbf{G}^*$, 
    reducing the variation in $h^2(\boldsymbol{\beta})$ across directions.
    (b)~The coefficient of variation of directional heritability, 
    $\text{CV}(h^2) = \sqrt{2/(p+2)} \times \text{CV}(\lambda^*)$, shown for 
    different levels of eigenvalue dispersion. Higher $\text{CV}(\lambda^*)$ 
    (more eccentric $\mathbf{G}^*$ ellipsoid) produces greater directional 
    variation in heritability, but this effect is attenuated as dimensionality 
    increases.
    (c)~Simulation verification. Blue points show empirical $\text{CV}(h^2)$ 
    computed from 5,000 random directions sampled uniformly on the unit sphere; 
    the dashed line shows the theoretical prediction. The near-perfect agreement 
    (correlation $r > 0.99$) validates the formula 
    $\text{CV}^2(h^2) = \frac{2}{p+2} \times V_{\text{rel}}(\mathbf{G}^*)$ 
    derived in Section~13.1.}
    \label{fig:cv_dimensionality}
\end{figure}

\section{The coefficient of variation of directional heritability\index[subject]{heritability}}

A natural measure of constraint heterogeneity is the coefficient of
variation of directional heritability\index[subject]{heritability}:
\[
  \text{CV}(h^2) = \frac{\text{SD}(h^2)}{\text{E}(h^2)}.
\]

Using the results above and the fact that $\text{E}(h^2) = \bar{\lambda}^*$
(the mean eigenvalue\index[subject]{eigenvalue}), we obtain:
\[
  \text{CV}(h^2) = \sqrt{\frac{2}{p+2}} \times \text{CV}(\lambda^*),
\]
where $\text{CV}(\lambda^*) = \text{SD}(\lambda^*) / \bar{\lambda}^*$ is
the coefficient of variation of the eigenvalue\index[subject]{eigenvalue}s.

This formula reveals two factors controlling heritability\index[subject]{heritability} variation:

\begin{enumerate}
  \item \textbf{eigenvalue\index[subject]{eigenvalue} dispersion:} $\text{CV}(\lambda^*)$ measures
        how different the principal heritabilities are. Large dispersion
        means some directions have much higher heritability\index[subject]{heritability} than others.
  \item \textbf{Dimensionality:} The factor $\sqrt{2/(p+2)}$ decreases
        with the number of traits $p$. In high dimensions, random
        directions tend to ``average out'' the eigenvalue\index[subject]{eigenvalue}s, reducing
        heritability\index[subject]{heritability} variation.
\end{enumerate}

\begin{figure}[ht]
  \centering
  \includegraphics[width=0.9\textwidth]{fig41_cv_formula.png}
  \caption[CV of directional heritability\index[subject]{heritability}]{
    The coefficient of variation of directional heritability\index[subject]{heritability} depends on
    two factors: the dispersion of G* eigenvalue\index[subject]{eigenvalue}s and the number of traits.
    Higher eigenvalue\index[subject]{eigenvalue} dispersion increases CV; more traits decrease CV
    through averaging.
  }
  \label{fig:cv-formula}
\end{figure}

\section[Relative eigenvalue\index[subject]{eigenvalue} variance]{Relative eigenvalue variance: \texorpdfstring{$V_{\text{rel}}(\mat{G}^*)$}{Vrel(G*)}}

A key quantity is the relative variance of the eigenvalue\index[subject]{eigenvalue}s:
\[
  V_{\text{rel}}(\mat{G}^*) = \frac{\Var(\lambda^*)}{\bar{\lambda}^{*2}}
    = \text{CV}(\lambda^*)^2.
\]

This dimensionless quantity measures the ``eccentricity'' of the G* ellipsoid.
When $V_{\text{rel}} = 0$, all eigenvalue\index[subject]{eigenvalue}s are equal (G* is spherical, no
constraint). As $V_{\text{rel}}$ increases, the ellipsoid becomes more
elongated and constraint heterogeneity increases.

The formula for $\text{CV}(h^2)$ becomes:
\[
  \text{CV}(h^2) = \sqrt{\frac{2}{p+2} \times V_{\text{rel}}(\mat{G}^*)}.
\]

This elegant relationship connects three quantities:
\begin{itemize}
  \item $\text{CV}(h^2)$: observable variation in heritability\index[subject]{heritability} across
        directions.
  \item $V_{\text{rel}}(\mat{G}^*)$: geometric property of the G-P
        relationship.
  \item $p$: the number of traits.
\end{itemize}

\begin{keyidea}
The formula $\text{CV}^2(h^2) = \frac{2}{p+2} \times V_{\text{rel}}(\mat{G}^*)$
connects the distributional property (heritability\index[subject]{heritability} variation) to the
geometric property (G* eccentricity). This is the bridge between theory
and empirical observation.
\end{keyidea}

\section{Constraint traps revisited}

In Chapter~21 we introduced constraint traps: directions where phenotypic
variance is normal but heritability\index[subject]{heritability} is low. Now we can quantify how severe
these traps are.

The minimum directional heritability\index[subject]{heritability} is $\lambda_p^*$, the smallest
eigenvalue\index[subject]{eigenvalue} of $\mat{G}^*$. If $\lambda_p^*$ is much smaller than the mean
$\bar{\lambda}^*$, there exist directions where selection will be
ineffective despite observable phenotypic variation.

Define the \textbf{constraint severity} as:
\[
  \text{Severity} = 1 - \frac{\lambda_p^*}{\bar{\lambda}^*}.
\]

This measures how much worse the worst direction is compared to the average.
Severity of 0 means all directions are equally heritable; severity
approaching 1 means the worst direction has near-zero heritability\index[subject]{heritability}.

\begin{figure}[ht]
  \centering
  \includegraphics[width=0.8\textwidth]{fig41_constraint_severity.png}
  \caption[Constraint severity]{
    Constraint severity measures how much worse the minimum heritability\index[subject]{heritability}
    is compared to the mean. High severity indicates the existence of
    severe constraint traps.
  }
  \label{fig:constraint-severity}
\end{figure}

\section{Selection strength and constraint risk}

Whether a constraint trap matters biologically depends on selection strength.
Consider a selection gradient $\boldsymbol{\beta}$ with magnitude
$\|\boldsymbol{\beta}\|$.

The expected response to selection is:
\[
  \text{E}[\Delta\bar{z}] \approx \bar{\lambda}^* \times \|\boldsymbol{\beta}\|
\]
for random direction (roughly speaking).

But the variance of response depends on heritability\index[subject]{heritability} variance:
\[
  \Var[\Delta\bar{z}] \propto \Var(h^2) \times \|\boldsymbol{\beta}\|^2.
\]

When selection is strong ($\|\boldsymbol{\beta}\|$ large), even modest
$\text{CV}(h^2)$ produces large variation in response. Some selection
directions produce robust response; others produce disappointing results.

This interaction matters for:
\begin{itemize}
  \item \textbf{Breeding programs:} Strong artificial selection may hit
        constraint traps.
  \item \textbf{Conservation:} Populations facing strong environmental
        change may be unable to respond if change targets constrained
        directions.
  \item \textbf{Evolutionary prediction:} Predicting long-term response
        requires knowing not just mean heritability\index[subject]{heritability} but its directional
        distribution.
\end{itemize}

\section{Empirical estimation}

To apply these concepts, we need to estimate G and P from data and compute
$\mat{G}^*$ and its eigenvalue\index[subject]{eigenvalue}s.

\subsection*{Estimation pipeline}

\begin{enumerate}
  \item \textbf{Estimate G and P.} Use standard quantitative genetics
        methods: half-sib designs, parent-offspring regression, animal
        models, or genomic approaches.
  \item \textbf{Check positive definite\index[subject]{matrix!positive definite}ness.} Ensure both G and P have all
        positive eigenvalue\index[subject]{eigenvalue}s. Sampling error can produce negative eigenvalue\index[subject]{eigenvalue}s
        in G, especially for the smallest values.
  \item \textbf{Compute P-whitening matrix.} Eigendecompose P and construct
        $\mat{P}^{-1/2}$.
  \item \textbf{Compute G*.} Form
        $\mat{G}^* = \mat{P}^{-1/2}\mat{G}\mat{P}^{-1/2}$.
  \item \textbf{Eigendecompose G*.} The eigenvalue\index[subject]{eigenvalue}s are directional
        heritabilities; eigenvector\index[subject]{eigenvector}s are the principal constraint axes.
  \item \textbf{Compute summary statistics.} Mean, variance, CV of
        eigenvalue\index[subject]{eigenvalue}s; constraint severity; $V_{\text{rel}}$.
\end{enumerate}

\subsection*{R implementation}

\begin{verbatim}
compute_h2_distribution <- function(G, P) {
  # Check positive definite\index[subject]{matrix!positive definite}ness
  if (any(eigen(G)$values <= 0)) 
    warning("G has non-positive eigenvalues")
  if (any(eigen(P)$values <= 0)) 
    stop("P must be positive definite\index[subject]{matrix!positive definite}")
  
  # P-whitening
  eig_P <- eigen(P)
  P_inv_sqrt <- eig_P$vectors %*% 
                diag(1/sqrt(eig_P$values)) %*% 
                t(eig_P$vectors)
  
  # Compute G*
  G_star <- P_inv_sqrt %*% G %*% P_inv_sqrt
  
  # Eigendecompose G*
  eig_Gstar <- eigen(G_star)
  lambda_star <- eig_Gstar$values
  
  # Summary statistics
  p <- nrow(G)
  mean_h2 <- mean(lambda_star)
  var_lambda <- var(lambda_star)
  cv_lambda <- sqrt(var_lambda) / mean_h2
  V_rel <- var_lambda / mean_h2^2
  cv_h2 <- sqrt(2/(p+2)) * cv_lambda
  
  h2_min <- min(lambda_star)
  severity <- 1 - h2_min / mean_h2
  
  list(
    lambda_star = lambda_star,
    mean_h2 = mean_h2,
    cv_h2 = cv_h2,
    V_rel = V_rel,
    h2_min = h2_min,
    h2_max = max(lambda_star),
    constraint_severity = severity,
    eigenvectors = eig_Gstar$vectors
  )
}
\end{verbatim}

\section{Case study: empirical G-P analysis}

Let us apply these methods to a real example. Consider a dataset of G and P
matrices from a published study of floral traits in a plant species.

\[
  \mat{G} = \begin{pmatrix}
    0.42 & 0.28 & 0.15 \\
    0.28 & 0.38 & 0.22 \\
    0.15 & 0.22 & 0.31
  \end{pmatrix},
  \qquad
  \mat{P} = \begin{pmatrix}
    0.85 & 0.35 & 0.20 \\
    0.35 & 0.72 & 0.28 \\
    0.20 & 0.28 & 0.55
  \end{pmatrix}.
\]

\subsection*{Results}

Eigenvalues of $\mat{G}^*$: $(0.71, 0.52, 0.35)$.

Summary statistics:
\begin{center}
\begin{tabular}{lc}
\hline
Statistic & Value \\
\hline
Mean $h^2$ & 0.53 \\
Max $h^2$ & 0.71 \\
Min $h^2$ & 0.35 \\
$\text{CV}(h^2)$ & 0.24 \\
$V_{\text{rel}}(\mat{G}^*)$ & 0.15 \\
Constraint severity & 0.34 \\
\hline
\end{tabular}
\end{center}

\subsection*{Interpretation}

The mean heritability\index[subject]{heritability} is moderate (0.53), but it ranges from 0.35 to 0.71
depending on direction. The CV of 24\% indicates substantial heterogeneity.
The constraint severity of 0.34 means the worst direction has heritability\index[subject]{heritability}
34\% below the average.

The eigenvector\index[subject]{eigenvector} for minimum heritability\index[subject]{heritability} reveals which trait combination
faces the strongest constraint. If this direction corresponds to a
biologically important target (e.g., a pollinator preference axis), the
breeding implications are significant.

\section{Implications for breeding and evolution}

\subsection*{Breeding programs}

Traditional breeding often focuses on individual-trait heritabilities. Our
framework reveals that:

\begin{enumerate}
  \item The direction of selection matters as much as its strength.
  \item Index selection that combines traits should be evaluated for
        directional heritability\index[subject]{heritability}, not just component heritabilities.
  \item Constraint traps can be identified in advance and avoided or
        addressed through introgression of new genetic variation.
\end{enumerate}

\subsection*{Evolutionary biology}

For understanding evolution in natural populations:

\begin{enumerate}
  \item The rate of adaptation depends on whether selection aligns with
        high-heritability\index[subject]{heritability} directions.
  \item Long-term evolutionary trajectories may be biased toward
        $\mathbf{g}_{\max}$ or its P-standardised equivalent.
  \item Phenotypic divergence among populations may reflect the geometry
        of G* as much as the direction of selection.
\end{enumerate}

\subsection*{Conservation}

For populations facing environmental change:

\begin{enumerate}
  \item The capacity to adapt depends on whether required trait changes
        align with heritable directions.
  \item Populations with high $\text{CV}(h^2)$ are more vulnerable---they
        may have adequate mean heritability\index[subject]{heritability} but face traps in critical
        directions.
  \item Assisted gene flow could be targeted to increase heritability\index[subject]{heritability} in
        constrained directions.
\end{enumerate}

\section{Extensions and open questions}

The framework presented here opens several research directions:

\begin{itemize}
  \item \textbf{Non-Gaussian distributions:} The formula
        $\text{CV}^2(h^2) = \frac{2}{p+2} V_{\text{rel}}$ assumes uniform
        sampling on the P-sphere\index[subject]{P-sphere}. For specific selection scenarios, the
        distribution may differ.
  
  \item \textbf{Estimation uncertainty:} G and P are estimated with error.
        How does this uncertainty propagate to $\text{CV}(h^2)$ and
        constraint severity?
  
  \item \textbf{Temporal stability:} Does the G-P geometry remain stable
        across generations, or does constraint heterogeneity itself evolve?
  
  \item \textbf{Environmental dependence:} Does P (and hence G*) change
        across environments? Are constraint traps consistent or
        context-dependent?
  
  \item \textbf{Genomic architecture:} Can we predict G* eigenstructure
        from knowledge of gene networks and pleiotropy?
\end{itemize}

\section{Summary}

In this chapter we have:

\begin{itemize}
  \item Derived the distribution of directional heritability\index[subject]{heritability} across random
        selection directions.
  \item Shown that $\text{CV}^2(h^2) = \frac{2}{p+2} V_{\text{rel}}(\mat{G}^*)$
        connects the distributional property to G* eigenstructure.
  \item Defined constraint severity as a measure of how bad the worst
        constraint trap is.
  \item Provided an estimation pipeline and R code for computing these
        quantities from G and P estimates.
  \item Illustrated the analysis with a case study.
  \item Discussed implications for breeding, evolutionary biology, and
        conservation.
  \item Outlined open research questions.
\end{itemize}

This chapter brings us full circle. We began with distance and the simple
question of why we square. We built up through covariance matrices,
Mahalanobis distance\index[subject]{distance!Mahalanobis} distance, eigendecomposition, and P-whitening\index[subject]{whitening transformation}. Now we can
answer sophisticated questions about evolutionary constraint: not just
``is this trait heritable?'' but ``how does heritability\index[subject]{heritability} vary across the
space of possible selection targets?''

The geometric perspective unifies it all. Matrices are shapes; eigenvalues
measure extent; eigenvector\index[subject]{eigenvector}s define axes. With these tools, the complexity
of multivariate evolution becomes tractable---not simple, but navigable.

The ellipse has been our guide throughout. May it serve you well in your
own research.

\backmatter

\chapter{Epilogue: The Shape of Things}


We began with a point in a paddock and a question about distance. We end
with ellipsoids in high-dimensional space and a framework for understanding
evolutionary constraint. The journey has been long, but the core idea has
remained simple: \emph{symmetric matrices describe shapes}.

This is worth dwelling on. A matrix is just a table of numbers---rows and
columns, entries that can be added and multiplied. Yet when that matrix is
symmetric and positive definite\index[subject]{matrix!positive definite}, it becomes something more. It becomes a
geometry. The eigenvalue\index[subject]{eigenvalue}s measure how far the shape extends; the
eigenvector\index[subject]{eigenvector}s point along its natural axes. The determinant captures the
volume; the trace captures the total extent. Invert the matrix, and you
flip the geometry inside out---what was long becomes short, what was wide
becomes narrow.

Once you see matrices as shapes, the tools of multivariate statistics
become intuitive. PCA is not a mysterious algorithm; it is the simple act
of finding the axes of an ellipse. MANOVA\index[subject]{MANOVA} is not an arbitrary test
statistic; it is a comparison of two shapes---one describing variation
among groups, the other describing variation within. The Mahalanobis distance\index[subject]{distance!Mahalanobis}
distance is not a formula to memorise; it is Euclidean distance\index[subject]{distance!Euclidean} distance measured
after reshaping the space to match the data.

And the G matrix---that central object of evolutionary quantitative
genetics---is not merely a summary of breeding values. It is the shape of
what evolution can do. Its long axes are the directions of easy change;
its short axes are the directions of constraint. When we ask whether a
population can respond to selection, we are asking whether the selection
gradient points along a long axis or a short one. When we compare G
matrices across species, we are comparing the shapes of evolutionary
possibility.

\section*{What geometry gives us}

The geometric perspective offers three gifts.

The first is \textbf{intuition}. Equations can be opaque; shapes are
visible. When you read that the response to selection is
$\Delta\bar{\vect{z}} = \mat{G}\boldsymbol{\beta}$, you might see symbols.
But when you visualise G as an ellipse and $\boldsymbol{\beta}$ as an
arrow, you see immediately that the response will be deflected toward
the long axis. The algebra confirms what the picture reveals.

The second gift is \textbf{unification}. The methods scattered across
textbooks---PCA\index[subject]{PCA}, discriminant analysis, canonical correlation, MANOVA\index[subject]{MANOVA},
Mahalanobis distance\index[subject]{distance!Mahalanobis} distance, the breeder's equation\index[subject]{breeder's equation} equation---are not separate techniques
requiring separate intuitions. They are all eigendecompositions, all ways
of finding the natural axes of some ellipsoid. Learn the geometry once,
and you understand them all.

The third gift is \textbf{new questions}. Once you see G as a shape, you
can ask: how does this shape vary across environments? How does it change
over evolutionary time? How does it compare to the shape of the fitness
surface? These questions were always implicit in the formalism, but the
geometric perspective makes them vivid. It turns parameter estimation into
shape description, and shape description invites comparison.

\section*{The limits of ellipses}

I have told a particular story in these notes---one centred on symmetric
matrices and their ellipsoids. This is not the only story that could be
told.

Ellipses assume linearity. The covariance matrix\index[subject]{covariance matrix} captures only the linear
relationships among traits; it misses curvature, thresholds, and
interactions. When the true relationships are nonlinear, the ellipse is an
approximation---sometimes good, sometimes misleading.

Ellipses assume multivariate normality, or at least that the second moments
are sufficient statistics. For heavy-tailed distributions or multimodal
populations, the covariance matrix\index[subject]{covariance matrix} tells only part of the story.

And ellipses assume stationarity. The G matrix is not fixed; it evolves.
Mutation introduces new variation, selection erodes it, drift reshapes it.
The geometry we estimate today may not be the geometry of tomorrow. The
ellipse is a snapshot, not a law.

These limitations are real, but they do not diminish the value of the
geometric perspective. They define its scope. Linearity is often a good
first approximation. Normality is often reasonable for quantitative traits.
And stationarity, while imperfect, is often sufficient for the timescales
we study. The ellipse is a model, and like all models, it is useful
precisely because it is simpler than reality.

\section*{The view from here}

If you have followed these notes from the beginning, you now possess a
way of seeing. When you encounter a covariance matrix\index[subject]{covariance matrix}, you see an ellipse.
When you read an eigenvalue\index[subject]{eigenvalue}, you see an axis length. When you compute a
Mahalanobis distance\index[subject]{distance!Mahalanobis} distance, you see a rescaling of space. This way of seeing is
not a trick or a shortcut; it is the thing itself. The geometry is not a
metaphor for the mathematics---it is the mathematics, visualised.

This perspective will serve you in many contexts. In quantitative genetics,
it illuminates the breeder's equation\index[subject]{breeder's equation} equation and the geometry of constraint. In
multivariate statistics, it unifies PCA\index[subject]{PCA}, MANOVA\index[subject]{MANOVA}, and discriminant analysis\index[subject]{discriminant analysis}.
In machine learning, it underlies principal component regression,
regularisation, and the geometry of high-dimensional data. In physics, it
connects to quadratic form\index[subject]{quadratic form}s, moment of inertia tensors, and the geometry
of configuration spaces.

The same ideas, the same shapes, appearing across fields. This is not
coincidence. It is the deep structure of linear algebra asserting itself
wherever quantities vary together.

\section*{An invitation}

These notes are an invitation, not an endpoint. I have shown you how to
see matrices as shapes; I have not shown you everything those shapes can
reveal.

There are questions I have not addressed. How do we estimate G matrices
reliably when sample sizes are small? How do we test whether two G matrices
differ? How do we model the evolution of G itself? How do we connect the
geometry of genetic variation to the molecular mechanisms that generate it?

These are research questions, not textbook exercises. They are the frontier.
And if these notes have done their job, you are now equipped to approach
that frontier with geometric intuition as your guide.

\section*{A final image}

Picture a high-dimensional ellipsoid---the G matrix of some population,
suspended in the space of all possible phenotypes. Its axes point in
directions we cannot visualise directly, but we can measure their lengths.
Some are long: abundant genetic variation, easy evolutionary change. Some
are short: scarce variation, constrained response.

Now imagine a selection gradient---an arrow pointing toward some optimum,
some direction that would increase fitness. The arrow may point anywhere.
It may align with a long axis, and evolution will be swift. It may point
toward a short axis, and evolution will be slow, frustrated, deflected.

The shape of the ellipsoid and the direction of the arrow---these two
things, together, determine what will happen. The G matrix is potential;
selection is actuality. Their interaction is evolution.

This is the image I hope you carry forward. Not a formula to be memorised,
but a shape to be seen. Not a technique to be applied, but a geometry to
be understood.

The ellipse has been our guide. May it serve you well.

\vspace{2em}
\hfill\emph{Daniel Ortiz-Barrientos}

\hfill\emph{Brisbane, 2025}
\chapter{References}

\begin{description}

\item[Blows, M. W. (2007)]
A tale of two matrices: multivariate approaches in evolutionary biology.
\emph{Journal of Evolutionary Biology}, 20(1), 1--8.

\item[Lande, R. (1979)]
Quantitative genetic analysis of multivariate evolution, applied to
brain:body size allometry.
\emph{Evolution}, 33(1), 402--416.

\item[Lande, R., \& Arnold, S. J. (1983)]
The measurement of selection on correlated characters.
\emph{Evolution}, 37(6), 1210--1226.

\item[Lynch, M., \& Walsh, B. (1998)]
\emph{Genetics and Analysis of Quantitative Traits}.
Sinauer Associates.

\item[Phillips, P. C., \& Arnold, S. J. (1989)]
Visualizing multivariate selection.
\emph{Evolution}, 43(6), 1209--1222.

\item[Schluter, D. (1996)]
Adaptive radiation along genetic lines of least resistance.
\emph{Evolution}, 50(5), 1766--1774.

\item[Strang, G. (2016)]
\emph{Introduction to Linear Algebra} (5th ed.).
Wellesley-Cambridge Press.

\item[Walsh, B., \& Lynch, M. (2018)]
\emph{Evolution and Selection of Quantitative Traits}.
Oxford University Press.

\end{description}
\chapter{Appendix: Mathematical and Statistical Background}

This appendix collects the minimum linear algebra and probability background
assumed in the main text. It is not a complete course. Instead, it provides
a compact reference for four key ingredients:

\begin{itemize}
  \item vectors and matrices, and how to multiply them;
  \item eigenvalues and eigenvectors;
  \item basic probability notions: variance, covariance, and the multivariate normal;
  \item selection gradients and Lande's equation.
\end{itemize}

Readers who have seen these topics before can skim or skip this appendix.
Readers for whom these ideas are new may find it helpful to read this
appendix alongside Chapters~0--2 and Chapters~10--12.

\section{Vectors and matrices}

\subsection*{Vectors}

A \emph{vector} is an ordered list of numbers that we usually picture as
an arrow in a trait space. For $p$ traits, we write a phenotype as a
column vector
\[
  \vect{z} =
  \begin{pmatrix}
    z_1 \\ z_2 \\ \vdots \\ z_p
  \end{pmatrix}.
\]

Each entry $z_i$ is the value of trait $i$. We use bold letters for
vectors (\vect{z}, \vect{x}, \vect{u}) and reserve plain italics
($z_i$, $x_i$) for individual components.

The \emph{length} (Euclidean norm) of a vector is
\[
  \|\vect{z}\| = \sqrt{z_1^2 + z_2^2 + \cdots + z_p^2}.
\]

The \emph{dot product} (inner product) of two vectors $\vect{x}$ and
$\vect{y}$ is
\[
  \langle \vect{x}, \vect{y} \rangle
    = \vect{x}^\top \vect{y}
    = x_1 y_1 + x_2 y_2 + \cdots + x_p y_p.
\]

Geometrically, this dot product relates to the angle $\theta$ between
$\vect{x}$ and \vect{y} via
\[
  \langle \vect{x}, \vect{y} \rangle
    = \|\vect{x}\| \, \|\vect{y}\| \cos \theta.
\]

\subsection*{Matrices as linear maps}

A \emph{matrix} is a rectangular array of numbers. An $m \times n$ matrix
has $m$ rows and $n$ columns. We write matrices in bold capitals;
for example,
\[
  \mat{A} =
  \begin{pmatrix}
    a_{11} & a_{12} & \cdots & a_{1n} \\
    a_{21} & a_{22} & \cdots & a_{2n} \\
    \vdots & \vdots & \ddots & \vdots \\
    a_{m1} & a_{m2} & \cdots & a_{mn}
  \end{pmatrix}.
\]

In this book, matrices are always used as \emph{linear maps}: machines
that take a vector in and return a vector out. If $\mat{A}$ is $m \times n$
and $\vect{x}$ is an $n \times 1$ column vector, the product
$\vect{y} = \mat{A}\vect{x}$ is an $m \times 1$ column vector whose
entries are
\[
  y_i = \sum_{j=1}^n a_{ij} x_j, \qquad i = 1,\dots,m.
\]

So each entry $y_i$ is a dot product between the $i$th row of $\mat{A}$
and $\vect{x}$.

\subsection*{Linear combinations view}

There is another useful way to see the matrix--vector product. Let the
columns of $\mat{A}$ be denoted by $\vect{a}_1,\dots,\vect{a}_n$. Then
\[
  \mat{A}\vect{x}
    = x_1 \vect{a}_1 + x_2 \vect{a}_2 + \cdots + x_n \vect{a}_n.
\]

So $\mat{A}\vect{x}$ is a \emph{linear combination} of the columns of
$\mat{A}$, with coefficients $x_1,\dots,x_n$. This viewpoint is important
when we interpret covariance matrices as sets of basis vectors that span
trait space.

\subsection*{A simple example}

Consider
\[
  \mat{A} =
  \begin{pmatrix}
    2 & 1 \\
    1 & 3
  \end{pmatrix},
  \qquad
  \vect{x} =
  \begin{pmatrix}
    1 \\ 2
  \end{pmatrix}.
\]

Then
\[
  \mat{A}\vect{x}
    = 
    \begin{pmatrix}
      2 \cdot 1 + 1 \cdot 2 \\
      1 \cdot 1 + 3 \cdot 2
    \end{pmatrix}
    =
    \begin{pmatrix}
      4 \\ 7
    \end{pmatrix}.
\]

Geometrically, the matrix $\mat{A}$ stretches and shears the plane.
The vector $(1,2)^\top$ is moved to $(4,7)^\top$.

\subsection*{Matrix--matrix multiplication}

If $\mat{A}$ is $m \times n$ and $\mat{B}$ is $n \times k$, we can form
the product $\mat{C} = \mat{A}\mat{B}$, which is $m \times k$. By
definition, the $(i,\ell)$ entry of $\mat{C}$ is
\[
  c_{i\ell} = \sum_{j=1}^n a_{ij} b_{j\ell}.
\]

Geometrically, applying $\mat{B}$ and then $\mat{A}$ to a vector
$\vect{x}$ gives
\[
  \mat{A}(\mat{B}\vect{x}) = (\mat{A}\mat{B})\vect{x}.
\]

Matrix multiplication corresponds to \emph{composition} of linear maps:
do $\mat{B}$ first, then $\mat{A}$. In general, we have
$\mat{A}\mat{B} \ne \mat{B}\mat{A}$; matrix multiplication is not
commutative.

\section{eigenvalue\index[subject]{eigenvalue}s and eigenvector\index[subject]{eigenvector}s}

\subsection*{Definition}

Let $\mat{A}$ be a square $p \times p$ matrix. A non-zero vector
$\vect{v}$ is an \emph{eigenvector\index[subject]{eigenvector}} of $\mat{A}$ with eigenvalue\index[subject]{eigenvalue}
$\lambda$ if
\[
  \mat{A}\vect{v} = \lambda \vect{v}.
\]

This means that when $\mat{A}$ acts on $\vect{v}$, it does not change
its direction, only its length. The factor $\lambda$ is the stretch
(or compression) factor along the direction $\vect{v}$.

\subsection*{Symmetric matrices}

In this book we work almost exclusively with symmetric matrices:
$\mat{A} = \mat{A}^\top$. Covariance matrices, the $\mat{G}$ and $\mat{P}$
matrices, and the quadratic selection matrix $\boldsymbol{\gamma}$ are
all symmetric by construction.

Symmetric matrices have two important properties:

\begin{itemize}
  \item All eigenvalue\index[subject]{eigenvalue}s are real.
  \item There exists an orthonormal basis of eigenvector\index[subject]{eigenvector}s.
\end{itemize}

This means we can write a symmetric matrix as
\[
  \mat{A} = \mat{Q} {\Lambda} \mat{Q}^\top,
\]
where $\mat{Q}$ is an orthogonal matrix whose columns are unit eigenvector\index[subject]{eigenvector}s
of $\mat{A}$, and ${\Lambda}$ is a diagonal matrix whose entries are
the corresponding eigenvalue\index[subject]{eigenvalue}s.

This decomposition is called an \emph{eigendecomposition} or
\emph{spectral decomposition}.

\subsection*{A two-dimensional example}

Consider
\[
  \mat{A} =
  \begin{pmatrix}
    3 & 1 \\
    1 & 3
  \end{pmatrix}.
\]

We look for $\lambda$ and non-zero $(v_1,v_2)^\top$ such that
$\mat{A}\vect{v} = \lambda\vect{v}$. The characteristic equation\index[subject]{characteristic equation} is
\[
  \det(\mat{A} - \lambda \mat{I})
    = \det
      \begin{pmatrix}
        3 - \lambda & 1 \\
        1 & 3 - \lambda
      \end{pmatrix}
    = (3-\lambda)^2 - 1 = 0.
\]

So $(3-\lambda)^2 = 1$ and therefore $\lambda = 4$ or $\lambda = 2$.

For $\lambda=4$, we solve
\[
  \begin{pmatrix}
    -1 & 1 \\
    1 & -1
  \end{pmatrix}
  \begin{pmatrix}
    v_1 \\ v_2
  \end{pmatrix}
  = \vect{0},
\]
which yields $v_1 = v_2$. A normalised eigenvector\index[subject]{eigenvector} is
\[
  \vect{v}_1 = \frac{1}{\sqrt{2}}
  \begin{pmatrix}
    1 \\ 1
  \end{pmatrix}.
\]

For $\lambda=2$, we solve
\[
  \begin{pmatrix}
    1 & 1 \\
    1 & 1
  \end{pmatrix}
  \begin{pmatrix}
    v_1 \\ v_2
  \end{pmatrix}
  = \vect{0},
\]
which yields $v_1 = -v_2$. A normalised eigenvector\index[subject]{eigenvector} is
\[
  \vect{v}_2 = \frac{1}{\sqrt{2}}
  \begin{pmatrix}
    1 \\ -1
  \end{pmatrix}.
\]

Geometrically, $\vect{v}_1$ points along the line $x=y$ and is stretched by
a factor of $4$, while $\vect{v}_2$ points along $x=-y$ and is stretched
by a factor of $2$. Any vector in the plane can be written as a
combination of these two directions.

\subsection*{Eigenstructure of covariance matrices}

A covariance matrix ${\Sigma}$ is symmetric and positive definite\index[subject]{matrix!positive definite}
(all eigenvalue\index[subject]{eigenvalue}\index[subject]{eigenvalue}s positive). Its eigendecomposition
\[
  {\Sigma} = \mat{Q}{\Lambda}\mat{Q}^\top
\]
has a clear geometric interpretation:

\begin{itemize}
  \item the columns of $\mat{Q}$ give the principal axes (directions)
        of variation;
  \item the eigenvalue\index[subject]{eigenvalue}s on the diagonal of ${\Lambda}$ give the
        variances along those axes.
\end{itemize}

In two dimensions, the contours of a multivariate normal distribution
with covariance matrix\index[subject]{covariance matrix} ${\Sigma}$ are ellipses whose axes are aligned
with the eigenvector\index[subject]{eigenvector}s of ${\Sigma}$ and whose axis lengths are
proportional to $\sqrt{\lambda_1}$ and $\sqrt{\lambda_2}$.

The same idea extends to the genetic covariance matrix\index[subject]{covariance matrix} $\mat{G}$ and
the phenotypic covariance matrix\index[subject]{covariance matrix} $\mat{P}$.

\section{Basic probability and covariance}

\subsection*{Random variables and expectation}

A \emph{random variable} $X$ is a numerical quantity that takes different
values with certain probabilities. We write its \emph{expectation}
(mean) as $\E[X]$. For a discrete variable taking values $x_i$ with
probabilities $p_i$,
\[
  \E[X] = \sum_i x_i p_i.
\]

For a continuous variable with density $f(x)$,
\[
  \E[X] = \int_{-\infty}^{\infty} x f(x)\,dx.
\]

In practice, we often estimate the mean from data $x_1,\dots,x_n$ as
the sample mean
\[
  \bar{x} = \frac{1}{n} \sum_{i=1}^n x_i.
\]

\subsection*{Variance and covariance}

The \emph{variance} of $X$ measures spread:
\[
  \Var(X) = \E[(X - \E[X])^2].
\]

The \emph{covariance} between two random variables $X$ and $Y$ is
\[
  \Cov(X,Y) = \E[(X - \E[X])(Y - \E[Y])].
\]

Covariance is positive when large values of $X$ tend to occur with large
values of $Y$, negative when large values of $X$ tend to occur with small
values of $Y$, and near zero when there is no consistent linear relation.

From data pairs $(x_i,y_i)$, we estimate covariance as
\[
  \widehat{\Cov}(X,Y)
    = \frac{1}{n-1} \sum_{i=1}^n (x_i - \bar{x})(y_i - \bar{y}).
\]

\subsection*{Covariance matrices}

For a random vector $\vect{Z} = (Z_1,\dots,Z_p)^\top$, the covariance
matrix ${\Sigma}$ is the $p \times p$ matrix with entries
\[
  \Sigma_{ij} = \Cov(Z_i,Z_j).
\]

So
\[
  {\Sigma} =
  \begin{pmatrix}
    \Var(Z_1) & \Cov(Z_1,Z_2) & \cdots & \Cov(Z_1,Z_p) \\
    \Cov(Z_2,Z_1) & \Var(Z_2) & \cdots & \Cov(Z_2,Z_p) \\
    \vdots & \vdots & \ddots & \vdots \\
    \Cov(Z_p,Z_1) & \Cov(Z_p,Z_2) & \cdots & \Var(Z_p)
  \end{pmatrix}.
\]

The diagonal entries record variances of individual traits; the
off-diagonals record covariances between traits. The matrices
$\mat{P}$ and $\mat{G}$ used throughout the book are examples of
covariance matrices.

\subsection*{The multivariate normal distribution}

A $p$-dimensional random vector $\vect{Z}$ is said to have a
multivariate normal distribution with mean vector ${\mu}$ and
covariance matrix\index[subject]{covariance matrix} ${\Sigma}$, written
$\vect{Z} \sim \mathcal{N}({\mu},{\Sigma})$, if its density is
\[
  f({\textbf{z}})
  = \frac{1}{(2\pi)^{p/2} \sqrt{\det({\Sigma})}}
    \exp\left(
      -\frac{1}{2}
      ({\textbf{z}} - {\mu})^\top
      {\Sigma}^{-1}
      ({\textbf{z}} - {\mu})
    \right).
\]

The quantity

  $$({\textbf{z}} - {\mu})^\top
  {\Sigma}^{-1}
  ({\textbf{z}} - {\mu})$$
  
is the squared Mahalanobis distance\index[subject]{distance!Mahalanobis} distance from ${\textbf{z}}$ to the mean ${\mu}$. Contours of equal density are ellipsoids defined by constant Mahalanobis distance\index[subject]{distance!Mahalanobis} distance.

The multivariate normal plays a central role in quantitative genetics because sums of many small, independent genetic and environmental effects tend to generate approximately normal trait distributions.

\section{Selection gradients and Lande's equation}

\subsection*{Phenotypes and fitness}

Let $\vect{z}$ be a vector of traits for an individual, and let $w$ be
its absolute fitness (for example, number of surviving offspring).

We define \emph{relative fitness} as
\[
  \tilde{w} = \frac{w}{\bar{w}},
\]
where $\bar{w}$ is the mean fitness in the population. Relative fitness
has mean 1. Using relative fitness simplifies many expressions.

\subsection*{Selection differentials and gradients}

The \emph{selection differential} for trait $i$ is the difference between
the mean of trait $i$ after selection and the mean before selection.
For a vector of traits, the vector of selection differentials is
\[
  \vect{s} = \Cov(\vect{z}, \tilde{w}),
\]
where the covariance is taken component-wise:
$s_i = \Cov(z_i,\tilde{w})$.

The \emph{directional selection gradient} $\boldsymbol{\beta}$ is defined
as the vector of partial regression coefficients of relative fitness on
traits. In matrix notation, if $\mat{P}$ is the phenotypic covariance
matrix of $\vect{z}$, then
\[
  \boldsymbol{\beta} = \mat{P}^{-1} \vect{s}.
\]

Equivalently, $\boldsymbol{\beta}$ is the vector that minimises the mean
squared error in the linear approximation
\[
  \tilde{w} \approx \alpha + \boldsymbol{\beta}^\top \vect{z}.
\]

The elements $\beta_i$ measure the strength and direction of linear
selection on trait $i$, holding the other traits constant.

\subsection*{The multivariate breeder's equation\index[subject]{breeder's equation} equation}

Let $\mat{G}$ be the additive genetic covariance matrix\index[subject]{covariance matrix} for the traits
$\vect{z}$. Under standard assumptions (additive gene action, weak
selection, random mating, no environmental change across generations),
the multivariate breeder's equation\index[subject]{breeder's equation} equation is
\[
  \Delta \bar{\vect{z}} = \mat{G}\mat{P}^{-1} \vect{s},
\]
where $\Delta \bar{\vect{z}}$ is the expected change in the mean trait
vector from one generation to the next.

Substituting $\vect{s} = \mat{P}\boldsymbol{\beta}$ gives
\[
  \Delta \bar{\vect{z}}
    = \mat{G} \mat{P}^{-1} \mat{P} \boldsymbol{\beta}
    = \mat{G} \boldsymbol{\beta}.
\]

This is \emph{Lande's equation}.

\begin{keyidea}
Lande's equation
\[
  \Delta \bar{\vect{z}} = \mat{G}\boldsymbol{\beta}
\]
states that the expected evolutionary response vector is obtained by
multiplying the directional selection gradient $\boldsymbol{\beta}$ by
the additive genetic covariance matrix\index[subject]{covariance matrix} $\mat{G}$. The pattern and amount
of genetic variance (encoded in $\mat{G}$) filter and redirect the effect
of selection.
\end{keyidea}

In the main chapters we interpret this equation geometrically: $\vect{\beta}$
is a direction in trait space describing how fitness changes; $\mat{G}$
rescales and rotates this direction according to the available additive
variance, yielding the response vector $\Delta \bar{\vect{z}}$.

\subsection*{Quadratic selection}

When selection is not purely directional, we can expand relative fitness
to second order in the traits:
\[
  \tilde{w}
    \approx \alpha
          + \boldsymbol{\beta}^\top \vect{z}
          + \frac{1}{2}
            \vect{z}^\top \boldsymbol{\gamma} \vect{z},
\]
where $\boldsymbol{\gamma}$ is a symmetric matrix of quadratic selection
gradients. The diagonal entries of $\boldsymbol{\gamma}$ describe
stabilising or disruptive selection on individual traits, and the
off-diagonal entries describe correlational selection between traits.

The matrix $\boldsymbol{\gamma}$ defines the local curvature of the
fitness surface near the mean. In later chapters we combine $\mat{G}$
and $\boldsymbol{\gamma}$ to study how genetic variance interacts with
this curvature to shape evolutionary trajectories and constraints.

\section*{Further reading}

For more detailed treatments of these topics, see standard texts in
linear algebra (for example, Strang's \emph{Introduction to Linear
Algebra}) and in quantitative genetics (for example, Walsh and Lynch's
\emph{Evolution and Selection of Quantitative Traits}). The aim of this
appendix is not to replace such texts, but to provide a compact yet formal reminder of the concepts and notation used throughout this book.

%%%%%%%%%%%%%%%%%%%%%%%%%%%%%%%%%%%%%%%%%%%%%%%%%%%%%%%%%%%%%%%%%%%%%%%%%%%%%%%
%%%%%%%%%%%%%%%%%%%%%%%%%%%%%%%%%%%%%%%%%%%%%%%%%%%%%%%%%%%%%%%%%%%%%%%%%%%%%%%
%
% APPENDIX: HINTS AND SELECTED SOLUTIONS
%
%%%%%%%%%%%%%%%%%%%%%%%%%%%%%%%%%%%%%%%%%%%%%%%%%%%%%%%%%%%%%%%%%%%%%%%%%%%%%%%
%%%%%%%%%%%%%%%%%%%%%%%%%%%%%%%%%%%%%%%%%%%%%%%%%%%%%%%%%%%%%%%%%%%%%%%%%%%%%%%

\chapter*{Hints and Selected Solutions}
\addcontentsline{toc}{chapter}{Hints and Selected Solutions}

This appendix provides hints and partial solutions for selected exercises.
Full solutions are not given---working through the problems yourself is
essential for developing intuition.

\section*{Chapter 0}

\paragraph{Exercise 0.1.}
The centroid is $(\bar{L}, \bar{W}) = (4.4, 2.3)$. After adding plant F,
the centroid shifts toward (6.0, 3.5); compute the new mean of all six
observations.

\paragraph{Exercise 0.4.}
Euclidean distance is $\sqrt{2^2 + 3^2} = \sqrt{13} \approx 3.6$ mm (if
both traits are in mm). This number has no direct biological interpretation;
it mixes millimetres of length with millimetres of width.

\section*{Chapter 1}

\paragraph{Exercise 1.1.}
(a) $\|(3,4)\| = 5$. (b) $\|(1,1,1)\| = \sqrt{3}$. (c) $\|(2,-2,1)\| = 3$.
(d) $\|(1,0,0,0,1)\| = \sqrt{2}$.

\paragraph{Exercise 1.3.}
(c) The dot product of $(1,2)$ and $(-2,1)$ is $1(-2) + 2(1) = 0$. A dot
product of zero means the vectors are orthogonal (perpendicular).

\paragraph{Exercise 1.4.}
(a) $\text{proj}_{(1,0)}(3,4) = (3, 0)$. This is the ``shadow'' of $(3,4)$
on the $x$-axis.

\section*{Chapter 2}

\paragraph{Exercise 2.1.}
$\mat{A}$ stretches vectors by factor 2 in the $x$-direction while leaving
the $y$-component unchanged. The unit circle becomes an ellipse with
semi-axes 2 (horizontal) and 1 (vertical).

\paragraph{Exercise 2.2.}
$\mat{R}$ rotates vectors by $90°$ counterclockwise. Applying $\mat{R}$
twice gives $\mat{R}^2 = -\mat{I}$, which is rotation by $180°$.

\paragraph{Exercise 2.4.}
$\mat{A}\mat{B} = \begin{pmatrix} 0 & 2 \\ 3 & 0 \end{pmatrix}$ and
$\mat{B}\mat{A} = \begin{pmatrix} 0 & 3 \\ 2 & 0 \end{pmatrix}$. They
are not equal: matrix multiplication is not commutative.

\section*{Chapter 10}

\paragraph{Exercise 10.2.}
Mean = 6. Deviations: $-4, -2, 0, 2, 4$. Squared deviations: $16, 4, 0, 4, 16$.
Variance = $(16+4+0+4+16)/5 = 8$ (population variance) or $40/4 = 10$
(sample variance with $n-1$).

\paragraph{Exercise 10.4.}
Sum of squared distances: at $c=2$: 12; at $c=3$: 8; at $c=4$: 12. The
minimum is at the mean, $c=3$.

\paragraph{Exercise 10.5.}
In cm: $d = \sqrt{(12-10)^2 + (2.5-2)^2} = \sqrt{4.25} \approx 2.06$. In
mm for wing: $d = \sqrt{(120-100)^2 + (2.5-2)^2} = \sqrt{400.25} \approx 20.0$.
The wing difference now dominates because it's measured in larger numbers.

\section*{Chapter 11}

\paragraph{Exercise 11.1.}
With height in metres: $d(A,B) \approx 2.00$, $d(A,C) \approx 20.0$. With
height in cm: $d(A,B) \approx 5.4$, $d(A,C) \approx 22.4$. Converting to
cm increases the height contribution, but since weight differences dominate
anyway, $d(A,C)$ changes less proportionally.

\paragraph{Exercise 11.4.}
Euclidean distance is appropriate when: (1) all traits are measured in
the same units with similar scales, and (2) traits are uncorrelated. Example
where it works: measurements all in mm on a single structure. Example where
it fails: mass (kg) and length (mm) of animals.

\section*{Chapter 12}

\paragraph{Exercise 12.1.}
$\bar{X} = 4$, $\bar{Y} = 6$. $\Var(X) = 2.5$, $\Var(Y) = 2.5$.
$\Cov(X,Y) = 2.5$. So
$\mat{S} = \begin{pmatrix} 2.5 & 2.5 \\ 2.5 & 2.5 \end{pmatrix}$.
(Note: this matrix is singular---the traits are perfectly correlated.)

\paragraph{Exercise 12.3.}
$\mat{S}^{-1} = \frac{1}{16}\begin{pmatrix} 5 & -2 \\ -2 & 4 \end{pmatrix}$.
$D^2 = (2, 1) \mat{S}^{-1} (2, 1)^\top = \frac{1}{16}(20 - 8 - 4 + 4) = 0.75$.
$D = \sqrt{0.75} \approx 0.87$. Euclidean distance is $\sqrt{5} \approx 2.24$.

\paragraph{Exercise 12.5.}
For a bivariate normal, about 39\% of observations lie within the $D=1$
ellipse (not 68\%). The univariate ``68\% within 1 SD'' rule does not
generalise directly because probability spreads across two dimensions.

\section*{Chapter 32}

\paragraph{Exercise 32.1.}
Eigenvalues: $\lambda_1 = 8$, $\lambda_2 = 2$. Eigenvectors:
$\vect{v}_1 = (1, 1)/\sqrt{2}$, $\vect{v}_2 = (1, -1)/\sqrt{2}$.
PC1 explains $8/10 = 80\%$ of variance. PC1 is ``both traits together''
(size); PC2 is ``traits in opposition'' (shape/contrast).

\paragraph{Exercise 32.3.}
Covariance PCA: PC1 will be dominated by Trait A (variance 100). After
standardising, all traits contribute more equally. Use covariance PCA when
traits are in the same units and you want variance differences to matter.
Use correlation PCA when traits are on different scales or you want to
weight them equally.

\paragraph{Exercise 32.6.}
Total variance = 10. PC1 explains 52\%, PC2 explains 21\%, PC3 explains 10\%.
Cumulative: 52\%, 73\%, 83\%. A common rule is to retain PCs until cumulative
variance exceeds 80\%, suggesting 3 PCs. The ``elbow'' in the scree plot
also suggests 3.

\printindex[subject]
\printindex[authors]

\end{document}
