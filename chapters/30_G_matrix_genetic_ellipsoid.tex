\chapter{The G Matrix and the Genetic Ellipsoid}

We now have a complete geometric toolkit: vectors, matrices, eigenvalue\index[subject]{eigenvalue}s,
the Mahalanobis distance\index[subject]{distance!Mahalanobis} distance, and the whitening transformation. In this chapter
we apply these tools to the central object of multivariate quantitative
genetics: the additive genetic covariance matrix\index[subject]{covariance matrix}, $\mat{G}$.

The G matrix is not merely a table of numbers. It is a shape---an ellipsoid
in trait space that determines how populations can and cannot evolve. By
understanding G geometrically, we gain insight into evolutionary constraint,
the ``line of least resistance,'' and why some trait combinations respond
readily to selection while others do not.

\section{What the G matrix represents}

The additive genetic covariance matrix\index[subject]{covariance matrix} $\mat{G}$ summarises the heritable
variation in a population. Its entries are:

\begin{itemize}
  \item \textbf{Diagonal entries} $G_{ii} = V_{A,i}$: the additive genetic
        variance of trait $i$.
  \item \textbf{Off-diagonal entries} $G_{ij} = \text{Cov}_A(z_i, z_j)$:
        the additive genetic covariance between traits $i$ and $j$.
\end{itemize}

These covariances arise from pleiotropy (single genes affecting multiple
traits) and linkage disequilibrium (non-random association of alleles at
different loci). When $G_{ij} > 0$, alleles that increase trait $i$ tend
also to increase trait $j$. When $G_{ij} < 0$, they tend to have opposite
effects.

\begin{keyidea}
The G matrix encodes the genetic architecture underlying multiple traits.
It determines which trait combinations can be easily assembled by selection
and which cannot.
\end{keyidea}

\section{The genetic ellipsoid}

Like any symmetric positive semi-definite\index[subject]{matrix!positive semi-definite} matrix, $\mat{G}$ defines an
ellipsoid. In two traits, this is an ellipse; in three traits, an
ellipsoid; in $p$ traits, a $p$-dimensional hyperellipsoid.

The ellipsoid has a concrete interpretation: it shows where genetic
variation extends in trait space. Directions along the long axes of the
ellipsoid have high genetic variance; directions along the short axes have
low genetic variance.

\begin{figure}[ht]
  \centering
  \includegraphics[width=0.8\textwidth]{fig30_genetic_ellipsoid.png}
  \caption[The genetic ellipsoid]{
    The G matrix defines an ellipsoid in trait space. The eigenvector\index[subject]{eigenvector}s
    point along the principal axes; the eigenvalue\index[subject]{eigenvalue}s are the genetic
    variances along those axes. The shape reveals which directions have
    abundant genetic variation and which are constrained.
  }
  \label{fig:genetic-ellipsoid}
\end{figure}

Formally, if we diagonalise $\mat{G} = \mat{V}\mat{\Lambda}\mat{V}^\top$:

\begin{itemize}
  \item The eigenvector\index[subject]{eigenvector}s $\vect{g}_1, \vect{g}_2, \ldots, \vect{g}_p$ are
        the principal axes of the genetic ellipsoid.
  \item The eigenvalue\index[subject]{eigenvalue}s $\lambda_1 \ge \lambda_2 \ge \cdots \ge \lambda_p$
        are the genetic variances along those axes.
  \item The semi-axis lengths are $\sqrt{\lambda_1}, \sqrt{\lambda_2},
        \ldots, \sqrt{\lambda_p}$.
\end{itemize}

\section[gmax: the line of least evolutionary resistance]{\texorpdfstring{$\mathbf{g}_{\max}$}{gmax}: the line of least evolutionary resistance}

The first eigenvector\index[subject]{eigenvector} of $\mat{G}$, denoted $\mathbf{g}_{\max}$, points in
the direction of maximum genetic variance. This direction has been called
the ``line of least evolutionary resistance'' because:

\begin{enumerate}
  \item Selection along $\mathbf{g}_{\max}$ produces the largest possible
        response per unit of selection intensity.
  \item Even when selection targets a different direction, the response
        tends to be deflected toward $\mathbf{g}_{\max}$.
  \item Over evolutionary time, populations may diverge primarily along
        $\mathbf{g}_{\max}$, not along the direction of selection.
\end{enumerate}

\begin{keyidea}
$\mathbf{g}_{\max}$ is the direction of maximum genetic variance. It
represents the path of least resistance for evolutionary change---the
direction the population ``wants'' to go, regardless of where selection
points.
\end{keyidea}

The smallest eigenvector\index[subject]{eigenvector}, $\mathbf{g}_{\min}$, represents the direction of
minimum genetic variance. Evolution in this direction is difficult: even
strong selection produces little response because the necessary genetic
variation is scarce.

\section{The multivariate breeder's equation\index[subject]{breeder's equation} equation revisited}

The importance of G geometry becomes clear through the breeder's equation\index[subject]{breeder's equation} equation.
In its multivariate form:
\[
  \Delta\bar{\vect{z}} = \mat{G}\boldsymbol{\beta},
\]
where $\boldsymbol{\beta} = \mat{P}^{-1}\vect{S}$ is the selection gradient
and $\vect{S}$ is the selection differential.

The response $\Delta\bar{\vect{z}}$ is not parallel to $\boldsymbol{\beta}$
unless $\mat{G}$ is a scalar multiple of the identity (equal variances, no
covariances). In general, $\mat{G}$ rotates and stretches the selection
gradient, deflecting the response toward directions of high genetic
variance.

\begin{figure}[ht]
  \centering
  \includegraphics[width=0.85\textwidth]{fig30_deflection.png}
  \caption[Selection deflected by G]{
    Selection (red arrow) aims in one direction, but the response (blue
    arrow) is deflected toward $\mathbf{g}_{\max}$. The more eccentric the
    G ellipse, the stronger the deflection.
  }
  \label{fig:deflection}
\end{figure}

\subsection*{A worked example}

Consider a G matrix with strong positive genetic correlation:
\[
  \mat{G} =
  \begin{pmatrix}
    1.0 & 0.8 \\
    0.8 & 1.0
  \end{pmatrix}.
\]

The eigenvalue\index[subject]{eigenvalue}s are $\lambda_1 = 1.8$ and $\lambda_2 = 0.2$. The first
eigenvector\index[subject]{eigenvector} points at 45° (both traits increasing together); the second
points at 135° (traits in opposition).

Suppose selection favours increased trait 2 only:
$\boldsymbol{\beta} = (0, 1)^\top$.

The response is:
\[
  \Delta\bar{\vect{z}} = \mat{G}\boldsymbol{\beta}
  = \begin{pmatrix} 1.0 & 0.8 \\ 0.8 & 1.0 \end{pmatrix}
    \begin{pmatrix} 0 \\ 1 \end{pmatrix}
  = \begin{pmatrix} 0.8 \\ 1.0 \end{pmatrix}.
\]

Selection targeted only trait 2, but the response includes an increase in
trait 1 as well. The genetic correlation\index[subject]{correlation}has ``dragged'' trait 1 along.
The response vector $(0.8, 1.0)$ is deflected toward the 45° direction of
$\mathbf{g}_{\max}$.

\begin{figure}[htbp]
    \centering
    \includegraphics[width=0.9\textwidth]{figures/fig_ch9_evolvability_vs_respondability.pdf}
    \caption{\textbf{Evolvability versus respondability.} Direction 
    $\vect{u}_1$ (red) has high evolvability (genetic variance = 1.21) but 
    lower heritability ($h^2 = 0.30$) because phenotypic variance is even 
    higher. Direction $\vect{u}_2$ (blue) has lower evolvability (genetic 
    variance = 0.40) but higher heritability ($h^2 = 0.37$). Evolvability 
    tells you how much genetic variance is available; respondability 
    (heritability) tells you what fraction of phenotypic variance is genetic. 
    Both matter for predicting response to selection.}
    \label{fig:evolvability-respondability}
\end{figure}

\section{Evolvability: genetic variance in the direction of selection}

Hansen and Houle introduced \textbf{evolvability\index[subject]{evolvability}} as the genetic variance
in the direction of selection, scaled appropriately. For a selection
gradient $\boldsymbol{\beta}$, the evolvability\index[subject]{evolvability} is:
\[
  e(\boldsymbol{\beta}) = \boldsymbol{\beta}^\top \mat{G} \boldsymbol{\beta}.
\]

When $\boldsymbol{\beta}$ is a unit vector, this is simply the genetic
variance in that direction. When $\boldsymbol{\beta}$ is standardised by
trait means (for mean-scaled evolvability\index[subject]{evolvability}), it measures the proportional
response to proportional selection.

\begin{keyidea}
evolvability\index[subject]{evolvability} $e(\boldsymbol{\beta}) = \boldsymbol{\beta}^\top\mat{G}\boldsymbol{\beta}$
measures the genetic variance available in the direction selection is
pushing. High evolvability\index[subject]{evolvability} means large response; low evolvability\index[subject]{evolvability} means
the population is constrained in that direction.
\end{keyidea}

From Chapter~20, we know that:
\[
  \lambda_{\min} \le \boldsymbol{\beta}^\top\mat{G}\boldsymbol{\beta} \le \lambda_{\max}.
\]

evolvability\index[subject]{evolvability} ranges from the smallest to the largest eigenvalue\index[subject]{eigenvalue} of $\mat{G}$.
The ratio $\lambda_{\max}/\lambda_{\min}$ quantifies how much evolvability\index[subject]{evolvability}
varies across directions---how ``eccentric'' the genetic ellipsoid is.

\section{Respondability and the comparison with P}

evolvability\index[subject]{evolvability} measures genetic variance alone. But response to selection
also depends on how much of the phenotypic variance is genetic. This leads
to \textbf{respondability}:
\[
  r(\boldsymbol{\beta}) 
    = \frac{\boldsymbol{\beta}^\top \mat{G} \boldsymbol{\beta}}
           {\boldsymbol{\beta}^\top \mat{P} \boldsymbol{\beta}}
    = h^2(\boldsymbol{\beta}),
\]
which is the directional heritability\index[subject]{heritability} from Chapter~21.

Respondability asks: of the phenotypic variance in direction
$\boldsymbol{\beta}$, what fraction is genetic? A direction can have high
evolvability\index[subject]{evolvability} (lots of genetic variance) but low respondability (even more
environmental variance), or vice versa.

The P-whitening\index[subject]{whitening transformation} framework from Chapter~21 lets us study respondability
systematically. The eigenvalue\index[subject]{eigenvalue}s of
$\mat{G}^* = \mat{P}^{-1/2}\mat{G}\mat{P}^{-1/2}$
are the respondabilities along the principal axes.

\section{Constraint: when G limits evolution}

The G matrix constrains evolution when:

\begin{enumerate}
  \item \textbf{Low eigenvalue\index[subject]{eigenvalue}s exist.} Some directions have little genetic
        variance. Evolution in those directions is slow or impossible.
  \item \textbf{eigenvalue\index[subject]{eigenvalue}s are unequal.} The genetic ellipsoid is
        eccentric. Response is channelled along $\mathbf{g}_{\max}$ even
        when selection points elsewhere.
  \item \textbf{G and selection are misaligned.} If selection targets a
        direction near $\mathbf{g}_{\min}$, response will be weak.
\end{enumerate}

\begin{figure}[ht]
  \centering
  \includegraphics[width=0.8\textwidth]{fig30_constraint_scenarios.png}
  \caption[Constraint scenarios]{
    Three scenarios. Left: G nearly spherical---little constraint, response
    tracks selection. Centre: G eccentric but aligned with selection---
    response is strong. Right: G eccentric and misaligned---response is
    weak and deflected.
  }
  \label{fig:constraint-scenarios}
\end{figure}

A useful summary statistic is the \textbf{eccentricity} of G, measured by
the relative variance of its eigenvalue\index[subject]{eigenvalue}s:
\[
  V_{\text{rel}}(\mat{G}) 
    = \frac{\Var(\lambda)}{\bar{\lambda}^2}
    = \frac{\sum_i (\lambda_i - \bar{\lambda})^2 / p}
           {\left(\sum_i \lambda_i / p\right)^2}.
\]

When $V_{\text{rel}} = 0$, all eigenvalue\index[subject]{eigenvalue}s are equal and G is spherical
(no constraint). When $V_{\text{rel}}$ is large, G is highly eccentric
(strong constraint).

\section{Effective dimensionality}

Another way to quantify constraint is through \textbf{effective
dimensionality}---how many independent directions of genetic variation
exist.

If all eigenvalue\index[subject]{eigenvalue}s were equal, $\lambda_i = \bar{\lambda}$, we would have
$p$ effective dimensions. If one eigenvalue\index[subject]{eigenvalue} dominates, the effective
dimensionality is closer to 1.

One common measure is:
\[
  n_{\text{eff}} = \frac{(\tr \mat{G})^2}{\tr(\mat{G}^2)}
                 = \frac{\left(\sum_i \lambda_i\right)^2}{\sum_i \lambda_i^2}.
\]

This equals $p$ when all eigenvalue\index[subject]{eigenvalue}s are equal and approaches 1 when one
eigenvalue\index[subject]{eigenvalue} dominates.

\begin{keyidea}
Effective dimensionality measures how many ``independent'' directions of
genetic variation the population has. Low effective dimensionality means
genetic variation is concentrated in a few directions---the population is
genetically constrained.
\end{keyidea}

\section{Empirical G matrices: what do they look like?}

Decades of empirical work have revealed some patterns:

\begin{itemize}
  \item \textbf{G matrices are often eccentric.} In many studies, the first
        few eigenvalue\index[subject]{eigenvalue}s account for most of the genetic variance.
        Effective dimensionality is typically much less than $p$.
  \item \textbf{$\mathbf{g}_{\max}$ often aligns with body size.} For
        morphological traits, the direction of maximum genetic variance
        frequently corresponds to overall size---all traits scaling
        together.
  \item \textbf{Genetic correlations can be strong.} Off-diagonal elements
        of G are often substantial, reflecting pervasive pleiotropy.
  \item \textbf{G varies among populations.} The G matrix is not fixed; it
        evolves and can differ between populations, species, and
        environments.
\end{itemize}

These patterns suggest that genetic constraint is common. Populations do
not have equal access to all directions in trait space; evolution is
channelled along particular paths.

\section{Stability and estimation of G}

Estimating G requires breeding designs (parent-offspring regression,
half-sib designs, animal models) or genomic data. Estimation is
challenging because:

\begin{itemize}
  \item \textbf{Sample sizes are often small.} Estimating a $p \times p$
        matrix requires estimating $p(p+1)/2$ unique elements. With many
        traits, sampling error can be severe.
  \item \textbf{G can be singular or nearly singular.} If some trait
        combinations have near-zero genetic variance, the estimated G may
        have zero or negative eigenvalue\index[subject]{eigenvalue}s due to sampling error.
  \item \textbf{G may not be stable.} If genetic architecture changes
        over time or differs among environments, a single G matrix may
        not capture the population's evolutionary potential.
\end{itemize}

These issues motivate careful statistical treatment: regularisation,
Bayesian estimation, and sensitivity analyses. We should interpret G
matrices with appropriate caution, especially their smaller eigenvalue\index[subject]{eigenvalue}s.

\section{G in the context of P}

Throughout these notes we have emphasised comparing G and P. The P matrix
describes total phenotypic variation; G describes the heritable component.
Their relationship determines:

\begin{itemize}
  \item \textbf{heritability\index[subject]{heritability}:} The ratio of G to P, generalised to multiple
        traits through directional heritability\index[subject]{heritability} $h^2(\boldsymbol{\beta})$.
  \item \textbf{Selection response:} The breeder's equation\index[subject]{breeder's equation} equation
        $\Delta\bar{\vect{z}} = \mat{G}\mat{P}^{-1}\vect{S}$ involves both.
  \item \textbf{Constraint traps:} Directions where G is small relative to
        P---phenotypic variation exists, but it is mostly environmental.
\end{itemize}

The P-whitening\index[subject]{whitening transformation} transformation $\mat{G}^* = \mat{P}^{-1/2}\mat{G}\mat{P}^{-1/2}$
places G and P on a common footing. In whitened space, P is the identity,
and G* directly reveals the heritability\index[subject]{heritability} structure.

\section{Summary}

In this chapter we have:

\begin{itemize}
  \item Interpreted the G matrix as an ellipsoid in trait space, with
        eigenvector\index[subject]{eigenvector}s as principal axes and eigenvalue\index[subject]{eigenvalue}s as genetic
        variances along those axes.
  \item Identified $\mathbf{g}_{\max}$ as the direction of maximum genetic
        variance---the line of least evolutionary resistance.
  \item Shown how the breeder's equation\index[subject]{breeder's equation} equation $\Delta\bar{\vect{z}} = \mat{G}\boldsymbol{\beta}$
        deflects selection response toward $\mathbf{g}_{\max}$.
  \item Defined evolvability\index[subject]{evolvability} as $\boldsymbol{\beta}^\top\mat{G}\boldsymbol{\beta}$
        and respondability (directional heritability\index[subject]{heritability}) as its ratio to
        phenotypic variance.
  \item Discussed constraint in terms of eigenvalue\index[subject]{eigenvalue} eccentricity and
        effective dimensionality.
  \item Summarised empirical patterns: G matrices are often eccentric,
        with $\mathbf{g}_{\max}$ frequently aligned with body size.
  \item Noted challenges in estimating G and the importance of comparing
        G to P through P-whitening\index[subject]{whitening transformation}.
\end{itemize}

The G matrix is the engine of evolutionary response. Its shape determines
which paths are open and which are blocked. In the next chapter, we turn
to the other side of the equation: the fitness surface, encoded in the
$\boldsymbol{\gamma}$ matrix, which determines where selection is pushing.