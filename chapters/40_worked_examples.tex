\chapter{Worked Examples: Complete Analyses}
\label{ch:worked-examples}

This chapter brings together everything we have learned. We work through
complete analyses from raw data to biological interpretation, showing each
step explicitly. The goal is not just to demonstrate techniques, but to
illustrate the thought process: when to use each tool, how to check
assumptions, and how to connect mathematical results to biological meaning.

We present three examples of increasing complexity:
\begin{enumerate}
  \item A two-trait analysis of a G matrix, computing directional
        heritabilities by hand.
  \item A four-trait analysis comparing G and P, with P-whitening.
  \item A selection analysis combining $\boldsymbol{\beta}$ and
        $\boldsymbol{\gamma}$ with the G matrix.
\end{enumerate}

Each example follows the same arc: state the data, check assumptions,
compute the relevant eigendecompositions, and interpret the results
biologically.

%-------------------------------------------------------------------
\section{Example 1: Two-trait G matrix analysis}
\index[subject]{worked example!two-trait G matrix}
%-------------------------------------------------------------------

\subsection*{The data}

A plant breeding program has estimated the following additive genetic
covariance matrix\index[subject]{covariance matrix}\index[subject]{G matrix@$\mathbf{G}$ matrix} for 
flowering time (days) and plant height (cm):
\[
  \mat{G} =
  \begin{pmatrix}
    25 & 15 \\
    15 & 36
  \end{pmatrix}.
\]

The phenotypic covariance matrix\index[subject]{covariance matrix}\index[subject]{P matrix@$\mathbf{P}$ matrix} is:
\[
  \mat{P} =
  \begin{pmatrix}
    50 & 20 \\
    20 & 60
  \end{pmatrix}.
\]

Our goals are to:
\begin{enumerate}
  \item Find the principal axes of genetic variation.
  \item Compute heritability in several directions.
  \item Identify the directions of maximum and minimum heritability\index[subject]{heritability}.
\end{enumerate}

\begin{figure}[htbp]
    \centering
    \includegraphics[width=\textwidth]{figures/fig_ch12_worked_example.pdf}
    \caption{Complete analysis of the two-trait worked example. (a) Genetic 
    ($\mathbf{G}$, blue) and phenotypic ($\mathbf{P}$, green) ellipses in 
    original trait space, with $\mathbf{g}_{\max}$ and $\mathbf{g}_{\min}$ 
    marked. (b) The whitened view: $\mathbf{G}^*$ ellipse inside the 
    $\mathbf{P}$-sphere, showing directions of extreme heritability. 
    (c) Directional heritability $h^2(\theta)$ as a function of direction, 
    ranging from 0.42 to 0.62. (d) Summary statistics from the analysis. 
    (e) The breeder's equation in action: selection gradient $\boldsymbol{\beta}$ 
    is deflected toward $\mathbf{g}_{\max}$ in the response $\Delta\bar{\mathbf{z}}$. 
    (f) Variance decomposition by direction, showing the gap between phenotypic 
    and genetic variance (environmental variance) varies with direction.}
    \label{fig:worked_example}
\end{figure}

\subsection*{Step 1: Eigendecompose G}
\index[subject]{eigendecomposition!of G matrix}

The characteristic equation\index[subject]{characteristic equation}\index[subject]{characteristic equation\index[subject]{characteristic equation}} for $\mat{G}$ is:
\[
  \det(\mat{G} - \lambda\mat{I}) = (25 - \lambda)(36 - \lambda) - 15^2 = 0.
\]

Expanding:
\[
  \lambda^2 - 61\lambda + (25 \times 36 - 225) = \lambda^2 - 61\lambda + 675 = 0.
\]

Using the quadratic form\index[subject]{quadratic form}ula:
\[
  \lambda = \frac{61 \pm \sqrt{61^2 - 4 \times 675}}{2}
          = \frac{61 \pm \sqrt{3721 - 2700}}{2}
          = \frac{61 \pm \sqrt{1021}}{2}
          = \frac{61 \pm 31.95}{2}.
\]

So $\lambda_1 = 46.48$ and $\lambda_2 = 14.52$.

For $\lambda_1 = 46.48$, the eigenvector satisfies:
\[
  \begin{pmatrix} 25 - 46.48 & 15 \\ 15 & 36 - 46.48 \end{pmatrix}
  \begin{pmatrix} v_1 \\ v_2 \end{pmatrix} = \vect{0}.
\]

From the first row: $-21.48 v_1 + 15 v_2 = 0$, so $v_2/v_1 = 21.48/15 = 1.43$.

Normalising: $\mathbf{g}_{\max} = (0.573, 0.820)^\top$.
\index[subject]{gmax@$\mathbf{g}_{\max}$}

Similarly, $\mathbf{g}_{\min} = (0.820, -0.573)^\top$ (orthogonal to
$\mathbf{g}_{\max}$).

\paragraph{Interpretation.}
The direction of maximum genetic variance points mostly toward height
(coefficient 0.820) with a positive contribution from flowering time
(0.573). Plants with high breeding values tend to be both tall and
late-flowering. The genetic variance along this axis is 46.48.

The direction of minimum genetic variance contrasts the traits: tall but
early-flowering, or short but late-flowering. Genetic variance along this
axis is only 14.52---about one-third of the maximum.

\subsection*{Step 2: Compute univariate heritabilities}
\index[subject]{heritability}

For each trait separately:
\[
  h^2_{\text{time}} = \frac{G_{11}}{P_{11}} = \frac{25}{50} = 0.50,
  \qquad
  h^2_{\text{height}} = \frac{G_{22}}{P_{22}} = \frac{36}{60} = 0.60.
\]

Both traits have moderate heritability\index[subject]{heritability}.

\subsection*{Step 3: Compute directional heritabilities}
\index[subject]{heritability!directional}

For an arbitrary direction $\boldsymbol{\beta}$, the directional
heritability\index[subject]{heritability} is:
\[
  h^2(\boldsymbol{\beta}) = 
    \frac{\boldsymbol{\beta}^\top \mat{G} \boldsymbol{\beta}}
         {\boldsymbol{\beta}^\top \mat{P} \boldsymbol{\beta}}.
\]

\paragraph{Along $\mathbf{g}_{\max}$.}
Let $\boldsymbol{\beta} = (0.573, 0.820)^\top$.

Numerator:
\begin{align*}
  \boldsymbol{\beta}^\top \mat{G} \boldsymbol{\beta}
    &= (0.573, 0.820) 
       \begin{pmatrix} 25 & 15 \\ 15 & 36 \end{pmatrix}
       \begin{pmatrix} 0.573 \\ 0.820 \end{pmatrix} \\
    &= (0.573, 0.820) \begin{pmatrix} 26.63 \\ 38.12 \end{pmatrix}
     = 46.48.
\end{align*}

Denominator:
\begin{align*}
  \boldsymbol{\beta}^\top \mat{P} \boldsymbol{\beta}
    &= (0.573, 0.820) 
       \begin{pmatrix} 50 & 20 \\ 20 & 60 \end{pmatrix}
       \begin{pmatrix} 0.573 \\ 0.820 \end{pmatrix} \\
    &= (0.573, 0.820) \begin{pmatrix} 45.05 \\ 60.66 \end{pmatrix}
     = 75.55.
\end{align*}

So $h^2(\mathbf{g}_{\max}) = 46.48 / 75.55 = 0.615$.

\paragraph{Along $\mathbf{g}_{\min}$.}
Let $\boldsymbol{\beta} = (0.820, -0.573)^\top$.

By similar calculation:
\[
  \boldsymbol{\beta}^\top \mat{G} \boldsymbol{\beta} = 14.52,
  \qquad
  \boldsymbol{\beta}^\top \mat{P} \boldsymbol{\beta} = 34.45.
\]

So $h^2(\mathbf{g}_{\min}) = 14.52 / 34.45 = 0.421$.

\paragraph{Along flowering time only.}
Let $\boldsymbol{\beta} = (1, 0)^\top$.

Then $h^2 = G_{11}/P_{11} = 25/50 = 0.50$.

\subsection*{Step 4: Find extreme heritabilities via G*}
\index[subject]{G* matrix@$\mathbf{G}^*$ matrix}

To find the true maximum and minimum heritabilities across all directions,
we compute $\mat{G}^* = \mat{P}^{-1/2}\mat{G}\mat{P}^{-1/2}$ and
find its eigenvalues.

First, eigendecompose $\mat{P}$:
\[
  \mat{P} = \mat{V}_P {\Lambda}_P \mat{V}_P^\top.
\]

The eigenvalue\index[subject]{eigenvalue}s of $\mat{P}$ are $\lambda_{P,1} = 75.62$ and
$\lambda_{P,2} = 34.38$, with corresponding eigenvector\index[subject]{eigenvector}s.

Then:
\[
  \mat{P}^{-1/2} = \mat{V}_P {\Lambda}_P^{-1/2} \mat{V}_P^\top.
\]

Computing $\mat{G}^* = \mat{P}^{-1/2}\mat{G}\mat{P}^{-1/2}$ and
finding its eigenvalue\index[subject]{eigenvalue}s gives:
\[
  \lambda_1^* = 0.619, \qquad \lambda_2^* = 0.419.
\]

These are the maximum and minimum directional heritabilities.

\begin{keyidea}
The eigenvalue\index[subject]{eigenvalue}s of $\mat{G}^* = \mat{P}^{-1/2}\mat{G}\mat{P}^{-1/2}$ give
the extreme directional heritabilities. Here, heritability\index[subject]{heritability} ranges from
0.42 to 0.62 depending on direction---a range of 0.20, or about 40\%
of the minimum value.
\end{keyidea}

\subsection*{Summary table}

\begin{center}
\begin{tabular}{lcc}
\hline
Direction & Genetic variance & heritability\index[subject]{heritability} \\
\hline
Flowering time only & 25.0 & 0.50 \\
Height only & 36.0 & 0.60 \\
$\mathbf{g}_{\max}$ & 46.5 & 0.62 \\
$\mathbf{g}_{\min}$ & 14.5 & 0.42 \\
Maximum $h^2$ direction & --- & 0.62 \\
Minimum $h^2$ direction & --- & 0.42 \\
\hline
\end{tabular}
\end{center}

In this example, the direction of maximum genetic variance 
($\mathbf{g}_{\max}$) is close to the direction of maximum heritability\index[subject]{heritability},
but they need not coincide in general. The former maximises
$\boldsymbol{\beta}^\top\mat{G}\boldsymbol{\beta}$; the latter maximises
the ratio
$\boldsymbol{\beta}^\top\mat{G}\boldsymbol{\beta} /
 \boldsymbol{\beta}^\top\mat{P}\boldsymbol{\beta}$.
When $\mat{G}$ and $\mat{P}$ have different orientations, these directions
can differ substantially.

%-------------------------------------------------------------------
\section{Example 2: Four-trait G-P comparison with whitening\index[subject]{whitening transformation}}
\index[subject]{worked example!four-trait G-P comparison}
\index[subject]{whitening\index[subject]{whitening transformation} transformation}
%-------------------------------------------------------------------

\subsection*{The data}

A study of a passerine bird population estimates G and P for four
morphological traits: wing length, tarsus length, bill depth, and bill
width. The matrices are:

\[
  \mat{G} =
  \begin{pmatrix}
    0.80 & 0.45 & 0.20 & 0.15 \\
    0.45 & 0.60 & 0.25 & 0.18 \\
    0.20 & 0.25 & 0.35 & 0.28 \\
    0.15 & 0.18 & 0.28 & 0.30
  \end{pmatrix},
\]
\[
  \mat{P} =
  \begin{pmatrix}
    1.20 & 0.55 & 0.30 & 0.22 \\
    0.55 & 0.95 & 0.35 & 0.25 \\
    0.30 & 0.35 & 0.55 & 0.40 \\
    0.22 & 0.25 & 0.40 & 0.50
  \end{pmatrix}.
\]

\subsection*{Step 1: Basic checks}
\index[subject]{positive definite\index[subject]{matrix!positive definite}ness}

Before analysis, we verify that both matrices are positive definite\index[subject]{matrix!positive definite}
(all eigenvalue\index[subject]{eigenvalue}s positive) and that $G_{ii} \le P_{ii}$ for each trait
(genetic variance should not exceed phenotypic variance).

eigenvalue\index[subject]{eigenvalue}s of $\mat{G}$: 1.36, 0.44, 0.21, 0.04.
All positive---$\mat{G}$ is positive definite\index[subject]{matrix!positive definite}.

eigenvalue\index[subject]{eigenvalue}s of $\mat{P}$: 1.95, 0.67, 0.46, 0.12.
All positive---$\mat{P}$ is positive definite\index[subject]{matrix!positive definite}.

Diagonal check: $G_{ii} \le P_{ii}$ for all $i$. Yes: 0.80 < 1.20,
0.60 < 0.95, 0.35 < 0.55, 0.30 < 0.50.

\subsection*{Step 2: Univariate heritabilities}

\begin{center}
\begin{tabular}{lccc}
\hline
Trait & $G_{ii}$ & $P_{ii}$ & $h^2$ \\
\hline
Wing length  & 0.80 & 1.20 & 0.67 \\
Tarsus length & 0.60 & 0.95 & 0.63 \\
Bill depth   & 0.35 & 0.55 & 0.64 \\
Bill width   & 0.30 & 0.50 & 0.60 \\
\hline
\end{tabular}
\end{center}

All traits have similar, moderately high heritabilities (0.60--0.67).
A univariate analysis would conclude that all four traits are roughly
equally heritable. But this masks important directional variation.

\subsection*{Step 3: Compute G* and its eigenstructure}

We compute $\mat{G}^* = \mat{P}^{-1/2}\mat{G}\mat{P}^{-1/2}$ using the
eigendecomposition of $\mat{P}$.

The eigenvalue\index[subject]{eigenvalue}s of $\mat{G}^*$ are the directional heritabilities along
the principal axes of P-whitened space:

\begin{center}
\begin{tabular}{cccc}
\hline
$\lambda_1^*$ & $\lambda_2^*$ & $\lambda_3^*$ & $\lambda_4^*$ \\
\hline
0.71 & 0.65 & 0.45 & 0.34 \\
\hline
\end{tabular}
\end{center}

\subsection*{Step 4: Interpret the heritability\index[subject]{heritability} distribution}

The eigenvalue\index[subject]{eigenvalue}s range from 0.34 to 0.71. This means:
\begin{itemize}
  \item Maximum directional heritability\index[subject]{heritability}: 0.71 (71\% of variance genetic).
  \item Minimum directional heritability\index[subject]{heritability}: 0.34 (34\% genetic).
  \item Range: 0.37 (more than double the minimum).
\end{itemize}

Mean heritability\index[subject]{heritability} (average of eigenvalue\index[subject]{eigenvalue}s): $\bar{h}^2 = 0.54$.

Coefficient of variation of eigenvalue\index[subject]{eigenvalue}s:
\[
  \text{CV}(\lambda^*) = \frac{\text{SD}(\lambda^*)}{\text{mean}(\lambda^*)}
    = \frac{0.150}{0.54} = 0.28.
\]

From the formula $\text{CV}(h^2) = \sqrt{2/(p+2)} \times \text{CV}(\lambda^*)$:
\[
  \text{CV}(h^2) = \sqrt{2/6} \times 0.28 = 0.577 \times 0.28 = 0.16.
\]

The coefficient of variation of directional heritability\index[subject]{heritability} is about 16\%.
This indicates moderate constraint heterogeneity---some directions are
substantially more heritable than others.

\subsection*{Step 5: Identify constraint traps}
\index[subject]{constraint!trap}

The eigenvector\index[subject]{eigenvector} corresponding to $\lambda_4^* = 0.34$ defines the direction
of minimum heritability\index[subject]{heritability}. Examining its loadings:
\[
  \vect{v}_4^* = (0.14, -0.19, 0.66, -0.71)^\top.
\]

This direction contrasts the bill traits (depth positive, width negative)
with small contributions from the body-size traits. Selection for birds
with deep but narrow bills, or vice versa, would face a constraint trap:
only 34\% of phenotypic variance in this direction is genetic.

In contrast, the direction of maximum heritability\index[subject]{heritability} ($\lambda_1^* = 0.71$)
has loadings:
\[
  \vect{v}_1^* = (0.79, 0.61, 0.05, 0.04)^\top.
\]

This direction loads heavily on wing and tarsus---overall body ``size.''
Selection for larger or smaller birds has high heritability\index[subject]{heritability}; 71\% of
variance is genetic.

\begin{keyidea}
Size variation has high heritability\index[subject]{heritability} (71\%); bill-shape variation has
lower heritability\index[subject]{heritability} (34\%). A breeding program targeting bill proportions
would face stronger constraints than one targeting overall body size.
\end{keyidea}

\subsection*{Step 6: Geometric interpretation}

Figure~\ref{fig:bird_case_study} illustrates these results geometrically.

In the original trait coordinates, the phenotypic covariance matrix\index[subject]{covariance matrix}
$\mat{P}$ defines an ellipsoid whose cross-sections in any two-trait
plane are ellipses. The genetic covariance matrix\index[subject]{covariance matrix} $\mat{G}$ defines a
second ellipsoid nested inside the first. For this bird population, the
$\mat{G}$ ellipsoid is somewhat narrower along directions involving bill
shape than along overall size.

whitening\index[subject]{whitening transformation} by $\mat{P}^{-1/2}$ maps the $\mat{P}$ ellipsoid to a sphere:
in whitened coordinates every direction has unit phenotypic variance.
In this whitened space, $\mat{G}^\ast$ appears as an ellipsoid whose
axes have lengths given by the square roots of the eigenvalue\index[subject]{eigenvalue}s
$\lambda_i^\ast$. The long axis corresponds to the high-heritability\index[subject]{heritability}
``size'' direction ($h^2 = 0.71$); the short axis corresponds to the
low-heritability\index[subject]{heritability} ``shape'' direction ($h^2 = 0.34$).

\begin{figure}[t]
  \centering
  \includegraphics[width=0.95\textwidth]{fig_bird_case_study.pdf}
  \caption[Geometric summary of the bird G--P example]{
    Schematic view of the four-trait bird example.
    (a) Genetic ($\mat{G}$) and phenotypic ($\mat{P}$) ellipses in a
        two-trait projection of the original trait space.
    (b) Whitened trait space: the P-sphere\index[subject]{P-sphere} (unit circle) and the
        $\mat{G}^\ast$ ellipse showing directional heritabilities. The
        long axis corresponds to the high-heritability\index[subject]{heritability} size direction
        ($h^2 = 0.71$); the short axis corresponds to the 
        low-heritability\index[subject]{heritability} bill-shape direction ($h^2 = 0.34$), a 
        constraint trap.
  }
  \label{fig:bird_case_study}
\end{figure}

%-------------------------------------------------------------------
\section[Example 3: Selection analysis with G and gamma]{Example 3: Selection analysis with G and \texorpdfstring{$\boldsymbol{\gamma}$}{γ}}
\index[subject]{worked example!selection analysis}
\index[subject]{gamma matrix@$\boldsymbol{\gamma}$ matrix}
%-------------------------------------------------------------------

\subsection*{The data}

A study measures survival in relation to two traits (standardised to mean
zero, unit variance). The estimated selection gradients are:
\[
  \boldsymbol{\beta} = \begin{pmatrix} 0.18 \\ 0.12 \end{pmatrix},
  \qquad
  \boldsymbol{\gamma} = \begin{pmatrix} -0.15 & 0.08 \\ 0.08 & -0.10 \end{pmatrix}.
\]

The G matrix (in standardised units) is:
\[
  \mat{G} = \begin{pmatrix} 0.45 & 0.30 \\ 0.30 & 0.35 \end{pmatrix}.
\]

\subsection*{Step 1: Interpret $\boldsymbol{\beta}$}
\index[subject]{selection gradient}

Both elements of $\boldsymbol{\beta}$ are positive: directional selection
favours increases in both traits. Trait 1 is under stronger directional
selection (0.18 vs.\ 0.12).

The magnitude $\|\boldsymbol{\beta}\| = \sqrt{0.18^2 + 0.12^2} = 0.216$
gives overall strength of directional selection.

The direction of selection:
\[
  \frac{\boldsymbol{\beta}}{\|\boldsymbol{\beta}\|} = (0.832, 0.555)^\top.
\]

\subsection*{Step 2: Interpret $\boldsymbol{\gamma}$}
\index[subject]{selection!stabilising}
\index[subject]{selection!correlational}

Both diagonal elements are negative: stabilising selection on each trait
individually. The off-diagonal is positive: correlational selection favours
positive trait combinations (both high or both low together).

Eigendecomposition of $\boldsymbol{\gamma}$:
\[
  \lambda_1^\gamma = -0.04, \qquad \lambda_2^\gamma = -0.21.
\]

Both negative, confirming overall stabilising selection. But selection is
much stronger ($|\lambda| = 0.21$) along the second eigenvector\index[subject]{eigenvector} than the
first ($|\lambda| = 0.04$).

The eigenvector\index[subject]{eigenvector}s are:
\begin{align*}
  \vect{v}_1^\gamma &= (0.59, 0.81)^\top \quad \text{(weak stabilising)}, \\
  \vect{v}_2^\gamma &= (0.81, -0.59)^\top \quad \text{(strong stabilising)}.
\end{align*}

\paragraph{Interpretation.}
Stabilising selection is weak along the direction where both traits
increase together---a ``size'' axis. Stabilising selection is strong
along the direction where traits oppose each other---a ``contrast'' axis.
The fitness surface is elongated: a ridge running along the positive
diagonal, with steep sides.

\subsection*{Step 3: Predict response to selection}

Using the Lande equation\index[subject]{breeder's equation}:
\[
  \Delta\bar{\vect{z}} = \mat{G}\boldsymbol{\beta}
    = \begin{pmatrix} 0.45 & 0.30 \\ 0.30 & 0.35 \end{pmatrix}
      \begin{pmatrix} 0.18 \\ 0.12 \end{pmatrix}
    = \begin{pmatrix} 0.117 \\ 0.096 \end{pmatrix}.
\]

The predicted response is $(0.117, 0.096)$---increases in both traits, with
trait 1 responding slightly more.

The response direction is:
\[
  \frac{\Delta\bar{\vect{z}}}{\|\Delta\bar{\vect{z}}\|}
    = (0.773, 0.634)^\top.
\]

Comparing to the selection direction $(0.832, 0.555)^\top$, we see the
response is deflected toward the direction of higher genetic variance
(trait 1), but the deflection is modest because the genetic correlation
is positive and selection favours both traits.

\subsection*{Step 4: Alignment of G and $\boldsymbol{\gamma}$}
\index[subject]{G-gamma alignment@G--$\gamma$ alignment}

Eigendecompose $\mat{G}$:
\[
  \lambda_1^G = 0.70, \quad \lambda_2^G = 0.10.
\]

The eigenvector\index[subject]{eigenvector} for $\lambda_1^G$ is $(0.76, 0.65)^\top$---similar to the
direction of weak stabilising selection ($\vect{v}_1^\gamma$).

\begin{keyidea}
The direction of maximum genetic variance ($\mathbf{g}_{\max}$) aligns
with the direction of weak stabilising selection. This is favourable:
the population can move along the fitness ridge without fighting strong
curvature. Evolution is channeled but not blocked.
\end{keyidea}

Conversely, the direction of minimum genetic variance aligns with the
direction of strong stabilising selection. Even if selection pushed toward
unusual trait combinations (one high, one low), the population would
struggle to respond because:
\begin{enumerate}
  \item Genetic variance is low in that direction ($\lambda_2^G = 0.10$).
  \item Stabilising selection is strong ($\lambda_2^\gamma = -0.21$).
\end{enumerate}

This alignment is likely not coincidental. Theory predicts that
mutation-selection balance\index[subject]{mutation-selection balance} tends to erode variance in directions of strong
stabilising selection while preserving variance along fitness ridges.

\subsection*{Step 5: Long-term prediction}

Under mutation-selection balance\index[subject]{mutation-selection balance}, the population is expected to maintain
variation primarily along $\mathbf{g}_{\max}$ (the fitness ridge). The
combination of high genetic variance, weak stabilising selection, and
positive correlational selection along the size axis suggests that size
variation will persist. Variation in trait contrast will be more rapidly
eroded.

This example illustrates why both $\mat{G}$ and $\boldsymbol{\gamma}$
matter. Knowing $\mat{G}$ alone tells us about evolutionary potential;
knowing $\boldsymbol{\gamma}$ alone tells us about the fitness landscape.
Only by comparing their geometries can we predict whether evolution will
be fast or slow, direct or deflected.

%-------------------------------------------------------------------
\section{Computational tools}
\index[subject]{R code}
%-------------------------------------------------------------------

All calculations in this chapter can be performed by hand for two traits,
but become tedious for more. Here is R code implementing the key steps:

\begin{verbatim}
# Given G and P matrices (bird example)
G <- matrix(c(0.80, 0.45, 0.20, 0.15,
              0.45, 0.60, 0.25, 0.18,
              0.20, 0.25, 0.35, 0.28,
              0.15, 0.18, 0.28, 0.30), 4, 4)

P <- matrix(c(1.20, 0.55, 0.30, 0.22,
              0.55, 0.95, 0.35, 0.25,
              0.30, 0.35, 0.55, 0.40,
              0.22, 0.25, 0.40, 0.50), 4, 4)

# Step 1: Basic checks
cat("G eigenvalues:", round(eigen(G)$values, 3), "\n")
cat("P eigenvalues:", round(eigen(P)$values, 3), "\n")
cat("All positive?", all(eigen(G)$values > 0) & 
                     all(eigen(P)$values > 0), "\n")

# Step 2: Compute P^{-1/2}
eig_P <- eigen(P)
V_P <- eig_P$vectors
P_inv_sqrt <- V_P %*% diag(1/sqrt(eig_P$values)) %*% t(V_P)

# Step 3: Compute G*
G_star <- P_inv_sqrt %*% G %*% P_inv_sqrt

# Step 4: Eigenvalues of G* are directional heritabilities
eig_Gstar <- eigen(G_star)
h2_dir <- eig_Gstar$values
cat("Directional heritabilities:", round(h2_dir, 3), "\n")
cat("Max h2:", round(max(h2_dir), 3), "\n")
cat("Min h2:", round(min(h2_dir), 3), "\n")

# Step 5: CV of directional heritability
mean_h2 <- mean(h2_dir)
sd_h2 <- sd(h2_dir)
cv_lambda <- sd_h2 / mean_h2
p <- length(h2_dir)
cv_h2 <- sqrt(2/(p+2)) * cv_lambda
cat("Mean h2:", round(mean_h2, 3), "\n")
cat("CV(h2):", round(cv_h2, 3), "\n")

# Step 6: Identify constraint directions
cat("\nMax h2 direction (loadings):\n")
print(round(eig_Gstar$vectors[, 1], 2))
cat("\nMin h2 direction (loadings):\n")
print(round(eig_Gstar$vectors[, p], 2))
\end{verbatim}

%-------------------------------------------------------------------
\section{Summary}
%-------------------------------------------------------------------

In this chapter we have:

\begin{itemize}
  \item Worked through a complete two-trait analysis: eigendecomposition
        of G, calculation of directional heritabilities, and identification
        of extreme values via G*.
  \item Extended to four traits, demonstrating P-whitening\index[subject]{whitening transformation} and
        interpretation of the eigenvalue\index[subject]{eigenvalue} spectrum as the distribution of
        directional heritability\index[subject]{heritability}.
  \item Combined G with $\boldsymbol{\gamma}$ to analyse how genetic
        constraint interacts with the fitness surface geometry.
  \item Shown that alignment between G and $\boldsymbol{\gamma}$ determines
        whether evolution is facilitated or frustrated.
  \item Provided R code for computing G* and its eigenstructure.
\end{itemize}

These examples illustrate the payoff of the geometric perspective. Matrices
are not just tables of numbers---they are shapes that constrain and channel
evolution. By visualising G, P, and $\boldsymbol{\gamma}$ as ellipsoids and
understanding their eigenstructure, we gain insight into evolutionary
potential and constraint that would be invisible from univariate analyses
alone.

%-------------------------------------------------------------------
\section*{Exercises}
%-------------------------------------------------------------------

\paragraph{Exercise 1 (Covariance ellipses with the same trace).}
Construct two $2 \times 2$ covariance matrices that have the same trace
(sum of diagonal elements) but different eigenvalue\index[subject]{eigenvalue}s. For each matrix:
\begin{enumerate}
  \item Compute the eigenvalue\index[subject]{eigenvalue}s and eigenvector\index[subject]{eigenvector}s.
  \item Sketch the corresponding covariance ellipse.
  \item Explain how the trace can be the same while the shape differs.
\end{enumerate}

\paragraph{Exercise 2 (G* and the P-sphere\index[subject]{P-sphere} for two traits).}
Using the plant example from Example~1:
\begin{enumerate}
  \item Compute $\mat{P}^{-1/2}$ explicitly.
  \item Compute $\mat{G}^\ast = \mat{P}^{-1/2} \mat{G} \mat{P}^{-1/2}$.
  \item Draw the unit circle (the P-sphere\index[subject]{P-sphere}) and sketch the ellipse
        defined by $\mat{G}^\ast$.
  \item Mark the directions of maximum and minimum heritability\index[subject]{heritability}.
\end{enumerate}

\paragraph{Exercise 3 (Directional heritability\index[subject]{heritability} in a chosen direction).}
In the bird example from Example~2:
\begin{enumerate}
  \item Define a direction corresponding to ``bill shape'' (e.g.,
        increasing bill depth while decreasing bill width).
  \item Compute $h^2(\vect{u})$ for this direction.
  \item Compare to the extreme values from $\mat{G}^\ast$.
\end{enumerate}

\paragraph{Exercise 4 (G-$\gamma$ alignment).}
Consider the selection example in Example~3:
\begin{enumerate}
  \item Compute the angle between $\mathbf{g}_{\max}$ and
        $\vect{v}_1^\gamma$.
  \item How would the evolutionary trajectory change if these were
        perpendicular?
  \item Describe a scenario where misalignment would strongly
        frustrate evolutionary change.
\end{enumerate}