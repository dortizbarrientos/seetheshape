\chapter{Points and Trait Space}

These notes will use pictures of points, arrows, and clouds in space
throughout. Before we mention vectors or matrices, we need a clear picture
of what the space is and what it means to move around in it.

\section{Traits as axes, individuals as points}

Start with a single quantitative trait, such as body length. We can draw a
line and mark each individual on that line according to its measured value.
This is the usual number line from school (\cref{fig:two-points}).

\begin{figure}[ht]
  \centering
  \includegraphics[width=0.7\textwidth]{fig01_two_points.png}
  \caption[Two points on a number line]{
    Distance begins with two points. On a single trait axis, the distance
    between individuals A and B is the absolute difference of their values.
  }
  \label{fig:two-points}
\end{figure}

Now take two traits, for example body length and wing span. Instead of a
single line, we draw a horizontal axis for body length and a vertical axis
for wing span. Every individual is now a point in the resulting plane. If
individual $i$ has body length $x_i$ and wing span $y_i$, we draw it at the
point $(x_i, y_i)$.

With three traits, we would have three axes and each individual would be a
point in three-dimensional space. We cannot draw that easily on paper, but
the idea is the same. In general, if we measure $p$ traits on each
individual, then each individual is a point in a $p$-dimensional
\emph{trait space}.

\begin{keyidea}
A multivariate phenotype is a point in trait space. The number of
dimensions equals the number of traits.
\end{keyidea}

This is our basic mental model: a sample is a cloud of points in trait
space. Much of quantitative genetics is about describing the shape of this
cloud and how it moves under selection, drift, and other processes.

\section{Differences between individuals as arrows}

We often care about how two individuals differ. On a single trait, the
difference between individuals $i$ and $j$ is just
\[
  x_j - x_i,
\]
a signed distance along the line. When we have a reference point, every
individual can be described by its displacement from that reference
(\cref{fig:reference-point}).

\begin{figure}[ht]
  \centering
  \includegraphics[width=0.7\textwidth]{fig02_reference_point.png}
  \caption[Distance from a reference point]{
    Choosing a reference point lets us describe each individual by its
    displacement. Here, individual A sits 2 units from the reference,
    B sits 7 units away. The distance between A and B is still $|7-2|=5$.
  }
  \label{fig:reference-point}
\end{figure}

In two traits, the situation is similar but now differences have two
components. If individual $i$ is at $(x_i, y_i)$ and individual $j$ is at
$(x_j, y_j)$, then the difference between them can be drawn as an arrow
from $i$ to $j$.

To construct this arrow, we subtract coordinates:
\[
  \text{change in body length} = x_j - x_i,
\]
\[
  \text{change in wing span} = y_j - y_i.
\]

The arrow itself records two pieces of information:

\begin{itemize}
  \item a \emph{direction} in the plane (where you would walk if you were
        trying to go from individual $i$ to individual $j$);
  \item a \emph{length} (how far you would need to walk).
\end{itemize}

We can describe the same arrow in words:

\begin{quote}
  ``Starting at individual $i$, add $(x_j - x_i)$ units of body length and
  $(y_j - y_i)$ units of wing span.''
\end{quote}

In coordinates, we might write this as
\[
  \begin{pmatrix}
    x_j - x_i \\
    y_j - y_i
  \end{pmatrix}.
\]

At this stage we do not need any new jargon. The important point is that
these arrows behave in a regular way.

\section{The mean as natural reference}

In statistics, we typically use the sample mean as our reference point
(\cref{fig:mean-reference}). This choice is not arbitrary: measuring
deviations from the mean is the foundation of variance.

\begin{figure}[ht]
  \centering
  \includegraphics[width=0.7\textwidth]{fig03_mean_as_reference.png}
  \caption[The mean as reference]{
    Using the mean as reference point. Each individual's position is now
    a deviation from the mean. The sum of all deviations (with signs) is
    zero---this is a defining property of the mean.
  }
  \label{fig:mean-reference}
\end{figure}

When we summarise a sample, we face a problem: how do we reduce many
individual deviations to a single number that captures the ``spread'' of
the cloud? (\cref{fig:summarizing-problem})

\begin{figure}[ht]
  \centering
  \includegraphics[width=0.7\textwidth]{fig04_summarizing_problem.png}
  \caption[The summarising problem]{
    Many individuals, many deviations. We need a single number to describe
    how spread out the cloud is. Simply averaging the deviations will not
    work---positive and negative values cancel.
  }
  \label{fig:summarizing-problem}
\end{figure}

\section{Adding and stretching arrows}

If arrows represent changes in phenotype, then we can combine changes.

Imagine one change that goes from phenotype A to phenotype B, and another
change that goes from phenotype B to phenotype C. If we draw both arrows
head-to-tail, the overall change from A to C is the arrow from the start of
the first to the end of the second.

Algebraically, if the first change has components $(\Delta x_1, \Delta y_1)$
and the second has components $(\Delta x_2, \Delta y_2)$, then the combined
change has components
\[
  (\Delta x_1 + \Delta x_2,\; \Delta y_1 + \Delta y_2).
\]

Similarly, we can stretch or shrink an arrow. If a change in phenotype is
described by $(\Delta x, \Delta y)$, then half that change is
$(\tfrac{1}{2}\Delta x, \tfrac{1}{2}\Delta y)$, and twice that change is
$(2\Delta x, 2\Delta y)$.

\begin{keyidea}
Changes in phenotype can be added and scaled. Geometrically, this means we
can join arrows head-to-tail and stretch or shrink them. Algebraically, this
corresponds to adding and scaling their coordinate pairs.
\end{keyidea}

This simple behaviour is what makes these objects so useful. It is the
reason we will eventually give them a special name.

\begin{figure}[htbp]
    \centering
    \includegraphics[width=\textwidth]{figures/fig_ch1_vector_addition.pdf}
    \caption{Vector addition by the head-to-tail method. (a) Two vectors 
    $\mathbf{u}$ and $\mathbf{v}$ originating from the origin, representing 
    independent changes in phenotype. (b) Head-to-tail construction: the tail 
    of $\mathbf{v}$ is placed at the head of $\mathbf{u}$. (c) The resulting 
    sum vector $\mathbf{u} + \mathbf{v}$ connects the origin to the endpoint.
    This construction underlies how selection differentials accumulate across 
    episodes of selection and how evolutionary responses combine across 
    generations.}
    \label{fig:vector_addition}
\end{figure}

\section{From arrows to vectors}

We are now ready to introduce the word ``vector''.

\begin{definition}
A \emph{vector} in trait space is a quantity that has both direction and
length and that can be added and scaled in the way just described.
\end{definition}

The difference between two phenotypes is a vector. A selection gradient is
also a vector: it points in the direction of steepest increase in fitness in
trait space. A response to selection is a vector: it describes how the mean
phenotype moves.

Each vector can be viewed in three equivalent ways:

\begin{itemize}
  \item as an arrow in trait space;
  \item as a verbal instruction for how to change each trait;
  \item as a list of numbers, one for each trait.
\end{itemize}

For example, in two traits we might write
\[
  \vect{v}
    =
  \begin{pmatrix}
    2 \\
    -1
  \end{pmatrix}
\]
to represent the change ``add 2 units of trait 1 and subtract 1 unit of
trait 2''. In three traits, a vector would have three components, and so
on.

When you see a bold symbol like $\vect{z}$ or $\boldsymbol{\beta}$ in later
chapters, it will always stand for such an arrow in trait space.

\section{Distances and lengths of vectors}

Once we have arrows, it is natural to ask how long they are. For a vector
with components $(\Delta x, \Delta y)$, the length is given by the
Pythagorean theorem:
\[
  \|\vect{v}\|
    = \sqrt{(\Delta x)^2 + (\Delta y)^2}.
\]

We can also use this to define the distance between two phenotypes:
draw the arrow from one to the other and take its length. This recovers
the usual Euclidean distance\index[subject]{distance!Euclidean} distance in the plane.

In higher dimensions, the same idea applies. If a vector has components
$(\Delta x_1, \Delta x_2, \dots, \Delta x_p)$, then its length is
\[
  \|\vect{v}\|
    = \sqrt{
        (\Delta x_1)^2 + (\Delta x_2)^2 + \dots + (\Delta x_p)^2
      }.
\]

The square of this length,
\[
  \|\vect{v}\|^2
    = (\Delta x_1)^2 + (\Delta x_2)^2 + \dots + (\Delta x_p)^2,
\]
is a sum of squared components. This will connect directly to variance
and, later, to matrix notation.

\section{Summary}

In this chapter we have:

\begin{itemize}
  \item represented multivariate phenotypes as points in trait space;
  \item represented differences as arrows, and named them vectors;
  \item linked vector length to sums of squared components;
  \item shown how to generate a simple trait space plot with code.
\end{itemize}

In the next chapters we will move from these arrows to distances from a
mean, and from there to variance and covariance.

\section*{Exercises}

\paragraph{Exercise 0.1 (Plotting a phenotype cloud).}
Five plants are measured for leaf length (cm) and leaf width (cm):

\begin{center}
\begin{tabular}{ccc}
\hline
Plant & Length & Width \\
\hline
A & 4.2 & 2.1 \\
B & 5.1 & 2.8 \\
C & 3.8 & 1.9 \\
D & 4.7 & 2.4 \\
E & 4.2 & 2.3 \\
\hline
\end{tabular}
\end{center}

\begin{enumerate}
  \item Plot these five individuals as points in a two-dimensional trait space.
  \item Estimate the centroid (mean phenotype) by eye from your plot.
  \item Calculate the centroid exactly. How close was your estimate?
  \item A sixth plant F has measurements (6.0, 3.5). Add it to your plot.
        How does the centroid shift?
\end{enumerate}

\paragraph{Exercise 0.2 (Trait space dimensions).}
A bird ecologist measures wing length, tarsus length, bill depth, and body
mass on each individual.

\begin{enumerate}
  \item How many dimensions does this trait space have?
  \item Can you visualise this space directly? If not, what strategies
        might help you understand the distribution of individuals?
  \item If you added bill width as a fifth trait, how would the
        dimensionality change?
\end{enumerate}

\paragraph{Exercise 0.3 (Phenotype as position).}
Consider two fish: Fish 1 has length 15 cm and mass 50 g; Fish 2 has
length 20 cm and mass 80 g.

\begin{enumerate}
  \item Represent each fish as a point in (length, mass) space.
  \item Draw the arrow from Fish 1 to Fish 2. What does this arrow
        represent biologically?
  \item If a third fish lies exactly halfway along this arrow, what are
        its length and mass?
\end{enumerate}

\paragraph{Exercise 0.4 (The meaning of ``distance'' in trait space).}
Two flowers differ in petal length by 2 mm and in petal width by 3 mm.

\begin{enumerate}
  \item What is the straight-line (Euclidean) distance between them in
        trait space?
  \item Does this number have a direct biological interpretation?
  \item What might make two flowers ``far apart'' in trait space but
        similar in fitness?
\end{enumerate}
