\chapter{Whitening and the P-sphere\index[subject]{P-sphere}}

In the previous chapter we learned to diagonalise symmetric matrices and
interpret eigenvalues geometrically. Now we apply these tools to a specific
problem: how do we compare genetic and phenotypic variation across
directions in trait space?

This chapter introduces the whitening\index[subject]{whitening transformation} transformation and the concept of the
P-sphere\index[subject]{P-sphere}. These ideas unify several threads from earlier chapters and set
the stage for understanding directional heritability and evolutionary
constraint.

\section{The problem: comparing G and P}

Quantitative genetics gives us two fundamental matrices:

\begin{itemize}
  \item The genetic covariance matrix\index[subject]{covariance matrix} $\mat{G}$, describing heritable
        variation.
  \item The phenotypic covariance matrix\index[subject]{covariance matrix} $\mat{P}$, describing total
        observed variation (genetic plus environmental).
\end{itemize}

For a single trait, heritability\index[subject]{heritability} is the ratio $h^2 = V_G / V_P$. But with
multiple traits, both $\mat{G}$ and $\mat{P}$ are matrices, not scalars.
How do we generalise heritability\index[subject]{heritability} to multiple dimensions?

One approach is to pick a direction $\boldsymbol{\beta}$ in trait space and
ask: what fraction of phenotypic variance in that direction is genetic?
This gives us the \textbf{directional heritability\index[subject]{heritability}}:
\[
  h^2(\boldsymbol{\beta}) 
    = \frac{\boldsymbol{\beta}^\top \mat{G} \boldsymbol{\beta}}
           {\boldsymbol{\beta}^\top \mat{P} \boldsymbol{\beta}}.
\]

The numerator is the genetic variance in direction $\boldsymbol{\beta}$;
the denominator is the phenotypic variance in the same direction. Their
ratio is a number between 0 and 1 (assuming $\mat{G}$ and $\mat{P}$ are
properly estimated).

\begin{keyidea}
Directional heritability\index[subject]{heritability} $h^2(\boldsymbol{\beta})$ measures what fraction
of phenotypic variance is genetic along a specific direction in trait
space. It generalises the scalar heritability\index[subject]{heritability} $h^2 = V_G/V_P$ to multiple
traits.
\end{keyidea}

But here is the difficulty: the value of $h^2(\boldsymbol{\beta})$ depends
on which direction we choose. Some directions may have high heritability\index[subject]{heritability}
(most variation is genetic), while others have low heritability\index[subject]{heritability} (most
variation is environmental). How do we summarise this variation across
directions? And how do we sample directions ``fairly''?

\section{The naive approach and its problem}

A natural idea is to sample directions uniformly from the unit sphere---all
directions equally likely---and compute $h^2(\boldsymbol{\beta})$ for each.

But ``uniform on the unit sphere'' is ambiguous when traits have different
scales. Consider two traits: body mass in kilograms and wing length in
millimetres. A ``uniform'' sample in the original coordinates would be
dominated by directions that emphasise the trait with larger numerical
values.

Even after standardising each trait to have unit variance, there is still
a problem. If traits are correlated, the phenotypic covariance matrix\index[subject]{covariance matrix}
$\mat{P}$ is not the identity. The directions that ``look uniform'' in
Euclidean distance\index[subject]{distance!Euclidean} space are not uniform with respect to phenotypic variation.

\begin{figure}[ht]
  \centering
  \includegraphics[width=0.85\textwidth]{fig21_sphere_vs_psphere.png}
  \caption[The Euclidean distance\index[subject]{distance!Euclidean} sphere versus the P-sphere\index[subject]{P-sphere}]{
    Left: The unit sphere in original coordinates. Uniform sampling here
    ignores the correlation\index[subject]{correlation}structure. Right: The P-sphere\index[subject]{P-sphere}, where points
    are equidistant in Mahalanobis distance\index[subject]{distance!Mahalanobis} distance. Uniform sampling on the
    P-sphere\index[subject]{P-sphere} respects the phenotypic covariance structure.
  }
  \label{fig:sphere-vs-psphere}
\end{figure}

\section{The P-sphere\index[subject]{P-sphere}: uniform with respect to phenotype}

The solution is to define ``uniform'' with respect to the phenotypic
covariance matrix\index[subject]{covariance matrix}. Instead of the Euclidean distance\index[subject]{distance!Euclidean} unit sphere
\[
  \{ \boldsymbol{\beta} : \|\boldsymbol{\beta}\|^2 = 1 \}
    = \{ \boldsymbol{\beta} : \boldsymbol{\beta}^\top \boldsymbol{\beta} = 1 \},
\]
we use the \textbf{P-sphere\index[subject]{P-sphere}}:
\[
  \{ \boldsymbol{\beta} : \boldsymbol{\beta}^\top \mat{P} \boldsymbol{\beta} = 1 \}.
\]

Points on the P-sphere\index[subject]{P-sphere} all have unit phenotypic variance. This is the
natural normalisation for comparing directions: we are asking ``per unit
of phenotypic variance, how much is genetic?''

The P-sphere\index[subject]{P-sphere} is an ellipsoid in the original coordinate system, but it
becomes a true sphere after the whitening\index[subject]{whitening transformation} transformation.

\begin{keyidea}
The P-sphere\index[subject]{P-sphere} is the set of all directions with unit phenotypic variance.
Sampling uniformly from the P-sphere\index[subject]{P-sphere} means treating all phenotypically
equivalent directions equally.
\end{keyidea}

\section{The whitening\index[subject]{whitening transformation} transformation}

In Chapter~12 we saw that the Mahalanobis distance\index[subject]{distance!Mahalanobis} distance can be understood as
Euclidean distance\index[subject]{distance!Euclidean} distance after a whitening\index[subject]{whitening transformation} transformation. Now we develop this
idea systematically.

The whitening\index[subject]{whitening transformation} transformation uses the matrix square root of $\mat{P}^{-1}$.
Define
\[
  \mat{P}^{-1/2} = \mat{V}_P {\Lambda}_P^{-1/2} \mat{V}_P^\top,
\]
where $\mat{V}_P$ contains the eigenvectors of $\mat{P}$ and
${\Lambda}_P$ is the diagonal matrix of eigenvalue\index[subject]{eigenvalue}s. The matrix
${\Lambda}_P^{-1/2}$ has entries $1/\sqrt{\lambda_i}$ on the diagonal.

Apply this transformation to both the genetic and phenotypic matrices:
\begin{align*}
  \mat{P}^* &= \mat{P}^{-1/2} \mat{P} \mat{P}^{-1/2} = \mat{I}, \\
  \mat{G}^* &= \mat{P}^{-1/2} \mat{G} \mat{P}^{-1/2}.
\end{align*}

The phenotypic matrix becomes the identity---this is what ``whitening\index[subject]{whitening transformation}''
means. The genetic matrix becomes $\mat{G}^*$, sometimes called the
\textbf{P-standardised genetic matrix} or the \textbf{G-P matrix}.

\begin{keyidea}
whitening\index[subject]{whitening transformation} by $\mat{P}^{-1/2}$ transforms the phenotypic matrix to the
identity. In whitened space, the P-sphere\index[subject]{P-sphere} becomes the ordinary unit sphere,
and uniform sampling is straightforward.
\end{keyidea}

\section{A remarkable fact: eigenvalue\index[subject]{eigenvalue}s of G* are directional heritabilities}

Here is the key result. In whitened coordinates, let
$\boldsymbol{\beta}^*$ be a unit vector (on the ordinary sphere, which is
now also the P-sphere\index[subject]{P-sphere}). The directional heritability\index[subject]{heritability} is
\begin{align*}
  h^2(\boldsymbol{\beta}^*)
    &= \frac{(\boldsymbol{\beta}^*)^\top \mat{G}^* \boldsymbol{\beta}^*}
            {(\boldsymbol{\beta}^*)^\top \mat{I} \boldsymbol{\beta}^*} \\
    &= (\boldsymbol{\beta}^*)^\top \mat{G}^* \boldsymbol{\beta}^*.
\end{align*}

This is a quadratic form\index[subject]{quadratic form} in $\mat{G}^*$. From Chapter~20, we know that:
\begin{itemize}
  \item The maximum value is the largest eigenvalue\index[subject]{eigenvalue} of $\mat{G}^*$.
  \item The minimum value is the smallest eigenvalue\index[subject]{eigenvalue} of $\mat{G}^*$.
  \item The eigenvector\index[subject]{eigenvector}s of $\mat{G}^*$ are the directions that achieve
        these extremes.
\end{itemize}

\begin{keyidea}
The eigenvalue\index[subject]{eigenvalue}s of $\mat{G}^* = \mat{P}^{-1/2}\mat{G}\mat{P}^{-1/2}$ are
the maximum and minimum directional heritabilities. The eigenvector\index[subject]{eigenvector}s are
the directions that achieve them.
\end{keyidea}

This is a powerful result. It tells us that to understand the range of
possible directional heritabilities, we need only diagonalise $\mat{G}^*$.
The eigenvalue\index[subject]{eigenvalue}s give us the bounds; the eigenvector\index[subject]{eigenvector}s tell us where those
bounds are achieved.

\begin{figure}[htbp]
    \centering
    \includegraphics[width=\textwidth]{figures/fig_ch8_h2_distribution.pdf}
    \caption{Distribution of directional heritability $h^2(\boldsymbol{\beta})$ 
    across directions. (a) The $\mathbf{G}^*$ ellipse (magenta) inside the 
    $\mathbf{P}$-sphere (green) in whitened space. Arrows indicate directions 
    of maximum and minimum heritability, which are the eigenvectors of 
    $\mathbf{G}^*$. (b) Histogram of $h^2$ values from 10,000 random directions 
    sampled uniformly from the $\mathbf{P}$-sphere. The shaded region indicates 
    the ``constraint trap zone'' where heritability is well below average. 
    (c) Heritability as a continuous function of direction angle, showing the 
    180° periodicity (opposite directions have identical $h^2$). The eigenvalues 
    of $\mathbf{G}^*$ bound the distribution.}
    \label{fig:h2_distribution}
\end{figure}

\section{The distribution of directional heritability\index[subject]{heritability}}

If we sample directions uniformly from the P-sphere\index[subject]{P-sphere} (equivalently, the
unit sphere in whitened space), what distribution of $h^2$ values do we
get?

From Chapter~20, the quadratic form\index[subject]{quadratic form}
$(\boldsymbol{\beta}^*)^\top \mat{G}^* \boldsymbol{\beta}^*$
is a weighted average of the eigenvalue\index[subject]{eigenvalue}s $\lambda_i^*$ of $\mat{G}^*$,
with weights given by squared projections onto the eigenvector\index[subject]{eigenvector}s.

For uniform random directions on the sphere, there is a known formula
for the variance of this quadratic form\index[subject]{quadratic form}:
\[
  \Var[h^2(\boldsymbol{\beta})] 
    = \frac{2}{p+2} \cdot \Var(\lambda^*),
\]
where $\Var(\lambda^*)$ is the variance of the eigenvalue\index[subject]{eigenvalue}s of $\mat{G}^*$,
and $p$ is the number of traits.

The coefficient of variation of directional heritability\index[subject]{heritability} is therefore
\[
  \text{CV}[h^2] = \sqrt{\frac{2}{p+2} \cdot V_{\text{rel}}(\mat{G}^*)},
\]
where $V_{\text{rel}}(\mat{G}^*) = \Var(\lambda^*) / \bar{\lambda}^{*2}$
is the relative variance of the eigenvalue\index[subject]{eigenvalue}s.

\begin{keyidea}
The variability of directional heritability\index[subject]{heritability} across directions depends on
two factors:
\begin{enumerate}
  \item The relative variance of the eigenvalue\index[subject]{eigenvalue}s of $\mat{G}^*$---how
        eccentric is the P-standardised genetic ellipsoid?
  \item The number of traits $p$---more traits mean more ``averaging''
        and less variability.
\end{enumerate}
\end{keyidea}

\section{Constraint traps}

A \textbf{constraint trap} occurs when a direction has low heritability\index[subject]{heritability}
despite having substantial phenotypic variance. Selection in that direction
produces little evolutionary response because the genetic variance is low
relative to environmental variance.

In the G* framework, constraint traps correspond to directions near the
eigenvector\index[subject]{eigenvector}s of $\mat{G}^*$ with small eigenvalue\index[subject]{eigenvalue}s. These are directions
where:
\begin{itemize}
  \item The phenotypic variance is typical (by construction, we are on the
        P-sphere\index[subject]{P-sphere}).
  \item The genetic variance is unusually low.
  \item The heritability\index[subject]{heritability} $h^2(\boldsymbol{\beta})$ is near its minimum.
\end{itemize}

\begin{figure}[ht]
  \centering
  \includegraphics[width=0.8\textwidth]{fig21_constraint_trap.png}
  \caption[Constraint traps]{
    In whitened space, the G* ellipse sits inside the P-sphere\index[subject]{P-sphere} (which is
    now the unit circle). Directions where G* is thin relative to the
    sphere are constraint traps: plenty of phenotypic variance, but little
    genetic variance.
  }
  \label{fig:constraint-trap}
\end{figure}

The danger is subtle. A breeder or natural selection might target a
direction with plenty of phenotypic variation, expecting a response. But
if that direction happens to be a constraint trap, the response will be
disappointing---the variation is mostly environmental, not genetic.

\section{Visualising G inside P}

A useful visualisation is to plot the G ellipse and P ellipse together,
centred at the same point. In two dimensions:

\begin{itemize}
  \item The P ellipse shows where phenotypic variation extends.
  \item The G ellipse shows where genetic variation extends.
  \item Directions where G is thin relative to P are low-heritability\index[subject]{heritability}
        directions.
  \item Directions where G nearly fills P are high-heritability\index[subject]{heritability} directions.
\end{itemize}

After whitening\index[subject]{whitening transformation}, P becomes the unit circle. The G* ellipse sits inside it
(assuming $h^2 \le 1$ in all directions). The shape of G* relative to the
circle reveals the constraint structure.

\begin{figure}[ht]
  \centering
  \includegraphics[width=0.85\textwidth]{fig21_g_inside_p.png}
  \caption[G inside P]{
    Left: In original coordinates, both G and P are ellipses. Right: After
    whitening\index[subject]{whitening transformation}, P becomes the unit circle, and G* reveals where heritability\index[subject]{heritability}
    is high (G* close to the circle) or low (G* far inside).
  }
  \label{fig:g-inside-p}
\end{figure}

\section{Connection to the breeder's equation\index[subject]{breeder's equation} equation}

Recall the multivariate breeder's equation\index[subject]{breeder's equation} equation:
\[
  \Delta\bar{\vect{z}} = \mat{G}\mat{P}^{-1}\vect{S} = \mat{G}\boldsymbol{\beta},
\]
where $\vect{S}$ is the selection differential and
$\boldsymbol{\beta} = \mat{P}^{-1}\vect{S}$ is the selection gradient.

The response to selection depends on both $\mat{G}$ and the direction of
$\boldsymbol{\beta}$. If $\boldsymbol{\beta}$ points in a high-heritability\index[subject]{heritability}
direction (large $\boldsymbol{\beta}^\top\mat{G}\boldsymbol{\beta}$ relative
to $\boldsymbol{\beta}^\top\mat{P}\boldsymbol{\beta}$), the response is
strong. If it points in a constraint trap, the response is weak.

The P-whitening\index[subject]{whitening transformation} framework makes this explicit. In whitened coordinates:
\[
  \Delta\bar{\vect{z}}^* = \mat{G}^* \boldsymbol{\beta}^*.
\]

The response in whitened space is simply $\mat{G}^*$ acting on the
whitened selection gradient. The eigenstructure of $\mat{G}^*$ directly
determines how selection translates to response.

\section{Computing G* in practice}

Given estimates of $\mat{G}$ and $\mat{P}$, here is how to compute
$\mat{G}^*$:

\paragraph{Step 1: Eigendecompose P.}
\[
  \mat{P} = \mat{V}_P{\Lambda}_P \mat{V}_P^\top.
\]

\paragraph{Step 2: Compute $\mat{P}^{-1/2}$.}
\[
  \mat{P}^{-1/2} = \mat{V}_P {\Lambda}_P^{-1/2} \mat{V}_P^\top,
\]
where ${\Lambda}_P^{-1/2}$ has diagonal entries $1/\sqrt{\lambda_i}$.

\paragraph{Step 3: Transform G.}
\[
  \mat{G}^* = \mat{P}^{-1/2} \mat{G} \mat{P}^{-1/2}.
\]

\paragraph{Step 4: Eigendecompose G*.}
The eigenvalue\index[subject]{eigenvalue}\index[subject]{eigenvalue}s of $\mat{G}^*$ are the directional heritabilities along
the principal axes. The eigenvector\index[subject]{eigenvector}s (transformed back to original
coordinates by $\mat{P}^{1/2}$) are those principal axes.

In R:
\begin{verbatim}
# Eigendecompose P
eig_P <- eigen(P)
V_P <- eig_P$vectors
Lambda_P <- diag(eig_P$values)

# Compute P^{-1/2}
P_inv_sqrt <- V_P %*% diag(1/sqrt(eig_P$values)) %*% t(V_P)

# Transform G
G_star <- P_inv_sqrt %*% G %*% P_inv_sqrt

# Eigendecompose G*
eig_Gstar <- eigen(G_star)
# eig_Gstar$values are directional heritabilities
\end{verbatim}

\section{Why whitening\index[subject]{whitening transformation} matters}

The whitening\index[subject]{whitening transformation} transformation is not just a mathematical convenience. It
changes how we think about constraint.

Without whitening\index[subject]{whitening transformation}, we might compare directions using Euclidean distance\index[subject]{distance!Euclidean} angles and
conclude that a direction is ``close to $\mathbf{g}_{\max}$'' when it is
actually far from it in phenotypic terms. whitening\index[subject]{whitening transformation} ensures that our
notion of ``close'' respects the phenotypic covariance structure.

It also simplifies sampling. To study the distribution of heritability\index[subject]{heritability}
across directions, we can sample uniformly from the ordinary sphere in
whitened space, which is easy. Sampling uniformly from the P-sphere\index[subject]{P-sphere} in
original coordinates would require accounting for the elliptical geometry.

\begin{keyidea}
whitening\index[subject]{whitening transformation} by $\mat{P}$ is the multivariate generalisation of dividing by
the phenotypic standard deviation. It puts all directions on an equal
footing, so that comparisons are fair.
\end{keyidea}

\section{Summary}

In this chapter we have:

\begin{itemize}
  \item Introduced directional heritability\index[subject]{heritability} $h^2(\boldsymbol{\beta})$ as
        the ratio of genetic to phenotypic variance in a given direction.
  \item Defined the P-sphere\index[subject]{P-sphere} as the set of directions with unit phenotypic
        variance, and explained why sampling uniformly from the P-sphere\index[subject]{P-sphere}
        is the right notion of ``uniform.''
  \item Developed the whitening\index[subject]{whitening transformation} transformation using $\mat{P}^{-1/2}$,
        which converts the P-sphere\index[subject]{P-sphere} to the ordinary unit sphere.
  \item Shown that the eigenvalue\index[subject]{eigenvalue}s of $\mat{G}^* = \mat{P}^{-1/2}\mat{G}\mat{P}^{-1/2}$
        are the extreme directional heritabilities, and the eigenvector\index[subject]{eigenvector}s
        are the directions that achieve them.
  \item Connected the variance of directional heritability\index[subject]{heritability} to the relative
        variance of the eigenvalue\index[subject]{eigenvalue}s of $\mat{G}^*$.
  \item Defined constraint traps as directions where phenotypic variance
        is normal but genetic variance (and hence heritability\index[subject]{heritability}) is low.
  \item Provided practical code for computing $\mat{G}^*$ from estimates
        of $\mat{G}$ and $\mat{P}$.
\end{itemize}

The whitening\index[subject]{whitening transformation} framework unifies the geometry of G and P into a single
picture. In Part~IV, we will apply these ideas to real biological
questions: the G matrix and its eigenstructure, fitness surfaces, and
the analysis of selection and response.
