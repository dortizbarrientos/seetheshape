\chapter{Diagonalisation\index[subject]{diagonalisation} and Natural Axes}

In Part~II we saw that covariance matrices define ellipses and that the Mahalanobis distance\index[subject]{distance!Mahalanobis} distance uses the inverse covariance matrix\index[subject]{covariance matrix} to measure ``unusualness.'' But we repeatedly invoked eigenvalue\index[subject]{eigenvalue}s and eigenvectors without explaining how to find them or what they mean.

This chapter fills that gap. We develop diagonalisation\index[subject]{diagonalisation}---the process of finding the natural axes of an ellipse---from first principles. By the end, you will understand eigenvalue\index[subject]{eigenvalue}s and eigenvector\index[subject]{eigenvector}s not as abstract algebra but as answers to a concrete geometric question: \emph{in which directions does a matrix act as pure stretching?}

\section{The question that leads to eigenvalue\index[subject]{eigenvalue}s}

Consider a symmetric matrix $\mat{A}$ acting on vectors in the plane. For most vectors $\vect{v}$, the output $\mat{A}\vect{v}$ points in a different direction from $\vect{v}$. The matrix rotates as well as stretches.

But for some special vectors, the output points in the \emph{same direction} as the input---or exactly opposite. The matrix stretches (or compresses) without rotating. These special vectors are called \textbf{eigenvector\index[subject]{eigenvector}s},
and the stretching factors are called \textbf{eigenvalue\index[subject]{eigenvalue}s}.

Formally, $\vect{v}$ is an eigenvector\index[subject]{eigenvector} of $\mat{A}$ with eigenvalue\index[subject]{eigenvalue}
$\lambda$ if
\[
  \mat{A}\vect{v} = \lambda\vect{v}.
\]

The matrix $\mat{A}$ acting on $\vect{v}$ produces the same result as simply multiplying $\vect{v}$ by the scalar $\lambda$.

\begin{keyidea}
eigenvector\index[subject]{eigenvector}s are the directions along which a matrix acts as pure scaling. eigenvalue\index[subject]{eigenvalue}s are the scaling factors. Finding them reveals the natural axes of the transformation.
\end{keyidea}

\section{A concrete example}

Let us find the eigenvector\index[subject]{eigenvector}s and eigenvalue\index[subject]{eigenvalue}s of
\[
  \mat{A} =
  \begin{pmatrix}
    3 & 1 \\
    1 & 3
  \end{pmatrix}.
\]

We seek vectors $\vect{v}$ and scalars $\lambda$ such that
$\mat{A}\vect{v} = \lambda\vect{v}$.

Rearranging:
\[
  \mat{A}\vect{v} - \lambda\vect{v} = \vect{0}
  \quad\Rightarrow\quad
  (\mat{A} - \lambda\mat{I})\vect{v} = \vect{0}.
\]

For a non-zero solution $\vect{v}$ to exist, the matrix
$(\mat{A} - \lambda\mat{I})$ must be singular. This happens when its determinant is zero:
\[
  \det(\mat{A} - \lambda\mat{I}) = 0.
\]

This is the \textbf{characteristic equation\index[subject]{characteristic equation}}. For our matrix:
\[
  \det
  \begin{pmatrix}
    3 - \lambda & 1 \\
    1 & 3 - \lambda
  \end{pmatrix}
  = (3 - \lambda)^2 - 1
  = \lambda^2 - 6\lambda + 8
  = (\lambda - 4)(\lambda - 2)
  = 0.
\]

The eigenvalue\index[subject]{eigenvalue}s are $\lambda_1 = 4$ and $\lambda_2 = 2$.

\subsection*{Finding the eigenvector\index[subject]{eigenvector}s}

For $\lambda_1 = 4$:
\[
  (\mat{A} - 4\mat{I})\vect{v} =
  \begin{pmatrix}
    -1 & 1 \\
    1 & -1
  \end{pmatrix}
  \begin{pmatrix}
    v_1 \\ v_2
  \end{pmatrix}
  = \vect{0}.
\]

This gives $-v_1 + v_2 = 0$, so $v_1 = v_2$. The eigenvector\index[subject]{eigenvector} (normalised to unit length) is
\[
  \vect{v}_1 = \frac{1}{\sqrt{2}}
  \begin{pmatrix}
    1 \\ 1
  \end{pmatrix}.
\]

For $\lambda_2 = 2$:
\[
  (\mat{A} - 2\mat{I})\vect{v} =
  \begin{pmatrix}
    1 & 1 \\
    1 & 1
  \end{pmatrix}
  \begin{pmatrix}
    v_1 \\ v_2
  \end{pmatrix}
  = \vect{0}.
\]

This gives $v_1 + v_2 = 0$, so $v_2 = -v_1$. The eigenvector\index[subject]{eigenvector} is
\[
  \vect{v}_2 = \frac{1}{\sqrt{2}}
  \begin{pmatrix}
    1 \\ -1
  \end{pmatrix}.
\]

\begin{figure}[ht]
  \centering
  \includegraphics[width=0.8\textwidth]{fig20_eigenvectors.png}
  \caption[Eigenvectors and eigenvalue\index[subject]{eigenvalue}s]{
    The matrix $\mat{A}$ stretches space by factor 4 along $\vect{v}_1$ (the diagonal where both traits increase together) and by factor 2 along $\vect{v}_2$ (the diagonal where traits move in opposite directions). These are the natural axes of the transformation.
  }
  \label{fig:eigenvectors}
\end{figure}

Notice that $\vect{v}_1$ and $\vect{v}_2$ are perpendicular (their dot product is zero). This is not a coincidence---it is guaranteed for symmetric matrices.

\section{The spectral theorem: why symmetric matrices are special}

Symmetric matrices have three remarkable properties that make them central to statistics and biology:

\begin{enumerate}
  \item \textbf{Real eigenvalue\index[subject]{eigenvalue}s.} The eigenvalue\index[subject]{eigenvalue}s of a symmetric matrix are always real numbers, never complex.
  \item \textbf{Orthogonal eigenvector\index[subject]{eigenvector}s.} eigenvector\index[subject]{eigenvector}s corresponding to different eigenvalue\index[subject]{eigenvalue}s are perpendicular.
  \item \textbf{Complete set.} A $p \times p$ symmetric matrix always has $p$ eigenvector\index[subject]{eigenvector}s that form an orthonormal basis for $\mathbb{R}^p$.
\end{enumerate}

These properties are collectively known as the \textbf{spectral theorem}.

\begin{keyidea}
Every symmetric matrix can be understood as pure stretching along perpendicular axes. There is no rotation mixed in, no shear, no reflection---just stretching (or compressing) along $p$ orthogonal directions.
\end{keyidea}

Covariance matrices are symmetric by construction (the covariance of $X$ with $Y$ equals the covariance of $Y$ with $X$). This is why ellipses, not parallelograms, describe their geometry.

\section{diagonalisation\index[subject]{diagonalisation}: the matrix factorisation}

Collect the eigenvector\index[subject]{eigenvector}s as columns of a matrix $\mat{V}$:
\[
  \mat{V} =
  \begin{pmatrix}
    | & | \\
    \vect{v}_1 & \vect{v}_2 \\
    | & |
  \end{pmatrix}.
\]

For our example:
\[
  \mat{V} = \frac{1}{\sqrt{2}}
  \begin{pmatrix}
    1 & 1 \\
    1 & -1
  \end{pmatrix}.
\]

Collect the eigenvalue\index[subject]{eigenvalue}s in a diagonal matrix ${\Lambda}$:
\[
  {\Lambda} =
  \begin{pmatrix}
    \lambda_1 & 0 \\
    0 & \lambda_2
  \end{pmatrix}
  =
  \begin{pmatrix}
    4 & 0 \\
    0 & 2
  \end{pmatrix}.
\]

The spectral theorem guarantees that
\[
  \mat{A} = \mat{V}{\Lambda}\mat{V}^\top.
\]

This is the \textbf{eigendecomposition} or \textbf{spectral decomposition}
of $\mat{A}$.

\begin{keyidea}
The eigendecomposition $\mat{A} = \mat{V}{\Lambda}\mat{V}^\top$
factorises a symmetric matrix into three parts:
\begin{itemize}
  \item $\mat{V}^\top$: rotate from original axes to eigenvector\index[subject]{eigenvector} axes;
  \item ${\Lambda}$: stretch along each eigenvector\index[subject]{eigenvector} axis;
  \item $\mat{V}$: rotate back to original axes.
\end{itemize}
\end{keyidea}

Let us verify this for our example:
\begin{align*}
  \mat{V}{\Lambda}\mat{V}^\top
  &= \frac{1}{\sqrt{2}}
     \begin{pmatrix} 1 & 1 \\ 1 & -1 \end{pmatrix}
     \begin{pmatrix} 4 & 0 \\ 0 & 2 \end{pmatrix}
     \frac{1}{\sqrt{2}}
     \begin{pmatrix} 1 & 1 \\ 1 & -1 \end{pmatrix} \\
  &= \frac{1}{2}
     \begin{pmatrix} 1 & 1 \\ 1 & -1 \end{pmatrix}
     \begin{pmatrix} 4 & 4 \\ 2 & -2 \end{pmatrix} \\
  &= \frac{1}{2}
     \begin{pmatrix} 6 & 2 \\ 2 & 6 \end{pmatrix}
  = \begin{pmatrix} 3 & 1 \\ 1 & 3 \end{pmatrix}
  = \mat{A}. \quad\checkmark
\end{align*}

\begin{figure}[htbp]
    \centering
    \includegraphics[width=\textwidth]{figures/fig_ch7_rotate_stretch_rotate.pdf}
    \caption{\textbf{The eigendecomposition as rotate--stretch--rotate.} The 
    matrix $\mat{A} = \mat{V}\mat{\Lambda}\mat{V}^\top$ acts in three stages. 
    (a)~Original unit circle with eigenvectors marked (dashed). 
    (b)~Multiply by $\mat{V}^\top$: rotate the coordinate system so that the 
    eigenvectors align with the axes. (c)~Multiply by $\mat{\Lambda}$: stretch 
    by $\lambda_1 = 4$ along the first axis and $\lambda_2 = 2$ along the 
    second. (d)~Multiply by $\mat{V}$: rotate back to the original coordinates. 
    The unit circle becomes an ellipse whose axes align with the eigenvectors 
    and whose semi-axis lengths are $\sqrt{\lambda_1}$ and $\sqrt{\lambda_2}$.}
    \label{fig:rotate-stretch-rotate}
\end{figure}

\section{Geometric interpretation: the ellipse revealed}

Now we can see exactly what a covariance matrix\index[subject]{covariance matrix} describes geometrically.

If ${\Sigma}$ is a covariance matrix\index[subject]{covariance matrix} with eigendecomposition
${\Sigma} = \mat{V}{\Lambda}\mat{V}^\top$, then:

\begin{itemize}
  \item The \textbf{eigenvector\index[subject]{eigenvector}s} $\vect{v}_1, \vect{v}_2, \ldots, \vect{v}_p$
        are the directions of the principal axes of the covariance ellipse.
  \item The \textbf{eigenvalue\index[subject]{eigenvalue}s} $\lambda_1, \lambda_2, \ldots, \lambda_p$
        are the variances along those axes.
  \item The \textbf{semi-axis lengths} of the ellipse are
        $\sqrt{\lambda_1}, \sqrt{\lambda_2}, \ldots, \sqrt{\lambda_p}$.
\end{itemize}

\begin{figure}[ht]
  \centering
  \includegraphics[width=0.8\textwidth]{fig20_ellipse_axes.png}
  \caption[eigenvalue\index[subject]{eigenvalue}s and eigenvector\index[subject]{eigenvector}s of a covariance matrix\index[subject]{covariance matrix}]{
    A covariance matrix\index[subject]{covariance matrix} defines an ellipse. The eigenvector\index[subject]{eigenvector}s point along the principal axes; the eigenvalue\index[subject]{eigenvalue}s are the variances (squared semi-axis lengths) along those axes.
  }
  \label{fig:ellipse-axes}
\end{figure}

For the covariance matrix\index[subject]{covariance matrix} from Chapter~12,
\[
  {\Sigma} =
  \begin{pmatrix}
    1.0 & 0.8 \\
    0.8 & 1.0
  \end{pmatrix},
\]
the eigenvalue\index[subject]{eigenvalue}s are $\lambda_1 = 1.8$ and $\lambda_2 = 0.2$. The ratio $\lambda_1 / \lambda_2 = 9$ tells us the ellipse is elongated: its major axis is three times longer than its minor axis ($\sqrt{9} = 3$).

The eigenvector\index[subject]{eigenvector}s are at 45° angles to the original axes---one along the direction where both traits increase together (the correlation\index[subject]{correlation}direction), and one perpendicular to it.

\section{Why diagonalisation\index[subject]{diagonalisation} simplifies everything}

In the eigenvector\index[subject]{eigenvector} coordinate system, the covariance matrix\index[subject]{covariance matrix} becomes diagonal:
\[
  {\Lambda} =
  \begin{pmatrix}
    \lambda_1 & 0 & \cdots & 0 \\
    0 & \lambda_2 & \cdots & 0 \\
    \vdots & & \ddots & \vdots \\
    0 & 0 & \cdots & \lambda_p
  \end{pmatrix}.
\]

A diagonal matrix is trivial to work with:

\begin{itemize}
  \item \textbf{Inverse:} ${\Lambda}^{-1}$ has entries
        $1/\lambda_1, 1/\lambda_2, \ldots$
  \item \textbf{Square root:} ${\Lambda}^{1/2}$ has entries
        $\sqrt{\lambda_1}, \sqrt{\lambda_2}, \ldots$
  \item \textbf{Powers:} ${\Lambda}^k$ has entries
        $\lambda_1^k, \lambda_2^k, \ldots$
  \item \textbf{Determinant:} $\det({\Lambda}) = \lambda_1 \lambda_2
        \cdots \lambda_p$
  \item \textbf{Trace:} $\tr({\Lambda}) = \lambda_1 + \lambda_2 +
        \cdots + \lambda_p$
\end{itemize}

Since ${\Sigma} = \mat{V}{\Lambda}\mat{V}^\top$, these operations
extend to the original matrix. For example:
\[
  {\Sigma}^{-1} = \mat{V}{\Lambda}^{-1}\mat{V}^\top,
  \qquad
  {\Sigma}^{1/2} = \mat{V}{\Lambda}^{1/2}\mat{V}^\top.
\]

\begin{keyidea}
diagonalisation\index[subject]{diagonalisation} converts hard matrix problems into easy scalar problems. Once you know the eigenvalue\index[subject]{eigenvalue}s and eigenvector\index[subject]{eigenvector}s, operations like inversion, taking square roots, and computing powers become simple.
\end{keyidea}

\section{The trace and determinant as summaries}

Two numbers summarise the eigenvalue\index[subject]{eigenvalue} spectrum:

\paragraph{The trace.}
The trace of a matrix is the sum of its diagonal entries, which equals the sum of its eigenvalue\index[subject]{eigenvalue}s:
\[
  \tr({\Sigma}) = \sigma_1^2 + \sigma_2^2 + \cdots + \sigma_p^2
    = \lambda_1 + \lambda_2 + \cdots + \lambda_p.
\]

For a covariance matrix\index[subject]{covariance matrix}, the trace is the \textbf{total variance}---the sum of variances across all traits. Geometrically, it measures the overall ``size'' of the ellipse.

\paragraph{The determinant.}
The determinant is the product of the eigenvalue\index[subject]{eigenvalue}s:
\[
  \det({\Sigma}) = \lambda_1 \lambda_2 \cdots \lambda_p.
\]

Geometrically, the determinant is the squared volume of the ellipsoid. For a covariance matrix\index[subject]{covariance matrix}, it measures \textbf{generalised variance}---a single number capturing the overall spread, accounting for correlations.

If any eigenvalue\index[subject]{eigenvalue} is zero, the determinant is zero, indicating that the data lie in a lower-dimensional subspace. The matrix is then singular and
cannot be inverted.

\section{Variance in any direction: the quadratic form\index[subject]{quadratic form} revisited}

In Chapter~2 we introduced the quadratic form\index[subject]{quadratic form} $\vect{v}^\top{\Sigma}\vect{v}$. Now we can interpret it precisely.

Let $\boldsymbol{\beta}$ be a unit vector representing a direction in trait space. The variance of the population in that direction is
\[
  \sigma^2_{\boldsymbol{\beta}} = \boldsymbol{\beta}^\top {\Sigma} \boldsymbol{\beta}.
\]

Using the eigendecomposition ${\Sigma} = \mat{V}{\Lambda}\mat{V}^\top$:
\begin{align*}
  \boldsymbol{\beta}^\top {\Sigma} \boldsymbol{\beta}
    &= \boldsymbol{\beta}^\top \mat{V}{\Lambda}\mat{V}^\top \boldsymbol{\beta} \\
    &= (\mat{V}^\top\boldsymbol{\beta})^\top {\Lambda} (\mat{V}^\top\boldsymbol{\beta}).
\end{align*}

Let $\vect{c} = \mat{V}^\top\boldsymbol{\beta}$ be the coordinates of $\boldsymbol{\beta}$ in the eigenvector\index[subject]{eigenvector} basis. Then
\[
  \boldsymbol{\beta}^\top {\Sigma} \boldsymbol{\beta}
    = \vect{c}^\top {\Lambda} \vect{c}
    = \sum_{i=1}^{p} \lambda_i c_i^2.
\]

\begin{keyidea}
The variance in direction $\boldsymbol{\beta}$ is a weighted average of the eigenvalue\index[subject]{eigenvalue}s, with weights $c_i^2 = (\boldsymbol{\beta} \cdot \vect{v}_i)^2$ ---the squared projections of $\boldsymbol{\beta}$ onto the eigenvector\index[subject]{eigenvector}s.
\end{keyidea}

This formula explains why:
\begin{itemize}
  \item Variance is maximised ($= \lambda_1$) when $\boldsymbol{\beta}$ 
        aligns with $\vect{v}_1$, the first eigenvector\index[subject]{eigenvector}.
  \item Variance is minimised ($= \lambda_p$) when $\boldsymbol{\beta}$
        aligns with $\vect{v}_p$, the last eigenvector\index[subject]{eigenvector}.
  \item For any other direction, variance is intermediate.
\end{itemize}

The eigenvalue\index[subject]{eigenvalue}s bound the variance: $\lambda_{\min} \le \boldsymbol{\beta}^\top{\Sigma}\boldsymbol{\beta} \le \lambda_{\max}$.

\section{Principal Component Analysis (PCA\index[subject]{PCA}) in one paragraph}

PCA\index[subject]{PCA} is simply this: project the data onto the eigenvector\index[subject]{eigenvector}s of the covariance matrix\index[subject]{covariance matrix}. The first principal component is the projection onto $\vect{v}_1$; it captures the direction of maximum variance. The second principal component is the projection onto $\vect{v}_2$; it captures the most variance in the subspace orthogonal to the first. And so on.

The eigenvalue\index[subject]{eigenvalue}s tell you how much variance each component captures. If $\lambda_1$ is much larger than the others, the first principal component carries most of the information, and the data are approximately one-dimensional despite having $p$ measured traits.

We will explore PCA\index[subject]{PCA} and related methods in Chapter~32.

\section{Computing eigenvalue\index[subject]{eigenvalue}s and eigenvector\index[subject]{eigenvector}s in practice}

For $2 \times 2$ matrices, you can solve the characteristic equation\index[subject]{characteristic equation} by hand. For larger matrices, use numerical algorithms. In R:

\begin{verbatim}
Sigma <- matrix(c(1.0, 0.8, 0.8, 1.0), nrow = 2)
eig <- eigen(Sigma)
eig$values      # eigenvalues
eig$vectors     # eigenvectors (columns)
\end{verbatim}

In Python:

\begin{verbatim}
import numpy as np
Sigma = np.array([[1.0, 0.8], [0.8, 1.0]])
eigenvalues, eigenvectors = np.linalg.eigh(Sigma)
\end{verbatim}

Note: \texttt{eigh} is for symmetric (Hermitian) matrices and guarantees real eigenvalue\index[subject]{eigenvalue}s and orthogonal eigenvector\index[subject]{eigenvector}s. Use it for covariance matrices.

\section{A biological example: the G matrix}

The additive genetic covariance matrix\index[subject]{covariance matrix} $\mat{G}$ describes the genetic architecture underlying multiple traits. Its eigendecomposition reveals:

\begin{itemize}
  \item \textbf{$\mathbf{g}_{\max}$:} The first eigenvector\index[subject]{eigenvector}, pointing in
        the direction of maximum genetic variance. This is the ``line of
        least resistance''---the direction evolution finds easiest.
  \item \textbf{Higher eigenvector\index[subject]{eigenvector}s:} Directions of progressively less
        genetic variance. Evolution in these directions requires stronger
        selection to achieve the same response.
  \item \textbf{eigenvalue\index[subject]{eigenvalue}s:} The genetic variances along each principal
        axis. Large eigenvalue\index[subject]{eigenvalue} ratios indicate that $\mat{G}$ strongly
        channels evolution along certain directions.
\end{itemize}

\begin{figure}[ht]
  \centering
  \includegraphics[width=0.8\textwidth]{fig20_gmax.png}
  \caption[The G matrix and $\mathbf{g}_{\max}$]{
    The G matrix defines a genetic ellipse. The first eigenvector\index[subject]{eigenvector}
    $\mathbf{g}_{\max}$ points along the direction of maximum genetic
    variance---the ``line of least evolutionary resistance.''
  }
  \label{fig:gmax}
\end{figure}

When the G matrix is highly eccentric (eigenvalue\index[subject]{eigenvalue}s very unequal), the population can respond quickly to selection along $\mathbf{g}_{\max}$ but responds sluggishly---or not at all---to selection perpendicular to it. This is the geometric foundation of evolutionary constraint.

\section{positive definite\index[subject]{matrix!positive definite}ness and what eigenvalue\index[subject]{eigenvalue}s tell us}

A matrix is \textbf{positive definite\index[subject]{matrix!positive definite}} if all its eigenvalue\index[subject]{eigenvalue}s are strictly positive. For covariance matrices, this means:

\begin{itemize}
  \item Every direction has positive variance.
  \item The matrix can be inverted.
  \item The ellipse is a proper ellipse, not degenerate.
\end{itemize}

A matrix is \textbf{positive semi-definite\index[subject]{matrix!positive semi-definite}} if all eigenvalue\index[subject]{eigenvalue}s are non-negative (some may be zero). A zero eigenvalue\index[subject]{eigenvalue} means:

\begin{itemize}
  \item Some linear combination of traits has zero variance.
  \item The data lie in a lower-dimensional subspace.
  \item The matrix cannot be inverted (it is singular).
\end{itemize}

In practice, estimated covariance matrices from finite samples may have small or even slightly negative eigenvalues due to sampling error. This can cause numerical problems and may require regularisation.

\section{The condition number: how ``ill-behaved'' is the matrix?}

The ratio of largest to smallest eigenvalue\index[subject]{eigenvalue} is the \textbf{condition number}:
\[
  \kappa = \frac{\lambda_{\max}}{\lambda_{\min}}.
\]

A large condition number indicates:

\begin{itemize}
  \item The ellipse is highly elongated.
  \item The matrix is nearly singular.
  \item Numerical computations (like inversion) may be unstable.
  \item Small errors in estimating the matrix can cause large errors in
        derived quantities.
\end{itemize}

In evolutionary terms, a G matrix with large condition number channels evolution strongly along certain directions. In statistical terms, it makes estimation difficult.

\section{Summary}

In this chapter we have:

\begin{itemize}
  \item Defined eigenvalues and eigenvector\index[subject]{eigenvector}s as the answers to ``in which directions does a matrix act as pure scaling?''
  \item Worked through a complete example: finding eigenvalue\index[subject]{eigenvalue}s from the characteristic equation\index[subject]{characteristic equation}, then finding eigenvector\index[subject]{eigenvector}s.
  \item Stated the spectral theorem: symmetric matrices have real eigenvalue\index[subject]{eigenvalue}s and orthogonal eigenvector\index[subject]{eigenvector}s.
  \item Introduced the eigendecomposition
        $\mat{A} = \mat{V}{\Lambda}\mat{V}^\top$ and interpreted it as rotate--stretch--rotate-back.
  \item Connected eigenvalues to ellipse geometry: eigenvector\index[subject]{eigenvector}s are axes, eigenvalue\index[subject]{eigenvalue}s are variances along those axes.
  \item Showed that variance in any direction is a weighted average of eigenvalue\index[subject]{eigenvalue}s, with weights given by squared projections.
  \item Applied these ideas to the G matrix: $\mathbf{g}_{\max}$ is the direction of maximum genetic variance, and eigenvalue\index[subject]{eigenvalue} ratios quantify constraint.
\end{itemize}

We now have the tools to understand any symmetric matrix geometrically. In the next chapter, we use diagonalisation\index[subject]{diagonalisation} to construct the whitening\index[subject]{whitening transformation} transformation---the key to understanding directional heritability\index[subject]{heritability} and the P-sphere\index[subject]{P-sphere}.

%%%%%%%%%%%%%%%%%%%%%%%%%%%%%%%%%%%%%%%%%%%%%%%%%%%%%%%%%%%%%%%%%%%%%%%%%%%%%%%
% CHAPTER 32: PCA, MANOVA, AND PROJECTIONS
%%%%%%%%%%%%%%%%%%%%%%%%%%%%%%%%%%%%%%%%%%%%%%%%%%%%%%%%%%%%%%%%%%%%%%%%%%%%%%%

\section*{Exercises}

\paragraph{Exercise 32.1 (PCA by hand).}
Consider the covariance matrix
\[
  \mat{S} = \begin{pmatrix} 5 & 3 \\ 3 & 5 \end{pmatrix}.
\]

\begin{enumerate}
  \item Find the eigenvalues of $\mat{S}$.
  \item Find the eigenvectors of $\mat{S}$ (normalised to unit length).
  \item What proportion of total variance does PC1 explain?
  \item Interpret PC1 and PC2 in terms of the original traits.
\end{enumerate}

\paragraph{Exercise 32.2 (When all loadings are positive).}
In many morphological data sets, PC1 has all positive loadings.

\begin{enumerate}
  \item What biological interpretation does this suggest?
  \item Give an example of a trait set where you would \emph{not} expect
        all PC1 loadings to be positive.
  \item If PC1 is ``size,'' what does PC2 typically represent?
\end{enumerate}

\paragraph{Exercise 32.3 (Covariance vs.\ correlation PCA).}
A data set has three traits with very different variances:
\begin{itemize}
  \item Trait A: variance = 100
  \item Trait B: variance = 10
  \item Trait C: variance = 1
\end{itemize}

\begin{enumerate}
  \item If you run PCA on the covariance matrix, which trait will
        dominate PC1?
  \item What happens if you standardise each trait to unit variance
        before computing PCA (equivalently, PCA on the correlation matrix)?
  \item When is covariance-based PCA appropriate? When is correlation-based
        PCA better?
\end{enumerate}

\paragraph{Exercise 32.4 (MANOVA intuition).}
Two species of iris are measured for sepal length and sepal width. The
within-species variation forms two overlapping ellipses; the between-species
variation is the distance between their centroids.

\begin{enumerate}
  \item Sketch this scenario with two partially overlapping ellipses.
  \item MANOVA compares the ``size'' of between-group variation to
        within-group variation. In your sketch, are the groups well
        separated?
  \item How would the separation change if the within-group ellipses
        were narrower?
  \item How would it change if the centroids were farther apart?
\end{enumerate}

\paragraph{Exercise 32.5 (Discriminant analysis).}
Using the iris scenario from Exercise 32.4:

\begin{enumerate}
  \item The first discriminant function (DF1) is the direction that
        maximises between-group variance relative to within-group variance.
        Sketch the direction of DF1 on your diagram.
  \item If you project all individuals onto DF1, what do you expect the
        resulting univariate distributions to look like?
  \item How does DF1 relate to the eigenvectors of $\mat{W}^{-1}\mat{B}$?
  \item If there are $g$ groups, what is the maximum number of non-trivial
        discriminant functions?
\end{enumerate}

\paragraph{Exercise 32.6 (Scree plots).}
A PCA of 10 traits yields eigenvalues: 5.2, 2.1, 1.0, 0.6, 0.4, 0.3, 0.2,
0.1, 0.08, 0.02.

\begin{enumerate}
  \item Compute the proportion of variance explained by each PC.
  \item Compute the cumulative proportion of variance.
  \item Sketch a scree plot (eigenvalue vs.\ PC number).
  \item How many PCs would you retain? Justify your choice.
  \item What percentage of total variance do your retained PCs explain?
\end{enumerate}
